\documentclass[12pt, a4paper]{article}

% Avoid useless warnings about included images
% Reference: https://tex.stackexchange.com/a/78020
%\pdfsuppresswarningpagegroup=1

% Use \hypersetup here to get rid of the ugly boxes around links
\usepackage{hyperref}
\hypersetup{
  colorlinks=true,
  linkcolor=blue,
  urlcolor=blue
}

% Avoid warning about missing font for \textbackslash character.
\usepackage[T1]{fontenc}

% For nicer paragraph spacing
\usepackage{parskip}

\usepackage[disable]{todonotes}
%\usepackage{todonotes}

\usepackage{amsmath}%
\usepackage{amsfonts}%
\usepackage{amssymb}%
\usepackage{graphicx}
%\usepackage[miktex]{gnuplottex}
%\ShellEscapetrue
\usepackage{epstopdf}
\usepackage{longtable}
\usepackage{floatrow}
\usepackage{makecell}

% Use \setminted here to get horizontal line above and below listings
\usepackage{minted}
\setminted{
   frame=lines,
   framesep=2mm
}

\usepackage{textcomp}
\usepackage{color,soul}
\usepackage[font={small,it}]{caption}
\floatsetup[listing]{style=Plaintop}    
\floatsetup[longlisting]{style=Plaintop}    

% Turn off indentation but allow \indent command to still work.
\newlength\tindent
\setlength{\tindent}{\parindent}
\setlength{\parindent}{0pt}
\renewcommand{\indent}{\hspace*{\tindent}}

\addtolength{\textwidth}{0.8in}
\addtolength{\oddsidemargin}{-.4in}
\addtolength{\evensidemargin}{-.4in}
\addtolength{\textheight}{1.6in}
\addtolength{\topmargin}{-.8in}

\usepackage{longtable,supertabular}
\usepackage{listings}
\lstset{
  frame=top,frame=bottom,
  basicstyle=\ttfamily,
  language=XML,
  tabsize=2,
  belowskip=2\medskipamount
}

% All listings have a left aligned caption. This looks better as the listings themselves are left aligned.
\captionsetup[listing]{justification=justified,singlelinecheck=false}

\usepackage{tabu}
\tabulinesep=1.0mm
\restylefloat{table}

\usepackage{siunitx}
\usepackage{tablefootnote}
\usepackage{multirow}
\usepackage{colortbl}
\usepackage{footmisc}
\newenvironment{longlisting}{\captionsetup{type=listing}}{}
%\usepackage[colorlinks=true]{hyperref}

% Inline code fragments can run over the page boundary without ragged right.
\AtBeginDocument{\raggedright}

\newcommand{\UserGuide}{}

\renewcommand\P{\ensuremath{\mathbb{P}}}
\newcommand\E{\ensuremath{\mathbb{E}}}
\newcommand\Q{\ensuremath{\mathbb{Q}}}
\newcommand\I{\mathds{1}}
\newcommand\F{\ensuremath{\mathcal F}}
\newcommand\V{\ensuremath{\mathbb{V}}}
\newcommand\YOY{{\rm YOY}}
\newcommand\Prob{\ensuremath{\mathbb{P}}}
\newcommand{\D}[1]{\mbox{d}#1}
\newcommand{\NPV}{\mathit{NPV}}
\newcommand{\CVA}{\mathit{CVA}}
\newcommand{\DVA}{\mathit{DVA}}
\newcommand{\FVA}{\mathit{FVA}}
\newcommand{\COLVA}{\mathit{COLVA}}
\newcommand{\FCA}{\mathit{FCA}}
\newcommand{\FBA}{\mathit{FBA}}
\newcommand{\KVA}{\mathit{KVA}}
\newcommand{\MVA}{\mathit{MVA}}
\newcommand{\PFE}{\mathit{PFE}}
\newcommand{\EE}{\mathit{EE}}
\newcommand{\EPE}{\mathit{EPE}}
\newcommand{\ENE}{\mathit{ENE}}
\newcommand{\EEPE}{\mathit{EEPE}}
\newcommand{\EEE}{\mathit{EEE}}
\newcommand{\EAD}{\mathit{EAD}}
\newcommand{\PE}{\mathit{PE}}
\newcommand{\NE}{\mathit{NE}}
\newcommand{\RC}{\mathit{RC}}
\newcommand{\RW}{\mathit{RW}}
\newcommand{\NS}{\mathit{NS}}
\newcommand{\PD}{\mathit{PD}}
\newcommand{\LGD}{\mathit{LGD}}
\newcommand{\DIM}{\mathit{DIM}}
\newcommand{\DF}{\mathit{DF}}
\newcommand{\MA}{\mathit{MA}}
\newcommand{\SCVA}{\mathit{SCVA}}
\newcommand{\bs}{\textbackslash}
\newcommand{\REDY}{\color{red}Y}
\newcommand{\IA}{\mathit{IA}}
\newcommand{\Th}{\mathit{TH}}
\newcommand{\CSA}{\mathit{CSA}}

\begin{document}

%\title{Open Source Risk Engine \\ User Guide  }
\title{ORE User Guide}
%\author{Quaternion Risk Management Ltd.}
%\author{Acadia Inc.}
\date{15 January 2024}
\maketitle

\newpage

%-------------------------------------------------------------------------------
\section*{Document History}

\begin{center}
\begin{supertabular}{|l|l|p{9cm}|}
\hline
Date & Author & Comment \\
\hline
7 October 2016 & Quaternion & initial release\\
28 April 2017 & Quaternion  & updates for release 2\\
7 December 2017 & Quaternion & updates for release 3\\
20 March 2019 & Quaternion & updates for release 4\\
19 June 2020 & Quaternion & updates for release 5\\
30 June 2021 & Acadia & updates for release 6\\
16 September 2022 & Acadia & updates for release 7\\
6 December 2022 & Acadia & updates for release 8\\
31 March 2023 & Acadia & updates for release 9\\
16 June 2023 & Acadia & updates for release 10\\
16 October 2023 & Acadia & updates for release 11\\
15 January 2024 & Acadia & updates for release 12\\
\hline
\end{supertabular}
\end{center}

\newpage

\tableofcontents
\newpage

\section{Introduction}

The {\em Open Source Risk Project} \cite{ORE} aims at providing a transparent platform for pricing and risk analysis
that serves as
%\medskip
\begin{itemize}
\item a benchmarking, validation, training, and teaching reference,
\item an extensible foundation for tailored risk solutions.
\end{itemize}

Its main software project is {\em Open Source Risk Engine} (ORE), an application that provides
\begin{itemize}
\item a Monte Carlo simulation framework for contemporary risk analytics and value adjustments
\item simple interfaces for trade data, market data and system configuration
\item simple launchers and result visualisation in Jupyter, Excel, LibreOffice
\item unit tests and various examples.  
\end{itemize}
ORE is open source software, provided under the Modified BSD License. It is based 
on QuantLib, the open source library for quantitative finance \cite{QL}.

%\medskip
\subsubsection*{Audience}
The project aims at reaching quantitative risk ma\-nage\-ment practitioners (be it in financial institutions, audit
firms, consulting companies or regulatory bodies) who are looking for accessible software solutions, and quant
developers in charge of the implementation of pricing and risk methods similar to those in ORE. Moreover, the project
aims at reaching academics and students who would like to teach or learn quantitative risk management using a freely
available, contemporary risk application.

\subsubsection*{Contributions}
Quaternion Risk Management \cite{QRM} has been committed to sponsoring the Open Source Risk project through ongoing project
administration, through providing an initial release and a series of subsequent releases in order to achieve a wide
analytics, product and risk factor class coverage. Since Quaternion's acquisition by Acadia Inc. in February 2021, Acadia \cite{acadia} is committed to continue the sponsorship. The Open Source Risk project works will continue with former Quaternion now operating as Acadia's Quantitative Services unit. 

The community is invited to contribute to ORE, for example through
feedback, discussions and suggested enhancement in the forum on the ORE site \cite{ORE}, as well as contributions of ORE
enhancements in the form of source code. See the FAQ section on the ORE site \cite{ORE} on how to get involved.

\subsection{Scope}

ORE currently provides portfolio pricing, cash flow generation, market risk analysis and a range of contemporary derivative portfolio analytics. The latter are based on a Monte Carlo simulation framework which yields 
the evolution of various exposure measures:
\begin{itemize}
\item EE aka EPE (Expected Exposure or Expected Positive Exposure)
\item ENE (Expected Negative Exposure, i.e. the counterparty's perspective)
\item 'Basel' exposure measures relevant for regulatory capital charges under internal model methods 
\item PFE (Potential Future Exposure at some user defined quantile)
\end{itemize}
and derivative value adjustments (xVA)
\begin{itemize}
\item CVA (Credit Value Adjustment)
\item DVA (Debit Value Adjustment)
\item FVA (Funding Value Adjustment)
\item COLVA (Collateral Value Adjustment)
\item MVA (Margin Value Adjustment)
\end{itemize}
for portfolios with netting, variation and initial margin agreements. 

\medskip
The market risk framework provides sensitivity analysis, stress testing and several parametric VaR versions (Delta VaR, Delta-Gamma Normal VaR, Delta-Gamma VaR with Cornish-Fisher expansion and Saddlepoint method),  across all asset classes and products. 

\medskip
Thanks to Acadia's open-source strategy, ORE's financial instrument scope was extended beyond the initial vanilla scope with quarterly releases since version 7 (September 2022) to cover
\begin{itemize}
\item "First Generation" Equity and FX Exotics, released September with ORE v7
\item Commodity products (Swaps, Basis Swaps, Average Price Options, Swaptions), released December 22 with ORE v8
\item Credit products (Index CDS and Index CDS Options, Credit-Linked Swaps, Synthetic CDOs), released March 23 with ORE v9
\item Bond products and Hybrids (Bond Options, Bond Repos, Bond TRS, Composite Trades, Convertible Bonds, Generic TRS with mixed basket underlyings, CFDs), released in June 23 with ORE v10
\end{itemize}
These contributions were accompanied by analytics extensions to enhance ORE usability
\begin{itemize}
\item Exposure simulation for xVA and PFE, adding Commodity to the asset class coverage, and adding American Monte Carlo for Exotics, released in December 22 with ORE v8
\item Market Risk including multi-threaded sensitivity analysis, par sensitivity conversion, parametric delta/gamma VaR with Cornish-Fisher expansion and Saddlepoint method, released in March 23 with ORE v9
\item Portfolio Credit Model, released in June 23 with ORE v10
\item ISDA's Standard Initial Margin Model (SIMM), released in June 23 with ORE v10
\end{itemize}

With ORE v11 the release of the {\em Scripted Trade} framework followed. This allows the modelling of complex hybrid payoffs such as Accumulators, TARFs, PRDCs, Basket Options, etc, across IR, FX, INF, EQ, COM classes.  Scripted Trades are fully integrated into the market risk and exposure simulation frameworks, supported by American Monte Carlo methods for pricing and exposure simulation. 
The user can now extend the instrument scope conveniently by adding payoff scripts (embedded into the trade XML or in separate script "library" XML) and without recompiling the code base.

\medskip 
The product coverage of the latest release of ORE is sketched in Table \ref{tab_coverage}.
\begin{table}[hbt]
\scriptsize
\begin{center}
\begin{tabular}{|l|p{1.5cm}|p{1.5cm}|p{1.2cm}|p{1.5cm}|}
\hline
Product & Pricing and Cashflows & Sensitivity Analysis & Stress Testing & Exposure Simulation \& XVA\\
\hline
Fixed and Floating Rate Bonds/Loans & Y & Y & Y & N \\
\hline
Interest Rate Swaps & Y & Y & Y & Y\\
\hline
Caps/Floors & Y & Y & Y & Y\\
\hline
Swaptions & Y & Y & Y &Y \\
\hline
Constant Maturity Swaps, CMS Caps/Floors & Y & Y & Y & Y\\
\hline
FX Forwards and Average Forwards & Y & Y & Y & Y \\
\hline
Cross Currency Swaps & Y & Y & Y & Y \\
\hline
FX European and Asian Options & Y & Y & Y & Y\\
\hline
FX Exotic Options (see below) & Y & Y & Y & Y\\
\hline
Equity Forwards & Y & Y & Y & Y\\
\hline
Equity Swaps & Y & Y & Y & N\\
\hline
Equity European and Asian Options & Y & Y & Y & Y \\
\hline
Equity Exotic Options (see below)  & Y & Y & Y & Y \\
\hline
Equity Future Options & Y & Y & Y & Y \\
\hline
Commodity Forwards and Swaps & Y & Y & Y & Y\\
\hline
Commodity European and Asian Options & Y & Y & Y & Y \\
\hline
Commodity Digital Options & Y & Y & Y & Y \\
\hline
Commodity Swaptions & Y & Y & Y & Y\\
\hline
CPI Swaps & Y & Y & N & Y \\
\hline
CPI Caps/Floors & Y & Y & N & N\\
\hline
Year-on-Year Inflation Swaps & Y & Y & N & Y \\
\hline
Year-on-Year Inflation Caps/Floors & Y & Y & N & N\\
\hline
Credit Default Swaps, Options & Y & Y & N & Y \\
\hline
Index Credit Default Swaps, Options & Y & Y & N & Y \\
\hline
Credit Linked Swaps & Y & Y & N & Y \\
\hline
Index Tranches, Synthetic CDOs & Y & Y & N & Y \\
\hline
Composite Trades & Y & Y & Y & Y \\
\hline
Total Return Swaps and Contracts for Difference & Y & Y & Y & Y \\
\hline
Convertible Bonds & Y & Y & Y & N \\
\hline
ASCOTs & Y & Y & Y & Y \\
\hline
Scripted Trades & Y & Y & Y & Y \\
\hline
\end{tabular}
\caption{ORE product coverage. FX/Equity Exotics include Barrier, Digital, Digital Barrier (FX only), Double Barrier, European Barrier, KIKO Barrier (FX only), Touch and Double Touch Options. Scripted Trades cover single and multi-asset products across all asset classes except Credit (so far), see Example\_52 and the separate documentation in Docs/ScriptedTrade.}
\label{tab_coverage}
\end{center}
\end{table}

\medskip The simulation models applied in ORE's risk factor evolution implement the models discussed in detail in {\em
  Modern Derivatives Pricing and Credit Exposure Analysis} \cite{Lichters}: The IR/FX/INF/EQ risk factor evolution is based on
a cross currency model consisting of an arbitrage free combination of Linear Gauss Markov models for all interest rates
and lognormal processes for FX rates and EQ prices, Dodgson-Kainth (or Jarrow-Yildirim) models for inflation. The model components are calibrated to cross currency discounting and forward curves, Swaptions, FX Options, EQ Options and CPI caps/floors. With the 8th release, Commodity simulation has been added, as well as the foundation for a multi-factor Hull-White based IR/FX/COM simulation model. 

\subsection{ORE in Python or Java}

ORE is written in C++ and comes with a command line executable {\tt ore.exe} that supports batch processes. 
But since early versions of ORE we also provide language bindings following QuantLib's example using SWIG, in ORE's case with focus on Python and Java modules. 
The ORE SWIG module extends (contains) the QuantLib SWIG module and offers moreover access to a part of ORE's functionality.
Since ORE v9, Python {\em wheels} are provided for each release, so that users can install the most recent ORE Python module by calling

\medskip
\centerline{\tt pip install open-source-risk-engine}
 
\medskip
See section \ref{example:42} on how to use ORE-Python. 

\subsection{Roadmap}

It is planned that subsequent ORE releases will also provide the calculation of {\bf regulatory capital charges} 
\begin{itemize}
\item for Counterparty Credit Risk under the standardised approach (SA-CCR)
\item for Market Risk (FRTB-SA)
\item for CVA Risk (BA-CVA, SA-CVA)
\end{itemize}

We also expect to contribute performance enhancements over time that are currently in development at Acadia, primarily based on the scripted trade framework
\begin{itemize}
\item AAD for fast calculation of trade sensitivities and xVAs
\item tailored interfaces for utilising GPUs
\end{itemize}

There is demand among our clients for extended coverage of the ORE-Python version, so that we also expect steady growth of the Python wrapper around ORE.

\subsection{Further Resources}
\begin{itemize}
\item Open Source Risk Project site: \url{http://www.opensourcerisk.org}
\item Frequently Asked Questions: \url{http://www.opensourcerisk.org/faqs}
\item Forum: \url{http://www.opensourcerisk.org/forum}
\item Source code and releases: \url{https://github.com/opensourcerisk/engine}
\item Language bindings: \url{https://github.com/opensourcerisk/ore-swig}
\item Follow ORE on Twitter {\tt @OpenSourceRisk} for updates on releases and events
\end{itemize}
 
\subsubsection*{Organisation of this document}

This document focuses on instructions how to use ORE to cover basic workflows from individual deal analysis to portfolio
processing. After an overview over the core ORE data flow in section \ref{sec:process} and installation instructions in
section \ref{sec:installation} we start in section \ref{sec:examples} with a series of examples that illustrate how to
launch ORE using its command line application, and we discuss typical results and reports. We then illustrate in section
\ref{sec:visualisation} interactive analysis of resulting 'NPV cube' data. The final sections of this text document ORE
parametrisation and the structure of trade and market data input.

%========================================================
\section{Release Notes}\label{sec:releasenotes}
%========================================================

See the full history of release notes in {\tt News.txt} in the top level directory of the ORE's github repository.

\medskip
This section summarises the notable changes between release 10 (June 2023) and 11 (October 2023).

\bigskip
INSTRUMENTS
\begin{itemize}
\item Add the Scripted Trade framework, see Example 52
\item Add support for fixings at trade level, see Example 51
\item Add Commodity Heat Rate Option
\item Add support for FRA on OIS
\item Add support for SIFMA Cap/Floor
\end{itemize}

\bigskip
MARKETS
\begin{itemize}
\item Support Optionlet volatility input
\item Add Dated OIS Rate Helper to the ORE yield curve to support instruments
  tailored to Central Bank meeting dates, see Example 53
\end{itemize}

\bigskip
ANALYTICS
\begin{itemize}
\item Add SIMM 2.6
\item Add support for SIMM with one-day horizon, see updated Example 44
\item Convert pre-computed zero sensitivities into par sensitivities, see Example 50 
\item Add support for Cross Currency MtM Reset Swaps to AMC exposure simulation
\end{itemize}

\bigskip
TEST
\begin{itemize}
\item QuantExt: 272 test functions (vs 267 in the previous release)
\item OREData: 257 test functions (vs. 206 in the previous release)
\item OREAnalytics: 78 test functions 
\end{itemize}

\bigskip
DOCUMENTATION
\begin{itemize}
\item A separate guide for the Scripted Trade has been added,
  see Docs/ScriptedTrade/scriptedtrade.tex
\end{itemize}

\bigskip
LANGUAGE BINDINGS
\begin{itemize}
\item Upgrade to QuantLib-SWIG-v1.31.1
\end{itemize}

\bigskip
OTHER
\begin{itemize}
\item Added an external compute device interface (to utilise GPUs)
\item Logging enhancements (progress, structured logs)
\item Introduced Blackduck and Coverity scans of the code base before releases
\item Upgrade of ORE's QuantLib fork to QuantLib-v1.31.1
\end{itemize}

%========================================================
\section{ORE Data Flow}\label{sec:process}
%========================================================

The core processing steps followed in ORE to produce risk analytics results are sketched in Figure \ref{fig_process}.
All ORE calculations and outputs are generated in three fundamental process steps as indicated in the three boxes in the
upper part of the figure. In each of these steps appropriate data (described below) is loaded and results are generated,
either in the form of a human readable report, or in an intermediate step as pure data files (e.g. NPV data, exposure data).
\begin{figure}[h]
\begin{center}
\includegraphics[scale=0.6]{process.pdf}
\end{center}
\caption{Sketch of the ORE process, inputs and outputs. }
\label{fig_process}
\end{figure}

The overall ORE process needs to be parametrised using a set of configuration XML files which is the subject of section
\ref{sec:configuration}. The portfolio is provided in XML format which is explained in detail in sections
\ref{sec:portfolio_data} and \ref{sec:nettingsetinput}. Note that ORE comes with 'Schema' files for all supported
products so that any portfolio xml file can be validated before running through ORE. Market data is provided in a simple
three-column text file with unique human-readable labelling of market data points, as explained in section
\ref{sec:market_data}.  \\

The first processing step (upper left box) then comprises 
\begin{itemize}
\item loading the portfolio to be analysed, 
\item building any yield curves or other 'term structures' needed for pricing, 
\item calibration of pricing and simulation models.
\end{itemize}

The second processing step (upper middle box) is then 
\begin{itemize}
\item portfolio valuation, cash flow generation,
\item going forward - conventional risk analysis such as sensitivity analysis and stress testing, standard-rule capital
  calculations such as SA-CCR, etc,
\item and in particular, more time-consuming, the market simulation and portfolio valuation through time under Monte
  Carlo scenarios.
\end{itemize}
This process step produces several reports (NPV, cashflows etc) and in particular an {\bf NPV cube}, i.e. NPVs per
trade, scenario and future evaluation date. The cube is written to a file in both condensed binary and human-readable
text format.  \\

The third processing step (upper right box) performs more 'sophisticated' risk ana\-ly\-sis by post-processing the NPV
cube data:
\begin{itemize}
\item aggregating over trades per netting set, 
\item applying collateral rules to compute simulated variation margin as well as simulated (dynamic) initial margin
  posting,
\item computing various XVAs including CVA, DVA, FVA, MVA for all netting sets, with and without taking collateral
  (variation and initial margin) into account, on demand with allocation to the trade level.
\end{itemize}
The outputs of this process step are XVA reports and the 'net' NPV cube, i.e. after aggregation, netting and collateral. \\

The example section \ref{sec:examples} demonstrates for representative product types how the described processing steps
can be combined in a simple batch process which produces the mentioned reports, output files and exposure evolution
graphs in one 'go'.

Moreover, both NPV cubes can be further analysed interactively using a visualisation tool introduced in section
\ref{sec:jupyter}. And finally, sections \ref{sec:calc} and \ref{sec:excel} demonstrate how ORE processes can be
launched in spreadsheets and key results presented automatically within the same sheet.

%========================================================
\section{Getting and Building ORE}\label{sec:installation}
%========================================================

You can get ORE in two ways, either by downloading a release bundle as described in section \ref{sec:release} (easiest if you just want to use ORE) or by
checking out the source code from the github repository as described in section \ref{sec:build_ore} (easiest if you want to build and develop ORE).

\subsection{ORE Releases}\label{sec:release}

ORE releases are regularly provided in the form of source code archives, Windows exe\-cutables {\tt ore.exe}, example
cases and documentation. Release archives will be provided at \url{https://github.com/opensourcerisk/engine/releases}.

The release contains the QuantLib source version that ORE depends on. This is the latest QuantLib release that precedes the ORE release including a small number of patches.

\medskip
The release consists of a single archive in zip format
\begin{itemize}
\item {\tt ORE-<VERSION>.zip}
\end{itemize}

When unpacked, it creates a directory {\tt ORE-<VERSION>} with the following files respectively subdirectories
\begin{enumerate}
%\item {\tt bin/win32/ore.exe}
%\item {\tt bin/x64/ore.exe}
\item {\tt App/}
\item {\tt Docs/}
\item {\tt Examples/}
\item {\tt FrontEnd/}
\item {\tt OREAnalytics/}
\item {\tt OREData/}
\item {\tt ORETest/}
\item {\tt QuantExt/}
\item {\tt QuantLib/}
\item {\tt ThirdPartyLibs/}
\item {\tt tools/}
\item {\tt xsd/}
\item {\tt userguide.pdf}
\end{enumerate} 

The first three items and {\tt userguide.pdf} are sufficient to run the compiled ORE application
on the list of examples described in the user guide (this works on Windows only). The Windows executables are located in {\tt App/bin/Win32/Release/} respectively {\tt App/bin/x64/Release/}. To continue with the compiled
executables:
\begin{itemize}
\item Ensure that the scripting language Python is installed on your computer, see also section \ref{sec:python}
  below;
\item Move on to the examples in section \ref{sec:examples}.
\end{itemize}

\medskip
The release bundle contains the ORE source code, which is sufficient to build ORE from sources as follows (if you build ORE for development purposes, we recommend using git though, see section \ref{sec:build_ore}):
\begin{itemize}
\item Set up Boost as described in section \ref{sec:boost}, unless already installed
\item Build QuantLib, QuantExt, OREData, OREAnalytics, App (in this order) as described in section \ref{sec:build}
\item Note that ThirdPartyLibs does not need to be built, it contains RapidXml, header only code for reading and
  writing XML files
\item Move on to section \ref{sec:python} and the examples in section \ref{sec:examples}.
\end{itemize}

Open {\tt Docs/html/index.html} to see the API documentation for QuantExt, OREData and OREAnalytics, generated by
doxygen.

\subsection{Building ORE}\label{sec:build_ore}

ORE's source code is hosted at \url{https://github.com/opensourcerisk/engine}.

\subsubsection{Git}

To access the current code base on GitHub, one needs to get {\tt git} installed first.
   
\begin{enumerate}
\item Install and setup Git on your machine following instructions at \cite{git-download}

\item Fetch ORE from github by running the following: 

{\tt\% git clone https://github.com/opensourcerisk/engine.git ore}      

This will create a folder 'ore' in your current directory that contains the codebase.

\item Initially, the QuantLib subdirectory under {\tt ore} is empty as it is a submodule pointing to the official
  QuantLib repository. To pull down locally, use the following commands:

{\tt
\% cd ore \\
\% git submodule init \\
\% git submodule update
}

\end{enumerate}

Note that one can also run 

{\footnotesize \tt\% git clone --recurse-submodules https://github.com/opensourcerisk/engine.git ore}

in step 2, which also performs the steps in 3.

\subsubsection{Boost}\label{sec:boost}

QuantLib and ORE depend on the boost C++ libraries. Hence these need to be installed before building QuantLib and
ORE. On all platforms the minimum required boost version is 1\_78.
%Other versions may work on some platforms and system configurations, but were not tested.

\subsubsection*{Windows}

\begin{enumerate}
\item Download the pre-compiled binaries for your MSVC version (e.g. MSVC-14.3 for MSVC2022) from \cite{boost-binaries}
%, any recent version should work
\begin{itemize}
\item 32-bit: \cite{boost-binaries}{\bs}VERSION{\bs}boost\_VERSION-msvc-14.3-32.exe{\bs}download 
\item 64-bit: \cite{boost-binaries}{\bs}VERSION{\bs}boost\_VERSION-msvc-14.3-64.exe{\bs}download
\end{itemize}
\item Start the installation file and choose an installation folder (the ``boost root directory''). Take a note of that folder as it will be needed later on.   
\item Finish the installation by clicking Next a couple of times.
\end{enumerate}
    
Alternatively, compile all Boost libraries directly from the source code:

\begin{enumerate}
\item Open a Visual Studio Tools Command Prompt
\begin{itemize}
\item 32-bit: VS2022 x86 Native Tools Command Prompt
\item 64-bit: VS2022 x64 Native Tools Command Prompt
\end{itemize}
\item Navigate to the boost root directory
\item Run bootstrap.bat
\item Build the libraries from the source code
\begin{itemize}
\item 32-bit: \\
  {\footnotesize\tt .{\bs}b2 --stagedir=.{\bs}lib{\bs}Win32{\bs}lib --build-type=complete toolset=msvc-14.3 \bs \\
    address-model=32 --with-test --with-system --with-filesystem  \bs \\
    --with-serialization --with-regex --with-date\_time stage}
\item 64-bit: \\
  {\footnotesize\tt .{\bs}b2 --stagedir=.{\bs}lib{\bs}x64{\bs}lib --build-type=complete toolset=msvc-14.3 \bs \\
    address-model=64 --with-test --with-system --with-filesystem \bs \\
    --with-serialization --with-regex --with-date\_time stage}
\end{itemize}
\end{enumerate}

\subsubsection*{Unix}

\begin{enumerate}
\item Download Boost from \cite{boost} and build following the instructions on the site
\item Define the environment variable BOOST that points to the boost directory
(so includes should be in BOOST and libs should be in BOOST/stage/lib)
\end{enumerate}

\subsubsection{ORE Libraries and Application}\label{sec:build}

\subsubsection*{Windows}

\begin{enumerate}

\item Download and install Visual Studio Community Edition (Version 2019 or later, 2022 is recommended). 
During the installation, make sure you install the Visual
C++ support under the Programming Languages features (disabled by default).

\item Configure boost paths: \\

Set environment variables, e.g.:
\begin{itemize}
  	\item  {\tt \%BOOST\%} pointing to your directory, e.g, {\tt C:{\bs}boost\_1\_72\_0} 
  	\item {\tt \%BOOST\_LIB32\%} pointing to your Win32 lib directory, e.g, {\tt C:{\bs}boost\_1\_72\_0{\bs}lib32\-msvc\-14.3} 
	\item  {\tt \%BOOST\_LIB64\%} pointing to your x64 lib directory, e.g, {\tt C:{\bs}boost\_1\_72\_0{\bs}lib64\-msvc\-14.3} 
 \end{itemize}
 
\item Download and install CMake for Windows (https://cmake.org/download/). Visual Studio Community Edition 2019 or later supports CMake and you can install the feature 'C++ CMake Tools for Windows' instead of installing CMake as standalone program.

\end{enumerate}

%\subsubsection*{Visual Studio with CMake}

Visual Studio 2019 and later supports CMake Projects.

\begin{enumerate}
\item Start Visual Studio 2019 or later.
\item Select "Open a local folder" from the start page or menu.
\item In the dialog window, select the ORE root directory.
\item Visual Studio will read the cmake presets from CMakePresets.json and the project file CMakeList.txt and configure the project.
\item Once the configuration is finished and one can build the project.
\item The executables are built in the subfolder {\tt /build/TARGET/CONFIGURATION/EXECUTABLE}, e.g. {\tt /build/App/Release/ore.exe}.
\end{enumerate}

ORE is shipped with configuration and build presets using Visual Studio 2022 and the Ninja build system. Those presets are configured in the CMakePreset.json which is read by Visual Studio by default when opening the CMake project. If you want to use Visual Studio 2019 instead, you would have to change the Generator in the CMakePreset.json from "Visual Studio 17 2022" to "Visual Studio 16 2019".

You can switch in the solution explorer from the file view to the projects view, where the CMake Targets View can be selected. In this view, the various target projects can be seen below "ORE Project" and be used in a similar manner as the usual VS projects.

%\subsubsection*{Generate Visual Studio Projects with CMake}
\medskip

Alternatively, Visual Studio project files can be auto-generated from the CMake project files or ORE can be built with the CMake command line tool, similar to UNIX / Mac systems.

\begin{enumerate}

\item Generate MSVC project files from CMake files:
\begin{itemize}
\item Open a Visual Studio Tools Command Prompt
\begin{itemize}
\item 32-bit: VS2022/x86 Native Tools Command Prompt for VS 2022
\item 64-bit: VS2022/x64 Native Tools Command Prompt for VS 2022
\end{itemize}
\item Navigate to the ORE root directory
\item Run CMake command:
\begin{itemize}
\item 64-bit: \\
{\tt cmake -G "Visual Studio 17 2022" -A x64 -DBOOST\_INCLUDEDIR=\%BOOST\% -DBOOST\_LIBRARYDIR=\%BOOST\_LIB64\% -DQL\_ENABLE\_SESSIONS=ON -DMSVC\_LINK\_DYNAMIC\_RUNTIME=true -B build}
\item 32-bit: \\
{\tt cmake -G "Visual Studio 17 2022" -A x32 -DBOOST\_INCLUDEDIR=\%BOOST\% -DBOOST\_LIBRARYDIR=\%BOOST\_LIB32\% -DQL\_ENABLE\_SESSIONS=ON -DMSVC\_LINK\_DYNAMIC\_RUNTIME=true -B build}
\end{itemize}
Replace the generator "Visual Studio 17 2022" with the actual installed version.
The solution and project files will be generated in the {\tt $\langle$ORE\_ROOT$\rangle${\bs}build} subdirectory.
\end{itemize}

\item build the cmake project with the command {\tt cmake --build build -v --config Release}, 

\item or open the MSVC solution file {\tt build{\bs}ORE.sln} and build the entire solution with Visual Studio (again, make sure to select the correct platform in the configuration manager first).
\end{enumerate}

\subsubsection*{Optional: Install optional dependencies with VCPKG}

VCPKG is an open source c++ library manager. ORE can be built optionally with ZLIB and Eigen library support. 

For both features the libraries needed to be installed on the system. On Windows one can use the VCPKG package manager to install those dependencies:

\begin{itemize}
\item Install vcpkg: https://vcpkg.io/en/getting-started.html
\item Install dependencies with invoking the command \\
\medskip
{\tt vcpkg install --triplet x64-windows zlib} \\
{\tt vcpkg install --triplet x64-windows eigen3} \\
\medskip
\end{itemize}

To make VCPKG visible to CMAKE, create an environment variable {\tt VCPKG\_ROOT} pointing to the root of the vcpkg directory and configure ORE with the flag {\tt -DCMAKE\_TOOLCHAIN\_FILE=\%VCPKG\_ROOT\%/scripts/buildsystems/vcpkg.cmake}. 

To use VCPKG with Visual Studio add the toolChainFile to the configurePresets in the CMakePresets.json:

{\tt "toolchainFile": "\$env\{VCPKG\_ROOT\}/scripts/buildsystems/vcpkg.cmake",}

\subsubsection*{Unix}

With the 5th release we have discontinued automake support so that ORE can only be built with CMake on Unix systems, as follows.

\begin{enumerate}
\item set environment variable to locate the boost include and boost library directories\\
\medskip
  {\tt export BOOST\_LIB=path/to/boost/lib}\\
  {\tt export BOOST\_INC=path/to/boost/include}
\medskip
\item Change to the ORE project directory that contains the {\tt QuantLib}, {\tt QuantExt}, etc, folders; create subdirectory {\tt build} and change to subdirectory {\tt build}
\item Configure CMake by invoking \\
\medskip
{\tt cmake -DBOOST\_ROOT=\${BOOST\_INC} -DBOOST\_LIBRARYDIR=\${BOOST\_LIB} -DQL\_ENABLE\_SESSIONS=ON ..} \\
\medskip
where the {\tt QL\_ENABLE\_SESSIONS} variable is set to ON in order to enable some multi-threading applications in ORE.

Alternatively, set environment variables {\tt BOOST\_ROOT} and {\tt BOOST\_LIBRARYDIR} directly and run \\
\medskip
{\tt cmake ..} \\
\medskip
\item Build all ORE libraries, QuantLib, as well as the doxygen API documentation for QuantExt, OREData and OREAnalytics, by invoking \\
\medskip
{\tt make -j4} \\
\medskip
using four threads in this example.
\medskip
\item Run all test suites by invoking \\
\medskip
{\tt ctest -j4}
\item Run Examples (see section \ref{sec:examples})
\end{enumerate}

Note: 
\begin{itemize}
\item If the boost libraries are not installed in a standard path they might not be found during runtime because of a missing rpath
tag in their path. Run the script {\tt rename\_libs.sh} to set the rpath tag in all libraries located in {\tt
  \${BOOST}/stage/lib}.
\item Unset {\tt LD\_LIBRARY\_PATH} respectively {\tt DYLD\_LIBRARY\_PATH} before running the ORE executable or the test suites, in order not to override the rpath information embedded into the libaries built with CMake
\item On Linux systems, the 'locale' settings can negatively affect the ORE process and output. To avoid this, we
recommend setting the environment variable {\tt LC\_NUMERIC} to {\tt C}, e.g. in a bash shell, do

{\tt\footnotesize
\% export LC\_NUMERIC=C
}

before running ORE or any of the examples below. This will suppress thousand separators in numbers when converted to
strings.

\item Generate {\tt CMakeLists.txt}:

The .cpp and .hpp files included in the build process need to be explicitly specified in the various {\tt CMakeLists.txt} 
files in the project directory. The python script (in {\tt Tools/update\_cmake\_files.py}) can be used to update all CMakeLists.txt files 
automatically. 

\end{itemize}
 
\subsubsection*{ZLIB support}

To enable zlib support configure CMake with the flag {\tt -DORE\_USE\_ZLIB=ON}. 

If zlib is not installed on the system, it can be installed on Windows with the package manager VCPKG.

\subsection{Python and Jupyter}\label{sec:python}

Python (version 3.5 or higher) is required to use the ORE Python language bindings in section \ref{sec:oreswig}, 
or to run the examples in section \ref{sec:examples} and plot exposure
evolutions. Moreover, we use Jupyter \cite{jupyter} in section \ref{sec:visualisation} to visualise simulation
results. Both are part of the 'Anaconda Open Data Science Analytics Platform' \cite{Anaconda}. Anaconda installation
instructions for Windows, OS X and Linux are available on the Anaconda site, with graphical installers for
Windows\footnote{With Windows, after a fresh installation of Python the user may have to run the {\tt python} command
  once in a command shell so that the Python executable will be found subsequently when running the example scripts in
  section \ref{sec:examples}.}, Linux and OS X.

With Linux and OS X, the following environment variable settings are required
\begin{itemize}
\item set {\tt LANG} and {\tt LC\_ALL } to {\tt en\_US.UTF-8} or {\tt en\_GB.UTF-8}
\item set {\tt LC\_NUMERIC} to {\tt C}. 
\end{itemize}
The former is required for both running the Python scripts in the examples section, as well as successful installation
of the following packages. \\

The full functionality of the Jupyter notebook introduced in section \ref{sec:jupyter} requires furthermore installing
\begin{itemize}
\item jupyter\_dashboards: \url{https://github.com/jupyter-incubator/dashboards}
\item ipywidgets: \url{https://github.com/ipython/ipywidgets}
\item pythreejs: \url{https://github.com/jovyan/pythreejs}
\item bqplot: \url{https://github.com/bloomberg/bqplot}
\end{itemize}
With Python and Anaconda already installed, this can be done by running these commands
\begin{itemize}
\item {\tt conda install -c conda-forge ipywidgets}
\item {\tt pip install jupyter\_dashboards}
\item {\tt jupyter dashboards quick-setup --sys-prefix}
\item {\tt conda install -c conda-forge bqplot}
\item {\tt conda install -c conda-forge pythreejs}
\end{itemize}
Note that the bqplot installation requires the environment settings mentioned above.

\subsection{Building ORE-SWIG and Python Wheels}\label{sec:oreswig}

Since release 4, ORE comes with Python and Java language bindings following the QuantLib-SWIG example.
The ORE bindings extend the QuantLib SWIG wrappers and allow calling ORE functionality in the 
QuantExt/OREData/OREAnalytics libraries alongside with functionality in QuantLib.  

\medskip
The ORE-SWIG source code is hosted in a separate git repository at \url{https://github.com/opensourcerisk/ore-swig}.
The {\tt README.md} in the top level directory of this git repository contains build instructions and refers to tutorials 
for installing and building Python wrappers and wheels.

\medskip
Typical usage of the Python wrapper is shown in ORE's {\tt Example\_42} and in ORE SWIG's {\tt OREAnalytics/Python/Examples} directory.

%
%\medskip
%To build the wrappers on Windows, Linux, mac OS
%\begin{enumerate}
%\item build ORE and QuantLib first, as shown in the previous section. It is strongly advised to switch the codebase to a release tag here (e.g. 1.8.7.0), as the versions of the master branch are not always aligned!
%\item check out the ORE-SWIG repository into a project directory {\tt oreswig}, change into that directory (the same advice as above applies here as well, switch to the same release tag as ORE)
%\item pull in the QuantLib-SWIG project by running \\
%{\tt git submodule init} \\
%{\tt git submodule update}
%%\item Edit the top-level {\tt oreswig/CMakeLists.txt} and uncomment e.g. the line {\tt add\_subdirectory(OREAnalytics-SWIG/Python)} to build the OREAnalytics Python wrappers
%\item Create a subdirectory {\tt build}, change into that directory and configure cmake and then build using Ninja as follows \\
%\medskip
%{\footnotesize
%{\tt cmake -G Ninja $\backslash$} \\
%\hspace{1cm} {\tt -D ORE=<ORE Root Directory> $\backslash$} \\
%\hspace{1cm} {\tt -D BOOST\_ROOT=<Top level boost include directory> $\backslash$}\\
%\hspace{1cm} {\tt -D BOOST\_LIBRARYDIR=<Location of the compiled boost libraries> $\backslash$}\\
%\hspace{1cm} {\tt -D Python\_ROOT\_DIR=<Root directory of the python installation>$\backslash$} \\
%\hspace{1cm} {\tt ..} \\
%{\tt ninja} \\
%}
%\medskip
%On Linux or mac OS one can also use {\tt make} instead of {\tt ninja}. 
%In that case, omit the {\tt -G Ninja} part in the configuration. 
%
%\medskip
%To build on Windows using CMake and an existing Visual Studio installation you can e.g. run
%this from the top-level oreswig directory
%
%{\footnotesize
%{\tt mkdir build $\backslash$} \\
%{\tt cmake -G "Visual Studio 17 2022" $\backslash$}\\
%\hspace{1cm} {\tt -A x64 $\backslash$}\\
%\hspace{1cm} {\tt -D SWIG\_DIR=C:$\backslash$dev$\backslash$swigwin$\backslash$Lib $\backslash$}\\
%\hspace{1cm} {\tt -D SWIG\_EXECUTABLE=C:$\backslash$dev$\backslash$swigwin$\backslash$swig.exe $\backslash$}\\
%\hspace{1cm} {\tt -D ORE:PATHNAME=C:$\backslash$dev$\backslash$ORE$\backslash$master $\backslash$}\\
%\hspace{1cm} {\tt -D BOOST\_ROOT=C:$\backslash$dev$\backslash$boost $\backslash$}\\
%\hspace{1cm} {\tt -S OREAnalytics-SWIG/Python $\backslash$}\\
%\hspace{1cm} {\tt -B build $\backslash$}\\
%{\tt cmake -{}-build build -v}
%}

%\medskip
%
%This builds the OREAnalytics-Python bindings which include the wrapped parts of QuantLib, QuantExt, OREData and OREAnalytics.
%
%\begin{itemize}
%\item Try a Python example: Update your PYTHONPATH environment variable to include directory {\tt oreswig/build/OREAnalytics-SWIG/Python} 
%which contains both the new python module and the associated native library loaded by the python module;
%change to the {\tt oreswig/OREAnalytics/Python/Examples} directory and run e.g.\\
%
%\medskip
%\centerline{\tt python ore.py}  
%
%There is also an IPython example in the same directory. To try it, launch 
%
%\medskip
%\centerline{\tt jupyter notebook}  
%
%wait for your browser to open, select {\tt ore.ipy} from the list of files and then run all cells.
%
%\item Try a Java example: Make sure that the line {\tt add\_subdirectory(OREAnalytics-SWIG/Java)} 
%is uncommented in the top-level CMakeLists.txt file when building; change to directory 
%{\tt ORE-SWIG/OREAnalytics-SWIG/Java/Examples} and run
%
%\medskip
%{\tt java -Djava.library.path=../../../build/OREAnalytics-SWIG/Java $\backslash$} \\
%\hspace{1cm} {\tt -jar ../../../build/OREAnalytics-SWIG/Java/ORERunner.jar $\backslash$}\\
%\hspace{1cm} {\tt Input/ore.xml}
%
%\end{enumerate}
%
%\subsection{How to build ORE Python Wheels on Windows}\label{sec:win_wheel}
%
%This section is a stand-alone HOWTO for building ore and oreswig, including Python wrappers and wheels, at the DOS command line using cmake and VS 2022.  
%
%\subsubsection*{Prerequisites}
% 
%\begin{itemize}
%\item python: The following tools need to be up to date: \\
% {\tt python -m ensurepip } \\
% {\tt pip install build }
%\item boost: Download the binaries from the link below.  This release was tested against boost version 1.72.  It is important to download these precompiled binaries (rather than compiling yourself from source code) because the binaries include ZLIB support which this build requires. \\
%\url{https://sourceforge.net/projects/boost/files/boost-binaries}
%\item  swig: You need to either install the binaries, or install the source code and build yourself
% \item ore and oreswig: You need to install the source code. Instructions for building with cmake are provided below.
% \end{itemize}
% 
%\subsubsection*{Environment variables}
% 
%Set the following environment variables to the paths where the above items live on your machine, e.g:
% 
%{\tt set BOOST\_INC=C:{\bs}repos{\bs}boost{\bs}boost\_1\_81\_0} \\
%{\tt set BOOST\_LIB=C:{\bs}repos{\bs}boost{\bs}boost\_1\_81\_0{\bs}lib{\bs}x64{\bs}lib} \\
%{\tt set SWIG=C:{\bs}repos{\bs}swigwin-4.1.1} \\
%{\tt set ORE=C:{\bs}repos{\bs}ore } \\
%{\tt set ORESWIG=C:{\bs}repos{\bs}oreswig}
% 
%\subsubsection*{Build ORE}
% 
%Generate the project files:
% 
%\medskip
%{\tt cd \%ORE\%} \\
%{\tt mkdir build }\\
%{\tt cd \%ORE\%{\bs}build} \\
%{\tt cmake .. $\backslash$\\
%\hspace{1cm} -DBoost\_NO\_WARN\_NEW\_VERSIONS=1 $\backslash$\\
%\hspace{1cm} -Wno-dev -G "Visual Studio 17 2022" $\backslash$ \\
%\hspace{1cm} -A x64 $\backslash$\\
%\hspace{1cm} -DMSVC\_LINK\_DYNAMIC\_RUNTIME=OFF $\backslash$ \\
%\hspace{1cm} -DBOOST\_ROOT=\$BOOST\_INC $\backslash$\\
%\hspace{1cm} -DBOOST\_LIBRARYDIR=\$BOOST\_LIB $\backslash$ \\
%\hspace{1cm} -DQL\_ENABLE\_SESSIONS=ON} 
%
%\medskip
%This will create \%ORE\%{\bs}build{\bs}ORE.sln
% 
%\medskip
%Build:
% 
%\medskip
%{\tt cd \%ORE\%{\bs}build} \\
%%{\tt "C:{\bs}Program Files{\bs}CMake{\bs}bin{\bs}cmake.exe" -{}-build . -{}-config Release}
%{\tt cmake -{}-build . -{}-config Release}
%
%\medskip
%This will create  \%ORE\%{\bs}build{\bs}OREAnalytics{\bs}orea{\bs}Release{\bs}OREAnalytics-x64-mt.lib
% 
%\subsubsection*{Build ORE-SWIG Wrapper and Wheel}
% 
%In contrast to the generic cmake-based SWIG build in section \ref{sec:oreswig}, we are now resorting to python's setup.py.
% 
%\medskip
%{\tt cd \%ORESWIG\%{\bs}OREAnalytics-SWIG{\bs}Python} \\
%{\tt set BOOST\_ROOT=\%BOOST\_INC\%} \\
%% set BOOST_LIB is needed but already done above
%% set ORE needed but already done above
%{\tt set PATH=\%PATH\%;\%SWIG\%} \\
%{\tt set ORE\_STATIC\_RUNTIME=1} \\
%{\tt python setup.py wrap} \\
%{\tt python setup.py build} \\
%{\tt python setup.py test} \\
%{\tt python -m build -{}-wheel} 
% 
%\medskip
%This will create 
%\begin{itemize}
%\item the Python module and static library in folder {\tt <PATH>{\bs}build{\bs}lib.win-amd64-cpython-310} (the directory name depends on the machine and python version)
%\item the wheel file (filename.whl) in folder {\tt <PATH>{\bs}dist}
%\end{itemize}
%where {\tt <PATH>} stands for {\tt \%ORESWIG\%{\bs}OREAnalytics-SWIG{\bs}Python}.
%
%\medskip
%To use the wrapper directly: \\
%\medskip
%{\tt cd <PATH>{\bs}Examples} \\
%{\tt set PYTHONPATH=<PATH>{\bs}build{\bs}lib.win-amd64-cpython-310} \\
%{\tt python swap.py}
%
%\medskip
%To use the wheel:
% 
%{\tt cd <PATH>{\bs}Examples} \\
%{\tt python -m venv env1} \\
%{\tt .{\bs}env1{\bs}Scripts{\bs}activate.bat} \\
%{\tt pip install <PATH>{\bs}dist{\bs}filename.whl} \\
%{\tt python swap.py} \\
%{\tt deactivate} \\
%{\tt rmdir /s /q env1} \\
%
%\subsection{How to build ORE Python Wheels on Linux and macOS}\label{sec:nix_wheel}
%
%This section is a stand-alone HOWTO for building and using a python wheel for OREAnalytics on Linux or macOS.
%
%\subsubsection*{Prerequisites}
%
%\begin{itemize}
%\item python: Ensure that python3, pip, build, and virtualenv are installed and up to date. \\
%
%\medskip
%For example, on ubuntu:\\
%{\tt sudo apt install python3-pip} \\
%{\tt sudo apt install python3.10-venv} \\
%{\tt pip3 install -{}-upgrade build} \\
%
%\medskip
%On macOS it is recommended to install jupyterlab (which contains python and pip) with \\
%{\tt brew install jupyterlab} \\
%followed by\\
%{\tt pip install -{}-upgrade build} \\
%{\tt pip install virtualenv} 
%
%\item python3-dev: You need to install the python header files and libs. On some platforms these come already installed with python.  
%
%\medskip
%On ubuntu they do not and the command to install them is: \\
%{\tt sudo apt install python3-dev }\\
%
%\medskip
%On macOS they come with the recommended installation of jupyterlab
%
%\item boost and swig: You need to either install the binaries, or install the source code and build yourself.
%\item ore and oreswig: You need to install the source code. Instructions for building with cmake are provided below.
%\end{itemize}
%
%\subsubsection*{Environment variables}
%
%Set the following environment variables to the paths where the ore and ore swig repos, as well as boost live on your machine, e.g:
%
%\medskip
%{\tt export ORE=\$HOME/dev/ore} \\
%{\tt export ORESWIG=\$HOME/dev/oreswig}\\
%{\tt export BOOST\_INC=/opt/homebrew/include} \\
%{\tt export BOOST\_LIB=/opt/homebrew/lib} 
%
%\subsubsection*{Build ORE}
%
%Use cmake to generate the project Makefiles
%
%\medskip
%{\tt cd \$ORE} \\
%{\tt mkdir build} \\
%{\tt cd \$ORE/build} \\
%{\tt cmake .. \\
%\hspace{1cm} -DQL\_ENABLE\_SESSIONS=ON $\backslash$\\
%\hspace{1cm} -DCMAKE\_POSITION\_INDEPENDENT\_CODE=ON $\backslash$\\
%\hspace{1cm} -DORE\_BUILD\_DOC=OFF $\backslash$ \\
%\hspace{1cm} -DORE\_BUILD\_EXAMPLES=OFF $\backslash$ \\
%\hspace{1cm} -DORE\_BUILD\_TESTS=OFF $\backslash$\\
%\hspace{1cm} -DORE\_BUILD\_APP=OFF $\backslash$\\
%\hspace{1cm} -DQL\_BUILD\_BENCHMARK=OFF $\backslash$\\
%\hspace{1cm} -DQL\_BUILD\_EXAMPLES=OFF $\backslash$\\
%\hspace{1cm} -DQL\_BUILD\_TEST\_SUITE=OFF $\backslash$\\
%\hspace{1cm} -DBoost\_NO\_WARN\_NEW\_VERSIONS=1 $\backslash$\\
%\hspace{1cm} -DBoost\_NO\_SYSTEM\_PATHS=1 $\backslash$\\
%\hspace{1cm} -DBOOST\_ROOT=\$BOOST\_INC $\backslash$\\
%\hspace{1cm} -DBOOST\_LIBRARYDIR=\$BOOST\_LIB}
%
%\medskip
%Execute the following to kick off the build:
%
%\medskip
%{\tt cd \$ORE/build} \\
%{\tt cmake -{}-build} . 
%
%\medskip
%This will generate \$ORE/build/OREAnalytics/orea/libOREAnalytics.so or .dylib
%
%\subsubsection*{Build ORE-SWIG Wrapper and Wheel}
%
%In contrast to the generic cmake-based SWIG build in section \ref{sec:oreswig}, we are now resorting to python's setup.py. 
%
%The commands below will try to link to boost library boost\_thread.  On some platforms this lib has a different name and if that is the case for you then you can override it, e.g:
%
%{\tt export BOOST\_THREAD=boost\_thread-mt} \\
%
%\medskip
%
%Ensure that environent variables ORE, BOOST\_INC and BOOST\_LIB are set (see above), then
%
%\medskip
%{\tt cd \$ORESWIG/OREAnalytics-SWIG/Python} \\
%{\tt python setup.py wrap} \\
%{\tt python setup.py build} \\
%{\tt python -m build -{}-wheel}
%
%\medskip
%will then generate the wheel file (filename.whl) in folder \$ORESWIG/OREAnalytics-SWIG/Python/dist.
%For the second step above you may need to modify the \$ORESWIG/OREAnalytics-SWIG/oreanalytics-config script to return the appropriate cflags and libs on your machine.
%
%\medskip
%To use the wheel:
%
%\medskip
%{\tt cd \$ORESWIG/OREAnalytics-SWIG/Python/Examples} \\
%{\tt python -m venv env1} \\
%{\tt . ./env1/bin/activate} \\
%{\tt pip install \$ORESWIG/OREAnalytics-SWIG/Python/dist/filename.whl} \\
%{\tt python commodityforward.py} \\
%{\tt deactivate} \\
%{\tt rm -rf env1}

%========================================================
\section{Examples}\label{sec:examples}
%========================================================

The examples shown in table \ref{tab_0} are intended to help with getting started with ORE, and to serve as plausibility
checks for the simulation results generated with ORE.

\begin{table}[hbt]
\scriptsize
\begin{center}
\begin{tabular}{|c|l|}
\hline
Example & Description \\
\hline
\hline
1 & Vanilla at-the-money Swap with flat yield curve \\
\hline
2 & Vanilla Swap with normal yield curve \\
\hline
3 & European Swaption \\
\hline
4 & Bermudan Swaption \\
\hline
5 & Callable Swap \\
\hline
6 & Cap/Floor \\
\hline
7 & FX Forward \\
  & European FX Option \\ 
\hline
8 & Cross Currency Swap without notional reset \\
\hline
9 & Cross Currency Swap with notional reset \\
\hline
10 & Three-Swap portfolio with netting and collateral \\
   & XVAs - CVA, DVA, FVA, MVA, COLVA \\
   & Exposure and XVA Allocation to trade level \\
\hline
11 & Basel exposure measures - EE, EPE, EEPE \\
\hline
12 & Long term simulation with horizon shift \\
\hline
13 & Dynamic Initial Margin and MVA \\
\hline
14 & Minimal Market Data Setup \\
\hline
15 & Sensitivity Analysis and Stress Testing \\
\hline
16 & Equity Derivatives Exposure \\
\hline
17 & Inflation Swap Exposure under Dodgson-Kainth\\
\hline
18 & Bonds and Amortisation Structures\\
\hline
19 & Swaption Pricing with Smile\\
\hline
20 & Credit Default Swap Pricing\\
\hline
21 & Constant Maturity Swap Pricing\\
\hline
22 & Option Sensitivity Analysis with Smile\\
\hline
23 & Forward Rate Agreement and Averaging OIS Exposure\\
\hline
24 & Commodity Forward and Option Pricing and Sensitivity\\
\hline
25 & CMS Spread with (Digital) Cap/Floor Pricing, Sensitivity and Exposures\\
\hline
26 & Bootstrap Consistency\\
\hline
27 & BMA Basis Swap Pricing and Sensitivity\\
\hline
28 & Discount Ratio Curves\\
\hline
29 & Curve Building using Fixed vs. Float Cross Currency Helpers\\
\hline
30 & USD-Prime Curve Building via Prime-LIBOR Basis Swap\\
\hline
31 & Exposure Simulation using a Close-out Grid\\
\hline
32 & Inflation Swap Exposure under Jarrow-Yildrim\\ 
\hline
33 & CDS Exposure Simulation \\
\hline
34 & Wrong Way Risk \\
\hline
35 & Flip View \\
\hline
36 & Choice of Measure \\
\hline
37 & Multifactor Hull-White scenario generation \\
\hline
38 & Cross Currency Exposure using Multifactor Hull-White Models \\
\hline
39 & Exposure Simulation using American Monte Carlo \\
\hline
40 & Par Sensitivity Analysis \\
\hline
41 & Multi-threaded Exposure Simulation \\
\hline
42 & ORE Python Module \\
\hline
43 & Credit Portfolio Model \\
\hline
44 & ISDA SIMM Model \\
\hline
45 & Collateralized Bond Obligation \\
\hline
46 & Generic Total Return Swap \\
\hline
47 & Composite Trade \\
\hline
48 & Convertible Bond and ASCOT \\
\hline
49 & Bond Yield Shifted \\
\hline
50 & Par Sensitivity Conversion of external "Raw" Sensis \\
\hline
51 & Custom Trade Fixings\\
\hline
52 & Scripted Trades \\
\hline
53 & Curve Building using FRAs tailored to Central Bank Meeting Dates\\
\hline
54 & Scripted Trade Exposure with AMC: Bermudan Swaption and LPI Swap \\
\hline
55 & Scripted Trade Exposure with AMC: Fx TaRF \\
\hline
56 & CVA Sensitivity using AAD \\
\hline
57 & Base Scenario Analytic \\
\hline
\end{tabular}
\caption{ORE examples.}
\label{tab_0}
\end{center}
\end{table}
\clearpage

All example results can be produced with the Python scripts {\tt run.py} in the ORE release's {\tt Examples/Example\_\#}
folders which work on both Windows and Unix platforms. In a nutshell, all scripts call ORE's command line application
with a single input XML file

\medskip
\centerline{\tt ore[.exe] ore.xml}
\medskip

They produce a number of standard reports and exposure graphs in PDF format. The structure of the input file and of the
portfolio, market and other configuration files referred to therein will be explained in section
\ref{sec:configuration}.

\medskip ORE is driven by a number of input files, listed in table \ref{tab_1} and explained in detail in sections
\ref{sec:configuration} to \ref{sec:fixings}. In all examples, these input files are either located in the example's sub
directory {\tt Examples/Example\_\#/Input} or the main input directory {\tt Examples/Input} if used across several
examples. The particular selection of input files is determined by the 'master' input file {\tt ore.xml}.

\begin{table}[h]
\scriptsize
\begin{center}
\begin{tabular}{|l|p{11cm}|}
  \hline
  File Name & Description \\
  \hline
  {\tt ore.xml}&   Master input file, selection of further inputs below and selection of analytics \\
  {\tt portfolio.xml} & Trade data \\
  {\tt netting.xml} &  Collateral (CSA) data \\
  {\tt simulation.xml} & Configuration of simulation model and market\\
  {\tt market.txt} &  Market data snapshot \\
  {\tt fixings.txt} &  Index fixing history \\
  {\tt dividends.txt} &  Dividends history \\
  {\tt curveconfig.xml} & Curve and term structure composition from individual market instruments\\
  {\tt conventions.xml} & Market conventions for all market data points\\
  {\tt todaysmarket.xml} &  Configuration of the market composition, relevant for the pricing of the given portfolio as
                           of today (yield curves, FX rates, volatility surfaces etc) \\
  {\tt pricingengines.xml} &  Configuration of pricing methods by product\\
  \hline
\end{tabular}
\end{center}
\caption{ORE input files}
\label{tab_1}
\end{table}

The typical list of output files and reports is shown in table \ref{tab_2}. The names of output files can be configured
through the master input file {\tt ore.xml}. Whether these reports are generated also depends on the setting in {\tt
  ore.xml}. For the examples, all output will be written to the directory {\tt Examples/Example\_\#/Output}.

\begin{table}[h]
\scriptsize
\begin{center}
\begin{tabular}{|l|p{11cm}|}
\hline
File Name & Description \\
\hline
{\tt npv.csv}&   NPV report \\
{\tt flows.csv} & Cashflow report \\
{\tt curves.csv} & Generated yield (discount) curves report \\
{\tt xva.csv} & XVA report, value adjustments at netting set and trade level \\
{\tt exposure\_trade\_*.csv} & Trade exposure evolution reports\\
{\tt exposure\_nettingset\_*.csv} &  Netting set exposure evolution reports\\
{\tt rawcube.csv} & NPV cube in readable text format \\
{\tt netcube.csv} & NPV cube after netting and colateral, in readable text format \\
{\tt *.csv.gz} & Intermediate storage of NPV cube and scenario data \\
{\tt *.pdf} &  Exposure graphics produced by the python script {\tt run.py} after ORE completed\\
\hline
\end{tabular}
\end{center}
\caption{ORE output files}
\label{tab_2}
\end{table}

Note: When building ORE from sources on Windows platforms, make sure that you copy your {\tt ore.exe} to the binary
directory {\tt App/bin/win32/} respectively {\tt App/bin/x64/}. Otherwise the examples may be run using the pre-compiled
executables which come with the ORE release.

%--------------------------------------------------------
\subsection{Interest Rate Swap Exposure, Flat Market}\label{sec:example1}
%--------------------------------------------------------

We start with a vanilla single currency Swap (currency EUR, maturity 20y, notional 10m, receive fixed 2\% annual, pay
6M-Euribor flat). The market yield curves (for both discounting and forward projection) are set to be flat at 2\% for
all maturities, i.e. the Swap is at the money initially and remains at the money on average throughout its life. Running
ORE in directory {\tt Examples/Example\_1} with

\medskip
\centerline{\tt python run.py } 
\medskip

yields the exposure evolution in 

\medskip
\centerline{\tt Examples/Example\_1/Output/*.pdf } 
\medskip

and shown in figure \ref{fig_1}. 
\begin{figure}[h!]
\begin{center}
%\includegraphics[scale=0.45]{mpl_swap_1_1m_sbb_100k.pdf}
\includegraphics[scale=0.45]{mpl_swap_1_1m_sbb_10k_flat.pdf}
\end{center}
\caption{Vanilla ATM Swap expected exposure in a flat market environment from both parties' perspectives. The symbols are European Swaption prices. The simulation was run with monthly time steps and 10,000 Monte Carlo samples to demonstrate the convergence of EPE and ENE profiles. A similar
outcome can be obtained more quickly with 5,000 samples on a quarterly time grid which is the default setting of Example\_1. }
\label{fig_1}
\end{figure}
Both Swap simulation and Swaption pricing are run with calls to the ORE executable, essentially 

\medskip
\centerline{\tt ore[.exe] ore.xml} 

\centerline{\tt ore[.exe] ore\_swaption.xml} 
\medskip

which are wrapped into the script {\tt Examples/Example\_1/run.py} provided with the ORE release.
It is instructive to look into the input folder in Examples/Example\_1, the content of the main input file {\tt
  ore.xml}, together with the explanations in section \ref{sec:configuration}. \\

This simple example is an important test case which is also run similarly in one of the unit test suites of ORE. The
expected exposure can be seen as a European option on the underlying netting set, see also appendix
\ref{sec:app_exposure}. In this example, the expected exposure at some future point in time, say 10 years, is equal to
the European Swaption price for an option with expiry in 10 years, underlying Swap start in 10 years and underlying Swap
maturity in 20 years. We can easily compute such standard European Swaption prices for all future points in time where
both Swap legs reset, i.e. annually in this case\footnote{Using closed form expressions for standard European Swaption
  prices.}. And if the simulation model has been calibrated to the points on the Swaption surface which are used for
European Swaption pricing, then we can expect to see that the simulated exposure matches Swaption prices at these annual
points, as in figure \ref{fig_1}.  In Example\_1 we used co-terminal ATM Swaptions for both model calibration and
Swaption pricing. Moreover, as the yield curve is flat in this example, the exposures from both parties'
perspectives (EPE and ENE) match not only at the annual resets, but also for the period between annual reset of both
legs to the point in time when the floating leg resets. Thereafter, between floating leg (only) reset and next joint
fixed/floating leg reset, we see and expect a deviation of the two exposure profiles.

%--------------------------------------------------------
\subsection{Interest Rate Swap Exposure, Realistic Market}\label{sec:example2}
%--------------------------------------------------------

Moving to {\tt Examples/Example\_2}, we see what changes when using a realistic (non-flat) market
environment. Running the example with

\medskip
\centerline{\tt python run.py } 
\medskip

yields the exposure evolution in 

\medskip
\centerline{\tt Examples/Example\_2/Output/*.pdf } 
\medskip

shown in figure \ref{fig_2}.
\begin{figure}[h!]
\begin{center}
\includegraphics[scale=0.45]{mpl_swap_3.pdf}
\end{center}
\caption{Vanilla ATM Swap expected exposure in a realistic market environment as of 05/02/2016 from both parties'
  perspectives. The Swap is the same as in figure \ref{fig_1} but receiving fixed 1\%, roughly at the money. The symbols
  are the prices of European payer and receiver Swaptions. Simulation with 5000 paths and monthly time steps.}
\label{fig_2}
\end{figure}
In this case, where the curves (discount and forward) are upward sloping, the receiver Swap is at the money at inception
only and moves (on average) out of the money during its life. Similarly, the Swap moves into the money from the
counterparty's perspective. Hence the expected exposure evolutions from our perspective (EPE) and the counterparty's
perspective (ENE) 'detach' here, while both can still be be reconciled with payer or respectively receiver Swaption
prices.

%--------------------------------------------------------
\subsection{European Swaption Exposure}\label{sec:european_swaption}
%--------------------------------------------------------

This demo case in folder {\tt Examples/Example\_3} shows the exposure evolution of European Swaptions with cash and
physical delivery, respectively, see figure \ref{fig_3}.
\begin{figure}[h!]
\begin{center}
\includegraphics[scale=0.45]{mpl_swaption.pdf}
\end{center}
\caption{European Swaption exposure evolution, expiry in 10 years, final maturity in 20 years, for cash and physical
  delivery. Simulation with 1000 paths and quarterly time steps. }
\label{fig_3}
\end{figure}
The delivery type (cash vs physical) yields significantly different valuations as of today due to the steepness of the
relevant yield curves (EUR). The cash settled Swaption's exposure graph is truncated at the exercise date, whereas the
physically settled Swaption exposure turns into a Swap-like exposure after expiry. For comparison, the example also
provides the exposure evolution of the underlying forward starting Swap which yields a somewhat higher exposure after
the forward start date than the physically settled Swaption. This is due to scenarios with negative Swap NPV at expiry
(hence not exercised) and positive NPVs thereafter. Note the reduced EPE in case of a Swaption with settlement of the option premium on exercise date.

%--------------------------------------------------------
\subsection{Bermudan Swaption Exposure}
%--------------------------------------------------------

This demo case in folder {\tt Examples/Example\_4} shows the exposure evolution of Bermudan rather than European
Swaptions with cash and physical delivery, respectively, see figure \ref{fig_3b}.
\begin{figure}[h!]
\begin{center}
\includegraphics[scale=0.45]{mpl_bermudan_swaption.pdf}
\end{center}
\caption{Bermudan Swaption exposure evolution, 5 annual exercise dates starting in 10 years, final maturity in 20 years,
  for cash and physical delivery. Simulation with 1000 paths and quarterly time steps.}
\label{fig_3b}
\end{figure}
The underlying Swap is the same as in the European Swaption example in section \ref{sec:european_swaption}. Note in
particular the difference between the Bermudan and European Swaption exposures with cash settlement: The Bermudan shows
the typical step-wise decrease due to the series of exercise dates. Also note that we are using the same Bermudan option
pricing engines for both settlement types, in contrast to the European case, so that the Bermudan option cash and
physical exposures are identical up to the first exercise date. When running this example, you will notice the
significant difference in computation time compared to the European case (ballpark 30 minutes here for 2 Swaptions, 1000
samples, 90 time steps). The Bermudan example takes significantly more computation time because we use an LGM grid
engine for pricing under scenarios in this case. In a realistic context one would more likely resort to American Monte
Carlo simulation, feasible in ORE, but not provided in the current release. However, this implementation can be used to
benchmark any faster / more sophisticated approach to Bermudan Swaption exposure simulation.

%--------------------------------------------------------
\subsection{Callable Swap Exposure}
%--------------------------------------------------------

This demo case in folder {\tt Examples/Example\_5} shows the exposure evolution of a European callable Swap, represented
as two trades - the non-callable Swap and a Swaption with physical delivery. We have sold the call option, i.e. the
Swaption is a right for the counterparty to enter into an offsetting Swap which economically terminates all future flows
if exercised. The resulting exposure evolutions for the individual components (Swap, Swaption), as well as the callable
Swap are shown in figure \ref{fig_4}.
\begin{figure}[h!]
\begin{center}
\includegraphics[scale=0.45]{mpl_callable_swap.pdf}
\end{center}
\caption{European callable Swap represented as a package consisiting of non-callable Swap and Swaption. The Swaption has
  physical delivery and offsets all future Swap cash flows if exercised. The exposure evolution of the package is shown
  here as 'EPE Netting Set' (green line). This is covered by the pink line, the exposure evolution of the same Swap but
  with maturity on the exercise date. The graphs match perfectly here, because the example Swap is deep in the money and
  exercise probability is close to one. Simulation with 5000 paths and quarterly time steps.}
\label{fig_4}
\end{figure}
The example is an extreme case where the underlying Swap is deeply in the money (receiving fixed 5\%), and hence the
call exercise probability is close to one. Modify the Swap and Swaption fixed rates closer to the money ($\approx$ 1\%)
to see the deviation between net exposure of the callable Swap and the exposure of a 'short' Swap with maturity on
exercise.

%--------------------------------------------------------
\subsection{Cap/Floor Exposure}\label{sec:capfloor}
%--------------------------------------------------------

The example in folder {\tt Examples/Example\_6} generates exposure evolutions of several Swaps, caps and floors. The
example shown in figure \ref{fig_capfloor_1} ('portfolio 1') consists of a 20y Swap receiving 3\% fixed and paying
Euribor 6M plus a long 20y Collar
with both cap and floor at 4\% so that the net exposure corresponds to a Swap paying 1\% fixed. \\

\begin{figure}[h!]
\begin{center}
\includegraphics[scale=0.45]{mpl_capfloor_1.pdf}
\end{center}
\caption{Swap+Collar, portfolio 1. The Collar has identical cap and floor rates at 4\% so that it corresponds to a
  fixed leg which reduces the exposure of the Swap, which receives 3\% fixed. Simulation with 1000 paths and quarterly
  time steps.}
\label{fig_capfloor_1}
\end{figure}

The second example in this folder shown in figure \ref{fig_capfloor_2} ('portfolio 2') consists of a short Cap, long
Floor and a long Collar that exactly offsets the netted Cap and Floor.

\begin{figure}[h!]
\begin{center}
\includegraphics[scale=0.45]{mpl_capfloor_2.pdf}
\end{center}
\caption{Short Cap and long Floor vs long Collar, portfolio 2. Simulation with 1000 paths and quarterly time steps.}
\label{fig_capfloor_2}
\end{figure}

Further three test portfolios are provided as part of this example. Run the example and inspect the respective output
directories {\tt Examples/Example\_6/Output/portfolio\_\#}. Note that these directories have to be present/created
before running the batch with {\tt python run.py}.

%--------------------------------------------------------
\subsection{FX Forward and FX Option Exposure}\label{sec:fxfwd}
%--------------------------------------------------------

The example in folder {\tt Examples/Example\_7} generates the exposure evolution for a EUR / USD FX Forward transaction
with value date in 10Y. This is a particularly simple show case because of the single cash flow in 10Y. On the other
hand it checks the cross currency model implementation by means of comparison to analytic limits - EPE and ENE at the
trade's value date must match corresponding Vanilla FX Option prices, as shown in figure \ref{fig_5}.
\begin{figure}[h]
\begin{center}
\includegraphics[scale=0.45]{mpl_fxforward.pdf}
\end{center}
\caption{EUR/USD FX Forward expected exposure in a realistic market environment as of 26/02/2016 from both parties'
  perspectives. Value date is obviously in 10Y. The flat lines are FX Option prices which coincide with EPE and ENE,
  respectively, on the value date. Simulation with 5000 paths and quarterly time steps.}
\label{fig_5}
\end{figure}

%--------------------------------------------------------
\subsection*{FX Option Exposure}\label{sec:fxoption}
%--------------------------------------------------------

This example (in folder {\tt Examples/Example\_7}, as the FX Forward example) illustrates the exposure evolution for an
FX Option, see figure \ref{fig_7}.
\begin{figure}[h!]
\begin{center}
\includegraphics[scale=0.45]{mpl_fxoption.pdf}
\end{center}
\caption{EUR/USD FX Call and Put Option exposure evolution, same underlying and market data as in section
  \ref{sec:fxfwd}, compared to the call and put option price as of today (flat line). Simulation with 5000 paths and
  quarterly time steps.}
\label{fig_7}
\end{figure}
Recall that the FX Option value $NPV(t)$ as of time $0 \leq t \leq T$ satisfies
\begin{align*}
\frac{NPV(t)}{N(t)} &= \mbox{Nominal}\times\E_t\left[\frac{(X(T) - K)^+}{N(T)}\right]\\
NPV(0) &= \E\left[\frac{NPV(t)}{N(t)}\right] = \E\left[\frac{NPV^+(t)}{N(t)} \right]= \EPE(t) 
\end{align*}
where $N(t)$ denotes the numeraire asset.
One would therefore expect a flat exposure evolution up to option expiry. The deviation from this in ORE's simulation is
due to the pricing approach chosen here under scenarios. A Black FX option pricer is used with deterministic Black
volatility derived from today's volatility structure (pushed or rolled forward, see section \ref{sec:sim_market}). The
deviation can be removed by extending the volatility modelling, e.g. implying model consistent Black volatilities in
each simulation step on each path.  
%\todo[inline]{Add exposure evolution graph with 'simulated' FX vol}

%--------------------------------------------------------
\subsection{Cross Currency Swap Exposure, without FX Reset}
%--------------------------------------------------------

The case in {\tt Examples/Example\_8} is a vanilla cross currency Swap. It shows the typical blend of an Interest Rate
Swap's saw tooth exposure evolution with an FX Forward's exposure which increases monotonically to final maturity, see
figure \ref{fig_6}.
\begin{figure}[h!]
\begin{center}
\includegraphics[scale=0.45]{mpl_ccswap.pdf}
\end{center}
\caption{Cross Currency Swap exposure evolution without mark-to-market notional reset. Simulation with 1000 paths and
  quarterly time steps.}
\label{fig_6}
\end{figure}

%--------------------------------------------------------
\subsection{Cross Currency Swap Exposure, with FX Reset}
%--------------------------------------------------------

The effect of the FX resetting feature, common in Cross Currency Swaps nowadays, is shown in {\tt Examples/Example\_9}.
The example shows the exposure evolution of a EUR/USD cross currency basis Swap with FX reset at each interest period
start, see figure \ref{fig_6b}. As expected, the notional reset causes an exposure collapse at each period start when
the EUR leg's notional is reset to match the USD notional.
\begin{figure}[h!]
\begin{center}
\includegraphics[scale=0.45]{mpl_xccy_reset.pdf}
\end{center}
\caption{Cross Currency Basis Swap exposure evolution with and without mark-to-market notional reset. Simulation with
  1000 paths and quarterly time steps.}
\label{fig_6b}
\end{figure}
  
%--------------------------------------------------------
\subsection{Netting Set, Collateral, XVAs, XVA Allocation}
%--------------------------------------------------------

In this example (see folder {\tt Examples/Example\_10}) we showcase a small netting set consisting of three Swaps in
different currencies, with different collateral choices
\begin{itemize}
\item no collateral - figure \ref{fig_8},
\item collateral with threshold (THR) 1m EUR, minimum transfer amount (MTA) 100k EUR, margin period of risk (MPOR) 2
  weeks - figure \ref{fig_9}
\item collateral with zero THR and MTA, and MPOR 2w - figure \ref{fig_10}
\end{itemize}
The exposure graphs with collateral and positive margin period of risk show typical spikes. What is causing these? As
sketched in appendix \ref{sec:app_collateral}, ORE uses a {\em classical collateral model} that applies collateral
amounts to offset exposure with a time delay that corresponds to the margin period of risk. The spikes are then caused
by instrument cash flows falling between exposure measurement dates $d_1$ and $d_2$ (an MPOR apart), so that a
collateral delivery amount determined at $d_1$ but settled at $d_2$ differs significantly from the closeout amount at
$d_2$ causing a significant residual exposure for a short period of time. See for example \cite{Andersen2016} for a
recent detailed discussion of collateral modelling. The approach currently implemented in ORE corresponds to {\em
  Classical+} in \cite{Andersen2016}, the more conservative approach of the classical methods. The less conservative
alternative, {\em Classical-}, would assume that both parties stop paying trade flows at the beginning of the MPOR, so
that the P\&L over the MPOR does not contain the cash flow effect, and exposure spikes are avoided. Note that the size
and position of the largest spike in figure \ref{fig_9} is consistent with a cash flow of the 40 million GBP Swap in the
example's portfolio that rolls over the 3rd of March and has a cash flow on 3 March 2020, a bit more than four years
from the evaluation date.
  
\begin{figure}[h!]
\begin{center}
\includegraphics[scale=0.45]{mpl_nocollateral_epe.pdf}
\end{center}
\caption{Three Swaps netting set, no collateral. Simulation with 5000 paths and bi-weekly time steps.}
\label{fig_8}
\end{figure}

\begin{figure}[htb]
\begin{center}
\includegraphics[scale=0.45]{mpl_threshold_break_epe.pdf}
\end{center}
\caption{Three Swaps netting set, THR=1m EUR, MTA=100k EUR, MPOR=2w. The red evolution assumes that the each trade is
  terminated at the next break date. The blue evolution ignores break dates. Simulation with 5000 paths and bi-weekly
  time steps.}
\label{fig_9}
\end{figure}

%\begin{figure}[h]
%\begin{center}
%\includegraphics[scale=1.0]{example_mta_epe.pdf}
%\end{center}
%\caption{Three swaps, threshold = 0, mta > 0.}
%\label{fig_7}
%\end{figure}

\begin{figure}[h!]
\begin{center}
\includegraphics[scale=0.45]{mpl_mpor_epe.pdf}
\end{center}
\caption{Three Swaps, THR=MTA=0, MPOR=2w. Simulation with 5000 paths and bi-weekly time steps.}
\label{fig_10}
\end{figure}

%--------------------------------------------------------
\subsection*{CVA, DVA, FVA, COLVA, MVA, Collateral Floor}
%--------------------------------------------------------

We use one of the cases in {\tt Examples/Example\_10} to demonstrate the
XVA outputs, see folder {\tt Examples/Example\_10/Output/collateral\_threshold\_dim}.

\medskip The summary of all value adjustments (CVA, DVA, FVA, COLVA, MVA, as well as the Collateral Floor) is provided
in file {\tt xva.csv}.  The file includes the allocated CVA and DVA numbers to individual trades as introduced in the
next section. The following table illustrates the file's layout, omitting the three columns containing allocated data.

\begin{center}
\resizebox{\columnwidth}{!}{%
\begin{tabular}{|l|l|r|r|r|r|r|r|r|r|r|}
\hline
TradeId & NettingSetId & CVA & DVA & FBA & FCA & COLVA & MVA & CollateralFloor & BaselEPE & BaselEEPE \\
\hline
 & CPTY\_A &  6,521  &  151,193  & -946  &  72,103  &  2,769  & -14,203  &  189,936  &  113,260  &  1,211,770 \\
Swap\_1 & CPTY\_A &  127,688  &  211,936  & -19,624  &  100,584  &  n/a  &  n/a  &  n/a   &  2,022,590  &  2,727,010 \\
Swap\_3 & CPTY\_A &  71,315  &  91,222  & -11,270  &  43,370  &  n/a  &  n/a  &  n/a   &  1,403,320  &  2,183,860 \\
Swap\_2 & CPTY\_A &  68,763  &  100,347  & -10,755  &  47,311  &  n/a  &  n/a  &  n/a   &  1,126,520  &  1,839,590 \\
\hline
\end{tabular}
}
\end{center}

The line(s) with empty TradeId column contain values at netting set level, the others contain uncollateralised
single-trade VAs.  Note that COLVA, MVA and Collateral Floor are only available at netting set level at which collateral
is posted.

\medskip
Detailed output is written for COLVA and Collateral Floor to file {\tt colva\_nettingset\_*.csv} which shows the 
incremental contributions to these two VAs through time.


%--------------------------------------------------------
\subsection*{Exposure Reports \& XVA Allocation to Trades}
%--------------------------------------------------------
Using the example in folder {\tt Examples/Example\_10} we illustrate here the layout of an exposure report produced by
ORE. The report shows the exposure evolution of Swap\_1 without collateral which - after running Example\_10 - is found
in folder \\
{\tt Examples/Example\_10/Output/collateral\_none/exposure\_trade\_Swap\_1.csv}:

\begin{center}
\resizebox{\columnwidth}{!}{%
\begin{tabular}{|l|l|r|r|r|r|r|r|r|r|}
\hline
TradeId & Date & Time & EPE & ENE & AllocEPE & AllocENE & PFE & BaselEE & BaselEEE \\
\hline
Swap\_1 & 05/02/16 & 0.0000 & 0  & 1,711,748  & 0  & 0  & 0  & 0  & 0 \\
Swap\_1 & 19/02/16 & 0.0383 & 38,203   & 1,749,913  & -1,200,677 & 511,033 & 239,504 & 38,202 & 38,202 \\
Swap\_1 & 04/03/16 & 0.0765 & 132,862  & 1,843,837 & -927,499 & 783,476 & 1,021,715 & 132,845 & 132,845 \\
%Swap\_1 & 18/03/16 & 0.1148 & 299,155  & 1,742,450  & -650,225  & 793,067  & 1,914,150  & 299,091  & 299,091 \\
%Swap\_1 & 01/04/16 & 0.1530 & 390,178  & 1,834,810  & -552,029  & 892,604  & 2,373,560  & 390,058  & 390,058 \\
%Swap\_1 & 15/04/16 & 0.1913 & 471,849  & 1,918,600  & -465,580  & 981,171  & 2,765,710  & 471,659  & 471,659 \\
%Swap\_1 & 29/04/16 & 0.2295 & 550,301  & 2,000,640  & -330,578  & 1,119,760  & 3,106,810  & 550,016  & 550,016 \\
%Swap\_1 & 13/05/16 & 0.2678 & 620,279  & 2,074,880  & -266,042  & 1,188,560  & 3,427,080  & 619,888  & 619,888 \\
%Swap\_1 & 27/05/16 & 0.3060 & 690,018  & 2,140,320  & -190,419  & 1,259,880  & 3,778,570  & 689,509  & 689,509 \\
%Swap\_1 & 10/06/16 & 0.3443 & 763,207  & 2,206,020  & -137,681  & 1,305,130  & 4,052,870  & 762,560  & 762,560 \\
Swap\_1 & ... & ...& ... & ... & ... & ... & ... & ... & ... \\
\hline
\end{tabular}
}
\end{center}

The exposure measures EPE, ENE and PFE, and the Basel exposure measures $EE_B$ and $EEE_B$, are defined in appendix
\ref{sec:app_exposure}. Allocated exposures are defined in appendix \ref{sec:app_allocation}. The PFE quantile and
allocation method are chosen as described in section \ref{sec:analytics}. \\

In addition to single trade exposure files, ORE produces an exposure file per netting set. The example from the same
folder as above is:

\begin{center}
\resizebox{\columnwidth}{!}{%
\begin{tabular}{|l|l|r|r|r|r|r|r|r|}
\hline
NettingSet & Date & Time & EPE & ENE & PFE & ExpectedCollateral & BaselEE & BaselEEE \\
\hline
CPTY\_A & 05/02/16 & 0.0000 & 1,203,836 & 0 & 1,203,836 & 0 & 1,203,836 & 1,203,836 \\%1,211,770 & 0 & 1,211,770 & 0 & 1,211,770 & 1,211,770\\
CPTY\_A & 19/02/16 & 0.0383 & 1,337,713 & 137,326 & 3,403,460 & 0 & 1,337,651 & 1,337,651 \\ %0.0383 & 1,344,220 & 137,776 & 3,414,000 & 0 & 1,344,160 & 1,344,160\\
%CPTY\_A & 04/03/16 & 0.0765 & 1,518,610 & 308,381 & 4,354,060 & 0 & 1,518,410 & 1,518,410\\
%CPTY\_A & 18/03/16 & 0.1148 & 1,846,900 & 382,068 & 5,200,730 & 0 & 1,846,500 & 1,846,500\\
%CPTY\_A & 01/04/16 & 0.1530 & 1,961,290 & 494,416 & 5,869,470 & 0 & 1,960,690 & 1,960,690\\
%CPTY\_A & 15/04/16 & 0.1913 & 2,067,240 & 598,283 & 6,384,140 & 0 & 2,066,400 & 2,066,400\\
%CPTY\_A & 29/04/16 & 0.2295 & 2,053,670 & 745,960 & 6,740,070 & 0 & 2,052,610 & 2,066,400\\
%CPTY\_A & 13/05/16 & 0.2678 & 2,149,190 & 845,507 & 6,930,230 & 0 & 2,147,840 & 2,147,840\\
%CPTY\_A & 27/05/16 & 0.3060 & 2,235,630 & 930,218 & 7,295,440 & 0 & 2,233,980 & 2,233,980\\
%CPTY\_A & 10/06/16 & 0.3443 & 2,314,470 & 1,014,690 & 7,753,190 & 0 & 2,312,510 & 2,312,510\\
CPTY\_A & ... & ...& ... & ... & ... & ... & ... & ...\\
%CPTY\_A & 07/07/17 & 1.4167 & 3,320,430 & 2,423,890 & 12,787,900 & 0 & 3,304,650 & 3,304,650\\
%CPTY\_A & 21/07/17 & 1.4551 & 3,351,780 & 2,452,640 & 12,964,200 & 0 & 3,335,420 & 3,335,420\\
%CPTY\_A & 04/08/17 & 1.4934 & 3,302,820 & 2,511,500 & 12,796,100 & 0 & 3,286,260 & 3,335,420\\
%CPTY\_A & 18/08/17 & 1.5318 & 3,339,840 & 2,545,850 & 13,120,000 & 0 & 3,322,640 & 3,335,420\\
%CPTY\_A & 01/09/17 & 1.5701 & 3,371,300 & 2,576,100 & 13,238,700 & 0 & 3,353,480 & 3,353,480\\
%CPTY\_A & 15/09/17 & 1.6085 & 3,279,670 & 2,555,370 & 13,041,300 & 0 & 3,261,880 & 3,353,480\\
%CPTY\_A & 29/09/17 & 1.6468 & 3,305,060 & 2,579,200 & 13,072,800 & 0 & 3,286,680 & 3,353,480\\
%CPTY\_A & 13/10/17 & 1.6852 & 3,332,830 & 2,604,200 & 13,225,600 & 0 & 3,313,850 & 3,353,480\\
%CPTY\_A & 27/10/17 & 1.7236 & 3,280,280 & 2,661,770 & 13,034,600 & 0 & 3,261,150 & 3,353,480\\
%CPTY\_A & 13/11/17 & 1.7701 & 3,316,800 & 2,701,060 & 13,331,600 & 0 & 3,296,880 & 3,353,480\\
%CPTY\_A & 24/11/17 & 1.8003 & 3,337,760 & 2,720,870 & 13,402,400 & 0 & 3,317,280 & 3,353,480\\
%CPTY\_A & ... & ...& ... & ... & ... & ... & ... & ...\\
\hline
\end{tabular}
}
\end{center}

Allocated exposures are missing here, as they make sense at the trade level only, and the expected collateral balance is
added for information (in this case zero as collateralisation is deactivated in this example).

\medskip The allocation of netting set exposure and XVA to the trade level is frequently required by finance
departments. This allocation is also featured in {\tt Examples/Example\_10}. We start again with the uncollateralised
case in figure \ref{fig_12}, followed by the case with threshold 1m EUR in figure \ref{fig_13}.
\begin{figure}[h!]
\begin{center}
\includegraphics[scale=0.45]{mpl_nocollateral_allocated_epe.pdf}
\end{center}
\caption{Exposure allocation without collateral. Simulation with 5000 paths and bi-weekly time steps.}
\label{fig_12}
\end{figure}
In both cases we apply the {\em marginal} (Euler) allocation method as published by Pykhtin and Rosen in 2010, hence we
see the typical negative EPE for one of the trades at times when it reduces the netting set exposure. The case with
collateral moreover shows the typical spikes in the allocated exposures.
\begin{figure}[h!]
\begin{center}
\includegraphics[scale=0.45]{mpl_threshold_allocated_epe.pdf}
\end{center}
\caption{Exposure allocation with collateral and threshold 1m EUR. Simulation with 5000 paths and bi-weekly time steps.}
\label{fig_13}
\end{figure}
The analytics results also feature allocated XVAs in file {\tt xva.csv} which are derived from the allocated exposure
profiles. Note that ORE also offers alternative allocation methods to the marginal method by Pykhtin/Rosen, which can be
explored with {\tt Examples/Example\_10}.

%--------------------------------------------------------
\subsection{Basel Exposure Measures}\label{sec:basel}
%--------------------------------------------------------

Example {\tt Example\_11} demonstrates the relation between the evolution of the expected exposure (EPE in our notation)
to the `Basel' exposure measures EE\_B, EEE\_B, EPE\_B and EEPE\_B as defined in appendix \ref{sec:app_exposure}. In
particular the latter is used in internal model methods for counterparty credit risk as a measure for the exposure at
default. It is a `derivative' of the expected exposure evolution and defined as a time average over the running maximum
of EE\_B up to the horizon of one year.
\begin{figure}[h!]
\begin{center}
\includegraphics[scale=0.45]{mpl_basel_exposures.pdf}
\end{center}
\caption{Evolution of the expected exposure of Vanilla Swap, comparison to the `Basel' exposure measures EEE\_B, EPE\_B and EEPE\_B. Note that the latter two are indistinguishable in this case, because the expected exposure is increasing for the first year.}
\label{fig_14}
\end{figure}

%--------------------------------------------------------
\subsection{Long Term Simulation with Horizon Shift}\label{sec:longterm}
%--------------------------------------------------------

The example in folder {\tt Example\_12} finally demonstrates an effect that, at first glance, seems to cause a serious
issue with long term simulations. Fortunately this can be avoided quite easily in the Linear Gauss Markov model setting
that is used here. \\

In the example we consider a Swap with maturity in 50 years in a flat yield curve environment. If we simulate this
naively as in all previous cases, we obtain a particularly noisy EPE profile that does not nearly reconcile with the
known exposure (analytical Swaption prices). This is shown in figure \ref{fig_15} (`no horizon shift'). The origin of
this issue is the width of the risk-neutral NPV distribution at long time horizons which can turn out to be quite small
so that the Monte Carlo simulation with finite number of samples does not reach far enough into the positive or negative
NPV range to adequately sample the distribution, and estimate both EPE and ENE in a single run.  Increasing the number
of samples may not solve the problem, and may not even be feasible in a realistic setting. \\

The way out is applying a `shift transformation' to the Linear Gauss Markov model, see {\tt
  Example\_12/Input/simulation2.xml} in lines 92-95:
\begin{listing}[H]
%\hrule\medskip
\begin{minted}[fontsize=\footnotesize]{xml}
        <ParameterTransformation>
          <ShiftHorizon>30.0</ShiftHorizon>
          <Scaling>1.0</Scaling>
        </ParameterTransformation>
\end{minted}
%\hrule
%\caption{LGM Shift transformation}
%\label{lst:shift_transformation}
\end{listing}

The effect of the 'ShiftHorizon' parameter $T$ is to apply a shift to the Linear Gauss Markov model's $H(t)$ parameter
(see appendix \ref{sec:app_rfe}) {\em after} the model has been calibrated, i.e. to replace:
$$ 
H(t) \rightarrow H(t) - H(T) 
$$ 
It can be shown that this leaves all expectations computed in the model (such as EPE and ENE) invariant. As explained in
\cite{Lichters}, subtracting an $H$ shift effectively means performing a change of measure from the `native' LGM measure
to a T-Forward measure with horizon $T$, here 30 years. Both negative and positive shifts are permissible, but only
negative shifts are connected with a T-Forward measure and improve numerical stability. \\

In our experience it is helpful to place the horizon in the middle of the portfolio duration to significantly improve
the quality of long term expectations. The effect of this change (only) is shown in the same figure \ref{fig_15}
(`shifted horizon').
\begin{figure}[h!]
\begin{center}
\includegraphics[scale=0.45]{mpl_longterm.pdf}
\end{center}
\caption{Long term Swap exposure simulation with and without horizon shift.}
\label{fig_15}
\end{figure}
Figure \ref{fig_15b} further illustrates the origin of the problem and its resolution: The rate distribution's mean
(without horizon shift or change of measure) drifts upwards due to convexity effects (note that the yield curve is flat
in this example), and the distribution's width is then too narrow at long horizons to yield a sufficient number of low
rate scenarios with contributions to the Swap's $\EPE$ (it is a floating rate payer). With the horizon shift (change of
measure), the distribution's mean is pulled 'back' at long horizons, because the convexity effect is effectively wiped
out at the chosen horizon, and the expected rate matches the forward rate.

\begin{figure}[h!]
\begin{center}
\includegraphics[scale=0.45]{mpl_rates.pdf}
\end{center}
\caption{Evolution of rate distributions with and without horizon shift (change of measure). Thick lines indicate mean
  values, thin lines are contours of the rate distribution at $\pm$ one standard deviation.}
\label{fig_15b}
\end{figure}

%--------------------------------------------------------
\subsection{Dynamic Initial Margin and MVA}\label{sec:dim}
%--------------------------------------------------------

This example in folder {\tt Examples/Example\_13} demonstrates Dynamic Initial Margin calculations (see also appendix
\ref{sec:app_dim}) for a number of elementary products:
\begin{itemize}
\item A single currency Swap in EUR (case A), 
\item a European Swaption in EUR with physical delivery (case B), 
\item a single currency Swap in USD (case C), and 
\item a EUR/USD cross currency Swap (case D).
\end{itemize}

The examples can be run as before with 

\medskip
\centerline{\tt python run\_A.py} 

\medskip
and likewise for cases B, C and D. The essential results of each run are are visualised in the form of 
\begin{itemize}
\item evolution of expected DIM
\item regression plots at selected future times 
\end{itemize}
illustrated for cases A and B in figures \ref{fig_ex13a_evolution} - \ref{fig_ex13b_regression}. 
In the three swap cases, the regression orders do make a noticeable difference in the respective expected DIM evolution. In the Swaption case B, first and second order polynomial choice makes a difference before option expiry. More details on this DIM model and its performance can be found in \cite{Anfuso2016,LichtersEtAl}.
 
\begin{figure}[h!]
\begin{center}
\includegraphics[scale=0.45]{mpl_dim_evolution_A_swap_eur.pdf}
\end{center}
\caption{Evolution of expected Dynamic Initial Margin (DIM) for the EUR Swap of Example 13 A. DIM is evaluated using
  regression of NPV change variances versus the simulated 3M Euribor fixing; regression polynomials are zero, first and
  second order (first and second order curves are not distinguishable here). The simulation uses 1000 samples and a time
  grid with bi-weekly steps in line with the Margin Period of Risk.}
\label{fig_ex13a_evolution}
\end{figure}

\begin{figure}[h!]
\begin{center}
\includegraphics[scale=0.45]{mpl_dim_regression_A_swap_eur.pdf}
\end{center}
\caption{Regression snapshot at time step 100 for the EUR Swap of Example 13 A.}
\label{fig_ex13a_regression}
\end{figure}

\begin{figure}[h!]
\begin{center}
\includegraphics[scale=0.45]{mpl_dim_evolution_B_swaption_eur.pdf}
\end{center}
\caption{Evolution of expected Dynamic Initial Margin (DIM) for the EUR Swaption of Example 13 B with expiry in 10Y
  around time step 100. DIM is evaluated using regression of NPV change variances versus the simulated 3M Euribor
  fixing; regression polynomials are zero, first and second order. The simulation uses 1000 samples and a time grid with
  bi-weekly steps in line with the Margin Period of Risk.}
\label{fig_ex13b_evolution}
\end{figure}

\begin{figure}[h!]
\begin{center}
\includegraphics[scale=0.45]{mpl_dim_regression_B_swaption_eur_t100.pdf}
\end{center}
\caption{Regression snapshot at time step 100 (before expiry) for the EUR Swaption of Example 13 B.}
\label{fig_ex13b_regression}
\end{figure}

%--------------------------------------------------------
\subsection{Minimal Market Data Setup}
%--------------------------------------------------------

The example in folder {\tt Examples/Example\_14} demonstrates using a minimal market data setup in order to rerun the vanilla Swap exposure simulation shown in {\tt Examples/Example\_1}. The minimal market data uses single points per curve where possible.

%--------------------------------------------------------
\subsection{Sensitivity Analysis, Stress Testing and Parametric Value-at-Risk}\label{ex:sensitivity_stress}
%--------------------------------------------------------

The example in folder {\tt Examples/Example\_15} demonstrates the calculation of sensitivities and stress scenarios. The
portfolio used in this example consists of

\begin{itemize}
\item a vanilla swap in EUR
\item a cross currency swap EUR-USD
\item a resettable cross currency swap EUR-USD
\item a FX forward EUR-USD
\item a FX call option on USD/GBP % commented out?
\item a FX put option on USD/EUR
\item an European swaption
\item a Bermudan swaption 
\item a cap and a floor in USD
\item a cap and a floor in EUR
\item a fixed rate bond
\item a floating rate bond with floor
\item an Equity call option, put option and forward on S\&P500
\item an Equity call option, put option and forward on Lufthansa
\item a CPI Swap referencing UKRPI
\item a Year-on-Year inflation swap referencing EUHICPXT
\item a USD CDS.
\end{itemize}

The sensitivity configuration in {\tt sensitivity.xml} aims at computing the following sensitivities

\begin{itemize}
\item discount curve sensitivities in EUR, USD; GBP, CHF, JPY, on pillars 6M, 1Y, 2Y, 3Y, 5Y, 7Y, 10Y, 15Y, 20Y (absolute shift of 0.0001)
\item forward curve sensitivities for EUR-EURIBOR 6M and 3M indices, EUR-EONIA, USD-LIBOR 3M and 6M, GBP-LIBOR 3M and
  6M, CHF-LIBOR-6M and JPY-LIBOR-6M indices (absolute shift of 0.0001)
\item yield curve shifts for a bond benchmark curve in EUR (absolute shift of 0.0001)
\item FX spot sensitivities for USD, GBP, CHF, JPY against EUR as the base currency (relative shift of 0.01)
\item FX vegas for USDEUR, GBPEUR, JPYEUR volatility surfaces (relative shift of 0.01)
\item swaption vegas for the EUR surface on expiries 1Y, 5Y, 7Y, 10Y and underlying terms 1Y, 5Y, 10Y (relative shift of 0.01)
\item caplet vegas for EUR and USD on an expiry grid 1Y, 2Y, 3Y, 5Y, 7Y, 10Y and strikes 0.01, 0.02, 0.03, 0.04,
  0.05. (absolute shift of 0.0001)
\item credit curve sensitivities on tenors 6M, 1Y, 2Y, 5Y, 10Y (absolute shift of 0.0001).
\item Equity spots for S\&P500 and Lufthansa
\item Equity vegas for S\&P500 and Lufthansa at expiries 6M, 1Y, 2Y, 3Y, 5Y
\item Zero inflation curve deltas for UKRPI and EUHICPXT at tenors 6M, 1Y, 2Y, 3Y, 5Y, 7Y, 10Y, 15Y, 20Y
\item Year on year inflation curve deltas for EUHICPXT at tenors 6M, 1Y, 2Y, 3Y, 5Y, 7Y, 10Y, 15Y, 20Y
\end{itemize}

Furthermore, mixed second order derivatives (``cross gammas'') are computed for discount-discount, discount-forward and
forward-forward curves in EUR.

By definition the sensitivities are zero rate sensitivities and optionlet sensitivities, no par sensitivities are
provided. The sensitivity analysis produces three output files.

The first, {\tt scenario.csv}, contains the shift
direction ({\tt UP}, {\tt DOWN}, {\tt CROSS}), the base NPV, the scenario NPV and the difference of these two for each
trade and sensitivity key. For an overview over the possible scenario keys see \ref{sec:sensitivity}.

The second file, {\tt sensitivity.csv}, contains the shift size (in absolute terms always) and first (``Delta'') and second
(``Gamma'') order finite differences computed from the scenario results. Note that the Delta and Gamma results are pure
differences, i.e. they are not divided by the shift size.

The second file also contains second order mixed differences according to the specified cross gamma filter, along with the shift sizes for the two factors involved.

The stress scenario definition in {\tt stresstest.xml} defines two stress tests:

\begin{itemize}
\item {\tt parallel\_rates}: Rates are shifted in parallel by 0.01 (absolute). The EUR bond benchmark curve is shifted by
  increasing amounts 0.001, ..., 0.009 on the pillars 6M, ..., 20Y. FX Spots are shifted by 0.01 (relative), FX vols by
  0.1 (relative), swaption and cap floor vols by 0.0010 (absolute).
  Credit curves are not yet shifted.
\item {\tt twist}: The EUR bond benchmark curve is shifted by amounts -0.0050, -0.0040, -0.0030, -0.0020, 0.0020,
  0.0040, 0.0060, 0.0080, 0.0100 on pillars 6M, 1Y, 2Y, 3Y, 5Y, 7Y, 10Y, 15Y, 20Y.
\end{itemize}

The corresponding output file {\tt stresstest.csv} contains the base NPV, the NPV under the scenario shifts and the
difference of the two for each trade and scenario label.

%\todo[inline]{Update after CDS has been added to the example.}
\medskip
Finally, this example demonstrates a parametric VaR calculation based on the sensitivity and cross gamma output from the sensitivity analysis (deltas, vegas, gammas, cross gammas) and an external covariance matrix input. The result in {\tt var.csv} shows a breakdown by portfolio, risk class (All, Interest Rate, FX, Inflation, Equity, Credit) and risk type (All, Delta \& Gamma, Vega). The results shown are Delta Gamma Normal VaRs for the 95\% and 99\% quantile, the holding period is incorporated into the input covariances. Alternatively, one can choose a Monte Carlo VaR which means that the sensitivity based P\&L distribution is evaluated with MC simulation assuming normal respectively log-normal risk factor distribution. 
 
%--------------------------------------------------------
\subsection{Equity Derivatives Exposure}\label{ex:equityderivatives}
%--------------------------------------------------------

The example in folder {\tt Examples/Example\_16} demonstrates the computation of NPV, sensitivities, exposures and XVA for a portfolio 
of OTC equity derivatives. The portfolio used in this example consists of:

\begin{itemize}
	\item an equity call option denominated in EUR (``Luft'')
	\item an equity put option denominated in EUR (``Luft'')
	\item an equity forward denominated in EUR (``Luft'')
	\item an equity call option denominated in USD (``SP5'')
	\item an equity put option denominated in USD (``SP5'')
	\item an equity forward denominated in USD (``SP5'')
	\item an equity Swap in USD with return type  ``price'' (``SP5'')
	\item an equity Swap in USD with return type ``total'' (``SP5'')
\end{itemize}

The step-by-step procedure for running ORE is identical for equities as for other asset classes; the same market and 
portfolio data files are used to store the equity market data and trade details, respectively. For the exposure 
simulation, the calibration parameters for the equity risk factors can be set in the usual {\tt simulation.xml} file.

Looking at the MtM results in the output file {\tt npv.csv} we observe that put-call parity ($V_{Fwd} = V_{Call} - 
V_{Put}$) is observed as expected. Looking at Figure \ref{fig_eq_call} we observe that the Expected Exposure profile of 
the equity call option trade is relatively smooth over time, while for the equity forward trade the Expected Exposure 
tends to increase as we approach maturity. This behaviour is similar to what we observe in sections \ref{sec:fxfwd} 
and \ref{sec:fxoption}. 

\begin{figure}[h!]
	\begin{center}
		\includegraphics[scale=0.45]{mpl_eq_call.pdf}
	\end{center}
	\caption{Equity (``Luft'') call option and OTC forward exposure evolution, maturity in approximately 2.5 years. 
	Simulation with 
	10000 paths and quarterly time steps.}
	\label{fig_eq_call}
\end{figure}

%--------------------------------------------------------
\subsection{Inflation Swap Exposure under Dodgson-Kainth}% Example 17
\label{example:17}
%--------------------------------------------------------

The example portfolio in folder {\tt Examples/Example\_17} contains two CPI Swaps and one Year-on-Year Inflation Swap.
The terms of the three trades are as follows:

\begin{itemize}
\item CPI Swap 1: Exchanges on 2036-02-05 a fixed amount of 20m GBP for a 10m GBP notional inflated with UKRPI with base CPI 210
\item CPI Swap 2: Notional 10m GBP, maturity 2021-07-18, exchanging GBP Libor for GBP Libor 6M vs. $2\%$ x CPI-Factor (Act/Act), inflated with index UKRPI with base CPI 210
\item YOY Swap: Notional 10m EUR, maturity 2021-02-05, exchanging fixed coupons for EUHICPXT year-on-year inflation coupons
\item YOY Swap with capped/floored YOY leg: Notional 10m EUR, maturity 2021-02-05, exchanging fixed coupons for EUHICPXT year-on-year inflation coupons, YOY leg capped with 0.03 and floored with 0.005
\item YOY Swap with scheduled capped/floored YOY leg: Notional 10m EUR, maturity 2021-02-05, exchanging fixed coupons for EUHICPXT year-on-year inflation coupons, YOY leg capped with cap schedule and floored with floor schedule
\end{itemize}

The example generates cash flows, NPVs, exposure evolutions, XVAs, as well as two exposure graphs for CPI Swap 1 respectively the YOY Swap. For the YOY Swap and the both YOY Swaps with capped/floored YOY leg, the example generates their cash flows, NPVs, exposure evolutions, XVAs and sensitivities. Figure \ref{fig_cpi_swap} shows the CPI Swap exposure evolution.

\begin{figure}[h!]
	\begin{center}
		\includegraphics[scale=0.45]{mpl_cpi_swap.pdf}
	\end{center}
	\caption{CPI Swap 1 exposure evolution. Simulation with 1000 paths and quarterly time steps.}
	\label{fig_cpi_swap}
\end{figure}

Figure \ref{fig_yoy_swap} shows the evolution of the 5Y maturity Year-on-Year inflation swap for comparison. Note that the inflation simulation model (Dodgson-Kainth, see appendix \ref{sec:app_rfe}) yields the evolution of inflation indices and inflation zero bonds which allows spanning future inflation zero curves and the pricing of CPI swaps. To price Year-on-Year inflation Swaps under future scenarios, we imply Year-on-Year inflation curves from zero inflation curves\footnote{Currently we discard the required (small) convexity adjustment. This will be supplemented in a subsequent release.}. Note that for pricing Year-on-Year Swaps as of today we use a separate inflation curve bootstrapped from quoted Year-on-Year inflation Swaps.
 
\begin{figure}[h!]
	\begin{center}
		\includegraphics[scale=0.45]{mpl_yoy_swap.pdf}
	\end{center}
	\caption{Year-on-Year Inflation Swap exposure evolution. Simulation with 1000 paths and quarterly time steps.}
	\label{fig_yoy_swap}
\end{figure}

%--------------------------------------------------------
\subsection{Bonds and Amortisation Structures}% Example 18
%--------------------------------------------------------

The example in folder {\tt Examples/Example\_18} computes NPVs and cash flow projections for a vanilla bond portfolio
consisting of a range of bond products, in particular demonstrating amortisation features:
\begin{itemize}
\item fixed rate bond
\item floating rate bond linked to Euribor 6M
\item bond switching from fixed to floating
\item bond with 'fixed amount' amortisation
\item bond with percentage amortisation relative to the initial notional
\item bond with percentage amortisation relative to the previous notional
\item bond with fixed annuity amortisation
\item bond with floating annuity amortisation (this example needs QuantLib 1.10 or higher to work, in particular the amount() method in the Coupon class needs to be virtual)
\item bond with fixed amount amortisation followed by percentage amortisation relative to previous notional
\end{itemize}

After running the example, the results of the computation can be found in the output files {\tt npv.csv} and {\tt
  flows.csv}, respectively.

\medskip
Note that the amortisation features used here are linked to the LegData structure, hence not limited to the Bond instrument, see section \ref{ss:amortisationdata}.

%--------------------------------------------------------
\subsection{Swaption Pricing with Smile}% Example 19
%--------------------------------------------------------

This example in folder {\tt Examples/Example\_19} demonstrates European Swaption pricing with and without smile. Calling

\medskip
\centerline{\tt python run.py}

\medskip
will launch two ORE runs using config files {\tt ore\_flat.xml} and {\tt ore\_smile.xml}, respectively. The only difference in these is referencing alternative market configurations {\tt todaymarket\_flat.xml} and {\tt todaysmarket\_smile.xml} using an ATM Swaption volatility matrix and a Swaption cube, respectively. NPV results are written to {\tt npv\_flat.cvs} and {\tt npv\_smile.csv}.

%--------------------------------------------------------
\subsection{Credit Default Swap Pricing}% Example 20
%--------------------------------------------------------

This example in folder {\tt Examples/Example\_20} demonstrates Credit Default Swap pricing via ORE. Calling

\medskip
\centerline{\tt python run.py}

\medskip
will launch a single ORE run to process a single name CDS example and to generate NPV and cash flows in the usual result files. 

\medskip
CDS can be included in sensitivity analysis and stress testing. Exposure simulation for credit derivatives will follow in the next ORE release.

%--------------------------------------------------------
\subsection{CMS and CMS Cap/Floor Pricing}% Example 21
%--------------------------------------------------------

This example in folder {\tt Examples/Example\_21} demonstrates the pricing of CMS and CMS Cap/Floor using a portfolio consisting of a CMS Swap (CMS leg vs. fixed leg) and a CMS Cap. Calling

\medskip
\centerline{\tt python run.py}

\medskip
will launch a single ORE run to process the portfolio and generate NPV and cash flows in the usual result files. 

\medskip
CMS structures can be included in sensitivity analysis, stress testing and exposure simulation. 

%--------------------------------------------------------
\subsection{Option Sensitivity Analysis with Smile}% Example 22
%--------------------------------------------------------

The example in folder {\tt Examples/Example\_22} demonstrates the current state of sensitivity calculation for European options where the volatility surface has a smile. 

\medskip
The portfolio used in this example consists of
\begin{itemize}
	\item an equity call option denominated in USD (``SP5'')
	\item an equity put option denominated in USD (``SP5'')
	\item a receiver swaption in EUR
	\item an FX call option on EUR/USD
\end{itemize}

\medskip
Refer to appendix \ref{sec:app_sensi} for the current status of sensitivity implementation with smile. In this example the setup is as follows
\begin{itemize}
\item today's market is configured with volatility smile for all three products above
\item simulation market has two configurations, to simulate ``ATM only'' or the ``full surface''; ``ATM only'' means that only ATM volatilities are to be simulated and shifts to ATM vols are propagated to the respective smile section (see appendix \ref{sec:app_sensi});  
\item the sensitivity analysis has two corresponding configurations as well, ``ATM only'' and ``full surface''; note that the ``full surface'' configuration leads to explicit sensitivities by strike only in the case of Swaption volatilities, for FX and Equity volatilities only ATM sensitivity can be specified at the moment and sensitivity output is currently aggregated to the ATM bucket (to be extended in subsequent releases).
\end{itemize}

The respective output files end with ``{\tt\_fullSurface.csv}'' respectively ``{\tt\_atmOnly.csv}''.

%ORE supports two methods of simulating equity volatility smile. The first method simulates the entire surface using specific moneyness levels configured in simulation.xml. The second method simulates only the ATM equity volatilities, the other strikes are shifted relative to this new ATM using the $t_{0}$ smile.  This example compares both methods using the same sensitivity configuration as in {\tt Examples/Example\_15}. For the first method {\tt simulation\_fullSurface.xml} is used and all output files are appended with ``\_fullSurface'', for the second method {\tt simulation\_atmOnly.xml} is used and all output files are appended with ``\_atmOnly''.


%--------------------------------------------------------
\subsection{FRA and Average OIS Exposure}% Example 23
%--------------------------------------------------------

This example in folder {\tt Examples/Example\_23} demonstrates pricing, cash flow projection and exposure simulation for two additional products
\begin{itemize}
\item Forward Rate Agreements
\item Averaging Overnight Index Swaps
\end{itemize}
using a minimal portfolio of four trades, one FRA and three OIS. The essential results are in {\tt npv.csv}, {\tt flows.csv} and 
four {\tt exposure\_trade\_*.csv} files.

%--------------------------------------------------------
\subsection{Commodity Derivatives, Pricing, Sensitivity, Exposure}% Example 24
\label{example:24}
%--------------------------------------------------------

Calling

\medskip
\centerline{\tt python run.py}

\medskip
in folder {\tt Examples/Example\_24} will launch two ORE runs. The first one determined by {\tt ore.xml} demonstrates pricing and sensitivity analysis for
\begin{itemize}
\item Commodity Forwards
\item European Commodity Options
\end{itemize}
using a minimal portfolio of four forwards and two options referencing WTI and Gold. 
The essential results are in {\tt npv.csv} and {\tt sensitivity.csv}.

The second run determined by {\tt ore\_wti.xml} demonstrates Commodity exposure simulation for a portfolio including a
\begin{itemize}
\item Commodity Forward
\item Commodity Swap
\item European Commodity Option
\item Commodity Average Price Option
\item Commodity Swaption
\end{itemize}
with the usual results, exposure reports and graphs. 

%--------------------------------------------------------
\subsection{CMS Spread with (Digital) Cap/Floor}% Example 25
\label{example:25}
%--------------------------------------------------------

The example in folder {\tt Examples/Example\_25}  demonstrates pricing, sensitivity analysis 
and exposure simulation for 
\begin{itemize}
\item Capped/Floored CMS Spreads
\item CMS Spreads with Digital Caps/Floors
\end{itemize}

The example can be run with

\medskip
\centerline{\tt python run.py}

\medskip
and results are in {\tt npv.csv}, {\tt sensitivity.csv}, {\tt exposure\_*.csv} 
and the exposure graphs in {\tt mpl\_cmsspread.pdf}.

%--------------------------------------------------------
\subsection{Bootstrap Consistency}% Example 26
%--------------------------------------------------------

The example in folder {\tt Examples/Example\_26} confirms that bootstrapped curves 
correctly reprice the bootstrap instruments (FRAs, Interest Rate Swaps, FX Forwards, Cross
Currency Basis Swaps) using three pricing setups with
\begin{itemize}
\item EUR collateral discounting (configuration xois\_eur)
\item USD collateral discounting (configuration xois\_usd)
\item in-currency OIS discounting (configuration collateral\_inccy)
\end{itemize}
all defined in {\tt Examples/Input/todaysmarket.xml}.

\medskip
The required portfolio files need to be generated from market data and conventions in
{\tt Examples/Input} and trade templates in 
{\tt Examples/Example\_26/Helpers}, calling

\medskip
\centerline{\tt python TradeGenerator.py}

\medskip
This will place three portfolio files {\tt *\_portfolio.xml} in the input folder.
Thereafter, the three consistency checks can be run calling

\medskip
\centerline{\tt python run.py}

\medskip
Results are in three files {\tt *\_npv.csv} and should show zero NPVs for all benchmark instruments.

%--------------------------------------------------------
\subsection{BMA Basis Swap}% Example 27
\label{example:27}
%--------------------------------------------------------

The example in folder {\tt Examples/Example\_27} demonstrates pricing 
and sensitivity analysis for a series of USD Libor 3M vs. Averaged BMA (SIFMA) 
Swaps that correspond to the instruments used to bootstrap the BMA curve. 

The example can be run with

\medskip
\centerline{\tt python run.py}

\medskip
and results are in {\tt npv.csv} and {\tt sensitivity.csv}.

%--------------------------------------------------------
\subsection{Discount Ratio Curves}% Example 28
\label{example:28}
%--------------------------------------------------------

The example in folder {\tt Examples/Example\_28} shows how to use a yield curve 
built from a DiscountRatio segment. 
In particular, it builds a GBP collateralized in EUR discount curve by referencing 
three other discount curves:
\begin{itemize}
\item a GBP collateralised in USD curve
\item a EUR collateralised in USD curve
\item a EUR OIS curve i.e. a EUR collateralised in EUR curve
\end{itemize}

The implicit assumption in building the curve this way is that EUR/GBP FX 
forwards collateralised in EUR have the same fair market rate as EUR/GBP 
FX forwards collateralised in USD. This assumption is illustrated in the 
example by the NPV of the two forward instruments in the portfolio returning 
exactly 0 under both discounting regimes i.e. under USD collateralization with 
direct curve building and under EUR collateralization with the discount ratio 
modified ``GBP-IN-EUR'' curve.

Also, in this example, an assumption is made that there are no direct GBP/EUR FX 
forward or cross currency quotes available which in general is false. The example 
s merely for illustration.

Both collateralizaton scenarios can be run calling {\tt python run.py}.

%--------------------------------------------------------
\subsection{Curve Building using Fixed vs. Float Cross Currency Helpers}% Example 29
\label{example:29}
%--------------------------------------------------------

The example in folder {\tt Examples/Example\_29} demonstrates using fixed vs. float 
cross currency swap helpers. In particular, it builds a TRY collateralised in USD 
discount curve using TRY annual fixed vs USD 3M Libor swap quotes.

The portfolio contains an at-market fixed vs. float cross currency swap that is 
included in the curve building. The NPV of this swap should be zero when the example is run,
using {\tt python run.py} or ``directly'' calling {\tt ore[.exe] ore.xml}.

%--------------------------------------------------------
\subsection{USD-Prime Curve Building via Prime-LIBOR Basis Swap}% Example 30
\label{example:30}
%--------------------------------------------------------

The example in folder {\tt Examples/Example\_30} demonstrates the implementation of the USD-Prime index in the ORE.
The USD-Prime yield curve is built from USD-Prime vs USD 3M Libor basis swap quotes.
The portfolio consists of two fair basis swaps (NPVs equal to 0):
\begin{itemize}
\item US Dollar Prime Rate vs 3 Month LIBOR
\item US Dollar 3 Month LIBOR vs Fed Funds + 0.027
\end{itemize}

In particular, it is confirmed that the bootstrapped curves USD-FedFunds and USD-Prime follow
the 3\% rule observed on the market: {\tt U.S. Prime Rate = (The Fed Funds Target Rate + 3\%)}.
(See \url{http://www.fedprimerate.com/}.)

Running ORE in directory {\tt Examples/Example\_30} with {\tt python run.py }
yields the USD-Prime curve in {\tt Examples/Example\_30/Output/curves.csv.}

%--------------------------------------------------------
\subsection{Exposure Simulation using a Close-Out Grid}% Example 31
\label{example:31}
%--------------------------------------------------------

In the previous examples we have used a ``lagged'' approach, described at the end of appendix \ref{sec:app_collateral}, to take the Margin Period of Risk into account in exposure modelling. This has the disadvantage in ORE that we need to use equally-spaced time grids with time steps that match the MPoR, e.g. 2W, out to final portfolio maturity. 

In this example we demonstrate an alternative approach supported by ORE since release 6. In this approach we use two nested grids: The (almost) arbitrary main simulation grid is used to compute ``default values'' which feed into the collateral balance $C(t)$ filtered by MTA and Threshold etc; an auxiliary ``close-out'' grid, offset from the main grid by the MPoR, is used to compute the delayed close-out values $V(t)$ associated with time default time $t$. The difference between $V(t)$ and $C(t)$ causes a residual exposure $[V(t)-C(t)]^+$ even if minimum transfer amounts and thresholds are zero.

The close-out date value can be computed in two ways in ORE
\begin{itemize}
\item as of default date, by just evolving the market from default date to close-out date
   (``sticky date''), or 
\item  as of close-out date, by evolving both valuation date and market over the 
   close-out period (``actual date''), i.e., the portfolio ages and cash flows might occur
   in the close-out period causing spikes in the evolution of exposures. 
\end{itemize}

We are reusing one case from Example 10 here, perfect CSA with zero threshold and
minimum transfer amount, so that the remaining exposure is solely due to the MPoR
effect. The portfolio consists of a single at-the-money Swap in GBP. 
The relevant configuration changes that trigger this modelling are in the Parameters section of {\tt simulation.xml} as shown in Listing \ref{lst:close_out_grid}

\begin{listing}[H]
\begin{minted}[fontsize=\footnotesize]{xml}
  <Parameters>
    <Grid> ... </Grid>
    <Calendar> ... </Calendar>
    <Sequence> ... </Sequence>
    <Scenario> ... </Scenario>
    <Seed> ... </Seed>
    <Samples> ... </Samples>
    <CloseOutLag> 2W </CloseOutLag>
    <MporMode> StickyDate </MporMode><!-- Alternative: ActualDate -->
  </Parameters>
\end{minted}
\caption{Close-out grid specification}
\label{lst:close_out_grid}
\end{listing}

and moreover in the XVA analytics section of {\tt ore\_mpor.xml} as shown in Listing \ref{lst:calctype_nolag}.

\begin{listing}[H]
\begin{minted}[fontsize=\footnotesize]{xml}
  <Analytic type="xva">
    ...
    <Parameter name="calculationType"> NoLag </Parameter>
    ...
  </Parameters>
\end{minted}
\caption{Close-out grid specification}
\label{lst:calctype_nolag}
\end{listing}

Run as usual calling {\tt python run.py}.

%--------------------------------------------------------------------
\subsection{Inflation Swap Exposure under Jarrow-Yildrim}% Example 32
\label{example:32}
%--------------------------------------------------------------------

The example here is similar to that in Section \ref{example:17} in that we are generating exposures for inflation swaps. The example in Section \ref{example:17} uses the Dodgson-Kainth model whereas this example uses the Jarrow-Yildrim model. The valuation date is 5 Oct 2020 and the portfolio contains four spot starting inflation swaps:

\begin{itemize}
\item trade\_01: 20Y standard UKRPI ZCIIS struck at the fair market rate of 3.1925\% giving an NPV of 0.0. 
\item trade\_02: 20Y standard EUHICPXT ZCIIS struck at the fair market rate of 1.16875\% giving an NPV of 0.0.
\item trade\_03: 20Y year on year EUHICPXT swap.
\item trade\_04: 20Y year on year UKRPI swap.
\end{itemize}

The example generates cash flows, NPVs, exposure evolutions and XVAs.

%--------------------------------------------------------------------
\subsection{CDS Exposure Simulation}% Example 33
\label{example:33}
%--------------------------------------------------------------------

The example in folder {\tt Examples/Example\_33} is the credit variant of the example in
\ref{sec:example1}. Running ORE in directory {\tt Examples/Example\_33} with

\medskip
\centerline{\tt python run.py } 
\medskip

yields the exposure evolution in 

\medskip
\centerline{\tt Examples/Example\_33/Output/*.pdf } 
\medskip

and shown in figure \ref{fig_33}. 
\begin{figure}[h!]
\begin{center}
\includegraphics[scale=0.45]{mpl_cds_33_2w_10k.pdf}
\end{center}
\caption{Credit Default Swap expected exposure in a flat market environment from both parties' perspectives. The symbols are CDS Option prices. The simulation was run with bi-weekly time steps and 10,000 Monte Carlo samples to demonstrate the convergence of EPE and ENE profiles. A similar
outcome can be obtained more quickly with 5,000 samples on a monthly time grid which is the default setting of Example\_33. }
\label{fig_33}
\end{figure}
Both CDS simulation and CDS Option pricing are run with calls to the ORE executable, essentially 

\medskip
\centerline{\tt ore[.exe] ore.xml} 

\centerline{\tt ore[.exe] ore\_cds\_option.xml} 
\medskip

which are wrapped into the script {\tt Examples/Example\_33/run.py} provided with the ORE release.

This example demonstrates credit simulation using the LGM model and the calculation of Wrong Way Risk due to credit
correlation between the underlying entity of the CDS and the counterparty of the CDS trade via dynamic credit.
Positive correlation between the two names weakens the protection of the CDS whilst
negative correlation strengthens the protection.

The following table lists the XVA result from the example at different levels of correlation.


\begin{table}[hbt]
\scriptsize
\begin{center}
\begin{tabular}{|r|l|r|r|r|r|}
\hline
Correlation & NettingSetId & CVA & DVA & FBA & FCA \\
\hline
-100\%  &  CPTY\_B  &  -2,638  &  2,906  &  486  &  -1,057 \\
 -90\%  &  CPTY\_B  &  -2,204  &  2,906  &  488  &  -1,053 \\
 -50\%  &  CPTY\_B  &    -485  &  2,906  &  493  &  -1,040 \\
 -40\%  &  CPTY\_B  &     -60  &  2,906  &  495  &  -1,037 \\
 -30\%  &  CPTY\_B  &     363  &  2,906  &  496  &  -1,033 \\
 -20\%  &  CPTY\_B  &     784  &  2,906  &  498  &  -1,030 \\
 -10\%  &  CPTY\_B  &   1,204  &  2,906  &  500  &  -1,027 \\
   0\%  &  CPTY\_B  &   1,621  &  2,906  &  501  &  -1,023 \\
  10\%  &  CPTY\_B  &   2,036  &  2,906  &  503  &  -1,020 \\
  20\%  &  CPTY\_B  &   2,450  &  2,906  &  504  &  -1,017 \\
  30\%  &  CPTY\_B  &   2,861  &  2,906  &  506  &  -1,013 \\
  40\%  &  CPTY\_B  &   3,271  &  2,906  &  507  &  -1,010 \\
  50\%  &  CPTY\_B  &   3,679  &  2,906  &  509  &  -1,017 \\
  90\%  &  CPTY\_B  &   5,290  &  2,906  &  515  &    -994 \\
 100\%  &  CPTY\_B  &   5,689  &  2,906  &  517  &    -991 \\
\hline
\end{tabular}
\caption{CDS XVA results with LGM model}
\end{center}
\end{table}

%--------------------------------------------------------------------
\subsection{Wrong Way Risk}% Example 34
\label{example:34}
%--------------------------------------------------------------------

The example in folder {\tt Examples/Example\_34} is an extension of the example in
\ref{sec:example1} with dynamic credit and IR-CR correlation. As we are paying
float, negative correlation implies that we pay more when the counterparty's credit
worsens, leading to a surge of CVA.

The following table lists the XVA result from the example at different levels of correlation.

\begin{table}[hbt]
\scriptsize
\begin{center}
\begin{tabular}{|r|l|r|r|r|r|}
\hline
Correlation & NettingSetId & CVA & DVA & FBA & FCA \\
\hline
 -30\%  &  CPTY\_A  & 105,146  &  68,061  &  31,519  &  -4,127 \\
 -20\%  &  CPTY\_A  &  88,442  &  68,061  &  30,976  &  -4,219 \\
 -10\%  &  CPTY\_A  &  71,059  &  68,061  &  30,439  &  -4,314 \\
   0\%  &  CPTY\_A  &  52,983  &  68,061  &  29,909  &  -4,411 \\
  10\%  &  CPTY\_A  &  34,199  &  68,061  &  29,386  &  -4,511 \\
  20\%  &  CPTY\_A  &  14,691  &  68,061  &  28,869  &  -4,614 \\
  30\%  &  CPTY\_A  &  -5,554  &  68,061  &  28,360  &  -4,719 \\
\hline
\end{tabular}
\caption{IR Swap XVA results with LGM model}
\end{center}
\end{table}

%--------------------------------------------------------------------
\subsection{Flip View}% Example 35
\label{example:35}
%--------------------------------------------------------------------

The example in folder {\tt Examples/Example\_35} demonstrates how ORE can be used to quickly switch perspectives in XVA calculations with minimal changes in the {\tt ore.xml} file only. In particular it does not involve manipulating the portfolio input or the netting set.

%--------------------------------------------------------------------
\subsection{Choice of Measure}% Example 36
\label{example:36}
%--------------------------------------------------------------------

The example in folder {\tt Examples/Example\_36} illustrates the effect of measure changes on simulated expected and peak exposures. For that purpose we reuse Example 1 (un-collateralized vanilla swap exposure) and run the simulation three times with different risk-neutral measures,
\begin{itemize}
\item in the LGM measure as in Example 1 (note {\tt <Measure>LGM</Measure>} in {\tt simulation\_lgm.xml}, this is the default also if the Measure tag is omitted)  
\item in the more common Bank Account measure (note {\tt <Measure>BA</Measure>} in {\tt simulation\_ba.xml})  
\item in the T-Forward measure with horizon T=20 at the Swap maturity (note {\tt <Measure>LGM</Measure>}  and {\tt <ShiftHorizon>20.0</ShiftHorizon>} in {\tt simulation\_fwd.xml})
\end{itemize}

The results are summarized in the exposure evolution graphs in figure \ref{fig:36}. As expected, the expected exposures evolutions match across measures, as these are expected discounted NPVs and hence measure independent.
However, peak exposures are dependent on the measure choice as confirmed graphically here. Many more measures are accessible with ORE, by way of varying the T-Forward horizon which was chosen arbitrarily here to match the Swap's maturity.

\begin{figure}[h!]
\begin{center}
\includegraphics[scale=0.45]{mpl_exposures_measures.pdf}
\end{center}
\caption{Evolution of expected exposures (EPE) and peak exposures (PFE at the 95\% quantile) in three measures, LGM, Bank Account, T-Forward with T=20, with 10k Monte Carlo samples.}
\label{fig:36}
\end{figure}

%--------------------------------------------------------------------
\subsection{Multifactor Hull-White Scenario Generation}% Example 37
\label{example:37}
%--------------------------------------------------------------------

The example in folder {\tt Examples/Example\_37} illustrates the scenario generation under a Hull-White multifactor
model. The model is driven by two independent Brownian motions and has four states. The diffusion matrix sigma is
therefore 2 x 4. The reversion matrix is a 4 x 4 diagonal matrix and entered as an array. Both diffusion and reversion
are constant in time. Their values are not calibrated to the option market, but hardcoded in simulation.xml.

The values for the diffusion and reversion matrices were fitted to the first two principal components of a
(hypothetical) analyis of absolute rate curve movements. These input principal components can be found in
inputeigenvectors.csv in the input folder. The tenor is given in years, and the two components are given as column
vectors, see table \ref{tab:ex37_1}.

\begin{table}[hbt]
\begin{center}
\begin{tabular}{r|r|r}
tenor & eigenvector 1  & eigenvector 2   \\
\hline      
1     & 0.353553390593 & -0.537955502871 \\
2     & 0.353553390593 & -0.374924478795 \\
3     & 0.353553390593 & -0.252916811525 \\
5     & 0.353553390593 & -0.087587539893 \\
10    & 0.353553390593 & 0.12267800393   \\
15    & 0.353553390593 & 0.240659435416  \\
20    & 0.353553390593 & 0.339148675322  \\
30    & 0.353553390593 & 0.552478951238
\end{tabular}
\caption{Input principal components}
\label{tab:ex37_1}
\end{center}
\end{table}

The first eigenvector represent perfectly parallel movements. The second eigenvector represent a rotation around the 7y
point of the curve. Furthermore we prescribe an annual volatility of 0.0070 for the first components and 0.0030 for the
second one. The values can be compared to normal (bp) volatilities.

We follow \cite{Andersen_Piterbarg_2010} chapter 12.1.5 ``Multi-Factor Statistical Gaussian Model'' to calibrate the
diffusion and reversion matrices to the prescribed components and volatilities. We do not detail the procedure here and
refer the interested reader to the given reference.

The example generates a single monte carlo path with 5000 daily steps and outputs the generated scenarios in
scenariodump.csv. The python script pca.py performs a principal component analysis on this output. The model implied
eigenvalues are given in table \ref{tab:ex37_2}.

\begin{table}[hbt]
\begin{center}
\begin{tabular}{r|r}
number & value                  \\
\hline      
1      & 4.9144936649319346e-05 \\
2      & 8.846877641067412e-06  \\
3      & 5.82566039467854e-10   \\
4      & 2.1298948225571415e-10 \\
5      & 9.254913949332787e-11  \\
6      & 1.0861256211767673e-11 \\
7      & 8.478795662698618e-14  \\
8      & 9.74468069377584e-13   \\
\end{tabular}
\caption{Input principal components}
\label{tab:ex37_2}
\end{center}
\end{table}

Only the first two values are relevant, the following are all close to zero. The square root of the first two
eigenvalues is given in table \ref{tab:ex37_3}.

\begin{table}[hbt]
\begin{center}
\begin{tabular}{r|r}
number & sqrt(value)                \\
\hline      
1      & 0.007010344973631422       \\
2      & 0.0029743701250966414      \\
\end{tabular}
\caption{Input principal components}
\label{tab:ex37_3}
\end{center}
\end{table}

matching the prescribed input values of 0.0070 and 0.0030 quite well. The correpsonding eigenvectors are given in etable
\ref{tab:ex37_4}.

\begin{table}[hbt]
\begin{center}
\begin{tabular}{r|r|r}
tenor & eigenvector 1       & eigenvector 2       \\
\hline      
1     & 0.34688826736335926 & 0.5441204725042812  \\
2     & 0.3489303472083185  & 0.380259707350115   \\
3     & 0.350362134519783   & 0.2581408080614405  \\
5     & 0.3523983915961889  & 0.09230899007104967 \\
10    & 0.3550169593982022  & -0.11856777284904292\\
15    & 0.35647835947136625 & -0.23676104168229614\\
20    & 0.3577146190751303  & -0.3354486339442275 \\
30    & 0.36042236352102563 & -0.549124709243042  \\
\end{tabular}
\caption{Input principal components}
\label{tab:ex37_4}
\end{center}
\end{table}

again matching the input principal components quite well. The second eigenvector is the negative of the input vector
here (the principal compoennt analysis can not distinguish these of course).

The example also produces a plot comparing the input eigenvectors and the model implied eigenvectors as shown in figure \ref{fig:ex37}.

\begin{figure}[h!]
\begin{center}
\includegraphics[scale=0.50]{mpl_eigenvectors_ex37.pdf}
\end{center}
\caption{Input and model implied eigenvectors for a Hull-White 4-factor model calibrated to 2 principal components of
  rate curve movements (parallel + rotation). Notice that the model implied 2nd eigenvector is the negative of the input
  vector.}
\label{fig:ex37}
\end{figure}

%--------------------------------------------------------------------
\subsection{Cross Currency Swap Exposure using Multifactor Hull-White Models}% Example 38
\label{example:38}
%--------------------------------------------------------------------

The example in folder {\tt Examples/Example\_38} is similar to Example 8 (EPE, ENE for xccy swap), but uses a
multifactor HW model for EUR and USD to generate scenarios. The parametrization of the HW models is taken from Example
37.

Each of the two factors of each HW model is correlated with each of the two factors of the other currency's HW model and
with the FX factors. Remember that the factors represent principal components of interest rate movements and so the
correlations can be interpreted as correlations of these principal components with each other and the fx rate processes.

%--------------------------------------------------------------------
\subsection{Exposure Simulation using American Monte Carlo}% Example 39
\label{example:39}
%--------------------------------------------------------------------

The example in folder {\tt Examples/Example\_39} demonstrates how to use American Monte Carlo simulation (AMC) to generate exposures in ORE.
For a sketch of the methodology and comments on its implementation in ORE see appendix \ref{sec:app_amc}.

Calling 

\medskip
\centerline {\tt python run.py} 

\medskip
performs two ORE runs, a 'classical' exposure simulation and an American Monte Carlo simulation, both on a quarterly simulation grid and for the same portfolio consisting of four trades:

\begin{itemize}
\item Bermudan swaption
\item Single Currency Swap
\item Cross Currency Swap
\item FX Option
\end{itemize}

We use a 'flat' market here (yield curve and Swaption volatility surface). The number of simulation paths is 2k in the classic simulations. If not stated otherwise below, the number of training paths and simulation paths is 10k in the AMC simulations. 

In the following we compare the AMC exposure profiles to those produced by the 'classic' valuation engine for each trade and the netting set. 

Figure \ref{epe_swaption} shows the EPE and ENE for a Bermudan Swaption 10y into 10y in (base ccy) EUR with physical settlement. The classic run uses
the LGM grid engine for valuation. We observe close agreement between the two runs. To achieve the observed agreement, it is essential to set the LGM model's mean reversion speed to zero in both
\begin{itemize}
\item the Bermudan Swaption LGM pricing model (see Input/pricingengine.xml), and
\item the Cross Asset Model's IR model components (see Input/simulation.xml and Input/simulation\_amc.xml) 
\end{itemize}
and to use a high order 6 of the regression polynomials (see Input/pricingengine\_amc.xml).
 
\begin{figure}
  \includegraphics[width=0.8\textwidth]{mpl_amc_bermudanswaption.pdf}
  \caption{EPE of a EUR Bermudan Swaption computed with the classic and AMC valuation engines, using 50k training paths for the AMC simulation.}
  \label{epe_swaption}
\end{figure}

Figure \ref{epe_swap} shows the EPE and ENE for a 20y vanilla Swap in USD. The currency of
the amc calculator is USD in this case, i.e. it is different from the base ccy of the simulation (EUR). The consistency
of the classic and amc runs in particular demonstrates the correct application of the currency conversion factor
\ref{currency_conversion_factor}. To get a better accuracy for purposes of the plot in this document we increased the
number of training paths for this example to 50k and the order of the basis functions to 6.

\begin{figure}
  \includegraphics[width=0.8\textwidth]{mpl_amc_vanillaswap_usd.pdf}
  \caption{EPE of a USD swap computed with the classic and AMC valuation engines}
  \label{epe_swap}
\end{figure}

Figure \ref{epe_ccyswap} shows the EPE and ENE for a 20y cross currency Swap EUR-USD. 

\begin{figure}
  \includegraphics[width=0.8\textwidth]{mpl_amc_xccyswap.pdf}
  \caption{EPE of a EUR-USD cross currency swap computed with the classic and AMC valuation engines}
  \label{epe_ccyswap}
\end{figure}

Figure \ref{epe_fxoption} shows the EPE and ENE for a vanilla FX Option EUR-USD with 10y1m expiry. 
For the classic run the FX volatility surface is not implied by the cross asset model but kept flat, which
yields a slight hump in the profile. The AMC profile is flat on the other hand which demonstrates the consistency of the
FX Option pricing with the risk factor evolution model.

\begin{figure}
  \includegraphics[width=0.8\textwidth]{mpl_amc_fxoption.pdf}
  \caption{EPE of a EUR-USD FX option computed with the classic and AMC valuation engines}
  \label{epe_fxoption}
\end{figure}

\subsubsection*{Analytic Configuration}
\label{sec:amc_applicationconfig}

To use the AMC engine for an XVA simulation the following needs to be added to the {\tt simulation} analytic in {\tt ore.xml}:

\begin{minted}[fontsize=\scriptsize]{xml}
<Analytic type="simulation">
  ...
  <Parameter name="amc">Y</Parameter>
  <Parameter name="amcPricingEnginesFile">pricingengine_amc.xml</Parameter>
  <Parameter name="amcTradeTypes">Swaption</Parameter>
  ...
</Analytic>
\end{minted}

The trades which have a trade type matching one of the types in the \verb+amcTradeTypes+ list, will be built against the
pricing engine config provided and processed in the AMC engine. As a naming convention, pricing engines with engine type
AMC provide the required functionality to be processed by the AMC engine, for technical details cf. \ref{sec:app_amc}.

All other trades are processed by the classic simulation engine in ORE. The resulting cubes from the classic and AMC
simulation are joined and passed to the post processor in the usual way.

Note that since sometimes the AMC pricing engines have a different base ccy than the risk factor evolution model (see
below), a horizon shift parameter in the simulation set up should be set for all currencies, so that the shift also
applies to these reduced models.

\subsubsection*{Pricing Engine Configuration}
\label{sec:amc_pricingengineconfig}

At this point we assume that the reader is generally familiar with the configuration section 
\ref{sec:configuration}, in particular pricing engine configuration in section \ref{sec:configuration_pricingengines}.

The pricing engine configuration is similar for all AMC enabled products, e.g. for Bermudan Swaptions:

\begin{minted}[fontsize=\scriptsize]{xml}
<Product type="BermudanSwaption">
  <Model>LGM</Model>
  <ModelParameters/>
  <Engine>AMC</Engine>
  <EngineParameters>
    <Parameter name="Training.Sequence">MersenneTwisterAntithetic</Parameter>
    <Parameter name="Training.Seed">42</Parameter>
    <Parameter name="Training.Samples">50000</Parameter>
    <Parameter name="Training.BasisFunction">Monomial</Parameter>
    <Parameter name="Training.BasisFunctionOrder">6</Parameter>
    <Parameter name="Pricing.Sequence">SobolBrownianBridge</Parameter>
    <Parameter name="Pricing.Seed">17</Parameter>
    <Parameter name="Pricing.Samples">0</Parameter>
    <Parameter name="BrownianBridgeOrdering">Steps</Parameter>
    <Parameter name="SobolDirectionIntegers">JoeKuoD7</Parameter>
    <Parameter name="MinObsDate">true</Parameter>
    <Parameter name="RegressionOnExerciseOnly">false</Parameter>
  </EngineParameters>
</Product>
\end{minted}

The \verb+Model+ differs by product type, table \ref{tbl:amcconfig} summarises the supported product types and model and
engine types. The engine parameters are the same for all products:

\begin{enumerate}
\item \verb+Training.Sequence+: The sequence type for the traning phase, can be \verb+MersenneTwister+,
  \verb+MersenneTwisterAntithetc+, \verb+Sobol+, \verb+Burley2020Sobol+, \verb+SobolBrownianBridge+,
  \verb+Burley2020SobolBrownianBridge+
\item \verb+Training.Seed+: The seed for the random number generation in the training phase
\item \verb+Training.Samples+: The number of samples to be used for the training phase
\item \verb+Pricing.Sequence+: The sequence type for the pricing phase, same values allowed as for training
\item \verb+Training.BasisFunction+: The type of basis function system to be used for the regression analysis, can be
  \verb+Monomial+, \verb+Laguerre+, \verb+Hermite+, \verb+Hyperbolic+, \verb+Legendre+, \verb+Chbyshev+,
  \verb+Chebyshev2nd+
\item \verb+BasisFunctionOrder+: The order of the basis function system to be used
\item \verb+Pricing.Seed+: The seed for the random number generation in the pricing
\item \verb+Pricing.Samples+: The number of samples to be used for the pricing phase. If this number is zero, no pricing
  run is performed, instead the (T0) NPV is estimated from the training phase (this result is used to fill the T0 slice
  of the NPV cube)
\item \verb+BrownianBridgeOrdering+: variate ordering for Brownian bridges, can be \verb+Steps+, \verb+Factors+,
  \verb+Diagonal+
\item \verb+SobolDirectionIntegers+: direction integers for Sobol generator, can be \verb+Unit+, \verb+Jaeckel+,
  \verb+SobolLevitan+, \verb+SobolLevitanLemieux+, \verb+JoeKuoD5+, \verb+JoeKuoD6+, \verb+JoeKuoD7+,
  \verb+Kuo+, \verb+Kuo2+, \verb+Kuo3+
\item \verb+MinObsDate+: if true the conditional expectation of each cashflow is taken from the minimum possible
  observation date (i.e. the latest exercise or simulation date before the cashflow's event date); recommended setting
  is \verb+true+
\item \verb+RegressionOnExerciseOnly+: if true, regression coefficients are computed only on exercise dates and
  extrapolated (flat) to earlier exercise dates; only for backwards compatibility to older versions of the AMC module,
  recommended setting is \verb+false+
\end{enumerate}

\begin{table}[hbt]
  \begin{tabular}{l|l|l}
    Product Type & Model & Engine \\ \hline
    Swap & CrossAssetModel & AMC \\
    CrossCurrencySwap & CrossAssetModel & AMC \\
    FxOption & CrossAssetModel & AMC \\
    BermudanSwaption & LGM & AMC \\
    MultiLegOption & CrossAssetModel & AMC \\
  \end{tabular}
  \caption{AMC enabled products with engine and model types}
  \label{tbl:amcconfig}
\end{table}

\subsubsection*{Additional Features}
\label{sec:amc_sideproducts}

As a side product the AMC module provides plain MC pricing engines for Bermudan Swaptions and a new trade type
\verb+MultiLegOption+ with a corresponding MC pricing engine.

\subsubsection*{MC pricing engine for Bermudan swaptions}\label{sec:mc_bermudan_engine}

The following listing shows a sample configuration for the MC Bermudan Swaption engine. The model parameters are
identical to the LGM Grid engine configuration. The engine parameters on the other hand are the same as for the AMC
engine, see \ref{sec:amc_pricingengineconfig}.

\begin{minted}[fontsize=\scriptsize]{xml}
<Product type="BermudanSwaption">
  <Model>LGM</Model>
  <ModelParameters>
    <Parameter name="Calibration">Bootstrap</Parameter>
    <Parameter name="CalibrationStrategy">CoterminalDealStrike</Parameter>
    <Parameter name="Reversion_EUR">0.0050</Parameter>
    <Parameter name="Reversion_USD">0.0030</Parameter>
    <Parameter name="ReversionType">HullWhite</Parameter>
    <Parameter name="VolatilityType">HullWhite</Parameter>
    <Parameter name="Volatility">0.01</Parameter>
    <Parameter name="ShiftHorizon">0.5</Parameter>
    <Parameter name="Tolerance">1.0</Parameter>
  </ModelParameters>
  <Engine>MC</Engine>
  <EngineParameters>
    <Parameter name="Training.Sequence">MersenneTwisterAntithetic</Parameter>
    <Parameter name="Training.Seed">42</Parameter>
    <Parameter name="Training.Samples">10000</Parameter>
    <Parameter name="Training.BasisFunction">Monomial</Parameter>
    <Parameter name="Training.BasisFunctionOrder">6</Parameter>
    <Parameter name="Pricing.Sequence">SobolBrownianBridge</Parameter>
    <Parameter name="Pricing.Seed">17</Parameter>
    <Parameter name="Pricing.Samples">25000</Parameter>
    <Parameter name="BrownianBridgeOrdering">Steps</Parameter>
    <Parameter name="SobolDirectionIntegers">JoeKuoD7</Parameter>
  </EngineParameters>
</Product>
\end{minted}

\subsubsection*{Multi Leg Options / MC pricing engine}

The following listing shows a sample MultiLegOption trade. It consists of

\begin{enumerate}
\item an option data block; this is optional, see below
\item a number of legs; in principle all leg types are supported, the number of legs is arbitrary and they can be in
  different currencies; if the payment currency of a leg is different from a floating index currency, this is
  interpreted as a quanto payoff
\end{enumerate}

If the option block is given, the trade represents a Bermudan swaption on the underlying legs. If the option block is
missing, the legs themselves represent the trade.

See \ref{sec:amc_limitations} for limitations of the multileg option pricing engine.

\begin{minted}[fontsize=\scriptsize]{xml}
<Trade id="Sample_MultiLegOption">
  <TradeType>MultiLegOption</TradeType>
  <Envelope>...</Envelope>
  <MultiLegOptionData>
    <OptionData>
      <LongShort>Long</LongShort>
      <OptionType>Call</OptionType>
      <Style>Bermudan</Style>
      <Settlement>Physical</Settlement>
      <PayOffAtExpiry>false</PayOffAtExpiry>
      <ExerciseDates>
        <ExerciseDate>2026-02-25</ExerciseDate>
        <ExerciseDate>2027-02-25</ExerciseDate>
        <ExerciseDate>2028-02-25</ExerciseDate>
      </ExerciseDates>
    </OptionData>
    <LegData>
      <LegType>Floating</LegType>
      <Payer>false</Payer>
      <Currency>USD</Currency>
      <Notionals>
        <Notional>100000000</Notional>
      </Notionals>
      ...
    </LegData>
    <LegData>
      <LegType>Floating</LegType>
      <Payer>true</Payer>
      <Currency>EUR</Currency>
      <Notionals>
        <Notional>100000000</Notional>
      </Notionals>
      ...
    </LegData>
  </MultiLegOptionData>
</Trade>
\end{minted}

The pricing engine configuration is similar to that of the MC Bermudan swaption engine, cf.
\ref{sec:mc_bermudan_engine}, also see the following listing.

\begin{minted}[fontsize=\scriptsize]{xml}
  <Product type="MultiLegOption">
  <Model>CrossAssetModel</Model>
  <ModelParameters>
    <Parameter name="Tolerance">0.0001</Parameter>
    <!-- IR -->
    <Parameter name="IrCalibration">Bootstrap</Parameter>
    <Parameter name="IrCalibrationStrategy">CoterminalATM</Parameter>
    <Parameter name="ShiftHorizon">1.0</Parameter>
    <Parameter name="IrReversion_EUR">0.0050</Parameter>
    <Parameter name="IrReversion_GBP">0.0070</Parameter>
    <Parameter name="IrReversion_USD">0.0080</Parameter>
    <Parameter name="IrReversion">0.0030</Parameter>
    <Parameter name="IrReversionType">HullWhite</Parameter>
    <Parameter name="IrVolatilityType">HullWhite</Parameter>
    <Parameter name="IrVolatility">0.0050</Parameter>
    <!-- FX -->
    <Parameter name="FxCalibration">Bootstrap</Parameter>
    <Parameter name="FxVolatility_EURUSD">0.10</Parameter>
    <Parameter name="FxVolatility">0.08</Parameter>
    <Parameter name="ExtrapolateFxVolatility_EURUSD">false</Parameter>
    <Parameter name="ExtrapolateFxVolatility">true</Parameter>
    <!-- Correlations IR-IR, IR-FX, FX-FX -->
    <Parameter name="Corr_IR:EUR_IR:GBP">0.80</Parameter>
    <Parameter name="Corr_IR:EUR_FX:GBPEUR">-0.50</Parameter>
    <Parameter name="Corr_IR:GBP_FX:GBPEUR">-0.15</Parameter>
  </ModelParameters>
  <Engine>MC</Engine>
  <EngineParameters>
    <Parameter name="Training.Sequence">MersenneTwisterAntithetic</Parameter>
    <Parameter name="Training.Seed">42</Parameter>
    <Parameter name="Training.Samples">10000</Parameter>
    <Parameter name="Pricing.Sequence">SobolBrownianBridge</Parameter>
    <Parameter name="Pricing.Seed">17</Parameter>
    <Parameter name="Pricing.Samples">25000</Parameter>
    <Parameter name="Training.BasisFunction">Monomial</Parameter>
    <Parameter name="Training.BasisFunctionOrder">4</Parameter>
    <Parameter name="BrownianBridgeOrdering">Steps</Parameter>
    <Parameter name="SobolDirectionIntegers">JoeKuoD7</Parameter>
  </EngineParameters>
</Product>
\end{minted}

Model Parameters special to that product are

\begin{enumerate}
\item \verb+IrCalibrationStrategy+ can be \verb+None+, \verb+CoterminalATM+, \verb+UnderlyingATM+
\item \verb+FXCalibration+ can be \verb+None+ or \verb+Bootstrap+
\item \verb+ExtrapolateFxVolatility+ can be \verb+true+ or \verb+false+; if false, no calibration instruments are used
  that require extrapolation of the market fx volatilty surface in option expiry direction
\item \verb+Corr_Key1_Key2+: These entries describe the cross asset model correlations to be used; the syntax for
  \verb+Key1+ and \verb+Key2+ is the same as in the simulation configuration for the cross asset model
\end{enumerate}

%--------------------------------------------------------------------
\subsection{Par Sensitivity Analysis}% Example 40
\label{example:40}
%--------------------------------------------------------------------

The example in folder {\tt Examples/Example\_40}  demonstrates ORE's par sensitivity analysis (e.g. to Swap rates) 
that is implemented  by means of a Jacobi transformation of the "raw" sensitivities (e.g. to zero rates), see a sketch of the 
methodology in appendix \ref{app:par_sensi} and section \ref{sec:sensitivity} for configuration details.

To perform a par sensitivity analysis, the following required change in {\tt ore.xml} is required

\begin{minted}[fontsize=\scriptsize]{xml}
    <Analytic type="sensitivity">
      <Parameter name="active">Y</Parameter>
      <Parameter name="marketConfigFile">simulation.xml</Parameter>
      <Parameter name="sensitivityConfigFile">sensitivity.xml</Parameter>
      <Parameter name="pricingEnginesFile">../../Input/pricingengine.xml</Parameter>
      <Parameter name="scenarioOutputFile">sensi_scenarios.csv</Parameter>
      <Parameter name="sensitivityOutputFile">sensitivity.csv</Parameter>
      <Parameter name="outputSensitivityThreshold">0.000001</Parameter>
      <!-- Additional parametrisation for par sensitivity analysis -->
      <Parameter name="parSensitivity">Y</Parameter>
      <Parameter name="parSensitivityOutputFile">parsensitivity.csv</Parameter>
      <Parameter name="outputJacobi">Y</Parameter>
      <Parameter name="jacobiOutputFile">jacobi.csv</Parameter>
      <Parameter name="jacobiInverseOutputFile">jacobi_inverse.csv</Parameter>
    </Analytic>
\end{minted}

The portfolio used in this example includes products sensitive to 
\begin{itemize}
\item Discount and index curves
\item Credit curves
\item Inflation curves
\item CapFloor volatilities
\end{itemize}

The usual sensitivity analysis is performed by bumping the "raw" rates (zero rates, hazard rates, inflation zero rates, optionlet vols).
This is followed by the Jacobi transformation that turns "raw" sensitivities  into sensitivities in the par domain (Deposit/FRA/Swap rates, FX Forwards, CC Basis Swap spreads, 
CDS spreads, ZC and YOY Inflation Swap rates, flat Cap/Floor vols). The conversion is controlled by the additional {\tt ParConversion} data blocks 
in {\tt sensitivity.xml} where the assumed par instruments and corresponding conventions are coded, as shown below for three types of discount curves.

\begin{minted}[fontsize=\scriptsize]{xml}
  <DiscountCurves>
  
    <DiscountCurve ccy="EUR">
      <ShiftType>Absolute</ShiftType>
      <ShiftSize>0.0001</ShiftSize>
      <ShiftTenors>2W,1M,3M,6M,9M,1Y,2Y,3Y,4Y,5Y,7Y,10Y,15Y,20Y,25Y,30Y</ShiftTenors>
      <ParConversion>
        <!--DEP, FRA, IRS, OIS, FXF, XBS -->
	<Instruments>OIS,OIS,OIS,OIS,OIS,OIS,OIS,OIS,OIS,OIS,OIS,OIS,OIS,OIS,OIS,OIS</Instruments>
	<SingleCurve>true</SingleCurve>
	<Conventions>
	  <Convention id="OIS">EUR-OIS-CONVENTIONS</Convention>
	</Conventions>
      </ParConversion>
    </DiscountCurve>   
    
    <DiscountCurve ccy="USD">
      <ShiftType>Absolute</ShiftType>
      <ShiftSize>0.0001</ShiftSize>
      <ShiftTenors>2W,1M,3M,6M,9M,1Y,2Y,3Y,4Y,5Y,7Y,10Y,15Y,20Y,25Y,30Y</ShiftTenors>
      <ParConversion>
	<Instruments>FXF,FXF,FXF,FXF,FXF,XBS,XBS,XBS,XBS,XBS,XBS,XBS,XBS,XBS,XBS,XBS</Instruments>
	<SingleCurve>true</SingleCurve>
	<Conventions>
	  <Convention id="XBS">EUR-USD-XCCY-BASIS-CONVENTIONS</Convention>
	  <Convention id="FXF">EUR-USD-FX-CONVENTIONS</Convention>
	</Conventions>
      </ParConversion>

    <DiscountCurve ccy="GBP">
      <ShiftType>Absolute</ShiftType>
      <ShiftSize>0.0001</ShiftSize>
      <ShiftTenors>2W,1M,3M,6M,9M,1Y,2Y,3Y,4Y,5Y,7Y,10Y,15Y,20Y,25Y,30Y</ShiftTenors>
      <ParConversion>
	<Instruments>DEP,DEP,DEP,DEP,DEP,IRS,IRS,IRS,IRS,IRS,IRS,IRS,IRS,IRS,IRS,IRS</Instruments>
	<SingleCurve>true</SingleCurve>
	<Conventions>
	  <Convention id="DEP">GBP-DEPOSIT</Convention>
	  <Convention id="IRS">GBP-6M-SWAP-CONVENTIONS</Convention>
	</Conventions>
      </ParConversion>
    </DiscountCurve>
  
  </DiscountCurves>
\end{minted}

Finally note that par sensitivity analysis requires that the shift tenor grid in the sensitivity data above matches the corresponding grid in the simulation (market) configuration.
See also section \ref{sec:sensitivity}.

%--------------------------------------------------------------------
\subsection{Multi-threaded Exposure Simultion}% Example 41
\label{example:41}
%--------------------------------------------------------------------

The example in folder {\tt Examples/Example\_41} demonstrates the multithreaded valuation engine to generate the exposure for a
portfolio of 8 copies of the vanilla swap in {\tt Example\_1}.

%--------------------------------------------------------------------
\subsection{ORE Python Module}% Example 42
\label{example:42}
%--------------------------------------------------------------------

Since release 9 (March 2023) we provide easy access to ORE via a pre-compiled Python module. Some example scripts using this ORE module are provided in this example, so change to this directory first

\medskip
{\tt cd Example\_42} 

\medskip
The examples require Python 3. The ORE Python module is then installed with a one-liner, see step 3 below. However, to separate ORE from any other Python environments on your machine, we recommend creating a virtual environment first. In that case the steps are as follows. 

\begin{enumerate}
\item To create a virtual environment: {\tt python -m venv env1} 
\item To activate this environment on Windows: {\tt .{\bs}env1{\bs}Scripts{\bs}activate.bat}  \\
or on macOS/Linux: {\tt ./env1/bin/activate }  
\item Then install the latest release of ORE:\\
{\tt pip install open-source-risk-engine } 
\item Try examples:\\
	\begin{itemize} 
	\item {\tt python ore.py} \\
	This demonstrates the Python-wrapped version of the ORE application that is also used in the command line application {\tt ore.exe}. We use it here to re-run the Swap exposure of {\tt Example\_1}. 
	\item {\tt python ore2.py} \\
	This extends the previous example and shows how to access and post-process ORE in-memory results in the Python framework without reading files. 
	\item {\tt python commodityforward.py} \\
	The ORE Python module also allows lower-level access to the QuantLib and QuantExt libraries, demonstrated here for a CommodityForward instrument defined in QuantExt. 
	Note that the ORE Python module contains the entire QuantLib Python functionality.
	\end{itemize}
	More use cases of the ORE Python module including Jupyter notebooks can be found in the ORE SWIG repository, in particular in folder OREAnalytics-SWIG/Python/Examples. 
\item You can deactivate the environment with {\tt deactivate} \\
or even fully remove the environment again by removing the {\tt env1} folder.
\end{enumerate}

Finally, you can build the Python module and installable packages yourself following the instructions in sections \ref{sec:oreswig}, \ref{sec:win_wheel}, \ref{sec:nix_wheel}
based on your local ORE code. 

%--------------------------------------------------------------------
\subsection{Credit Portfolio Model}% Example 43
\label{example:43}
%--------------------------------------------------------------------

The purpose of the credit portfolio model in ORE is to generate an integrated portfolio gain/loss distribution at a given future horizon which is driven by 
\begin{itemize}
\item credit defaults and rating migrations in Bonds and CDS, and 
\item the PnL of a portfolio of derivatives over the specified time horizon.
\end{itemize}
The model integrates Credit and Market Risk by jointly evolving systemic credit risk drivers alongside the usual risk factors in ORE's Cross Asset Model.
See also the separate documentation in Docs/UserGuide/creditmodel.tex.

By running \\
\medskip
\centerline{{\tt python run.py}} 

\medskip
this example demonstrates the model's outcome for seven demo portfolios

\begin{center}
\begin{tabular}{|l|l|l|l|}
\hline
Case & Credit Mode & Exposure Mode & Evaluation \\
\hline
\hline
Single Bond & Migration & Value & Analytic \\
\hline
Bond and Swap & Migration & Value & Analytic \\
\hline
3 Bonds & Migration & Value & Analytic \\
\hline
10 Bonds & Migration & Value & Analytic \\
\hline
10 Bonds & Migration & Value & Terminal Simulation \\
\hline
Bonds and CDS & Migration & Notional & Analytic \\
\hline
100 Bonds & Default & Notional & Analytic \\
\hline
\end{tabular}
\end{center}
The last demo case in this table can be activated by uncommenting the corresponding section at the end of the {\tt run.py} script.
 
%--------------------------------------------------------------------
\subsection{ISDA SIMM Model}% Example 44
\label{example:44}
%--------------------------------------------------------------------

This example demonstrates the calculation of initial margin using ISDA's Standard Initial Margin Model (SIMM) based on a provided 
sensitivity file in ISDA's Common Risk Interchange Format (CRIF).
ORE covers all SIMM versions since inception to date, i.e.\ 1.0, 1.1, 1.2, 1.3, 1.3.38, 2.0, 2.1, 2.2, 2.3, 2.4 (=2.3.8), 2.5, 2.5A, 2.6 (=2.5.6).
All versions have been tested against the respective ISDA SIMM model unit test suites and pass these tests.
Any new SIMM versions will be added with each ORE release.

For SIMM versions >= 2.2 we support SIMM calculation for both MPoR horizons, 1d and 10d.
 
Note that you need to purchase a SIMM model license from ISDA if you want to use the model in production, and the unit test
suites mentioned above are provided to licensed vendors only. Therefore we unfortunately cannot share our ORE SIMM model 
test suite here either. 

By running \\
\medskip
\centerline{{\tt python run.py}} 

\medskip
ORE will pick up the small example CRIF file in {\tt Input/crif.csv} (i.e.\ par sensitivities rebucketed and reformatted to match the ISDA CRIF template) and generate the resulting SIMM report in a {\tt simm.csv} file.
This report shows ISDA SIMM results with the usual breakdown by product class, risk class, margin type, bucket and SIMM ``side'' (IM to call or post).
The SIMM calculation in this example is done for SIMM version 2.4 and 2.6, with MPoR 1d and 10d:

\begin{itemize}
  \item SIMM 2.4, 1-day MPoR
  \item SIMM 2.4, 10-day MPoR
  \item SIMM 2.6, 1-day MPoR
  \item SIMM 2.6, 10-day MPoR
\end{itemize}

\medskip
There are four input files -- {\tt ore\_SIMM2.4\_1D.xml}, {\tt ore\_SIMM2.4\_10D.xml}, {\tt ore\_SIMM2.6\_1D.xml}, {\tt ore\_SIMM2.6\_10D.xml} -- with corresponding folders in the {\tt Output/} directory.
The relevant inputs in the files are:

\begin{itemize}
\item SIMM version
\item name of the CRIF file to be loaded
\item calculation currency - this determines which Risk\_FX entries of the CRIF will be ignored in the SIMM calculation
\item result currency (optional) - currency of the resulting SIMM amounts in the report, by default equal to the calculation currency
\item MPoR horizon, in terms of days
\end{itemize}

The market data input and todays's market configuration required here is minimal - limited to FX rates for conversions from base/calculation currency into USD and into the result currency.

\bigskip
If the ORE Python module is installed, as shown in Example 42, then you can also run the SIMM example using

\medskip
\centerline{\tt python ore.py} 

%--------------------------------------------------------
\subsection{Collateralized Bond Obligation}% Example 45
%--------------------------------------------------------

This example in folder {\tt Examples/Example\_45} demonstrates a Cashflow CDO or Collateralized Bond Obligation (CBO) via ORE. Calling

\medskip
\centerline{\tt python run.py}

\medskip
will launch a single ORE run to process a CBO example, referencing underyling bond portfolio of 20 trades. 
The CBO is represented by a CBO reference datum specified in the reference data file. 
NPV results are calculated for the investment in the junior tranche. 

%--------------------------------------------------------
\subsection{Generic Total Return Swap}% Example 46
%--------------------------------------------------------

This example in folder {\tt Examples/Example\_46} demonstrates ORE's generic Total Return Swap referencing a CBO. 
Calling

\medskip
\centerline{\tt python run.py}

\medskip
will launch a single ORE run to process a TRS example and to generate NPV and cash flows in the usual result files.
As opposed to example 45, the CBO and its bondbasket are represented explicitly in the CBO node.

%--------------------------------------------------------
\subsection{Composite Trade}% Example 47
%--------------------------------------------------------

This example in folder {\tt Examples/Example\_47} demonstrates the input of ORE's Composite Trade that can consist on any number 
and type of products covered by ORE. In this case the composite consists of two Equity Swaps.
Calling

\medskip
\centerline{\tt python run.py}

\medskip
runs ORE and generates an NPV report.

%--------------------------------------------------------
\subsection{Convertible Bond and ASCOT}% Example 48
%--------------------------------------------------------

This example in folder {\tt Examples/Example\_48} demonstrates the input of 
\begin{itemize}
\item a ConvertibleBond  trade
\item a related Asset Swapped Convertible Option Transaction (ASCOT)
\item a vanilla Swap that represents the package of Convertible Bond position and ASCOT
\end{itemize}

Calling
\medskip
\centerline{\tt python run.py}

\medskip
runs ORE and generates an NPV report.

%--------------------------------------------------------
\subsection{Bond Yield Shifted}% Example 49
\label{example:49}
%--------------------------------------------------------

The example in folder {\tt Examples/Example\_49} shows how to use a yield curve
built from a BondYieldShifted segment, as described in section \ref{sec:bond_yield_shifted}.

In particular, it builds the curve {\tt USD.BMK.GVN.CURVE\_SHIFTED} shifted by three liquid Bonds:

\begin{itemize}
\item Fixed rate USD Bond maturing in August 2023 with id {\tt EJ7706660}.
\item Fixed rate USD Bond maturing in September 2049 with id {\tt ZR5330686}.
\item Floating Rate Bond maturing in May 2025 with id {\tt AS064441}.
\end{itemize}

The resulting curve is exhibited in the {\tt curves.csv} output file.
Moreover, the results can be crosschecked against the NPVs, i.e. prices, of the ZeroBonds comprised in the portfolio.
\begin{itemize}
\item {\tt ZeroBond\_long}, maturing 2052-06-03 shows a price of 0.2022 akin to the 0.2022 in the curves output at the same date.
\item {\tt ZeroBond\_short}, maturing 2032-06-01 shows a price of 0.5754 aktin to the 0.5754 in the curves output at the same date.
\end{itemize}

The example can be run calling {\tt python run.py}.


%--------------------------------------------------------------------
\subsection{Zero to Par sensitivity Conversion Analysis}% Example 50
\label{example:50}
%--------------------------------------------------------------------

The example in folder {\tt Examples/Example\_50} demonstrates ORE's capability to convert external computed zero sensitivities (e.g Zero rates) to par sensitivities (e.g. to Swap rates) 
that is implemented  by means of a Jacobi transformation of the "raw" sensitivities (e.g. to zero rates), see a sketch of the 
methodology in appendix \ref{app:par_sensi} and section \ref{sec:sensitivity} for configuration details.

To perform a par sensitivity analysis, the following required change in {\tt ore.xml} is required

\begin{minted}[fontsize=\scriptsize]{xml}
    <Analytic type="zeroToParSensiConversion">
      <Parameter name="active">Y</Parameter>
      <Parameter name="marketConfigFile">simulation.xml</Parameter>
      <Parameter name="sensitivityConfigFile">sensitivity.xml</Parameter>
      <Parameter name="pricingEnginesFile">../../Input/pricingengine.xml</Parameter>
	  <!-- Input file with the raw sensitivities -->
      <Parameter name="sensitivityInputFile">sensitivity.csv</Parameter>
      <Parameter name="idColumn">TradeId</Parameter>
      <Parameter name="riskFactorColumn">Factor_1</Parameter>
      <Parameter name="deltaColumn">Delta</Parameter>
	  <Parameter name="currencyColumn">Currency</Parameter>
	  <Parameter name="baseNpvColumn">Base NPV</Parameter>
 	  <Parameter name="shiftSizeColumn">ShiftSize_1</Parameter>
      <Parameter name="outputThreshold">0.000001</Parameter>
      <Parameter name="outputFile">parconversion_sensitivity.csv</Parameter>
      <Parameter name="outputJacobi">Y</Parameter>
      <Parameter name="jacobiOutputFile">jacobi.csv</Parameter>
      <Parameter name="jacobiInverseOutputFile">jacobi_inverse.csv</Parameter>
    </Analytic>
\end{minted}

The portfolio used in this example includes zero sensitivities of 
\begin{itemize}
\item Discount and index curves
\item Credit curves
\item Inflation curves
\item CapFloor volatilities
\end{itemize}

ORE reads the raw sensitivites from the csv input file *sensitivityInputFile*. The input file needs to have six  columns, the column names can be user configured. Here is a description of each of the columns:

\begin{enumerate}
\item idColumn : Column with a unique identifier for the trade / nettingset / portfolio.
\item riskFactorColumn: Column with the identifier of the zero/raw sensitiviy. The risk factor name needs to follow the ORE naming convention, e.g. DiscountCurve/EUR/5/1Y (the 6th bucket in EUR discount curve as specified in the sensitivity.xml)\
\item deltaColumn: The raw sensitivity of the trade/nettingset / portfolio with respect to the risk factor
\item currencyColumn: The currency in which the raw sensitivity is expressed, need to be the same as the BaseCurrency in the simulation settings.
\item shiftSizeColumn: The shift size applied to compute the raw sensitivity, need to be consistent to the sensitivity configuration.
\item baseNpvColumn: The base npv of the trade / nettingset / portfolio in currency.
\end{enumerate}

This is followed by the Jacobi transformation that turns "raw" sensitivities  into sensitivities in the par domain (Deposit/FRA/Swap rates, FX Forwards, CC Basis Swap spreads, 
CDS spreads, ZC and YOY Inflation Swap rates, flat Cap/Floor vols). The conversion is controlled by the additional {\tt ParConversion} data blocks 
in {\tt sensitivity.xml} where the assumed par instruments and corresponding conventions are coded, as shown below for three types of discount curves.

\begin{minted}[fontsize=\scriptsize]{xml}
  <DiscountCurves>
  
    <DiscountCurve ccy="EUR">
      <ShiftType>Absolute</ShiftType>
      <ShiftSize>0.0001</ShiftSize>
      <ShiftTenors>2W,1M,3M,6M,9M,1Y,2Y,3Y,4Y,5Y,7Y,10Y,15Y,20Y,25Y,30Y</ShiftTenors>
      <ParConversion>
        <!--DEP, FRA, IRS, OIS, FXF, XBS -->
	<Instruments>OIS,OIS,OIS,OIS,OIS,OIS,OIS,OIS,OIS,OIS,OIS,OIS,OIS,OIS,OIS,OIS</Instruments>
	<SingleCurve>true</SingleCurve>
	<Conventions>
	  <Convention id="OIS">EUR-OIS-CONVENTIONS</Convention>
	</Conventions>
      </ParConversion>
    </DiscountCurve>   
    
    <DiscountCurve ccy="USD">
      <ShiftType>Absolute</ShiftType>
      <ShiftSize>0.0001</ShiftSize>
      <ShiftTenors>2W,1M,3M,6M,9M,1Y,2Y,3Y,4Y,5Y,7Y,10Y,15Y,20Y,25Y,30Y</ShiftTenors>
      <ParConversion>
	<Instruments>FXF,FXF,FXF,FXF,FXF,XBS,XBS,XBS,XBS,XBS,XBS,XBS,XBS,XBS,XBS,XBS</Instruments>
	<SingleCurve>true</SingleCurve>
	<Conventions>
	  <Convention id="XBS">EUR-USD-XCCY-BASIS-CONVENTIONS</Convention>
	  <Convention id="FXF">EUR-USD-FX-CONVENTIONS</Convention>
	</Conventions>
      </ParConversion>

    <DiscountCurve ccy="GBP">
      <ShiftType>Absolute</ShiftType>
      <ShiftSize>0.0001</ShiftSize>
      <ShiftTenors>2W,1M,3M,6M,9M,1Y,2Y,3Y,4Y,5Y,7Y,10Y,15Y,20Y,25Y,30Y</ShiftTenors>
      <ParConversion>
	<Instruments>DEP,DEP,DEP,DEP,DEP,IRS,IRS,IRS,IRS,IRS,IRS,IRS,IRS,IRS,IRS,IRS</Instruments>
	<SingleCurve>true</SingleCurve>
	<Conventions>
	  <Convention id="DEP">GBP-DEPOSIT</Convention>
	  <Convention id="IRS">GBP-6M-SWAP-CONVENTIONS</Convention>
	</Conventions>
      </ParConversion>
    </DiscountCurve>
  
  </DiscountCurves>
\end{minted}

Finally note that par sensitivity analysis requires that the shift tenor grid in the sensitivity data above matches the corresponding grid in the simulation (market) configuration. 
See also section \ref{sec:sensitivity}.


%--------------------------------------------------------------------
\subsection{Custom Trade Fixings}% Example 51
\label{example:51}
%--------------------------------------------------------------------

The example in folder {\tt Examples/Example\_51} demonstrates ORE's capability to use custom trade specific fixings. For OIS and Ibor floating legs one can specify historical fixing on a trade level, see \ref{ss:floatingleg_data}. Those trade level fixings will be only use for the specific trade, all other trades will use the global fixings.

%--------------------------------------------------------------------
\subsection{Scripted Trade}% Example 52
\label{example:52}
%--------------------------------------------------------------------

The scripted trade was added to ORE to gain more flexibility in representing exotic products, with hyprid payoffs across
asset classes, path-dependence, multiple kinds of early termination options. The scripted trade module uses Monte Carlo and
Finite Difference pricing approaches, it is an evolving interface to implement parallel processing with GPUs and a central
interface to implement AD methods in ORE. See the separate documentation in folder Docs/ScriptedTrade for an introduction to trade
representation, scripting language, model and pricing engine configuration. 

\medskip
The example in this folder {\tt Examples/Example\_52} is a basic demonstration of ORE's scripted trade functionality.
In this example we provide a self-contained case that can be run as usual calling

\medskip
\centerline{\tt python run.py}

\medskip

This generates an NPV and cash flow report for the following portfolio
\begin{itemize}
\item Trade 1: Vanilla European Equity Option, represented as standard ORE XML with analytical pricing
\item Trade 2: Same Option as above, represented as ``generic'' scripted trade with scripted payoff embedded into the trade XML,
  pricing via Monte Carlo
\item Trade 3: Same Option as above, same representation, pricing via Finite Differences triggered by a {\tt ProductTag} assigned
  to the script and used in {\tt pricingengine.xml} 
\item Trade 4: Same Option as above, the scripted trade now refers to an ``external'' script in {\tt scriptlibrary.xml},
  MC pricing
\item Trade 4b: Same as trade 4, but ``compact'' scripted trade representation (uncomment trade 4b in {\tt portfolio.xml})
\item Trade 5: Barrier Option with single continuously observed Up \& Out barrier, represented as standard ORE XML with
  analytical pricing
\item Trade 6: Same Barrier Option as above, approximated as generic scripted trade with daily barrier observation
\item Trade 6b: Same Barrier Option as above, approximated as ``compact'' scripted trade with daily barrier observation
  (uncomment trade 6b in {\tt portfolio.xml})
\item Trade 7: Same Barrier Option as above, represented as generic scripted trade with continuously observed barrier,
  i.e. adjusting for the probability of knock-out between daily observations
\item Trade 7b: Same Barrier Option as of above, represented as ``compact'' scripted trade
  (uncomment trade 7b in {\tt portfolio.xml})
\item Trade 8: Equity Accumulator, represented as generic scripted trade with external payoff script
\item Trade 8b: Same Equity Accumulator as above, represented as compact scripted trade with external payoff script
  (uncomment trade 8b in {\tt portfolio.xml})
\end{itemize}

Note:
\begin{itemize}
\item In all cases we use the Black-Scholes model to drive the Equity process.
\item The Barrier Option pricing using the scripted trade deviates noticeably from the analytical pricing when we use daily
  observations (trade 6 and 6b), but matches quite closely when we adjust for the probability of knock-out between observation
  dates (trade 7 and 7b)
\item We are not aware of analytical pricing for the Accumulator product in trade 8 to benchmark against; trade 8 is priced with MC,
  FD pricing of the Accumulator is possible as well but requires a separate payoff script, only in the vanilla European option case
  we can utilize the same script for both MC and FD pricing
\end{itemize}

Though this initial Example\_52 shows only single-asset Equity cases, the scripted trade in its current version is
  significantly more versatile, more examples and scripts to follow.

%--------------------------------------------------------------------
\subsection{Curve Building around Central Bank Meeting Dates}% Example 53
\label{example:53}
%--------------------------------------------------------------------

This example demonstrates the build of a GBP OIS curve using MPC Swaps at the short end.

%--------------------------------------------------------------------
\subsection{Scripted Trade Exposure with AMC: Bermudan Swaption and LPI Swap}% Example 54
\label{example:54}
%--------------------------------------------------------------------

This example demonstrates exposure simulation using AMC for selected scripted trade types
\begin{itemize}
\item Bermudan Swaption
\item LPI Swap
\end{itemize}
Both payoffs are defined in {\tt scriptlibrary.xml} which is referenced in {\tt portfolio.xml}. \\

To enable the AMC processing requires the following highlighted settings in {\tt ore.xml}.

\begin{minted}[fontsize=\scriptsize]{xml}
    <Analytic type="simulation">
      <Parameter name="active">Y</Parameter>
      <!-- Set to Y to trigger AMC processing -->
      <Parameter name="amc">Y</Parameter>
      <Parameter name="simulationConfigFile">simulation.xml</Parameter>
      <Parameter name="pricingEnginesFile">pricingengine.xml</Parameter>
      <!-- Specify a separate pricing engine file for AMC engines -->
      <Parameter name="amcPricingEnginesFile">pricingengine\_amc.xml</Parameter>
      <!-- Specify trade types to be covered by the AMC processing -->
      <Parameter name="amcTradeTypes">ScriptedTrade</Parameter>
      <Parameter name="baseCurrency">EUR</Parameter>
      <Parameter name="storeScenarios">N</Parameter>
      <Parameter name="cubeFile">cube.csv.gz</Parameter>
      <Parameter name="aggregationScenarioDataFileName">scenariodata.csv.gz</Parameter>
      <Parameter name="aggregationScenarioDataDump">scenariodata.csv</Parameter>
    </Analytic>
\end{minted}

Note that ORE can handle a mix of trades covered by AMC simulation and covered by ``classic'' simulation.
The respective NPV cubes are combined before generating results such as exposures or XVAs.

%--------------------------------------------------------------------
\subsection{Scripted Trade Exposure with AMC: Target Redemption Forward}% Example 55
\label{example:55}
%--------------------------------------------------------------------

This example demonstrates exposure simulation and XVA for another scripted product, an
FX Target Redemption Forward (TaRF). In contrast to the cases presented above, you won't see
the payoff script library in the Input folder, nor is the script embedded into the trade XML file.
The trade type in this case is {\tt FxTARF} which has its own implementation in OREData/ored/portfolio/tarf.xpp
and a separate trade schema. However, the scipted trade framework is used under the hood, and the payoff
script is embedded into the C++ code in OREData/ored/portfolio/tarf.cpp.

%--------------------------------------------------------------------
\subsection{CVA Sensitivity using AD}% Example 56
\label{example:56}
%--------------------------------------------------------------------

This example demonstrates a prototype CVA sensitivity calculation applying Adjoint Algorithmic Differentiation (AAD)
to a Swap instrument represented as scripted trade. 

%--------------------------------------------------------------------
\subsection{Base Scenario Analytic}% Example 57
\label{example:57}
%--------------------------------------------------------------------

Demonstration of the {\tt Scenario} analytic which has been added to export the simulation market's base scenario
as a file.

\clearpage
%========================================================
\section{Launchers and Visualisation}\label{sec:visualisation}
%========================================================

\subsection{Jupyter}\label{sec:jupyter}

ORE comes with an experimental Jupyter notebook for launching ORE batches and in particular for drilling into NPV cube
data.  The notebook is located in directory {\tt FrontEnd/Python/Visualization/npvcube}. To launch the notebook, change
to this directory and follow instructions in the {\tt Readme.txt}. In a nutshell, type\footnote{With Mac OS X, you may
  need to set the environment variable {\tt LANG} to {\tt en\_US.UTF-8} before running jupyter, as mentioned in the
  installation section \ref{sec:python}.}

\medskip
\centerline{\tt jupyter notebook}
\medskip

to start the ipython console and open a browser window. From the list of files displayed in the browser then click

\medskip
\centerline{\tt ore\_jupyter\_dashboard.ipynb} 
\medskip

to open the ORE notebook. The notebook offers
\begin{itemize}
\item launching an ORE job
\item selecting an NPV cube file and netting sets or trades therein
\item plotting a 3d exposure probability density surface
\item viewing exposure probability density function at a selected future time
\item viewing expected exposure evolution through time  
\end{itemize}

The cube file loaded here by default when processing all cells of the notebook (without changing it or launching a ORE
batch) is taken from {\tt Example\_7} (FX Forwards and FX Options).

%\todo[inline]{Add Jupyter section}

\subsection{Calc}\label{sec:calc}

ORE comes with a simple LibreOffice Calc \cite{LO} sheet as an ORE launcher and basic result viewer. This is
demonstrated on the example in section \ref{sec:example1}. It is currently based on the stable LibreOffice version 5.0.6
and tested on OS X. \\

To launch Calc, open a terminal, change to directory {\tt Examples/Example\_1}, and run

\medskip
{\centerline{\tt ./launchCalc.sh} }
\medskip

%This will show the blank sheet in figure \ref{fig_14}.
%\begin{figure}[h]
%\begin{center}
%\includegraphics[scale=0.4]{demo_calc_1}
%\end{center}
%\caption{Calc sheet after launching.}
%\label{fig_14}
%\end{figure}
The user can choose a configuration (one of the {\tt ore*.xml} files in Example\_1's subfolder Input) by hitting the
'Select' button. Initially Input/ore.xml is pre-selected. The ORE process is then kicked off by hitting 'Run'. Once
completed, standard ORE reports (NPV, Cashflow, XVA) are loaded into several sheets. Moreover, exposure evolutions can
then be viewed by hitting 'View' which shows the result in figure \ref{fig_16}.  \\
\begin{figure}[h]
\begin{center}
\includegraphics[scale=0.4]{demo_calc_2}
\end{center}
\caption{Calc sheet after hitting 'Run'.}
\label{fig_16}
\end{figure}

This demo uses simple Libre Office Basic macros which call Python scripts to execute ORE. The Libre Office Python uno
module (which comes with Libre Office) is used to communicate between Python and Calc to upload results into the sheets.

%\todo[inline]{Remove hard-coded file names from Python scripts}
%\todo[inline]{Calc example on Windows and Linux} 
%\todo[inline]{Harmonise layout with Excel launcher} 

\subsection{Excel}\label{sec:excel}

ORE also comes with a basic Excel sheet to demonstrate launching ORE and presenting results in Excel. This demo is more
self-contained than the Calc demo in the previous section, as it uses VBA only rather than calls to external Python
scripts. The Excel demo is available in Example\_1. Launch {\tt Example\_1.xlsm}. Then modify the paths on the first
sheet, and kick off the ORE process.

%========================================================
\section{Parameterisation}\label{sec:configuration}
%========================================================

A run of ORE is kicked off with a single command line parameter 

\medskip
\centerline{\tt ore[.exe] ore.xml}
\medskip

which points to the 'master input file' referred to  as {\tt ore.xml} subsequently. 
This file is the starting point of the engine's configuration explained in the following sub section.
An overview of all input configuration files respectively all output files is shown in Table \ref{tab_1} respectively Table \ref{tab_2}.
To set up your own ORE configuration, it might be not be necessary to start from scratch, but instead use any of the examples discussed in section \ref{sec:examples} as a boilerplate and just change the folders, see section \ref{sec:master_input}, and the trade data, see section \ref{sec:portfolio_data}, together with the netting definitions, see section \ref{sec:nettingsetinput}.

\subsection{Master Input File: {\tt ore.xml}}\label{sec:master_input}

The master input file contains general setup information (paths to configuration, trade data and market data), as well
as the selection and configuration of analytics. The file has an opening and closing root element {\tt <ORE>}, {\tt
  </ORE>} with three sections
\begin{itemize}
\item Setup
\item Logging
\item Markets
\item Analytics
\end{itemize}
which we will explain in the following.

\subsubsection{Setup}

This subset of data is easiest explained using an example, see listing \ref{lst:ore_setup}.
\begin{listing}[H]
%\hrule\medskip
\begin{minted}[fontsize=\footnotesize]{xml}
<Setup>
  <Parameter name="asofDate">2016-02-05</Parameter>
  <Parameter name="inputPath">Input</Parameter>
  <Parameter name="outputPath">Output</Parameter>
  <Parameter name="logFile">log.txt</Parameter>
  <Parameter name="logMask">255</Parameter>
  <Parameter name="marketDataFile">../../Input/market_20160205.txt</Parameter>
  <Parameter name="fixingDataFile">../../Input/fixings_20160205.txt</Parameter>
  <Parameter name="dividendDataFile">../../Input/dividends_20160205.txt</Parameter> <!-- Optional -->
  <Parameter name="implyTodaysFixings">Y</Parameter>
  <Parameter name="curveConfigFile">../../Input/curveconfig.xml</Parameter>
  <Parameter name="conventionsFile">../../Input/conventions.xml</Parameter>
  <Parameter name="marketConfigFile">../../Input/todaysmarket.xml</Parameter>
  <Parameter name="pricingEnginesFile">../../Input/pricingengine.xml</Parameter>
  <Parameter name="portfolioFile">portfolio.xml</Parameter>
  <Parameter name="calendarAdjustment">../../Input/calendaradjustment.xml</Parameter>
  <Parameter name="currencyConfiguration">../../Input/currencies.xml</Parameter>
  <Parameter name="referenceDataFile">../../Input/referencedata.xml</Parameter>
  <Parameter name="iborFallbackConfig">../../Input/iborFallbackConfig.xml</Parameter>
  <!-- None, Unregister, Defer or Disable -->
  <Parameter name="observationModel">Disable</Parameter>
  <Parameter name="lazyMarketBuilding">false</Parameter>
  <Parameter name="continueOnError">false</Parameter>
  <Parameter name="buildFailedTrades">true</Parameter>
  <Parameter name="nThreads">4</Parameter>
</Setup>
\end{minted}
%\hrule
\caption{ORE setup example}
\label{lst:ore_setup}
\end{listing}

Parameter names are self explanatory: Input and output path are interpreted relative from the directory where the ORE
executable is executed, but can also be specified using absolute paths. All file names are then interpreted relative to the
'inputPath' and 'outputPath', respectively. The files starting with {\tt ../../Input/} then point to files in the global
Example input directory {\tt Example/Input/*}, whereas files such as {\tt portfolio.xml} are local inputs in {\tt 
Example/Example\_\#/Input/}. 

Parameter {\tt logMask} determines the verbosity of log file output. Log messages are 
internally labelled as Alert, Critical, Error, Warning, Notice, Debug, associated with logMask values 1, 2, 4, 8, ..., 64. 
The logMask allows filtering subsets of these categories and controlling the verbosity of log file output\footnote{by bitwise comparison of the external logMask value with each message's log level}. LogMask 255 ensures maximum verbosity. \\

When ORE starts, it will initialise today's market, i.e. load market data, fixings and dividends, and build all term
structures as specified in {\tt todaysmarket.xml}.  Moreover, ORE will load the trades in {\tt portfolio.xml} and link
them with pricing engines as specified in {\tt pricingengine.xml}. When parameter {\tt implyTodaysFixings} is set to Y,
today's fixings would not be loaded but implied, relevant when pricing/bootstrapping off hypothetical market data as
e.g. in scenario analysis and stress testing. The curveConfigFile {\tt curveconfig.xml}, the conventionsFile {\tt
  conventions.xml}, the referenceDataFile {\tt referencedata.xml}, the iborFallbackConfig, the marketDataFile and the
fixingDataFile are explained in the sections below.

\medskip Parameter {\tt calendarAdjustment} includes the {\tt calendarAdjustment.xml} which lists out additional holidays and
business days to be added to specified calendars.

\medskip The optional parameter {\tt currencyConfiguration} points to a configuration file that contains additional currencies
to be added to ORE's setup, see {\tt Examples/Input/currencies.xml} for a full list of ISO currencies and a few unofficial currency
codes that can thus be made available in ORE. Note that the external configuration does not override any currencies that are
hard-coded in the QuantLib/QuantExt libraries, only currencies not present already are added from the external configuration file.

\medskip The last parameter {\tt observationModel} can be used to control ORE performance during simulation. The choices
{\em Disable } and {\em Unregister } yield similarly improved performance relative to choice {\em None}. For users
familiar with the QuantLib design - the parameter controls to which extent {\em QuantLib observer notifications} are
used when market and fixing data is updated and the evaluation date is shifted:
\begin{itemize}
\item The 'Unregister' option limits the amount of notifications by unregistering floating rate coupons from indices;
\item Option 'Defer' disables all notifications during market data and fixing updates with
{\tt ObservableSettings::instance().disableUpdates(true)}
and kicks off updates afterwards when enabled again
\item The 'Disable' option goes one step further and disables all notifications during market data and fixing updates,
  and in particular when the evaluation date is changed along a path, with \\
  {\tt ObservableSettings::instance().disableUpdates(false)} \\
  Updates are not deferred here. Required term structure and instrument recalculations are triggered explicitly.
\end{itemize}
%\todo[inline]{Expand the technical description of observationModel}

\medskip If the parameter {\tt lazyMarketBuilding} is set to true, the build of the curves in the TodaysMarket is
delayed until they are actually requested. This can speed up the processing when some curves configured in TodaysMarket
are not used. If not given, the parameter defaults to {\tt true}.

\medskip If the parameter {\tt continueOnError} is set to true, the application will not exit on an error, but try to
continue the processing. If not given, the parameter defaults to {\tt false}.

\medskip If the parameter {\tt buildFailedTrades} is set to true, the application will build a dummy trade if loading or
building the original trade fails. The dummy trade has trade type ``Failed'', zero notional and NPV.
If not given, the parameter defaults to {\tt false}.

\medskip If the parameter {\tt nThreads} is given, multiple threads will be used for valuation engine runs where
applicable (Sensitivity, Exposure Classic, Exposure AMC). If not given, the parameter defaults to $1$.

\subsubsection{Logging}\label{sec:master_input_logging}

The {\tt Logging} section (see listing \ref{lst:ore_logging}) is used to configure some ORE logging options.

\begin{listing}[H]
%\hrule\medskip
\begin{minted}[fontsize=\footnotesize]{xml}
<Logging>
  <Parameter name="logFile">log.txt</Parameter>
  <Parameter name="logMask">31</Parameter>
  <Parameter name="progressLogFile">my_log_progress_%N.json</Parameter>
  <Parameter name="progressLogRotationSize">102400</Parameter>
  <Parameter name="progressLogToConsole">false</Parameter>
  <Parameter name="structuredLogFile">my_structured_logs_%N.txt</Parameter>
  <Parameter name="structuredLogRotationSize">102400</Parameter>
</Logging>
\end{minted}
%\hrule
\caption{ORE logging}
\label{lst:ore_logging}
\end{listing}

Parameter {\tt logFile} and {\tt logMask} will override the same parameters in the {\tt Setup} section.

Parameters {\tt progressLogFile} and {\tt structuredLogFile} are the filename where progress log messages
and structured log messages are written out to, respectively, which supports Boost string patterns.This defaults to ``log\_progress\_\%N.json'' and ``log\_structured\_\%N.json'', respectively, where {\tt N} will be an integer (beginning at 0) used for log file rotation.

Parameters {\tt progressLogRotationSize} and {\tt structuredLogRotationSize} are the size limit (in bytes)
of each log file before applying log file rotation to the progress log file and structured log message file,
respectively.. For example, $10 * 1024 * 1024 = 10 \text{MiB}$. Defaults to 100 MiB.

If the parameter {\tt progressLogToConsole} is set to true, then progress logs will be written to std::cout.
This can be used simultaneously with {\tt progressLogFile}, i.e.\ progress logs can be written out
to both file and std::cout.

\subsubsection{Markets}\label{sec:master_input_markets}

The {\tt Markets} section (see listing \ref{lst:ore_markets}) is used to choose market configurations for calibrating
the IR, FX and EQ simulation model components, pricing and simulation, respectively. These configurations have to be 
defined
in {\tt todaysmarket.xml} (see section \ref{sec:market}).

\begin{listing}[H]
%\hrule\medskip
\begin{minted}[fontsize=\footnotesize]{xml}
<Markets>
  <Parameter name="lgmcalibration">collateral_inccy</Parameter>
  <Parameter name="fxcalibration">collateral_eur</Parameter>
  <Parameter name="eqcalibration">collateral_inccy</Parameter>
  <Parameter name="pricing">collateral_eur</Parameter>
  <Parameter name="simulation">collateral_eur</Parameter>
</Markets>
\end{minted}
%\hrule
\caption{ORE markets}
\label{lst:ore_markets}
\end{listing}

For example, the calibration of the simulation model's interest rate components requires local OIS discounting whereas
the simulation phase requires cross currency adjusted discount curves to get FX product pricing right. So far, the
market configurations are used only to distinguish discount curve sets, but the market configuration concept in ORE
applies to all term structure types.

\subsubsection{Analytics}\label{sec:analytics}

The {\tt Analytics} section lists all permissible analytics using tags {\tt <Analytic type="..."> ... </Analytic>} where
type can be (so far) in
\begin{itemize}
\item npv
\item cashflow
\item curves
\item simulation
\item xva
\item sensitivity
\item stress
\item parametricVar
\item simm
\end{itemize}

Each {\tt Analytic} section contains a list of key/value pairs to parameterise the analysis of the form {\tt <Parameter
  name="key">value</Parameter>}. Each analysis must have one key {\tt active} set to Y or N to activate/deactivate this
analysis.  The following listing \ref{lst:ore_analytics} shows the parametrisation of the first four basic analytics in
the list above.

\begin{listing}[H]
%\hrule\medskip
\begin{minted}[fontsize=\footnotesize]{xml}
<Analytics>    
  <Analytic type="npv">
    <Parameter name="active">Y</Parameter>
    <Parameter name="baseCurrency">EUR</Parameter>
    <Parameter name="outputFileName">npv.csv</Parameter>
    <Parameter name="additionalResults">Y</Parameter>
  </Analytic>      
  <Analytic type="cashflow">
    <Parameter name="active">Y</Parameter>
    <Parameter name="outputFileName">flows.csv</Parameter>
    <Parameter name="includePastCashflows">N</Parameter>
  </Analytic>      
  <Analytic type="curves">
    <Parameter name="active">Y</Parameter>
    <Parameter name="configuration">default</Parameter>
    <Parameter name="grid">240,1M</Parameter>
    <Parameter name="outputFileName">curves.csv</Parameter>
    <Parameter name="outputTodaysMarketCalibration">N</Parameter>
  </Analytic>
  <Analytic type="...">
    <!-- ... -->
  </Analytic>      
</Analytics>      
\end{minted}
\caption{ORE analytics: npv, cashflow, curves, additional results, todays market calibration}
\label{lst:ore_analytics}
\end{listing}

The cashflow analytic writes a report containing all future cashflows of the portfolio. Table \ref{cashflowreport} shows
a typical output for a vanilla swap.

\begin{table}[hbt]
\scriptsize
\begin{center}
  \begin{tabular}{l|l|l|l|r|l|r|r|l|r|r}
\hline
\#ID & Type & LegNo & PayDate & Amount & Currency & Coupon & Accrual & fixingDate & fixingValue & Notional \\
\hline
\hline
tr123 & Swap & 0 & 13/03/17 & -111273.76 & EUR & -0.00201 & 0.50556 & 08/09/16 & -0.00201 & 100000000.00 \\
tr123 & Swap & 0 & 12/09/17 & -120931.71 & EUR & -0.002379 & 0.50833 & 09/03/17 & -0.002381 & 100000000.00 \\
\ldots
\end{tabular}
\caption{Cashflow Report}
\label{cashflowreport}
\end{center}
\end{table}

The amount column contains the projected amount including embedded caps and floors and convexity (if applicable), the
coupon column displays the corresponding rate estimation. The fixing value on the other hand is the plain fixing
projection as given by the forward value, i.e. without embedded caps and floors or convexity.

Note that the fixing value might deviate from the coupon value even for a vanilla coupon, if the QuantLib library was
compiled {\em without} the flag \verb+QL_USE_INDEXED_COUPON+ (which is the default case). In this case the coupon value
uses a par approximation for the forward rate assuming the index estimation period to be identical to the accrual
period, while the fixing value is the actual forward rate for the index estimation period, i.e. whenever the index estimation
period differs from the accrual period the values will be slightly different.

The Notional column contains the underlying notional used to compute the amount of each coupon. It contains \verb+#NA+
if a payment is not a coupon payment.

The curves analytic exports all yield curves that have been built according to the specification in {\tt
  todaysmarket.xml}. Key {\tt configuration} selects the curve set to be used (see explanation in the previous Markets
section).  Key {\tt grid} defines the time grid on which the yield curves are evaluated, in the example above a grid of
240 monthly time steps from today. The discount factors for all curves with configuration default will be exported on
this monthly grid into the csv file specified by key {\tt outputFileName}. The grid can also be specified explicitly by
a comma separated list of tenor points such as {\tt 1W, 1M, 2M, 3M, \dots}.

The additionalResults analytic writes a report containing any additional results generated for the portfolio. The results are pricing engine specific but Table \ref{additionalreport} shows the output for a vanilla swaption.

\begin{table}[hbt]
\scriptsize
\begin{center}
  \begin{tabular}{l|l|l|l}
\hline
\#TradeId & ResultId & ResultType & ResultValue \\
example\_swaption & annuity & double & 2123720984 \\
example\_swaption & atmForward & double & 0.01664135 \\
example\_swaption & spreadCorrection & double & 0 \\
example\_swaption & stdDev & double & 0.00546015 \\
example\_swaption & strike & double & 0.024 \\
example\_swaption & swapLength & double & 4 \\
example\_swaption & vega & double & 309237709.5 \\
\hline
\hline
\ldots
\end{tabular}
\caption{AdditionalResults Report}
\label{additionalreport}
\end{center}
\end{table}

The todaysMarketCalibration analytic writes a report containing information on the build of the t0 market.

\medskip The purpose of the {\tt simulation} 'analytics' is to run a Monte Carlo simulation which evolves the market as
specified in the simulation config file. The primary result is an NPV cube file, i.e. valuations of all trades in the
portfolio file (see section Setup), for all future points in time on the simulation grid and for all paths. Apart from
the NPV cube, additional scenario data (such as simulated overnight rates etc) are stored in this process which are
needed for subsequent XVA analytics.

\begin{listing}[H]
%\hrule\medskip
\begin{minted}[fontsize=\footnotesize]{xml}
<Analytics>
  <Analytic type="simulation">
    <Parameter name="active">Y</Parameter>
    <Parameter name="simulationConfigFile">simulation.xml</Parameter>
    <Parameter name="pricingEnginesFile">../../Input/pricingengine.xml</Parameter>
    <Parameter name="baseCurrency">EUR</Parameter>
    <Parameter name="storeFlows">Y</Parameter>
    <Parameter name="storeSurvivalProbabilities">Y</Parameter>
    <Parameter name="cubeFile">cube_A.csv.gz</Parameter>
    <Parameter name="nettingSetCubeFile">nettingSetCube_A.csv.gz</Parameter>
    <Parameter name="cptyCubeFile">cptyCube_A.csv.gz</Parameter>
    <Parameter name="aggregationScenarioDataFileName">scenariodata.csv.gz</Parameter>
    <Parameter name="aggregationScenarioDump">scenariodump.csv</Parameter>
  </Analytic>
</Analytics>      
\end{minted}
\caption{ORE analytic: simulation}
\label{lst:ore_simulation}
\end{listing}

The pricing engines file specifies how trades are priced under future scenarios which can differ from pricing as of
today (specified in section Setup).  Key base currency determines into which currency all NPVs will be converted. Key
store scenarios (Y or N) determines whether the market scenarios are written to a file for later reuse. Key
`store flows' (Y or N) controls whether cumulative cash flows between simulation dates are stored in the (hyper-)
cube for post processing in the context of Dynamic Initial Margin and Variation Margin calculations. And finally, the
key `store survival probabilities' (Y or N) controls whether survival probabilities on simulation dates are stored in the
cube for post processing in the context of Dynamic Credit XVA calculation. The additional
scenario data (written to the specified file here) is likewise required in the post processor step. These data comprise
simulated index fixing e.g. for collateral compounding and simulated FX rates for cash collateral conversion into base
currency. The scenario dump file, if specified here, causes ORE to write simulated market data to a human-readable csv
file. Only those currencies or indices are written here that are stated in the AggregationScenarioDataCurrencies and 
AggregationScenarioDataIndices subsections of the simulation files market section, see also section
\ref{sec:sim_market}.
 
\medskip The XVA analytic section offers CVA, DVA, FVA and COLVA calculations which can be selected/deselected here
individually. All XVA calculations depend on a previously generated NPV cube (see above) which is referenced here via
the {\tt cubeFile} parameter. This means one can re-run the XVA analytics without regenerating the cube each time. The
XVA reports depend in particular on the settings in the {\tt csaFile} which determines CSA details such as margining
frequency, collateral thresholds, minimum transfer amounts, margin period of risk. By splitting the processing into
pre-processing (cube generation) and post-processing (aggregation and XVA analysis) it is possible to vary these CSA
details and analyse their impact on XVAs quickly without re-generating the NPV cube. The cube file is usually a
compressed csv file (using gzip compression, with file ending .csv.gz), except when the file extension is set explicitly
to txt or csv in which case an uncompressed version of the file is written to disk.

\begin{listing}[H]
%\hrule\medskip
\begin{minted}[fontsize=\footnotesize]{xml}
<Analytics>
  <Analytic type="xva">
    <Parameter name="active">Y</Parameter>
    <Parameter name="csaFile">netting.xml</Parameter>
    <Parameter name="cubeFile">cube.csv.gz</Parameter>
    <Parameter name="scenarioFile">scenariodata.csv.gz</Parameter>
    <Parameter name="baseCurrency">EUR</Parameter>
    <Parameter name="exposureProfiles">Y</Parameter>
    <Parameter name="exposureProfilesByTrade">Y</Parameter>
    <Parameter name="quantile">0.95</Parameter>
    <Parameter name="calculationType">NoLag</Parameter>      
    <Parameter name="allocationMethod">None</Parameter>    
    <Parameter name="marginalAllocationLimit">1.0</Parameter>
    <Parameter name="exerciseNextBreak">N</Parameter>
    <Parameter name="cva">Y</Parameter>
    <Parameter name="dva">N</Parameter>
    <Parameter name="dvaName">BANK</Parameter>
    <Parameter name="fva">N</Parameter>
    <Parameter name="fvaBorrowingCurve">BANK_EUR_BORROW</Parameter>
    <Parameter name="fvaLendingCurve">BANK_EUR_LEND</Parameter>
    <Parameter name="colva">Y</Parameter>
    <Parameter name="collateralFloor">Y</Parameter>
    <Parameter name="dynamicCredit">N</Parameter>
    <Parameter name="kva">Y</Parameter>
    <Parameter name="kvaCapitalDiscountRate">0.10</Parameter>
    <Parameter name="kvaAlpha">1.4</Parameter>
    <Parameter name="kvaRegAdjustment">12.5</Parameter>
    <Parameter name="kvaCapitalHurdle">0.012</Parameter>
    <Parameter name="kvaOurPdFloor">0.03</Parameter>
    <Parameter name="kvaTheirPdFloor">0.03</Parameter>
    <Parameter name="kvaOurCvaRiskWeight">0.005</Parameter>
    <Parameter name="kvaTheirCvaRiskWeight">0.05</Parameter>
    <Parameter name="dim">Y</Parameter>
    <Parameter name="mva">Y</Parameter>
    <Parameter name="dimQuantile">0.99</Parameter>
    <Parameter name="dimHorizonCalendarDays">14</Parameter>
    <Parameter name="dimRegressionOrder">1</Parameter>
    <Parameter name="dimRegressors">EUR-EURIBOR-3M,USD-LIBOR-3M,USD</Parameter>
    <Parameter name="dimLocalRegressionEvaluations">100</Parameter>
    <Parameter name="dimLocalRegressionBandwidth">0.25</Parameter>
    <Parameter name="dimScaling">1.0</Parameter>
    <Parameter name="dimEvolutionFile">dim_evolution.txt</Parameter>
    <Parameter name="dimRegressionFiles">dim_regression.txt</Parameter>
    <Parameter name="dimOutputNettingSet">CPTY_A</Parameter>      
    <Parameter name="dimOutputGridPoints">0</Parameter>
    <Parameter name="rawCubeOutputFile">rawcube.csv</Parameter>
    <Parameter name="netCubeOutputFile">netcube.csv</Parameter>
    <Parameter name="fullInitialCollateralisation">true</Parameter>
    <Parameter name="flipViewXVA">N</Parameter>
    <Parameter name="flipViewBorrowingCurvePostfix">_BORROW</Parameter>
    <Parameter name="flipViewLendingCurvePostfix">_LEND</Parameter>
    <Parameter name="mporCashFlowMode">NonePay</Parameter>
  </Analytic>
</Analytics>
\end{minted}
\caption{ORE analytic: xva}
\label{lst:ore_xva}
\end{listing}

Parameters:
\begin{itemize}
\item {\tt csaFile:} Netting set definitions file covering CSA details such as margining frequency, thresholds, minimum
transfer amounts, margin period of risk
\item {\tt cubeFile:} NPV cube file previously generated and to be post-processed here
\item {\tt scenarioFile:} Scenario data previously generated and used in the post-processor (simulated index fixings and
FX rates)
\item {\tt baseCurrency:} Expression currency for all NPVs, value adjustments, exposures
\item {\tt exposureProfiles:} Flag to enable/disable exposure output for each netting set
\item {\tt exposureProfilesByTrade:} Flag to enable/disable stand-alone exposure output for each trade
\item {\tt quantile:} Confidence level for Potential Future Exposure (PFE) reporting
\item {\tt calculationType:} Determines the settlement of margin calls. The admissible choices depend on having a close-out grid, see table \ref{tab:calcTypes}; \\
	\begin{itemize}
		\item if there isn't any ``close-out'' grid -see section \ref{sec:simulation}-, the choices are:
		\begin{itemize}
			\item {\em Symmetric} - margin for both counterparties settled after the margin period of risk;
			\item {\em AsymmetricCVA} - margin requested from the counterparty settles with delay,
			margin requested from us settles immediately;
			\item {\em AsymmetricDVA} - vice versa.
		\end{itemize}
		\item If there is a ``close-out'' grid -see section \ref{sec:simulation}-, only choice is:
		\begin{itemize}
			\item {\em NoLag} - used to disable any delayed settlement of the margin. 
		\end{itemize}
	\end{itemize}
	\todo[inline]{Move calculationType into the {\tt csaFile}?}
%
NoLag is the default configuration.
%
\begin{table}[!h]
\centering
\arrayrulecolor{black}
\begin{tabular}{!{\color{black}\vrule}c!{\color{black}\vrule}c!{\color{black}\vrule}l!{\color{black}\vrule}} 
\hline
\multicolumn{1}{!{\color{black}\vrule}l!{\color{black}\vrule}}{Grid Type} & \multicolumn{1}{l!{\color{black}\vrule}}{{\tt calculationType}} & Comment                                                                                                                                                                                           \\ 
\hline
\multirow{4}{*}{without close-out grid}                                    & {\em NoLag}                                                      & Not Supported                                                                                                                                                                                     \\ 
\cline{2-3}
                                                                          & {\em Symmetric}                                                  & Supported\tablefootnote{\label{note1} The calculations are correct only if the simulation grid (see section \ref{sec:simulation}) is equally-spaced with time steps that match the MPoR defined in netting-set definition (see section \ref{sec:CollNettingSet}). See section \ref{sec:mpor} for further explanation.}  \\ 
\cline{2-3}
                                                                          & {\em AsymmetricCVA}                                              & Supported \footref{note1} \\ 
\cline{2-3}
                                                                          & {\em AsymmetricDVA}                                              & Supported \footref{note1}\\ 
\hline
\multirow{4}{*}{with close-out grid}                                       & {\em NoLag}                                                      & Supported\tablefootnote{Close-out lag (see section \ref{sec:simulation}) must be equal to MPoR defined in netting-set definition (see section \ref{sec:CollNettingSet}). Otherwise, an error will be thrown.}                                           \\ 
\cline{2-3}
                                                                          & {\em Symmetric}                                                  & Not Supported                                                                                                                                                                                     \\ 
\cline{2-3}
                                                                          & {\em AsymmetricCVA}                                              & Not Supported                                                                                                                                                                                     \\ 
\cline{2-3}
                                                                          & {\em AsymmetricDVA}                                              & Not Supported                                                                                                                                                                                     \\
\hline
\end{tabular}
\arrayrulecolor{black}
\caption{Overview of admissible calculation types with combination of grid types.} \label{tab:calcTypes}
\end{table}
%
\item {\tt allocationMethod:} XVA allocation method, choices are {\em None, Marginal, RelativeXVA, RelativeFairValueGross, RelativeFairValueNet}
\item {\tt marginalAllocationLimit:} The marginal allocation method a la Pykhtin/Rosen breaks down when the netting set
value vanishes while the exposure does not. This parameter acts as a cutoff for the marginal allocation when the
absolute netting set value falls below this limit and switches to equal distribution of the exposure in this case.
\item {\tt exerciseNextBreak:} Flag to terminate all trades at their next break date before aggregation and the
subsequent analytics
\item {\tt cva, dva, fva, colva, collateralFloor, dim, mva:} Flags to enable/disable these analytics. \todo[inline]{Add
collateral rates floor to the collateral model file (netting.xml)}
\item {\tt dvaName:} Credit name to look up the own default probability curve and recovery rate for DVA calculation
\item {\tt fvaBorrowingCurve:} Identifier of the borrowing yield curve
\item {\tt fvaLendingCurve:} Identifier of the lending yield curve
%\item {\tt collateralSpread:} Deviation between collateral rate and overnight rate, expressed in absolute terms (one
%basis point is 0.0001) assuming the day count convention of the
%collateral rate. 
%basis point is 0.0001) assuming the day count convention of the collateral rate.
\item {\tt dynamicCredit:} Flag to enable using pathwise survival probabilities when calculating CVA, DVA, FVA and MVA increments from exposures. If set to N the survival probabilities are extracted from T0 curves.
\item {\tt kva:} Flag to enable setting the kva ccr parameters.
\item {\tt kvaCapitalDiscountRate, kvaAlpha, kvaRegAdjustment, kvaCapitalHurdle, kvaOurPdFloor, kvaTheirPdFloor kvaOurCvaRiskWeight, kvaTheirCvaRiskWeight:} the kva CCR parameters (see \ref{sec:app_kva} and \ref{sec:app_kva_cva}.
\item {\tt dimQuantile:} Quantile for Dynamic Initial Margin (DIM) calculation
\item {\tt dimHorizonCalendarDays:} Horizon for DIM calculation, 14 calendar days for 2 weeks, etc.
\item {\tt dimRegressionOrder:} Order of the regression polynomial (netting set clean NPV move over the simulation
period versus netting set NPV at period start)
\item {\tt dimRegressors:} Variables used as regressors in a single- or multi-dimensional regression; these variable
  names need to match entries in the {\tt simulation.xml}'s AggregationScenarioDataCurrencies and
  AggregationScenarioDataIndices sections (only these scenario data are passed on to the post processor); if the list is
  empty, the NPV will be used as a single regressor
\item {\tt dimLocalRegressionEvaluations:} Nadaraya-Watson local regression evaluated at the given number of points to
validate polynomial regression. Note that Nadaraya-Watson needs a large number of samples for meaningful
results. Evaluating the local regression at many points (samples) has a significant performance impact, hence the option
here to limit the number of evaluations.
\item {\tt dimLocalRegressionBandwidth:} Nadaraya-Watson local regression bandwidth in standard deviations of the
independent variable (NPV)
\item {\tt dimScaling:} Scaling factor applied to all DIM values used, e.g. to reconcile simulated DIM with actual IM at
$t_0$
\item {\tt dimEvolutionFile:} Output file name to store the evolution of zero order DIM and average of nth order DIM
through time
\item {\tt dimRegressionFiles:} Output file name(s) for a DIM regression snapshot, comma separated list
\item {\tt dimOutputNettingSet:} Netting set for the DIM regression snapshot
\item {\tt dimOutputGridPoints:} Grid point(s) (in time) for the DIM regression snapshot, comma separated list
\item {\tt rawCubeOutputFile:} File name for the trade NPV cube in human readable csv file format (per trade, date,
sample), leave empty to skip generation of this file.
\item {\tt netCubeOutputFile:} File name for the aggregated NPV cube in human readable csv file format (per netting set,
date, sample) {\em after} taking collateral into account. Leave empty to skip generation of this file.
\item {\tt fullInitialCollateralisation:} If set to {\tt true}, then for every netting set, the collateral balance at $t=0$ will be set to the NPV of the setting set. The resulting effect is that EPE, ENE and PFE are all zero at $t=0$. If set to {\tt false} (default value), then the collateral balance at $t=0$ will be set to zero.
\item {\tt flipViewXVA:} If set to {\tt Y}, the perspective in XVA calculations is switched to the cpty view, the npvs and the netting sets being reverted during calculation. In order to get the lending/borrowing curve, the calculation assumes these curves being set up with the cptyname + the postfix given in the next two settings.
\item {\tt flipViewBorrowingCurvePostfix:} postfix for the borrowing curve, the calculation assumes this is curves being set up with cptyname + postfix given.
\item {\tt flipViewLendingCurvePostfix:} postfix for the lending curve, the calculation assumes this is curve being set up with cptyname + postfix given.
\item {\tt mporCashFlowMode:} Assumption about payment of cashflows within mpor period. One of NonePay, BothPay, WePay,
  TheyPay, Unspecified. Defaults to Unspecified, in this case PP will assume NonePay if mpor sticky date is used,
  otherwise to BothPay.
\end{itemize}

The two cube file outputs {\tt rawCubeOutputFile} and {\tt netCubeOutputFile} are provided for interactive analysis and visualisation purposes, see section
\ref{sec:visualisation}.

\medskip The {\tt sensitivity} and {\tt stress} 'analytics' provide computation of bump and revalue (zero rate
resp. optionlet) sensitivities and NPV changes under user defined stress scenarios. Listing \ref{lst:ore_sensitivity}
shows a typical configuration for sensitivity calculation.

\begin{listing}[H]
%\hrule\medskip
\begin{minted}[fontsize=\footnotesize]{xml}
<Analytics>
 <Analytic type="sensitivity">
   <Parameter name="active">Y</Parameter>
   <Parameter name="marketConfigFile">simulation.xml</Parameter>
   <Parameter name="sensitivityConfigFile">sensitivity.xml</Parameter>
   <Parameter name="pricingEnginesFile">../../Input/pricingengine.xml</Parameter>
   <Parameter name="scenarioOutputFile">scenario.csv</Parameter>
   <Parameter name="sensitivityOutputFile">sensitivity.csv</Parameter>
   <Parameter name="crossGammaOutputFile">crossgamma.csv</Parameter>
   <Parameter name="outputSensitivityThreshold">0.000001</Parameter>
   <Parameter name="recalibrateModels">Y</Parameter>
   <!-- Additional parametrisation for par sensitivity analysis -->
   <Parameter name="parSensitivity">Y</Parameter>
   <Parameter name="parSensitivityOutputFile">parsensitivity.csv</Parameter>
   <Parameter name="outputJacobi">Y</Parameter>
   <Parameter name="jacobiOutputFile">jacobi.csv</Parameter>
   <Parameter name="jacobiInverseOutputFile">jacobi_inverse.csv</Parameter>
 </Analytic>
</Analytics>
\end{minted}
\caption{ORE analytic: sensitivity}
\label{lst:ore_sensitivity}
\end{listing}
%   <Parameter name="parRateSensitivityOutputFile">parsensi.csv</Parameter>

The parameters have the following interpretation:

\begin{itemize}
\item {\tt marketConfigFile:} Configuration file defining the simulation market under which sensitivities are computed,
  see \ref{sec:simulation}. Only a subset of the specification is needed (the one given under {\tt Market}, see
  \ref{sec:sim_market} for a detailed description).
\item {\tt sensitivityConfigFile:} Configuration file  for the sensitivity calculation, see section \ref{sec:sensitivity}.
\item {\tt pricingEnginesFile:} Configuration file for the pricing engines to be used for sensitivity calculation.
\item {\tt scenarioOutputFile:} File containing the results of the sensitivity calculation in terms of the base scenario
  NPV, the scenario NPV and their difference.
\item {\tt sensitivityOutputFile:} File containing the results of the sensitivity calculation in terms of the base scenario
  NPV, the shift size together with the risk-factor and the resulting first and (pure) second order finite differences.
  Also included is a second set of shift sizes together with the risk-factor with a (mixed) second order finite difference associated to a cross gamma calculation
%\item {\tt parRateSensitivityOutputFile:} File containing par sensitivities (only available in ORE+)
\item {\tt outputSensitivityThreshold:} Only finite differences with absolute value greater than this number are written
  to the output files.
\item {\tt recalibrateModels:} If set to Y, then recalibrate pricing models after each shift of relevant term structures; otherwise do not recalibrate
\item {\tt parSensitivity}: If set to Y, par sensitivity analysis is performed following the "raw" sensitivity analysis; note that in this case the 
{\tt sensitivityConfigFile} needs to contain {\tt ParConversion} sections, see {\tt Example\_40}   
\item {\tt parSensitivityOutputFile}: Output file name for the par sensitivity report
\item {\tt outputJacobi}: If set to Y, then the relevant Jacobi and inverse Jacobi matrix is written to a file, see below
\item {\tt jacobiOutputFile}: Output file name for the Jacobi matrx
\item {\tt jacobiInverseOutputFile}: Output file name for the inverse Jacobi matrix
\end{itemize}


The zero to par sensitivity conversion analytics configuration is similar to the one of the sensitivity calculation. Listing \ref{lst:ore_zerotoparconversion}
shows an example.

\begin{listing}[H]
%\hrule\medskip
\begin{minted}[fontsize=\footnotesize]{xml}
<Analytics>
 <Analytic type="zeroToParSensiConversion">
      <Parameter name="active">Y</Parameter>
      <Parameter name="marketConfigFile">simulation.xml</Parameter>
      <Parameter name="sensitivityConfigFile">sensitivity.xml</Parameter>
      <Parameter name="pricingEnginesFile">../../Input/pricingengine.xml</Parameter>
      <Parameter name="sensitivityInputFile">sensitivity.csv</Parameter>
      <Parameter name="outputThreshold">0.000001</Parameter>
      <Parameter name="outputFile">parconversion_sensitivity.csv</Parameter>
      <Parameter name="outputJacobi">Y</Parameter>
      <Parameter name="jacobiOutputFile">jacobi.csv</Parameter>
      <Parameter name="jacobiInverseOutputFile">jacobi_inverse.csv</Parameter>
    </Analytic>
</Analytics>
\end{minted}
\caption{ORE analytic: Zero to Par Sensitivity Conversion}
\label{lst:ore_zerotoparconversion}
\end{listing}

The parameters have the same interpretation as for the sensitivity analytic. There is one new parameter *sensitivityInputFile* which points to a csv file with the raw (zero)sensitivites. Those raw sensitivites will be converted into par sensitivities, using the same methodology described in \ref{app:par_sensi} and the configuration is described in \ref{sec:sensitivity}.

The raw sensitivites csv input file *sensitivityInputFile* needs to have at least six columns, the column names can be user configured in the master input file. Here is a description of each of the columns:

\begin{enumerate}
\item idColumn : Column with a unique identifier for the trade / nettingset / portfolio.
\item riskFactorColumn: Column with the identifier of the zero/raw sensitiviy. The risk factor name needs to follow the ORE naming convention, e.g. DiscountCurve/EUR/5/1Y (the 6th bucket in EUR discount curve as specified in the sensitivity.xml)\
\item deltaColumn: The raw sensitivity of the trade/nettingset / portfolio with respect to the risk factor
\item currencyColumn: The currency in which the raw sensitivity is expressed, need to be the same as the BaseCurrency in the simulation settings.
\item shiftSizeColumn: The shift size applied to compute the raw sensitivity, need to be consistent to the sensitivity configuration.
\item baseNpvColumn: The base npv of the trade / nettingset / portfolio in currency.
\end{enumerate}

Here is an example for an input file:

\begin{table}[hbt]
\scriptsize
\begin{center}
\begin{tabular}{lllrlrr}
\hline
{} & \#TradeId &                Factor\_1 &  ShiftSize\_1 & Currency &  Base NPV &  Delta \\
\hline
0 &     Swap &  DiscountCurve/EUR/3/6M &       0.0001 &      EUR &   1335.27 &   5.05 \\
1 &     Swap &  DiscountCurve/EUR/4/9M &       0.0001 &      EUR &   1335.27 &   0.35 \\
2 &     Swap &  DiscountCurve/EUR/5/1Y &       0.0001 &      EUR &   1335.27 &  -5.41 \\
3 &     Swap &  DiscountCurve/EUR/6/2Y &       0.0001 &      EUR &   1335.27 &  -0.22 \\
4 &     Swap &  DiscountCurve/EUR/7/3Y &       0.0001 &      EUR &   1335.27 &  -0.32 \\
\hline
\end{tabular}
\end{center}
\end{table}

The stress analytics configuration is similar to the one of the sensitivity calculation. Listing \ref{lst:ore_stress}
shows an example.

\begin{listing}[H]
%\hrule\medskip
\begin{minted}[fontsize=\footnotesize]{xml}
<Analytics>
 <Analytic type="stress">
   <Parameter name="active">Y</Parameter>
   <Parameter name="marketConfigFile">simulation.xml</Parameter>
   <Parameter name="stressConfigFile">stresstest.xml</Parameter>
   <Parameter name="pricingEnginesFile">../../Input/pricingengine.xml</Parameter>
   <Parameter name="scenarioOutputFile">stresstest.csv</Parameter>
   <Parameter name="outputThreshold">0.000001</Parameter>
 </Analytic>
</Analytics>
\end{minted}
\caption{ORE analytic: stress}
\label{lst:ore_stress}
\end{listing}

The parameters have the same interpretation as for the sensitivity analytic. The configuration file for the stress
scenarios is described in more detail in section \ref{sec:stress}.

\medskip The {\tt VaR} 'analytics' provide computation of Value-at-Risk measures based on the sensitivity (delta, gamma, cross gamma) data above. Listing \ref{lst:ore_var} shows a configuration example.

\begin{listing}[H]
%\hrule\medskip
\begin{minted}[fontsize=\footnotesize]{xml}
<Analytics>
    <Analytic type="parametricVar"> 
      <Parameter name="active">Y</Parameter> 
      <Parameter name="portfolioFilter">PF1|PF2</Parameter>
      <Parameter name="sensitivityInputFile">
         ../Output/sensitivity.csv,../Output/crossgamma.csv
      </Parameter> 
      <Parameter name="covarianceInputFile">covariance.csv</Parameter> 
      <Parameter name="salvageCovarianceMatrix">N</Parameter>
      <Parameter name="quantiles">0.01,0.05,0.95,0.99</Parameter> 
      <Parameter name="breakdown">Y</Parameter> 
      <!-- Delta, DeltaGammaNormal, Cornish-Fisher, Saddlepoint, MonteCarlo --> 
      <Parameter name="method">DeltaGammaNormal</Parameter> 
      <Parameter name="mcSamples">100000</Parameter> 
      <Parameter name="mcSeed">42</Parameter> 
      <Parameter name="outputFile">var.csv</Parameter> 
    </Analytic> 
</Analytics>
\end{minted}
\caption{ORE analytic: VaR}
\label{lst:ore_var}
\end{listing}

The parameters have the following interpretation:

\begin{itemize}
\item {\t portfolioFilter:} Regular expression used to filter the portfolio for which VaR is computed; if the filter is not provided, then the full portfolio is processed
\item {\tt sensitivityInputFile:} Reference to the sensitivity (deltas, vegas, gammas) and cross gamma input as generated by ORE in a comma separated list
\item {\tt covarianceFile:} Reference to the covariances input data; these are currently not calculated in ORE and need to be provided externally, in a blank/tab/comma separated file with three columns (factor1, factor2, covariance), where factor1 and factor2 follow the naming convention used in ORE's sensitivity and cross gamma output files. Covariances need to be consistent with the sensitivity data provided. For example, if sensitivity to factor1 is computed by absolute shifts and expressed in basis points, then the covariances with factor1 need to be based on absolute basis point shifts of factor1; if sensitivity is due to a relative factor1 shift of 1\%, then covariances with factor1 need to be based on relative shifts expressed in percentages to, etc. Also note that covariances are expected to include the desired holding period, i.e. no scaling with square root of time etc is performed in ORE; 
\item {\tt salvageCovarianceMatrix:} If set to Y, turn the input covariance matrix into a valid (positive definite) matrix applying a Salvaging algorithm; if set to N, throw an exception if the matrix is not positive definite
\item {\tt quantiles:} Several desired quantiles can be specified here in a comma separated list; these lead to several columns of results in the output file, see below. Note that e.g. the 1\% quantile corresponds to the lower tail of the P\&L distribution (VaR), 99\% to the upper tail.
\item {\tt breakdown:} If yes, VaR is computed by portfolio, risk class (All, Interest Rate, FX, Inflation, Equity, Credit) and risk type (All, Delta \& Gamma, Vega)
\item {\tt method:} Choices are {\em Delta, DeltaGammaNormal, Cornish-Fisher, Saddlepoint, MonteCarlo}, see appendix \ref{sec:app_var}
\item {\tt mcSamples:} Number of Monte Carlo samples used when the {\em MonteCarlo} method is chosen 
\item {\tt mcSeed:} Random number generator seed when the {\em MonteCarlo} method is chosen
\item {\tt outputFile:} Output file name
\end{itemize}

\medskip The {\tt simm} 'analytic' provides computation of initial margin using ISDA's Standard Initial Margin Model (SIMM) based on sensitivities in the Common Risk Interchange Format (CRIF) defined by ISDA. Listing \ref{lst:ore_simm} shows a configuration example.

\begin{listing}[H]
\begin{minted}[fontsize=\footnotesize]{xml}
<Analytics>
   <Analytic type="simm">
      <Parameter name="active">Y</Parameter>
      <Parameter name="version">2.1</Parameter>
      <Parameter name="crif">crif.csv</Parameter>
      <Parameter name="calculationCurrency">USD</Parameter>
      <Parameter name="resultCurrency">USD</Parameter>
      <Parameter name="enforceIMRegulations">true</Parameter>
      <Parameter name="mporDays">1</Parameter>
      <Parameter name="simmCalibration">simmcalibration.xml</Parameter>
    </Analytic>
<Analytics>
\end{minted}
\caption{ORE analytic: SIMM}
\label{lst:ore_simm}
\end{listing}

The parameters have the following interpretation:

\begin{itemize}
\item {\tt version:} SIMM model version string \\
Allowable values: 1.0, 1.1, 1.2, 1.3, 1.3.38, 2.0, 2.1, 2.2, 2.3, 2.4 (or 2.3.8), 2.5, 2.5A, 2.6 (or 2.5.6) \\
Note that any new SIMM model versions are integrated into ORE with each release, tested against the official ISDA SIMM unit tests.
\item {\tt crif:} Name of the CRIF file to be loaded
\item {\tt calculationCurrency:} Determines the {\tt Risk\_FX} CRIF entry that is ignored in ISDA SIMM calculation \\
Allowable values: See Table \ref{tab:currency} \lstinline!Currency!.
\item {\tt resultCurrency} (optional): Currency for expressing the amounts in the resulting SIMM report, by default set to the calculationCurrency. \\
Allowable values: See Table \ref{tab:currency} \lstinline!Currency!.
\item {\tt enforceIMRegulations} (optional): Whether to take collect/post regulations into account. \\
Allowable values: Allowable boolean values are given in Table \ref{tab:boolean_allowable}. Defaults to \emph{False} if omitted.
\item {\tt mporDays} (optional): Currency for expressing the amounts in the resulting SIMM report, by default set to the calculationCurrency. \\
Allowable values: See Table \ref{tab:currency} \lstinline!Currency!.
\item {\tt simmCalibration} (optional): Name of the SIMM calibration configuration file. See Section \ref{sec:simmcalibration} \lstinline!SIMM Calibration!.
\end{itemize}

The SIMM analytic requires minimal market data input and today's market configuration - FX rates for conversions calculation currency, USD and result currency.

%--------------------------------------------------------
\subsection{Market: {\tt todaysmarket.xml}}\label{sec:market}
%--------------------------------------------------------

This configuration file determines the subset of the 'market' universe which is going to be built by ORE. It is the
user's responsibility to make sure that this subset is sufficient to cover the portfolio to be analysed. If it is not,
the application will complain at run time and exit.

\medskip We assume that the market configuration is provided in file {\tt todaysmarket.xml}, however, the file name can
be chosen by the user. The file name needs to be entered into the master configuration file {\tt ore.xml}, see section
\ref{sec:master_input}.

\medskip 
The file starts and ends with the opening and closing tags {\tt <TodaysMarket>} 
and {\tt </TodaysMarket>}. The file then contains configuration blocks for
\begin{itemize}
\item Discounting curves
\item Index curves (to project index fixings)
\item Yield curves (for other purposes, e.g. as benchmark curve for bond pricing)
\item Swap index curves (to project Swap rates)
\item FX spot rates
\item Inflation index curves (to project zero or yoy inflation fixings)
\item Equity curves (to project forward prices)
\item Default curves
\item Swaption volatility structures
\item Cap/Floor volatility structures
\item FX volatility structures
\item Inflation Cap/Floor volatility surfaces
\item Equity volatility structures
\item CDS volatility structures
\item Base correlation structures
\item Correlation structures
\item Securities
\end{itemize}

There can be alternative versions of each block each labeled with a unique identifier (e.g. Discount curve block with ID
'default', discount curve block with ID 'ois', another one with ID 'xois', etc). The purpose of these IDs will be
explained at the end of this section. We now discuss each block's layout.

\subsubsection{Discounting Curves} 

We pick one discounting curve block as an example here (see {\tt Examples/Input/todaysmarket.xml}), the one with ID 'ois' 

\begin{listing}[H]
%\hrule\medskip
\begin{minted}[fontsize=\footnotesize]{xml}
  <DiscountingCurves id="ois">
    <DiscountingCurve currency="EUR">Yield/EUR/EUR1D</DiscountingCurve>
    <DiscountingCurve currency="USD">Yield/USD/USD1D</DiscountingCurve>
    <DiscountingCurve currency="GBP">Yield/GBP/GBP1D</DiscountingCurve>
    <DiscountingCurve currency="CHF">Yield/CHF/CHF6M</DiscountingCurve>
    <DiscountingCurve currency="JPY">Yield/JPY/JPY6M</DiscountingCurve>
    <!-- ... -->
  </DiscountingCurves>
\end{minted}
\caption{Discount curve block with ID 'ois'}
\label{lst:discountcurve_spec}
\end{listing}

This block instructs ORE to build five discount curves for the indicated currencies. The string within the tags,
e.g. Yield/EUR/EUR1D, uniquely identifies the curve to be built.  Curve Yield/EUR/EUR1D is defined in the curve
configuration file explained in section \ref{sec:curveconfig} below. In this case ORE is instructed to build an Eonia
Swap curve made of Overnight Deposit and Eonia Swap quotes. The right most token of the string Yield/EUR/EUR1D (EUR1D)
is user defined, the first two tokens Yield/EUR have to be used to point to a yield curve in currency EUR.
 
\subsubsection{Index Curves} 

See an excerpt of the index curve block with ID 'default' from the same example file:

\begin{listing}[H]
%\hrule\medskip
\begin{minted}[fontsize=\footnotesize]{xml}
<IndexForwardingCurves id="default">
  <Index name="EUR-EURIBOR-3M">Yield/EUR/EUR3M</Index>
  <Index name="EUR-EURIBOR-6M">Yield/EUR/EUR6M</Index>
  <Index name="EUR-EURIBOR-12M">Yield/EUR/EUR12M</Index>
  <Index name="EUR-EONIA">Yield/EUR/EUR1D</Index>
  <Index name="USD-LIBOR-3M">Yield/USD/USD3M</Index>
  <!-- ... -->
</IndexForwardingCurves>
\end{minted}
\caption{Index curve block with ID 'default'}
\label{lst:indexcurve_spec}
\end{listing}

This block of curve specifications instructs ORE to build another set of yield curves, unique strings
(e.g. Yield/EUR/EUR6M etc.) point to the {\tt curveconfig.xml} file where these curves are defined. Each curve is then
associated with an index name (of format Ccy-IndexName-Tenor, e.g. EUR-EURIBOR-6M) so that ORE will project the
respective index using the selected curve (e.g. Yield/EUR/EUR6M).

\subsubsection{Yield Curves}

See an excerpt of the yield curve block with ID 'default' from the same example file:

\begin{listing}[H]
%\hrule\medskip
\begin{minted}[fontsize=\footnotesize]{xml}
<YieldCurves id="default">
  <YieldCurve name="BANK_EUR_LEND">Yield/EUR/BANK_EUR_LEND</YieldCurve>
  <YieldCurve name="BANK_EUR_BORROW">Yield/EUR/BANK_EUR_BORROW</YieldCurve>
  <!-- ... -->
</YieldCurves>
\end{minted}
\caption{Yield curve block with ID 'default'}
\label{lst:yieldcurve_spec}
\end{listing}

This block of curve specifications instructs ORE to build another set of yield curves, unique strings
(e.g. Yield/EUR/EUR6M etc.) point to the {\tt curveconfig.xml} file where these curves are defined. Other than
discounting and index curves the yield curves in this block are not tied to a particular purpose. The curves defined in
this block typically include

\begin{itemize}
\item additional curves needed in the XVA post processor, e.g. for the FVA calculation
\item benchmark curves used for bond pricing
\end{itemize}

\subsubsection{Swap Index Curves}

The following is an excerpt of the swap index curve block with ID 'default' from the same example file:

\begin{listing}[H]
%\hrule\medskip
\begin{minted}[fontsize=\footnotesize]{xml}
<SwapIndexCurves id="default">
  <SwapIndex name="EUR-CMS-1Y">
    <Index>EUR-EURIBOR-6M</Index>
    <Discounting>EUR-EONIA</Discounting>
  </SwapIndex>
  <SwapIndex name="EUR-CMS-30Y">
    <Index>EUR-EURIBOR-6M</Index>
    <Discounting>EUR-EONIA</Discounting>
  </SwapIndex>
  <!-- ... -->
</SwapIndexCurves>
\end{minted}
\caption{Swap index curve block with ID 'default'}
\label{lst:swapindexcurve_spec}
\end{listing}

These instructions do not build any additional curves. They only build the respective swap index objects and associate
them with the required index forwarding and discounting curves already built above. This enables a swap index to project
the fair rate of forward starting Swaps. Swap indices are also containers for conventions. Swaption volatility surfaces
require two swap indices each available in the market object, a long term and a short term swap index. The curve
configuration file below will show that in particular the required short term index has term 1Y, and the required long
term index has 30Y term. This is why we build these two indices at this point.

\subsubsection{FX Spot}

The following is an excerpt of the FX spot block with ID 'default' from the same example file:

\begin{listing}[H]
%\hrule\medskip
\begin{minted}[fontsize=\footnotesize]{xml}
<FxSpots id="default">
  <FxSpot pair="EURUSD">FX/EUR/USD</FxSpot>
  <FxSpot pair="EURGBP">FX/EUR/GBP</FxSpot>
  <FxSpot pair="EURCHF">FX/EUR/CHF</FxSpot>
  <FxSpot pair="EURJPY">FX/EUR/JPY</FxSpot>
  <!-- ... -->
</FxSpots>
\end{minted}
\caption{FX spot block with ID 'default'}
\label{lst:fxspot_spec}
\end{listing}

This block instructs ORE to provide four FX quotes, all quoted with target currency EUR so
that foreign currency amounts can be converted into EUR via multiplication with that rate.
 
\subsubsection{FX Volatilities}

The following is an excerpt of the FX Volatilities block with ID 'default' from the same example file:

\begin{listing}[H]
%\hrule\medskip
\begin{minted}[fontsize=\footnotesize]{xml}
<FxVolatilities id="default">
  <FxVolatility pair="EURUSD">FXVolatility/EUR/USD/EURUSD</FxVolatility>
  <FxVolatility pair="EURGBP">FXVolatility/EUR/GBP/EURGBP</FxVolatility>
  <FxVolatility pair="EURCHF">FXVolatility/EUR/CHF/EURCHF</FxVolatility>
  <FxVolatility pair="EURJPY">FXVolatility/EUR/JPY/EURJPY</FxVolatility>
  <!-- ... -->
</FxVolatilities>
\end{minted}
\caption{FX volatility block with ID 'default'}
\label{lst:fxvol_spec}
\end{listing}

This instructs ORE to build four FX volatility structures for all FX pairs with target currency EUR, see curve
configuration file for the definition of the volatility structure.

\subsubsection{Swaption Volatilities}

The following is an excerpt of the Swaption Volatilities block with ID 'default' from the same example file:

\begin{listing}[H]
%\hrule\medskip
\begin{minted}[fontsize=\footnotesize]{xml}
<SwaptionVolatilities id="default">
  <SwaptionVolatility currency="EUR">SwaptionVolatility/EUR/EUR_SW_N</SwaptionVolatility>
  <SwaptionVolatility currency="USD">SwaptionVolatility/USD/USD_SW_N</SwaptionVolatility>
  <SwaptionVolatility currency="GBP">SwaptionVolatility/GBP/GBP_SW_N</SwaptionVolatility>
  <SwaptionVolatility currency="CHF">SwaptionVolatility/CHF/CHF_SW_N</SwaptionVolatility>
  <SwaptionVolatility currency="JPY">SwaptionVolatility/CHF/JPY_SW_N</SwaptionVolatility>
</SwaptionVolatilities>
\end{minted}
\caption{Swaption volatility block with ID 'default'}
\label{lst:swaptionvol_spec}
\end{listing}

This instructs ORE to build five Swaption volatility structures, see the curve configuration file for the definition of
the volatility structure. The latter token (e.g. EUR\_SW\_N) is user defined and will be found in the curve
configuration's CurveId tag.

\subsubsection{Cap/Floor Volatilities}

The following is an excerpt of the Cap/Floor Volatilities block with ID 'default' from the same example file:

\begin{listing}[H]
%\hrule\medskip
\begin{minted}[fontsize=\footnotesize]{xml}
<CapFloorVolatilities id="default">
  <CapFloorVolatility currency="EUR">CapFloorVolatility/EUR/EUR_CF_N</CapFloorVolatility>
  <CapFloorVolatility currency="USD">CapFloorVolatility/USD/USD_CF_N</CapFloorVolatility>
  <CapFloorVolatility currency="GBP">CapFloorVolatility/GBP/GBP_CF_N</CapFloorVolatility>
</CapFloorVolatilities>
\end{minted}
\caption{Cap/Floor volatility block with ID 'default'}
\label{lst:capfloorvol_spec}
\end{listing}

This instructs ORE to build three Cap/Floor volatility structures, see the curve configuration file for the definition
of the volatility structure. The latter token (e.g. EUR\_CF\_N) is user defined and will be found in the curve
configuration's CurveId tag.

\subsubsection{Default Curves}

The following is an excerpt of the Default Curves block with ID 'default' from the same example file:

\begin{listing}[H]
%\hrule\medskip
\begin{minted}[fontsize=\footnotesize]{xml}
<DefaultCurves id="default">
  <DefaultCurve name="BANK">Default/USD/BANK_SR_USD</DefaultCurve>
  <DefaultCurve name="CPTY_A">Default/USD/CUST_A_SR_USD</DefaultCurve>
  <DefaultCurve name="CPTY_B">Default/USD/CUST_A_SR_USD</DefaultCurve>
  <!-- ... -->
</DefaultCurves>
\end{minted}
\caption{Default curves block with ID 'default'}
\label{lst:defaultcurve_spec}
\end{listing}

This instructs ORE to build a set of default probability curves, again defined in the curve configuration file. Each
curve is then associated with a name (BANK, CUST\_A) for subsequent lookup.  As before, the last token
(e.g. BANK\_SR\_USD) is user defined and will be found in the curve configuration's CurveId tag.

\subsubsection{Securities}\label{sssec:securities}

The following is an excerpt of the Security block with ID 'default' from the same example file:

\begin{listing}[H]
	%\hrule\medskip
	\begin{minted}[fontsize=\footnotesize]{xml}
<Securities id="default">
  <Security name="SECURITY_1">Security/SECURITY_1</Security>
</Securities>
	\end{minted}
	\caption{Securities block with ID 'default'}
	\label{lst:secspread_spec}
\end{listing}

The pricing of bonds includes (among other components) a security specific spread and rate. 
This block links a security name to a spread and rate pair defined in the curve configuration file. This name may then be referenced 
as the security id in the bond trade definition.

\subsubsection{Equity Curves}
The following is an excerpt of the Equity curves block with ID 'default' from the same example file:

\begin{listing}[H]
%\hrule\medskip
\begin{minted}[fontsize=\footnotesize]{xml}
<EquityCurves id="default">
  <EquityCurve name="SP5">Equity/USD/SP5</EquityCurve>
  <EquityCurve name="Lufthansa">Equity/EUR/Lufthansa</EquityCurve>
</EquityCurves>
\end{minted}
\caption{Equity curves block with ID 'default'}
\label{lst:eqcurve_spec}
\end{listing}

This instructs ORE to build a set of equity curves, again defined in the curve configuration file. Each equity curve 
after construction will consist of a spot equity price, as well as a term structure of dividend yields, which can be 
used to determine forward prices. This object is then associated with a name (e.g. SP5) for subsequent lookup. 

\subsubsection{Equity Volatilities}

The following is an excerpt of the equity volatilities block with ID 'default' from the same example file:

\begin{listing}[H]
%\hrule\medskip
\begin{minted}[fontsize=\footnotesize]{xml}
<EquityVolatilities id="default">
  <EquityVolatility name="SP5">EquityVolatility/USD/SP5</EquityVolatility>
  <EquityVolatility name="Lufthansa">EquityVolatility/EUR/Lufthansa</EquityVolatility>
</EquityVolatilities>
\end{minted}
\caption{EQ volatility block with ID 'default'}
\label{lst:eqvol_spec}
\end{listing}

This instructs ORE to build two equity volatility structures for SP5 and Lufthansa, respectively. See the curve
configuration file for the definition of the equity volatility structure.


\subsubsection{Inflation Index Curves}

The following is an excerpt of the Zero Inflation Index Curves block with ID 'default' from the sample example file:

\begin{listing}[H]
%\hrule\medskip
\begin{minted}[fontsize=\footnotesize]{xml}
<ZeroInflationIndexCurves id="default">
    <ZeroInflationIndexCurve name="EUHICPXT">
        Inflation/EUHICPXT/EUHICPXT_ZC_Swaps
    </ZeroInflationIndexCurve>
    <ZeroInflationIndexCurve name="FRHICP">
        Inflation/FRHICP/FRHICP_ZC_Swaps
    </ZeroInflationIndexCurve>
    <ZeroInflationIndexCurve name="UKRPI">
        Inflation/UKRPI/UKRPI_ZC_Swaps
    </ZeroInflationIndexCurve>
    <ZeroInflationIndexCurve name="USCPI">
        Inflation/USCPI/USCPI_ZC_Swaps
    </ZeroInflationIndexCurve>
    ...
</ZeroInflationIndexCurves>
\end{minted}
\caption{Zero Inflation Index Curves block with ID 'default'}
\label{lst:zeroinflationindexcurve_spec}
\end{listing}

This instructs ORE to build a set of zero inflation index curves, which are defined in the curve configuration
file. Each curve is then associated with an index name (like e.g. EUHICPXT or UKRPI). The last token
(e.g. EUHICPXT\_ZC\_Swap) is user defined and will be found in the curve configuration's CurveId tag.

In a similar way, Year on Year index curves are specified:

\begin{listing}[H]
%\hrule\medskip
\begin{minted}[fontsize=\footnotesize]{xml}
  <YYInflationIndexCurves id="default">
      <YYInflationIndexCurve name="EUHICPXT">
          Inflation/EUHICPXT/EUHICPXT_YY_Swaps
      </YYInflationIndexCurve>
      ...
  </YYInflationIndexCurves>
\end{minted}
\caption{YoY Inflation Index Curves block with ID 'default'}
\label{lst:yoyinflationindexcurve_spec}
\end{listing}

Note that the index name is the same as in the corresponding zero index curve definition, but the token corresponding to
the CurveId tag is different. This is because the actual underlying index (and in particular its fixings) are shared
between the two index types, while different projection curves are used to forecast future index realisations.

\subsubsection{Inflation Cap/Floor Volatility Surfaces}

The following is an excerpt of the Inflation Cap/Floor Volatility Surfaces blocks with ID 'default' from the sample example
file:

{
\begin{listing}[H]
%\hrule\medskip
\begin{minted}[fontsize=\footnotesize]{xml}
  <YYInflationCapFloorVolatilities id="default">
    <YYInflationCapFloorVolatility name="EUHICPXT">
        InflationCapFloorVolatility/EUHICPXT/EUHICPXT_YY_CF
    </InflationCapFloorVolatility>
  </YYInflationCapFloorVolatilities>

  <ZeroInflationCapFloorVolatilities id="default">
    <ZeroInflationCapFloorVolatility name="UKRPI">
        InflationCapFloorVolatility/UKRPI/UKRPI_ZC_CF
    </ZeroInflationCapFloorVolatility>
    <ZeroInflationCapFloorVolatility name="EUHICPXT">
        InflationCapFloorVolatility/EUHICPXT/EUHICPXT_ZC_CF
    </ZeroInflationCapFloorVolatility>
    <ZeroInflationCapFloorVolatility name="USCPI">
        InflationCapFloorVolatility/USCPI/USCPI_ZC_CF
    </ZeroInflationCapFloorVolatility>
  </ZeroInflationCapFloorVolatilities>
\end{minted}
\caption{Inflation Cap/Floor Volatility Surfaces block with ID 'default'}
\label{lst:inflation_cap_floor_surface_spec}
\end{listing}

This instructs ORE to build a set of year-on-year and zero inflation cap floor volatility surfaces, which are defined in the curve
configuration file. Each surface is associated with an index name. The last token (e.g. EUHICPXT\_ZC\_CF) is user defined
and will be found in the curve configuration's CurveId tag.

\subsubsection{CDS Volatility Structures}

CDS volatility structures are configured as follows
\begin{listing}[H]
%\hrule\medskip
\begin{minted}[fontsize=\footnotesize]{xml}
  <CDSVolatilities id="default">
   <CDSVolatility name="CDSVOL_A">CDSVolatility/CDXIG</CDSVolatility>
   <CDSVolatility name="CDSVOL_B">CDSVolatility/CDXHY</CDSVolatility>
  </CDSVolatilities>
\end{minted}
\caption{CDS volatility structure block with ID 'default'}
\label{lst:cdsvol_spec}
\end{listing}

The composition of the CDS volatility structures is defined in the curve configuration.

\subsubsection{Base Correlation Structures}

Base correlation structures are configured as follows
\begin{listing}[H]
%\hrule\medskip
\begin{minted}[fontsize=\footnotesize]{xml}
  <BaseCorrelations id="default">
   <BaseCorrelation name="CDXIG">BaseCorrelation/CDXIG</BaseCorrelation>
  </BaseCorrelations>
\end{minted}
\caption{Base Correlations block with ID 'default'}
\label{lst:basecorr_spec}
\end{listing}

The composition of the base correlation structure is defined in the curve configuration.

\subsubsection{Correlation Structures}

Correlation structures are configured as follows
\begin{listing}[H]
%\hrule\medskip
\begin{minted}[fontsize=\footnotesize]{xml}
 <Correlations id="default">
      <Correlation name="EUR-CMS-10Y:EUR-CMS-1Y">Correlation/EUR-CORR</Correlation>  
      <Correlation name="USD-CMS-10Y:USD-CMS-1Y">Correlation/USD-CORR</Correlation>
 </Correlations>
\end{minted}
\caption{Correlations block with ID 'default'}
\label{lst:corr_spec}
\end{listing}

The composition of the correlation structure is defined in the curve configuration.
\subsubsection{Market Configurations}

Finally, representatives of each type of block (Discount Curves, Index Curves, Volatility structures, etc, up to
Inflation Cap/Floor Price Surfaces) can be bundled into a market configuration. This is done by adding the following to
the {\tt todaysmarket.xml} file:

\begin{listing}[H]
%\hrule\medskip
\begin{minted}[fontsize=\footnotesize]{xml}
<Configuration id="default">
  <DiscountingCurvesId>xois_eur</DiscountingCurvesId>
</Configuration>
<Configuration id="collateral_inccy">
  <DiscountingCurvesId>ois</DiscountingCurvesId>
</Configuration>
<Configuration id="collateral_eur">
  <DiscountingCurvesId>xois_eur</DiscountingCurvesId>
</Configuration>
<Configuration id="libor">
  <DiscountingCurvesId>inccy_swap</DiscountingCurvesId>
</Configuration>
\end{minted}
\caption{Market configurations}
\label{lst:config_spec}
\end{listing}

When ORE constructs the market object, all market configurations will be build and labelled using the 'Configuration
Id'.  This allows configuring a market setup for different alternative purposes side by side in the same {\tt
  todaysmarket.xml} file. Typical use cases are
\begin{itemize}
\item different discount curves needed for model calibration and risk factor evolution, respectively
\item different discount curves needed for collateralised and uncollateralised derivatives pricing.
\end{itemize}
The former is actually used throughout the {\tt Examples} section. Each master input file {\tt ore.xml} has a Markets
section (see \ref{sec:master_input}) where four market configuration IDs have to be provided - the ones used for
'lgmcalibration', 'fxcalibration', 'pricing' and 'simulation' (i.e. risk factor evolution).

\medskip The configuration ID concept extends across all curve and volatility objects though currently used only to
distinguish discounting.

%--------------------------------------------------------
\subsection{Pricing Engines: {\tt pricingengine.xml}}
%--------------------------------------------------------
\label{sec:configuration_pricingengines}

The pricing engine configuration file is provided to select pricing models and pricing engines by product type.

%--------------------------------------------------------
\subsubsection{Product Type: Ascot}
%--------------------------------------------------------

Used by trade type: Ascot

Available Model/Engine pairs:

\begin{itemize}
\item BlackScholes/Intrinsic
\end{itemize}

Engine description:

BlackScholes/Intrinsic builds a IntrinsicAscotEngine. A sample configuration is shown
in listing \ref{lst:peconfig_Ascot_BlackScholes_Intrinsic}.

The parameters have the following meaning:

\begin{itemize}
\item SensitivityTemplate [optional]: the sensitivity template to use 
\end{itemize}

\begin{longlisting}
\begin{minted}[fontsize=\footnotesize]{xml}
 <Product type="Ascot">
    <Model>BlackScholes</Model>
    <ModelParameters/>
    <Engine>Intrinsic</Engine>
    <EngineParameters>
        <Parameter name="SensitivityTemplate">EQ_FD</Parameter>
     </EngineParameters>
</Product>
\end{minted}
\caption{Configuration for Product Ascot, Model: BlackScholes, Engine: Intrinsic}
\label{lst:peconfig_Ascot_BlackScholes_Intrinsic}
\end{longlisting}

%--------------------------------------------------------
\subsubsection{Product Type: Bond}
%--------------------------------------------------------

Used by trade type: Bond

Available Model/Engine pairs:

\begin{itemize}
\item DiscountedCashflows/DiscountingRiskyBondEngine
\item DiscountedCashflows/DiscountingRiskyBondEngineMultiState
\end{itemize}

Engine description:

DiscountedCashflows/DiscountingRiskyBondEngine builds a DiscountingRiskyBondEngine. A sample configuration is shown
in listing \ref{lst:peconfig_Bond_DiscountedCashflows_DiscountingRiskyBondEngine}.

The parameters have the following meaning:

\begin{itemize}
\item TimestepPeriod: discretization interval for zero bond pricing
\item SensitivityTemplate [optional]: the sensitivity template to use 
\end{itemize}

\begin{longlisting}
\begin{minted}[fontsize=\footnotesize]{xml}
<Product type="Bond">
    <Model>DiscountedCashflows</Model>
    <ModelParameters/>
    <Engine>DiscountingRiskyBondEngine</Engine>
    <EngineParameters>
        <Parameter name="TimestepPeriod">3M</Parameter>
        <Parameter name="SensitivityTemplate">IR_Analytical</Parameter>
    </EngineParameters>
</Product>
\end{minted}
\caption{Configuration for Product Bond, Model DiscountedCashflows, Engine DiscountingRiskyBondEngine}
\label{lst:peconfig_Bond_DiscountedCashflows_DiscountingRiskyBondEngine}
\end{longlisting}

DiscountedCashflows/DiscountingRiskyBondEngineMultiState builds a DiscountingRiskyBondEngineMultiState for use in the
Credit Model. We refer to the credit model documentation for further details.

%--------------------------------------------------------
\subsubsection{Product Type: BondOption}
%--------------------------------------------------------

Used by trade type: BondOption

Available Model/Engine pairs:

\begin{itemize}
\item Black/BlackBondOptionEngine
\end{itemize}

Engine description:

Black/BlackBondOptionEngine builds a BlackBondOptionEngine. A sample configuration is shown in listing
\ref{lst:peconfig_BondOption_Black_BlackBondOptionEngine}.

The parameters have the following meaning:

\begin{itemize}
\item TimestepPeriod: discretization interval for zero bond pricing
\item SensitivityTemplate [optional]: the sensitivity template to use 
\end{itemize}

\begin{longlisting}
\begin{minted}[fontsize=\footnotesize]{xml}
<Product type="BondOption">
    <Model>Black</Model>
    <ModelParameters/>
    <Engine>BlackBondOptionEngine</Engine>
    <EngineParameters>
        <Parameter name="TimestepPeriod">3M</Parameter>
        <Parameter name="SensitivityTemplate">IR_Analytical</Parameter>
    </EngineParameters>
</Product>
\end{minted}
\caption{Configuration for Product BondOption, Model Black, Engine BlackBondOptionEngine}
\label{lst:peconfig_BondOption_Black_BlackBondOptionEngine}
\end{longlisting}

%--------------------------------------------------------
\subsubsection{Product Type: ConvertibleBond}
%--------------------------------------------------------

Used by trade type: ConvertibleBond

Available Model/Engine pairs: DefaultableEquityJumpDiffusion/FD

\begin{itemize}
\item DefaultableEquityJumpDiffusion/FD
\end{itemize}

Engine description:

DefaultableEquityJumpDiffusion/FD builds a FdDefaultableEquityJumpDiffusionConvertibleBondEngine following Andersen, L.,
and Buffum, D.: Calibration and Implementation of Convertible Bond Models (2002):

The model dynamics for the stock price $S(t)$ is given by

\begin{equation}
  dS / S(t^-) = (r(t) - q(t) + \eta h(t, S(t^-))) dt + \sigma(t) dW(t) - \eta dN(t)
\end{equation}

with a risk free rate $r(t)$, a continuous dividend yield $q(t)$, a default intensity $h(t,S)$, a volatility
$\sigma(t)$, a default loss fraction for the equity $\eta \in [0,1]$ and a Cox process $N(t)$ with

\begin{equation}
  E_t(dN(t)) = h(t,S(t^-)) dt
\end{equation}

The notation $S(t^-)$ is shorthand for $\lim_{\epsilon\downarrow 0} S(t-\epsilon)$. The first jump of $N(t)$ represents
the default of the equity. See equation (1) in Andersen, Buffum. We support a local default intensity of the form

\begin{equation}
h(t,S(t)) = h_0(t) \left( \frac{S(0)}{S(t)} \right)^p
\end{equation}

with a deterministic function $h_0(t)$ that is independent from $S(t)$ and a parameter $p \geq 0$. The parameters $p$
and $\eta$ can be set in the pricing engine configuration. More details on the pricing model is available in a separate
model documentation. A sample configuration is shown in listing
\ref{lst:peconfig_ConvertibleBond_DefaultableEquityJumpDiffusion_FD}.

The parameters have the following meaning:

\begin{itemize}
\item p: the model parameter p
\item eta: the model parameter eta
\item AdjustEquityForward: If false, the term $\eta h(t,e^z)$ in the coefficient of $v_z$ in the convertible bond
  pricing pde (see separate model documentation) is set to zero, i.e. the hazard rate $h$ is still used in the
  discounting term, but the equity drift is not corrected upwards accordingly. The default value is true.
\item AdjustEquityVolatility: If false, the market equity volatility input is not adjusted, but directly used in the
  pricing model. This setting is only possible if $p=0$. It will then set the weighting with the market survival
  probability $S(0,t_i)$ in the context of formula for the equity volatility match (see separate model documentation) to
  zero, i.e. $V$ is taken as the market implied volatility without adjustment. The default value is true.
\item AdjustDiscounting: If false, the adjustment of the discounting rate $r$ to the benchmark curve $b$ is suppressed,
  i.e. the change of the bond pricing PDE to the bond pricing pde with benchmark curve (see separate model
  documentation) is {\em not} made (see also section ``Curves used in practice'' in separate model documentation). The
  default value is true.
\item AdjustCreditSpreadToRR: If true, the credit curve $h(\cdot)$ is adjusted by a factor $\frac{1-R}{1-\rho}$ where
  $R$ is the recovery rate associated to the market default curve and $\rho$ is the recovery rate of the bond. Usually,
  $R=\rho$, i.e. the bond recovery rate is the same as the recovery rate of the associated credit curve. In this case,
  the flag has no effect, since the multiplier is $1$. However, when $\rho$ is overwritten with zero due to the flag
  ZeroRecoveryOverwrite set to true, the flag AdjustCreditSpreadToRR should also be set to true. The default value is
  false.
\item ZeroRecoveryOverwrite: If true, the recovery rate $\rho$ of the convertible bond is overwritten with zero. This
  option is usually used in conjunction with AdjustCreditSpreadToRR set to true (see below). The default value is false.
\item TreatSecuritySpreadAsCreditSpread: If true, the security spread is not incoroporated into the benchmark curve $b$
  as described above, but rather added as a spread on top of the credit curve, i.e. it is added to $h(t,e^z)$. For
  exchangeables, the security spread is added to {\em both} $h^B$ and $h^S$, i.e. it simultaneously increases the credit
  pread of both the equity and the bond component. Since the security spread is understood as an effective discounting
  spread, it is scaled by $s \rightarrow s / (1-\rho)$ before it is added to $h$, where $\rho$ is the recovery rate of
  the bond.
\item MesherIsStatic: whether to use the same finite-difference mesher under scenario / sensi calculations
\item Bootstrp.CalibrationGrid: The model is calibrated on a configurable set of times . All tenors from the specified
  grid before the maturity date of the convertible bond are kept and the maturity date itself is added to the resulting
  grid to avoid calibration for times beyond the bond maturity and at the same time ensuring that we do not need to
  extrapolate model functions beyond the last calibration point in the pricing.
\item Bootstrap.StateGridPoints: The number of state grid points of the Fokker-Planck PDE in the calibration phase
\item Bootstrap.MesherEpsilon: The mesher epsilon of the Fokker-Planck PDE in the calibration phase
\item Bootstrap.MesherScaling: The mesher scaling multiplier of the Fokker-Planck PDE in the calibration phase
\item Bootstrap.Mode: bootstrap strategy: ``Simultaneously'' or  ``Alternating'', see separate model docs for further details 
\item Pricing.TimeStepsPerYear: The number of time steps per year to be used for the pricing PDE
\item Pricing.StateGridPoints: The number of state grid points to be used for the pricing PDE
\item Pricing.MesherEpsilon: The mesher epsilon for the pricing PDE
\item Pricing.MesherScaling: The mesher scaling multiplier for the pricing PDE
\item ConversionRatioDiscretizationGrid: Multipliers to be used for conversion ratio discretization in the presence of
  conversion resets / adjustments. See separate model documentation for more details.
\item SensitivityTemplate [optional]: the sensitivity template to use
\end{itemize}

\begin{longlisting}
\begin{minted}[fontsize=\footnotesize]{xml}
 <Product type="ConvertibleBond">
    <Model>DefaultableEquityJumpDiffusion</Model>
    <ModelParameters>
        <Parameter name="p">0.0</Parameter>
        <Parameter name="eta">1.0</Parameter>
        <Parameter name="AdjustEquityForward">true</Parameter>
        <Parameter name="AdjustEquityVolatility">false</Parameter>
        <Parameter name="AdjustDiscounting">false</Parameter>
        <Parameter name="AdjustCreditSpreadToRR">true</Parameter>
        <Parameter name="ZeroRecoveryOverwrite">true</Parameter>
        <Parameter name="TreatSecuritySpreadAsCreditSpread">true</Parameter>
    </ModelParameters>
    <Engine>FD</Engine>
    <EngineParameters>
        <Parameter name="MesherIsStatic">true</Parameter>
        <Parameter name="Bootstrap.CalibrationGrid">
          6M,1Y,2Y,3Y,4Y,5Y,7Y,10Y,15Y,20Y,25Y,30Y,40Y,50Y
        </Parameter>
        <Parameter name="Bootstrap.TimeStepsPerYear">24</Parameter>
        <Parameter name="Bootstrap.StateGridPoints">400</Parameter>
        <Parameter name="Bootstrap.MesherEpsilon">1E-5</Parameter>
        <Parameter name="Bootstrap.MesherScaling">1.5</Parameter>
        <Parameter name="Bootstrap.Mode">Alternating</Parameter>
        <Parameter name="Pricing.TimeStepsPerYear">24</Parameter>
        <Parameter name="Pricing.StateGridPoints">100</Parameter>
        <Parameter name="Pricing.MesherEpsilon">1E-4</Parameter>
        <Parameter name="Pricing.MesherScaling">1.5</Parameter>
        <Parameter name="Pricing.ConversionRatioDiscretisationGrid">
          0.5,0.55,0.6,0.65,0.7,0.75,
          0.8,0.85,0.9,0.95,1.0,1.05,
          1.1,1.15,1.2,1.25,1.5,1.75,2.0
        </Parameter>
        <Parameter name="SensitivityTemplate">EQ_FD</Parameter>
    </EngineParameters>
</Product>
\end{minted}
\caption{Configuration for Product ConvertibleBond, Model DefaultableEquityJumpDiffusion, Engine FD}
\label{lst:peconfig_ConvertibleBond_DefaultableEquityJumpDiffusion_FD}
\end{longlisting}

%--------------------------------------------------------
\subsubsection{Product Type: CreditLinkedSwap}
%--------------------------------------------------------

Used by trade type: CreditLinkedSwap

Available Model/Engine pairs:

\begin{itemize}
\item DiscountedCashflows/DiscountingCreditLinkedSwapEngine
\end{itemize}

Engine description:

DiscountedCashflows/DiscountingCreditLinkedSwapEngine builds a DiscountingCreditLinkedSwapEngine. A sample configuration is shown
in listing \ref{lst:peconfig_CreditLinkedSwap_DiscountedCashflows_DiscountingCreditLinkedSwapEngine}.

The parameters have the following meaning:

\begin{itemize}
\item SensitivityTemplate [optional]: the sensitivity template to use 
\end{itemize}

\begin{longlisting}
\begin{minted}[fontsize=\footnotesize]{xml}
<Product type="CreditLinkedSwap">
    <Model>DiscountedCashflows</Model>
    <ModelParameters/>
    <Engine>DiscountingCreditLinkedSwapEngine</Engine>
    <EngineParameters>
        <Parameter name="SensitivityTemplate">IR_Analytical</Parameter>
    </EngineParameters>
</Product>
\end{minted}
\caption{Configuration for Product CreditLinkedSwap, Model DiscountedCashflows, Engine DiscountingCreditLinkedSwapEngine}
\label{lst:peconfig_CreditLinkedSwap_DiscountedCashflows_DiscountingCreditLinkedSwapEngine}
\end{longlisting}

%--------------------------------------------------------
\subsubsection{Product Type: EuropeanSwaption}
%--------------------------------------------------------

Used by trade type: Swaption, for European exercise on vanilla underlying coupon types

Available Model/Engine pairs:

\begin{itemize}
\item BlackBachelier/BlackBachelierSwaptionEngine
\item LGM/Grid
\item LGM/FD
\item LGM/MC
\item LGM/AMC
\end{itemize}

Engine description:

BlackBachelier/BlackBachelierSwaptionEngine builds a BlackMultiLegOptionEngine . A sample configuration is shown in
listing \ref{lst:peconfig_EuropeanSwaption_BlackBachelier_BlackBachelierSwaptionEngine}

The parameters have the following meaning:

\begin{itemize}
\item SensitivityTemplate [optional]: the sensitivity template to use 
\end{itemize}

\begin{longlisting}
\begin{minted}[fontsize=\footnotesize]{xml}
<Product type="EuropeanSwaption">
    <Model>BlackBachelier</Model>
    <ModelParameters/>
    <Engine>BlackBachelierSwaptionEngine</Engine>
    <EngineParameters>
        <Parameter name="SensitivityTemplate">IR_Analytical</Parameter>
    </EngineParameters>
</Product>
\end{minted}
\caption{Configuration for Product EuropeanSwaption, Model BlackBachelier, Engine BlackBachelierSwaptionEngine}
\label{lst:peconfig_EuropeanSwaption_BlackBachelier_BlackBachelierSwaptionEngine}
\end{longlisting}

LGM/Grid builds a NumericLgmMultiLegOptionEngine using LgmConvoluationSolver as a solver. The rollback follows the paper
``Hagan, P: Methodology for callable swaps and Bermudan exercise into swaptions''. A sample configuration is shown in
listing \ref{lst:peconfig_EuropeanSwaption_LGM_Grid}

The parameters have the following meaning:

\begin{itemize}
\item Calibration: Bootstrap, BestFit, None
\item CalibrationStrategy: CoterminalDealStrike, CoterminalATM
\item ReferenceCalibrationGrid: An optional grid, only one calibration instrument per interval is kept
\item Reversion: The mean reversion
\item ReversionType: Hagan, HullWhite
\item Volatility: The volatility (start value for calibration if calibrated)
\item VolatilityType: Hagan, HullWhite
\item ShiftHorizon: Shift horizon for LGM model as fraction of deal maturity
\item Tolerance: Error tolerance for calibration
\item sy, sx: Number of covered standard deviations (notation as in Hagan's paper)
\item ny, nx: Number of grid points for numerical integration (notation as in Hagan's paper)
\item SensitivityTemplate [optional]: the sensitivity template to use 
\end{itemize}

\begin{longlisting}
\begin{minted}[fontsize=\footnotesize]{xml}
<Product type="EuropeanSwaption">
    <Model>LGM</Model>
    <ModelParameters>
        <Parameter name="Calibration">Bootstrap</Parameter>
        <Parameter name="CalibrationStrategy">CoterminalDealStrike</Parameter>
        <Parameter name="ReferenceCalibrationGrid">400,3M</Parameter>
        <Parameter name="Reversion">0.0</Parameter>
        <Parameter name="ReversionType">HullWhite</Parameter>
        <Parameter name="Volatility">0.01</Parameter>
        <Parameter name="VolatilityType">Hagan</Parameter>
        <Parameter name="ShiftHorizon">0.5</Parameter>
        <Parameter name="Tolerance">0.20</Parameter>
    </ModelParameters>
    <Engine>Grid</Engine>
    <EngineParameters>
        <Parameter name="sy">5.0</Parameter>
        <Parameter name="ny">30</Parameter>
        <Parameter name="sx">5.0</Parameter>
        <Parameter name="nx">30</Parameter>
        <Parameter name="SensitivityTemplate">IR_FD</Parameter>
    </EngineParameters>
</Product>
\end{minted}
\caption{Configuration for Product EuropeanSwaption, Model LGM, Engine Grid}
\label{lst:peconfig_EuropeanSwaption_LGM_Grid}
\end{longlisting}

LGM/FD builds a NumericLgmMultiLegOptionEngine using LgmFdSolver as a solver using finite difference. A sample
configuration is shown in listing \ref{lst:peconfig_EuropeanSwaption_LGM_FD}

The parameters have the following meaning:

\begin{itemize}
\item Calibration: Bootstrap, BestFit, None
\item CalibrationStrategy: CoterminalDealStrike, CoterminalATM
\item ReferenceCalibrationGrid: An optional grid, only one calibration instrument per interval is kept
\item Reversion: The mean reversion
\item ReversionType: Hagan, HullWhite
\item Volatility: The volatility (start value for calibration if calibrated)
\item VolatilityType: Hagan, HullWhite
\item ShiftHorizon: Shift horizon for LGM model as fraction of deal maturity
\item Tolerance: Error tolerance for calibration
\item Scheme: The finite difference scheme to use
\item StateGridPoints: The number of grid points in state direction
\item TimeStepsPerYear: The number of time steps per year to use
\item MesherEpsilon: determines the covered probability mass, mass outside state grid is $\Phi^{-1}(1-2\epsilon)$
\item SensitivityTemplate [optional]: the sensitivity template to use 
\end{itemize}

\begin{longlisting}
\begin{minted}[fontsize=\footnotesize]{xml}
<Product type="EuropeanSwaption">
    <Model>LGM</Model>
    <ModelParameters>
        <Parameter name="Calibration">Bootstrap</Parameter>
        <Parameter name="CalibrationStrategy">CoterminalDealStrike</Parameter>
        <Parameter name="ReferenceCalibrationGrid">400,3M</Parameter>
        <Parameter name="Reversion">0.0</Parameter>
        <Parameter name="ReversionType">HullWhite</Parameter>
        <Parameter name="Volatility">0.01</Parameter>
        <Parameter name="VolatilityType">Hagan</Parameter>
        <Parameter name="ShiftHorizon">0.5</Parameter>
        <Parameter name="Tolerance">0.20</Parameter>
    </ModelParameters>
    <Engine>FD</Engine>
        <EngineParameters>
            <Parameter name="Scheme">Douglas</Parameter>
            <Parameter name="StateGridPoints">64</Parameter>
            <Parameter name="TimeStepsPerYear">24</Parameter>
            <Parameter name="MesherEpsilon">1E-4</Parameter>
        </EngineParameters>
    </EngineParameters>
</Product>
\end{minted}
\caption{Configuration for Product EuropeanSwaption, Model LGM, Engine FD}
\label{lst:peconfig_EuropeanSwaption_LGM_FD}
\end{longlisting}

LGM/MC builds a McMultiLegOptionEngine. A sample configuration is shown in
listing \ref{lst:peconfig_EuropeanSwaption_LGM_MC}

The parameters have the following meaning:

\begin{itemize}
\item Calibration: Bootstrap, BestFit, None
\item CalibrationStrategy: CoterminalDealStrike, CoterminalATM
\item ReferenceCalibrationGrid: An optional grid, only one calibration instrument per interval is kept
\item Reversion: The mean reversion
\item ReversionType: Hagan, HullWhite
\item Volatility: The volatility (start value for calibration if calibrated)
\item VolatilityType: Hagan, HullWhite
\item ShiftHorizon: Shift horizon for LGM model as fraction of deal maturity
\item Tolerance: Error tolerance for calibration
\item Training.Sequence: The sequence type for the traning phase, can be MersenneTwister+, MersenneTwisterAntithetc+,
  Sobol+, Burley2020Sobol+, SobolBrownianBridge+, Burley2020SobolBrownianBridge+
\item Training.Seed: The seed for the random number generation in the training phase
\item Training.Samples: The number of samples to be used for the training phase
\item Pricing.Sequence: The sequence type for the pricing phase, same values allowed as for training
\item Training.BasisFunction: The type of basis function system to be used for the regression analysis, can be
  Monomial+, Laguerre+, Hermite+, Hyperbolic+, Legendre+, Chbyshev+, Chebyshev2nd+
\item BasisFunctionOrder: The order of the basis function system to be used
\item Pricing.Seed: The seed for the random number generation in the pricing
\item Pricing.Samples: The number of samples to be used for the pricing phase. If this number is zero, no pricing run is
  performed, instead the (T0) NPV is estimated from the training phase (this result is used to fill the T0 slice of the
  NPV cube)
\item BrownianBridgeOrdering: variate ordering for Brownian bridges, can be Steps+, Factors+, Diagonal+
\item SobolDirectionIntegers: direction integers for Sobol generator, can be Unit+, Jaeckel+, SobolLevitan+,
  SobolLevitanLemieux+, JoeKuoD5+, JoeKuoD6+, JoeKuoD7+, Kuo+, Kuo2+, Kuo3+
\item MinObsDate: if true the conditional expectation of each cashflow is taken from the minimum possible observation
  date (i.e. the latest exercise or simulation date before the cashflow's event date); recommended setting is true+
\item RegressorModel: Simple, LaggedFX. If not given, it defaults to Simple. Depending on the choice the regressor is
  built as follows:
  \begin{itemize}
    \item Simple: For an observation date the full model state observed on this date is included in the regressor. No
      past states are included though.
    \item LaggedFX: For an observation date the full model state observed on this date is included in the regressor. In
      addition, past FX states that are relevant for future cashflows are included. For example, for a FX resettable
      cashflow the FX state observed on the FX reset date is included.
  \end{itemize}
\item SensitivityTemplate [optional]: the sensitivity template to use
\end{itemize}

\begin{longlisting}
\begin{minted}[fontsize=\footnotesize]{xml}
<Product type="EuropeanSwaption">
    <Model>LGM</Model>
    <ModelParameters>
        <Parameter name="Calibration">Bootstrap</Parameter>
        <Parameter name="CalibrationStrategy">CoterminalDealStrike</Parameter>
        <Parameter name="ReferenceCalibrationGrid">400,3M</Parameter>
        <Parameter name="Reversion">0.0</Parameter>
        <Parameter name="ReversionType">HullWhite</Parameter>
        <Parameter name="Volatility">0.01</Parameter>
        <Parameter name="VolatilityType">Hagan</Parameter>
        <Parameter name="ShiftHorizon">0.5</Parameter>
        <Parameter name="Tolerance">0.20</Parameter>
    </ModelParameters>
    <Engine>MC</Engine>
    <EngineParameters>
        <Parameter name="Training.Sequence">MersenneTwisterAntithetic</Parameter>
        <Parameter name="Training.Seed">42</Parameter>
        <Parameter name="Training.Samples">10000</Parameter>
        <Parameter name="Training.BasisFunction">Monomial</Parameter>
        <Parameter name="Training.BasisFunctionOrder">6</Parameter>
        <Parameter name="Pricing.Sequence">SobolBrownianBridge</Parameter>
        <Parameter name="Pricing.Seed">17</Parameter>
        <Parameter name="Pricing.Samples">0</Parameter>
        <Parameter name="BrownianBridgeOrdering">Steps</Parameter>
        <Parameter name="SobolDirectionIntegers">JoeKuoD7</Parameter>
        <Parameter name="MinObsDate">true</Parameter>
        <Parameter name="RegressorModel">Simple</Parameter>
        <Parameter name="SensitivityTemplate">IR_MC</Parameter>
    </EngineParameters>
</Product>
\end{minted}
\caption{Configuration for Product EuropeanSwaption, Model BlackBachelier, Engine BlackBachelierSwaptionEngine}
\label{lst:peconfig_EuropeanSwaption_LGM_MC}
\end{longlisting}

LGM/AMC builds a McMultiLegOptionEngine for use in AMC simulations. We refer to the AMC module documentation for further
details.

%--------------------------------------------------------
\subsubsection{Product Type: BermudanSwaption}
%--------------------------------------------------------

Used by trade type: Swaption, for Bermudan exercise or European exercise on non-vanilla underlying coupon types

Available Model/Engine pairs:

\begin{itemize}
\item LGM/Grid
\item LGM/FD
\item LGM/MC
\item LGM/AMC
\end{itemize}

Engine description:

LGM/Grid builds a NumericLgmMultiLegOptionEngine using LgmConvoluationSolver as a solver. The rollback follows the paper
``Hagan, P: Methodology for callable swaps and Bermudan exercise into swaptions''. A sample configuration is shown in
listing \ref{lst:peconfig_BermudanSwaption_LGM_Grid}

The parameters have the following meaning:

\begin{itemize}
\item Calibration: Bootstrap, BestFit, None
\item CalibrationStrategy: CoterminalDealStrike, CoterminalATM
\item ReferenceCalibrationGrid: An optional grid, only one calibration instrument per interval is kept
\item Reversion: The mean reversion
\item ReversionType: Hagan, HullWhite
\item Volatility: The volatility (start value for calibration if calibrated)
\item VolatilityType: Hagan, HullWhite
\item ShiftHorizon: Shift horizon for LGM model as fraction of deal maturity
\item Tolerance: Error tolerance for calibration
\item sy, sx: Number of covered standard deviations (notation as in Hagan's paper)
\item ny, nx: Number of grid points for numerical integration (notation as in Hagan's paper)
\item SensitivityTemplate [optional]: the sensitivity template to use 
\end{itemize}

\begin{longlisting}
\begin{minted}[fontsize=\footnotesize]{xml}
<Product type="BermudanSwaption">
    <Model>LGM</Model>
    <ModelParameters>
        <Parameter name="Calibration">Bootstrap</Parameter>
        <Parameter name="CalibrationStrategy">CoterminalDealStrike</Parameter>
        <Parameter name="ReferenceCalibrationGrid">400,3M</Parameter>
        <Parameter name="Reversion">0.0</Parameter>
        <Parameter name="ReversionType">HullWhite</Parameter>
        <Parameter name="Volatility">0.01</Parameter>
        <Parameter name="VolatilityType">Hagan</Parameter>
        <Parameter name="ShiftHorizon">0.5</Parameter>
        <Parameter name="Tolerance">0.20</Parameter>
    </ModelParameters>
    <Engine>Grid</Engine>
    <EngineParameters>
        <Parameter name="sy">5.0</Parameter>
        <Parameter name="ny">30</Parameter>
        <Parameter name="sx">5.0</Parameter>
        <Parameter name="nx">30</Parameter>
        <Parameter name="SensitivityTemplate">IR_FD</Parameter>
    </EngineParameters>
</Product>
\end{minted}
\caption{Configuration for Product BermudanSwaption, Model LGM, Engine Grid}
\label{lst:peconfig_BermudanSwaption_LGM_Grid}
\end{longlisting}

LGM/FD builds a NumericLgmMultiLegOptionEngine using LgmFdSolver as a solver using finite difference. A sample
configuration is shown in listing \ref{lst:peconfig_BermudanSwaption_LGM_FD}

The parameters have the following meaning:

\begin{itemize}
\item Calibration: Bootstrap, BestFit, None
\item CalibrationStrategy: CoterminalDealStrike, CoterminalATM
\item ReferenceCalibrationGrid: An optional grid, only one calibration instrument per interval is kept
\item Reversion: The mean reversion
\item ReversionType: Hagan, HullWhite
\item Volatility: The volatility (start value for calibration if calibrated)
\item VolatilityType: Hagan, HullWhite
\item ShiftHorizon: Shift horizon for LGM model as fraction of deal maturity
\item Tolerance: Error tolerance for calibration
\item Scheme: The finite difference scheme to use
\item StateGridPoints: The number of grid points in state direction
\item TimeStepsPerYear: The number of time steps per year to use
\item MesherEpsilon: determines the covered probability mass, mass outside state grid is $\Phi^{-1}(1-2\epsilon)$
\item SensitivityTemplate [optional]: the sensitivity template to use 
\end{itemize}

\begin{longlisting}
\begin{minted}[fontsize=\footnotesize]{xml}
<Product type="BermudanSwaption">
    <Model>LGM</Model>
    <ModelParameters>
        <Parameter name="Calibration">Bootstrap</Parameter>
        <Parameter name="CalibrationStrategy">CoterminalDealStrike</Parameter>
        <Parameter name="ReferenceCalibrationGrid">400,3M</Parameter>
        <Parameter name="Reversion">0.0</Parameter>
        <Parameter name="ReversionType">HullWhite</Parameter>
        <Parameter name="Volatility">0.01</Parameter>
        <Parameter name="VolatilityType">Hagan</Parameter>
        <Parameter name="ShiftHorizon">0.5</Parameter>
        <Parameter name="Tolerance">0.20</Parameter>
    </ModelParameters>
    <Engine>FD</Engine>
    <EngineParameters>
        <Parameter name="Scheme">Douglas</Parameter>
        <Parameter name="StateGridPoints">64</Parameter>
        <Parameter name="TimeStepsPerYear">24</Parameter>
        <Parameter name="MesherEpsilon">1E-4</Parameter>
    </EngineParameters>
</Product>
\end{minted}
\caption{Configuration for Product BermudanSwaption, Model LGM, Engine FD}
\label{lst:peconfig_BermudanSwaption_LGM_FD}
\end{longlisting}

LGM/MC builds a McMultiLegOptionEngine. A sample configuration is shown in
listing \ref{lst:peconfig_BermudanSwaption_LGM_MC}

The parameters have the following meaning:

\begin{itemize}
\item Calibration: Bootstrap, BestFit, None
\item CalibrationStrategy: CoterminalDealStrike, CoterminalATM
\item ReferenceCalibrationGrid: An optional grid, only one calibration instrument per interval is kept
\item Reversion: The mean reversion
\item ReversionType: Hagan, HullWhite
\item Volatility: The volatility (start value for calibration if calibrated)
\item VolatilityType: Hagan, HullWhite
\item ShiftHorizon: Shift horizon for LGM model as fraction of deal maturity
\item Tolerance: Error tolerance for calibration
\item Training.Sequence: The sequence type for the traning phase, can be MersenneTwister+, MersenneTwisterAntithetc+,
  Sobol+, Burley2020Sobol+, SobolBrownianBridge+, Burley2020SobolBrownianBridge+
\item Training.Seed: The seed for the random number generation in the training phase
\item Training.Samples: The number of samples to be used for the training phase
\item Pricing.Sequence: The sequence type for the pricing phase, same values allowed as for training
\item Training.BasisFunction: The type of basis function system to be used for the regression analysis, can be
  Monomial+, Laguerre+, Hermite+, Hyperbolic+, Legendre+, Chbyshev+, Chebyshev2nd+
\item BasisFunctionOrder: The order of the basis function system to be used
\item Pricing.Seed: The seed for the random number generation in the pricing
\item Pricing.Samples: The number of samples to be used for the pricing phase. If this number is zero, no pricing run is
  performed, instead the (T0) NPV is estimated from the training phase (this result is used to fill the T0 slice of the
  NPV cube)
\item BrownianBridgeOrdering: variate ordering for Brownian bridges, can be Steps+, Factors+, Diagonal+
\item SobolDirectionIntegers: direction integers for Sobol generator, can be Unit+, Jaeckel+, SobolLevitan+,
  SobolLevitanLemieux+, JoeKuoD5+, JoeKuoD6+, JoeKuoD7+, Kuo+, Kuo2+, Kuo3+
\item MinObsDate: if true the conditional expectation of each cashflow is taken from the minimum possible observation
  date (i.e. the latest exercise or simulation date before the cashflow's event date); recommended setting is true+
\item RegressorModel: Simple, LaggedFX. If not given, it defaults to Simple. Depending on the choice the regressor is
  built as follows:
  \begin{itemize}
    \item Simple: For an observation date the full model state observed on this date is included in the regressor. No
      past states are included though.
    \item LaggedFX: For an observation date the full model state observed on this date is included in the regressor. In
      addition, past FX states that are relevant for future cashflows are included. For example, for a FX resettable
      cashflow the FX state observed on the FX reset date is included.
  \end{itemize}
\item SensitivityTemplate [optional]: the sensitivity template to use
\end{itemize}

\begin{longlisting}
\begin{minted}[fontsize=\footnotesize]{xml}
<Product type="BermudanSwaption">
    <Model>LGM</Model>
    <ModelParameters>
        <Parameter name="Calibration">Bootstrap</Parameter>
        <Parameter name="CalibrationStrategy">CoterminalDealStrike</Parameter>
        <Parameter name="ReferenceCalibrationGrid">400,3M</Parameter>
        <Parameter name="Reversion">0.0</Parameter>
        <Parameter name="ReversionType">HullWhite</Parameter>
        <Parameter name="Volatility">0.01</Parameter>
        <Parameter name="VolatilityType">Hagan</Parameter>
        <Parameter name="ShiftHorizon">0.5</Parameter>
        <Parameter name="Tolerance">0.20</Parameter>
    </ModelParameters>
    <Engine>MC</Engine>
    <EngineParameters>
        <Parameter name="Training.Sequence">MersenneTwisterAntithetic</Parameter>
        <Parameter name="Training.Seed">42</Parameter>
        <Parameter name="Training.Samples">10000</Parameter>
        <Parameter name="Training.BasisFunction">Monomial</Parameter>
        <Parameter name="Training.BasisFunctionOrder">6</Parameter>
        <Parameter name="Pricing.Sequence">SobolBrownianBridge</Parameter>
        <Parameter name="Pricing.Seed">17</Parameter>
        <Parameter name="Pricing.Samples">0</Parameter>
        <Parameter name="BrownianBridgeOrdering">Steps</Parameter>
        <Parameter name="SobolDirectionIntegers">JoeKuoD7</Parameter>
        <Parameter name="MinObsDate">true</Parameter>
        <Parameter name="RegressorModel">Simple</Parameter>
        <Parameter name="SensitivityTemplate">IR_MC</Parameter>
    </EngineParameters>
</Product>
\end{minted}
\caption{Configuration for Product BermudanSwaption, Model BlackBachelier, Engine BlackBachelierSwaptionEngine}
\label{lst:peconfig_BermudanSwaption_LGM_MC}
\end{longlisting}

LGM/AMC builds a McMultiLegOptionEngine for use in AMC simulations. We refer to the AMC module documentation for further
details.

%--------------------------------------------------------
\subsubsection{Product Type: AmericanSwaption}
%--------------------------------------------------------

Used by trade type: Swaption, for American exercise on vanilla underlying coupon types

Available Model/Engine pairs:

\begin{itemize}
\item LGM/FD
\item LGM/Grid (Not recommended due to inferior performance)
\item LGM/MC
\item LGM/AMC
\end{itemize}

Engine description:

LGM/FD builds a NumericLgmMultiLegOptionEngine using LgmFdSolver as a solver using finite difference. A sample
configuration is shown in listing \ref{lst:peconfig_AmericanSwaption_LGM_FD}

The parameters have the following meaning:

\begin{itemize}
\item Calibration: Bootstrap, BestFit, None
\item CalibrationStrategy: CoterminalDealStrike, CoterminalATM
\item ReferenceCalibrationGrid: An optional grid, only one calibration instrument per interval is kept
\item Reversion: The mean reversion
\item ReversionType: Hagan, HullWhite
\item Volatility: The volatility (start value for calibration if calibrated)
\item VolatilityType: Hagan, HullWhite
\item ShiftHorizon: Shift horizon for LGM model as fraction of deal maturity
\item Tolerance: Error tolerance for calibration
\item Scheme: The finite difference scheme to use
\item StateGridPoints: The number of grid points in state direction
\item TimeStepsPerYear: The number of time steps per year to use
\item MesherEpsilon: determines the covered probability mass, mass outside state grid is $\Phi^{-1}(1-2\epsilon)$
\item SensitivityTemplate [optional]: the sensitivity template to use 
\end{itemize}

\begin{longlisting}
\begin{minted}[fontsize=\footnotesize]{xml}
<Product type="AmericanSwaption">
    <Model>LGM</Model>
    <ModelParameters>
        <Parameter name="Calibration">Bootstrap</Parameter>
        <Parameter name="CalibrationStrategy">CoterminalDealStrike</Parameter>
        <Parameter name="ReferenceCalibrationGrid">400,3M</Parameter>
        <Parameter name="Reversion">0.0</Parameter>
        <Parameter name="ReversionType">HullWhite</Parameter>
        <Parameter name="Volatility">0.01</Parameter>
        <Parameter name="VolatilityType">Hagan</Parameter>
        <Parameter name="ShiftHorizon">0.5</Parameter>
        <Parameter name="Tolerance">0.20</Parameter>
    </ModelParameters>
    <Engine>FD</Engine>
    <EngineParameters>
        <Parameter name="Scheme">Douglas</Parameter>
        <Parameter name="StateGridPoints">64</Parameter>
        <Parameter name="TimeStepsPerYear">24</Parameter>
        <Parameter name="MesherEpsilon">1E-4</Parameter>
    </EngineParameters>
</Product>
\end{minted}
\caption{Configuration for Product AmericanSwaption, Model LGM, Engine FD}
\label{lst:peconfig_AmericanSwaption_LGM_FD}
\end{longlisting}

LGM/Grid builds a NumericLgmMultiLegOptionEngine using LgmConvoluationSolver as a solver. The rollback follows the paper
``Hagan, P: Methodology for callable swaps and American exercise into swaptions''. A sample configuration is shown in
listing \ref{lst:peconfig_AmericanSwaption_LGM_Grid}. Not recommended due to inferior performance.

The parameters have the following meaning:

\begin{itemize}
\item Calibration: Bootstrap, BestFit, None
\item CalibrationStrategy: CoterminalDealStrike, CoterminalATM
\item ReferenceCalibrationGrid: An optional grid, only one calibration instrument per interval is kept
\item Reversion: The mean reversion
\item ReversionType: Hagan, HullWhite
\item Volatility: The volatility (start value for calibration if calibrated)
\item VolatilityType: Hagan, HullWhite
\item ShiftHorizon: Shift horizon for LGM model as fraction of deal maturity
\item Tolerance: Error tolerance for calibration
\item sy, sx: Number of covered standard deviations (notation as in Hagan's paper)
\item ny, nx: Number of grid points for numerical integration (notation as in Hagan's paper)
\item SensitivityTemplate [optional]: the sensitivity template to use 
\end{itemize}

\begin{longlisting}
\begin{minted}[fontsize=\footnotesize]{xml}
<Product type="AmericanSwaption">
    <Model>LGM</Model>
    <ModelParameters>
        <Parameter name="Calibration">Bootstrap</Parameter>
        <Parameter name="CalibrationStrategy">CoterminalDealStrike</Parameter>
        <Parameter name="ReferenceCalibrationGrid">400,3M</Parameter>
        <Parameter name="Reversion">0.0</Parameter>
        <Parameter name="ReversionType">HullWhite</Parameter>
        <Parameter name="Volatility">0.01</Parameter>
        <Parameter name="VolatilityType">Hagan</Parameter>
        <Parameter name="ShiftHorizon">0.5</Parameter>
        <Parameter name="Tolerance">0.20</Parameter>
    </ModelParameters>
    <Engine>Grid</Engine>
    <EngineParameters>
        <Parameter name="sy">5.0</Parameter>
        <Parameter name="ny">30</Parameter>
        <Parameter name="sx">5.0</Parameter>
        <Parameter name="nx">30</Parameter>
        <Parameter name="SensitivityTemplate">IR_FD</Parameter>
    </EngineParameters>
</Product>
\end{minted}
\caption{Configuration for Product AmericanSwaption, Model LGM, Engine Grid (Not recommended due to inferior performance)}
\label{lst:peconfig_AmericanSwaption_LGM_Grid}
\end{longlisting}

LGM/MC builds a McMultiLegOptionEngine. A sample configuration is shown in
listing \ref{lst:peconfig_AmericanSwaption_LGM_MC}

The parameters have the following meaning:

\begin{itemize}
\item Calibration: Bootstrap, BestFit, None
\item CalibrationStrategy: CoterminalDealStrike, CoterminalATM
\item ReferenceCalibrationGrid: An optional grid, only one calibration instrument per interval is kept
\item Reversion: The mean reversion
\item ReversionType: Hagan, HullWhite
\item Volatility: The volatility (start value for calibration if calibrated)
\item VolatilityType: Hagan, HullWhite
\item ShiftHorizon: Shift horizon for LGM model as fraction of deal maturity
\item Tolerance: Error tolerance for calibration
\item Training.Sequence: The sequence type for the traning phase, can be MersenneTwister+, MersenneTwisterAntithetc+,
  Sobol+, Burley2020Sobol+, SobolBrownianBridge+, Burley2020SobolBrownianBridge+
\item Training.Seed: The seed for the random number generation in the training phase
\item Training.Samples: The number of samples to be used for the training phase
\item Pricing.Sequence: The sequence type for the pricing phase, same values allowed as for training
\item Training.BasisFunction: The type of basis function system to be used for the regression analysis, can be
  Monomial+, Laguerre+, Hermite+, Hyperbolic+, Legendre+, Chbyshev+, Chebyshev2nd+
\item BasisFunctionOrder: The order of the basis function system to be used
\item Pricing.Seed: The seed for the random number generation in the pricing
\item Pricing.Samples: The number of samples to be used for the pricing phase. If this number is zero, no pricing run is
  performed, instead the (T0) NPV is estimated from the training phase (this result is used to fill the T0 slice of the
  NPV cube)
\item BrownianBridgeOrdering: variate ordering for Brownian bridges, can be Steps+, Factors+, Diagonal+
\item SobolDirectionIntegers: direction integers for Sobol generator, can be Unit+, Jaeckel+, SobolLevitan+,
  SobolLevitanLemieux+, JoeKuoD5+, JoeKuoD6+, JoeKuoD7+, Kuo+, Kuo2+, Kuo3+
\item MinObsDate: if true the conditional expectation of each cashflow is taken from the minimum possible observation
  date (i.e. the latest exercise or simulation date before the cashflow's event date); recommended setting is true+
\item RegressorModel: Simple, LaggedFX. If not given, it defaults to Simple. Depending on the choice the regressor is
  built as follows:
  \begin{itemize}
    \item Simple: For an observation date the full model state observed on this date is included in the regressor. No
      past states are included though.
    \item LaggedFX: For an observation date the full model state observed on this date is included in the regressor. In
      addition, past FX states that are relevant for future cashflows are included. For example, for a FX resettable
      cashflow the FX state observed on the FX reset date is included.
  \end{itemize}
\item SensitivityTemplate [optional]: the sensitivity template to use
\end{itemize}

\begin{longlisting}
\begin{minted}[fontsize=\footnotesize]{xml}
<Product type="AmericanSwaption">
    <Model>LGM</Model>
    <ModelParameters>
        <Parameter name="Calibration">Bootstrap</Parameter>
        <Parameter name="CalibrationStrategy">CoterminalDealStrike</Parameter>
        <Parameter name="ReferenceCalibrationGrid">400,3M</Parameter>
        <Parameter name="Reversion">0.0</Parameter>
        <Parameter name="ReversionType">HullWhite</Parameter>
        <Parameter name="Volatility">0.01</Parameter>
        <Parameter name="VolatilityType">Hagan</Parameter>
        <Parameter name="ShiftHorizon">0.5</Parameter>
        <Parameter name="Tolerance">0.20</Parameter>
    </ModelParameters>
    <Engine>MC</Engine>
    <EngineParameters>
        <Parameter name="Training.Sequence">MersenneTwisterAntithetic</Parameter>
        <Parameter name="Training.Seed">42</Parameter>
        <Parameter name="Training.Samples">10000</Parameter>
        <Parameter name="Training.BasisFunction">Monomial</Parameter>
        <Parameter name="Training.BasisFunctionOrder">6</Parameter>
        <Parameter name="Pricing.Sequence">SobolBrownianBridge</Parameter>
        <Parameter name="Pricing.Seed">17</Parameter>
        <Parameter name="Pricing.Samples">0</Parameter>
        <Parameter name="BrownianBridgeOrdering">Steps</Parameter>
        <Parameter name="SobolDirectionIntegers">JoeKuoD7</Parameter>
        <Parameter name="MinObsDate">true</Parameter>
        <Parameter name="RegressorModel">Simple</Parameter>
        <Parameter name="SensitivityTemplate">IR_MC</Parameter>
    </EngineParameters>
</Product>
\end{minted}
\caption{Configuration for Product AmericanSwaption, Model BlackBachelier, Engine BlackBachelierSwaptionEngine}
\label{lst:peconfig_AmericanSwaption_LGM_MC}
\end{longlisting}

LGM/AMC builds a McMultiLegOptionEngine for use in AMC simulations. We refer to the AMC module documentation for further
details.

%--------------------------------------------------------
\subsubsection{Product Type: BondRepo}
%--------------------------------------------------------

Used by trade type: BondRepo

Available Model/Engine pairs:

\begin{itemize}
\item DiscountedCashflows/DiscountingRepoEngine
\item Accrual/AccrualRepoEngine
\end{itemize}

Engine description:

DiscountedCashflows/DiscountingRepoEngine builds a DiscountingBondRepoEngine. A sample configuration is shown in listing
\ref{lst:peconfig_BondRepo_DiscountedCashflows_DiscountingRepoEngine}.

The parameters have the following meaning:

\begin{itemize}
\item IncludeSecurityLeg: include the security leg in the valuation
\item SensitivityTemplate [optional]: the sensitivity template to use 
\end{itemize}

\begin{longlisting}
\begin{minted}[fontsize=\footnotesize]{xml}
<Product type="BondRepo">
    <Model>DiscountedCashflows</Model>
    <ModelParameters>
        <Parameter name="IncludeSecurityLeg">true</Parameter>
    </ModelParameters>
    <Engine>DiscountingRepoEngine</Engine>
    <EngineParameters>
        <Parameter name="SensitivityTemplate">IR_Analytical</Parameter>
    </EngineParameters>
</Product>
\end{minted}
\caption{Configuration for Product BondRepo, Model DiscountedCashflows, Engine DiscountingRepoEngine}
\label{lst:peconfig_BondRepo_DiscountedCashflows_DiscountingRepoEngine}
\end{longlisting}

Accrual/AccrualRepoEngine builds a AccrualBondRepoEngine. A sample configuration is shown in listing
\ref{lst:peconfig_BondRepo_Accrual_AccrualBondRepoEngine}.

The parameters have the following meaning:

\begin{itemize}
\item IncludeSecurityLeg: include the security leg in the valuation
\item SensitivityTemplate [optional]: the sensitivity template to use 
\end{itemize}

\begin{longlisting}
\begin{minted}[fontsize=\footnotesize]{xml}
Product type="BondRepo">
    <Model>Accrual</Model>
    <ModelParameters>
        <Parameter name="IncludeSecurityLeg">true</Parameter>
    </ModelParameters>
    <Engine>AccrualRepoEngine</Engine>
    <EngineParameters>
        <Parameter name="SensitivityTemplate">IR_Analytical</Parameter>
    </EngineParameters>
</Product>
\end{minted}
\caption{Configuration for Product BondRepo, Model Accrual, Engine DiscountingBondRepoEngine}
\label{lst:peconfig_BondRepo_Accrual_AccrualBondRepoEngine}
\end{longlisting}

%--------------------------------------------------------
\subsubsection{Product Type: BondTRS}
%--------------------------------------------------------

Used by trade type: BondTRS

Available Model/Engine pairs: DiscountedCashflows/DiscountingBondTRSEngine

Engine description:

DiscountedCashflows/DiscountingBondTRSEngine builds a DisountingBondTRSEngine. A sample configuration is shown in listing \ref{lst:peconfig_BondTRS_DiscountedCashflows_DiscountedCashflows/DiscountingBondTRSEngine}.

The parameters have the following meaning:

\begin{itemize}
\item SensitivityTemplate [optional]: the sensitivity template to use 
\end{itemize}

\begin{longlisting}
\begin{minted}[fontsize=\footnotesize]{xml}
<Product type="BondTRS">
    <Model>DiscountedCashflows</Model>
    <ModelParameters/>
    <Engine>DiscountingBondTRSEngine</Engine>
    <EngineParameters>
        <Parameter name="SensitivityTemplate">IR_Analytical</Parameter>
    </EngineParameters>
</Product>
\end{minted}
\caption{Configuration for Product BondTRS , Model DiscountedCashflows, Engine DiscountingBondTRSEngine}
\label{lst:peconfig_BondTRS_DiscountedCashflows_DiscountedCashflows/DiscountingBondTRSEngine}
\end{longlisting}

%--------------------------------------------------------
\subsubsection{Product Type: CapFloor}
%--------------------------------------------------------

Used by trade type: CapFloor on underlying Ibor with subperiods (all other underlying use their respective coupon pricers)

Available Model/Engine pairs: IborCapModel/IborCapEngine

Engine description:

IborCapModel/IborCapEngine builds a BlackCapFloorEngine or BachelierCapFloorEngine depending on the input volatility
type. A sample configuration is shown in listing \ref{lst:peconfig_CapFloor_IborCapModel_IborCapEngine}

The parameters have the following meaning:

\begin{itemize}
\item SensitivityTemplate [optional]: the sensitivity template to use 
\end{itemize}

\begin{longlisting}
\begin{minted}[fontsize=\footnotesize]{xml}
<Product type="CapFloor">
    <Model>IborCapModel</Model>
    <ModelParameters/>
    <Engine>IborCapEngine</Engine>
    <EngineParameters>
        <Parameter name="SensitivityTemplate">IR_Analytical</Parameter>
    </EngineParameters>
</Product>
\end{minted}
\caption{Configuration for Product CapFloor, Model IborCapModel, Engine IborCapEngine}
\label{lst:peconfig_CapFloor_IborCapModel_IborCapEngine}
\end{longlisting}

%--------------------------------------------------------
\subsubsection{Product Type: CapFlooredIborLeg}
%--------------------------------------------------------

Used by trade type: any trade with a cap / floored ibor / rfr term rate leg

Available Model/Engine pairs: BlackOrBachelier/BlackIborCouponPricer

Engine description:

BlackOrBachelier/BlackIborCouponPricer builds a BlackIborCouponPricer. A sample configuration is
shown in listing \ref{lst:peconfig_CapFlooredIborLeg_BlackOrBachelier_BlackIborCouponPricer}.

The parameters have the following meaning:

\begin{itemize}
\item SensitivityTemplate [optional]: the sensitivity template to use 
\end{itemize}

\begin{longlisting}
\begin{minted}[fontsize=\footnotesize]{xml}
<Product type="CapFlooredIborCouponLeg">
    <Model>BlackOrBachelier</Model>
    <ModelParameters/>
    <Engine>BlackIborCouponPricer</Engine>
    <EngineParameters>
        <Parameter name="SensitivityTemplate">IR_Analytical</Parameter>
    </EngineParameters>
</Product>
\end{minted}
\caption{Configuration for Product CapFlooredIborLeg, Model BlackOrBachelier, Engine BlackIborCouponPricer}
\label{lst:peconfig_CapFlooredIborLeg_BlackOrBachelier_BlackIborCouponPricer}
\end{longlisting}

%--------------------------------------------------------
\subsubsection{Product Type: CapFlooredOvernightIndexedCouponLeg}
%--------------------------------------------------------

Used by trade type: any trade with a cap / floored OIS leg

Available Model/Engine pairs: BlackOrBachelier/BlackOvernightIndexedCouponPricer

Engine description:

BlackOrBachelier/BlackOvernightIndexedCouponPricer builds a BlackOvernightIndexedCouponPricer. A sample configuration is
shown in listing \ref{lst:peconfig_CapFlooredOvernightIndexedCouponLeg_BlackOrBachelier_BlackOvernightIndexedCouponPricer}.

The parameters have the following meaning:

\begin{itemize}
\item SensitivityTemplate [optional]: the sensitivity template to use 
\end{itemize}

\begin{longlisting}
\begin{minted}[fontsize=\footnotesize]{xml}
<Product type="CapFlooredOvernightIndexedCouponLeg">
    <Model>BlackOrBachelier</Model>
    <ModelParameters/>
    <Engine>BlackOvernightIndexedCouponPricer</Engine>
    <EngineParameters>
        <Parameter name="SensitivityTemplate">IR_Analytical</Parameter>
    </EngineParameters>
</Product>
\end{minted}
\caption{Configuration for Product CapFlooredOvernightIndexedCouponLeg, Model BlackOrBachelier, Engine BlackOvernightIndexedCouponPricer}
\label{lst:peconfig_CapFlooredOvernightIndexedCouponLeg_BlackOrBachelier_BlackOvernightIndexedCouponPricer}
\end{longlisting}

%--------------------------------------------------------
\subsubsection{Product Type: CapFlooredAverageONIndexedCouponLeg}
%--------------------------------------------------------

Used by trade type: any trade with a cap / floored OIS leg

Available Model/Engine pairs: BlackOrBachelier/BlackAverageONIndexedCouponPricer

Engine description:

BlackOrBachelier/BlackAverageONIndexedCouponPricer builds a BlackAverageONIndexedCouponPricer. A sample configuration is
shown in listing \ref{lst:peconfig_CapFlooredAverageONIndexedCouponLeg_BlackOrBachelier_BlackAverageONIndexedCouponPricer}.

The parameters have the following meaning:

\begin{itemize}
\item SensitivityTemplate [optional]: the sensitivity template to use 
\end{itemize}

\begin{longlisting}
\begin{minted}[fontsize=\footnotesize]{xml}
<Product type="CapFlooredAverageONIndexedCouponLeg">
    <Model>BlackOrBachelier</Model>
    <ModelParameters/>
    <Engine>BlackAverageONIndexedCouponPricer</Engine>
    <EngineParameters>
        <Parameter name="SensitivityTemplate">IR_Analytical</Parameter>
    </EngineParameters>
</Product>
\end{minted}
\caption{Configuration for Product CapFlooredAverageONIndexedCouponLeg, Model BlackOrBachelier, Engine BlackAverageONIndexedCouponPricer}
\label{lst:peconfig_CapFlooredAverageONIndexedCouponLeg_BlackOrBachelier_BlackAverageONIndexedCouponPricer}
\end{longlisting}

%--------------------------------------------------------
\subsubsection{Product Type: CapFlooredAverageBMAIndexedCouponLeg}
%--------------------------------------------------------

Used by trade type: any trade with a cap / floored OIS leg

Available Model/Engine pairs: BlackOrBachelier/BlackAverageBMAIndexedCouponPricer

Engine description:

BlackOrBachelier/BlackAverageBMAIndexedCouponPricer builds a BlackAverageBMAIndexedCouponPricer. A sample configuration is
shown in listing \ref{lst:peconfig_CapFlooredAverageBMAIndexedCouponLeg_BlackOrBachelier_BlackAverageBMAIndexedCouponPricer}.

The parameters have the following meaning:

\begin{itemize}
\item SensitivityTemplate [optional]: the sensitivity template to use 
\end{itemize}

\begin{longlisting}
\begin{minted}[fontsize=\footnotesize]{xml}
<Product type="CapFlooredAverageBMAIndexedCouponLeg">
    <Model>BlackOrBachelier</Model>
    <ModelParameters/>
    <Engine>BlackAverageBMAIndexedCouponPricer</Engine>
    <EngineParameters>
        <Parameter name="SensitivityTemplate">IR_Analytical</Parameter>
    </EngineParameters>
</Product>
\end{minted}
\caption{Configuration for Product CapFlooredAverageBMAIndexedCouponLeg, Model BlackOrBachelier, Engine BlackAverageBMAIndexedCouponPricer}
\label{lst:peconfig_CapFlooredAverageBMAIndexedCouponLeg_BlackOrBachelier_BlackAverageBMAIndexedCouponPricer}
\end{longlisting}

%--------------------------------------------------------
\subsubsection{Product Type: CappedFlooredCpiLegCoupons}
%--------------------------------------------------------

Used by trade type: any trade with a cap / floored CPI leg (coupons)

Available Model/Engine pairs: Black/BlackAnalytic

Engine description:

Black/BlackAnalytic builds a BlackCPICouponPricer or BachelierCPICouponPricer, depending on the volatility input. A sample configuration is
shown in listing \ref{lst:peconfig_CapFlooredCpiLegCoupons_Black_BlackAnalytic}

The parameters have the following meaning:

\begin{itemize}
\item useLastFixingDate: if true, use the last known fixing date as the base date of the volatility structure, otherwise
  use observation lag        
\item SensitivityTemplate [optional]: the sensitivity template to use 
\end{itemize}

\begin{longlisting}
\begin{minted}[fontsize=\footnotesize]{xml}
<Product type="CapFlooredCpiLegCoupons">
    <Model>Black</Model>
    <ModelParameters/>
    <Engine>BlackAnalytic</Engine>
    <EngineParameters>
          <Parameter name="useLastFixingDate">true</Parameter>
      <Parameter name="SensitivityTemplate">IR_Analytical</Parameter>
    </EngineParameters>
</Product>
\end{minted}
\caption{Configuration for Product CapFlooredCpiLegCoupons, Model Black, Engine BlackAnalytic}
\label{lst:peconfig_CapFlooredCpiLegCoupons_Black_BlackAnalytic}
\end{longlisting}

%--------------------------------------------------------
\subsubsection{Product Type: CappedFlooredCpiLegCashFlows}
%--------------------------------------------------------

Used by trade type: any trade with a cap / floored CPI leg (cashflows)

Available Model/Engine pairs: Black/BlackAnalytic

Engine description:

Black/BlackAnalytic builds a BlackCPICashFLowPricer or BachelierCPICashFlowPricer, depending on the volatility input. A sample configuration is
shown in listing \ref{lst:peconfig_CappedFlooredCpiLegCashFlows_Black_BlackAnalytic}

The parameters have the following meaning:

\begin{itemize}
\item useLastFixingDate: if true, use the last known fixing date as the base date of the volatility structure, otherwise
  use observation lag        
\item SensitivityTemplate [optional]: the sensitivity template to use 
\end{itemize}

\begin{longlisting}
\begin{minted}[fontsize=\footnotesize]{xml}
<Product type="CappedFlooredCpiLegCashFlows">
    <Model>Black</Model>
    <ModelParameters/>
    <Engine>BlackAnalytic</Engine>
    <EngineParameters>
          <Parameter name="useLastFixingDate">true</Parameter>
      <Parameter name="SensitivityTemplate">IR_Analytical</Parameter>
    </EngineParameters>
</Product>
\end{minted}
\caption{Configuration for Product CapFlooredCpiLegCoupons, Model Black, Engine BlackAnalytic}
\label{lst:peconfig_CappedFlooredCpiLegCashFlows_Black_BlackAnalytic}
\end{longlisting}

%--------------------------------------------------------
\subsubsection{Product Type: CommodityAveragePriceOption}
%--------------------------------------------------------

Used by trade type: CommodityAveragePriceOption

Available Model/Engine pairs:

\begin{itemize}
\item Black/AnalyticalApproximation
\item Black/MonteCarlo
\end{itemize}

Engine description:

Black/AnalyticalApproximation builds a CommodityAveragePriceOptionAnalyticalEngine. The correlation between two future
contracts is parametrized as

$$\rho(s, t) = e^{-\beta |s-t|}$$

where $s$ and $t$ are times to futures expiry. A sample configuration is shown in listing
\ref{lst:peconfig_CommodityAveragePriceOption_Black_AnalyticalApproximation}.

The parameters have the following meaning:

\begin{itemize}
\item beta: parameter in correlation parametrization
\item SensitivityTemplate [optional]: the sensitivity template to use 
\end{itemize}

\begin{longlisting}
\begin{minted}[fontsize=\footnotesize]{xml}
<Product type="CommodityAveragePriceOption">
    <Model>Black</Model>
    <ModelParameters/>
    <Engine>AnalyticalApproximation</Engine>
    <EngineParameters>
        <Parameter name="beta">0</Parameter>
        <Parameter name="SensitivityTemplate">COMM_Analytical</Parameter>
    </EngineParameters>
</Product>
\end{minted}
\caption{Configuration for Product CommodityAveragePriceOption, Model Black, Engine AnalyticalApproximation}
\label{lst:peconfig_CommodityAveragePriceOption_Black_AnalyticalApproximation}
\end{longlisting}

Black/MonteCarlo builds a CommodityAveragePriceOptionMonteCarloEngine. A sample configuration is shown in listing
\ref{lst:peconfig_CommodityAveragePriceOption_Black_MonteCarlo}.

The parameters have the following meaning:

\begin{itemize}
\item samples: the number of Monte Carlo Samples
\item beta: parameter in correlation parametrization
\item SensitivityTemplate [optional]: the sensitivity template to use 
\end{itemize}

\begin{longlisting}
\begin{minted}[fontsize=\footnotesize]{xml}
<Product type="CommodityAveragePriceOption">
    <Model>Black</Model>
    <ModelParameters/>
    <Engine>MonteCarlo</Engine>
    <EngineParameters>
        <Parameter name="samples">10000</Parameter>
        <Parameter name="beta">0</Parameter>
        <Parameter name="SensitivityTemplate">COMM_MC</Parameter>
    </EngineParameters>
</Product>
\end{minted}
\caption{Configuration for Product CommodityAveragePriceOption, Model Black, Engine MonteCarlo}
\label{lst:peconfig_CommodityAveragePriceOption_Black_MonteCarlo}
\end{longlisting}

%--------------------------------------------------------
\subsubsection{Product Type: CommodityAveragePriceBarrierOption}
%--------------------------------------------------------

Used by trade type: CommodityAveragePriceOption with BarrierData

Available Model/Engine pairs:

\begin{itemize}
\item Black/MonteCarlo
\end{itemize}

Engine description:

Black/MonteCarlo builds a CommodityAveragePriceOptionAnalyticalEngine. The correlation between two future contracts is
parametrized as

$$\rho(s, t) = e^{-\beta |s-t|}$$

where $s$ and $t$ are times to futures expiry. A sample configuration is shown in listing
\ref{lst:peconfig_CommodityAveragePriceBarrierOption_Black_MonteCarlo}.

The parameters have the following meaning:

\begin{itemize}
\item samples: the number of Monte Carlo Samples
\item beta: parameter in correlation parametrization
\item SensitivityTemplate [optional]: the sensitivity template to use 
\end{itemize}

\begin{longlisting}
\begin{minted}[fontsize=\footnotesize]{xml}
<Product type="CommodityAveragePriceBarrierOption">
    <Model>Black</Model>
    <ModelParameters/>
    <Engine>MonteCarlo</Engine>
    <EngineParameters>
        <Parameter name="samples">10000</Parameter>
        <Parameter name="beta">0</Parameter>
        <Parameter name="SensitivityTemplate">COMM_MC</Parameter>
    </EngineParameters>
</Product>
\end{minted}
\caption{Configuration for Product CommodityAveragePriceBarrierOption, Model Black, Engine MonteCarlo}
\label{lst:peconfig_CommodityAveragePriceBarrierOption_Black_MonteCarlo}
\end{longlisting}

%--------------------------------------------------------
\subsubsection{Product Type: CommodityForward}
%--------------------------------------------------------

Used by trade type: CommodityForward

Available Model/Engine pairs:

\begin{itemize}
\item DiscountedCashflows/DiscountingCommodityForwardEngine
\end{itemize}

Engine description:

DiscountedCashflows/DiscountingCommodityForwardEngine builds a DiscountingCommodityForwardEngine. A sample configuration is shown in listing
\ref{lst:peconfig_CommodityForward_DiscountedCashflows_DiscountingCommodityForwardEngine}.

The parameters have the following meaning:

\begin{itemize}
\item SensitivityTemplate [optional]: the sensitivity template to use 
\end{itemize}

\begin{longlisting}
\begin{minted}[fontsize=\footnotesize]{xml}
<Product type="CommodityForward">
    <Model>DiscountedCashflows</Model>
    <ModelParameters/>
    <Engine>DiscountingCommodityForwardEngine</Engine>
    <EngineParameters>
        <Parameter name="SensitivityTemplate">COMM_Analytical</Parameter>
    </EngineParameters>
</Product>
\end{minted}
\caption{Configuration for Product CommodityForward, Model DiscountedCashflows, Engine DiscountingCommodityForwardEngine}
\label{lst:peconfig_CommodityForward_DiscountedCashflows_DiscountingCommodityForwardEngine}
\end{longlisting}

%--------------------------------------------------------
\subsubsection{Product Type: CreditDefaultSwap}
%--------------------------------------------------------

Used by trade type: CreditDefaultSwap

Available Model/Engine pairs:

\begin{itemize}
\item DiscountedCashflows/MidPointCdsEngine
\item DiscountedCashflows/MidPointCdsEngineMultiState
\end{itemize}

Engine description:

DiscountedCashflows/MidPointCdsEngine builds a MidPointCdsEngine. A sample configuration is shown in listing
\ref{lst:peconfig_CreditDefaultSwap_DiscountedCashflows_MidPointCdsEngine}.

The parameters have the following meaning:

\begin{itemize}
\item SensitivityTemplate [optional]: the sensitivity template to use 
\end{itemize}

\begin{longlisting}
\begin{minted}[fontsize=\footnotesize]{xml}
<Product type="CreditDefaultSwap">
    <Model>DiscountedCashflows</Model>
    <ModelParameters/>
    <Engine>MidPointCdsEngine</Engine>
    <EngineParameters>
        <Parameter name="SensitivityTemplate">IR_Analytical</Parameter>
    </EngineParameters>
</Product>
\end{minted}
\caption{Configuration for Product CreditDefaultSwap, Model DiscountedCashflows, Engine MidPointCdsEngine}
\label{lst:peconfig_CreditDefaultSwap_DiscountedCashflows_MidPointCdsEngine}
\end{longlisting}

DiscountedCashflows/MidPointCdsEngineMultiState builds a MidPointCdsEngineMultiState. This engine is only used in the
context of the Credit Model. We refer to the documentation of this module for further details.

%--------------------------------------------------------
\subsubsection{Product Type: CreditDefaultSwapOption}
%--------------------------------------------------------

Used by trade type: CreditDefaultSwapOption

Available Model/Engine pairs:

\begin{itemize}
\item Black/BlackCdsOptionEngine
\end{itemize}

Engine description:

Black/BlackCdsOptionEngine builds a BlackCdsOptionEngine. A sample configuration is shown in listing
\ref{lst:peconfig_CreditDefaultSwapOption_Black_BlackCdsOptionEngine}.

The parameters have the following meaning:

\begin{itemize}
\item SensitivityTemplate [optional]: the sensitivity template to use 
\end{itemize}

\begin{longlisting}
\begin{minted}[fontsize=\footnotesize]{xml}
<Product type="CreditDefaultSwapOption">
    <Model>Black</Model>
    <ModelParameters/>
    <Engine>BlackCdsOptionEngine</Engine>
    <EngineParameters>
        <Parameter name="SensitivityTemplate">IR_Analytical</Parameter>
    </EngineParameters>
</Product>
\end{minted}
\caption{Configuration for Product CreditDefaultSwap, Model DiscountedCashflows, Engine MidPointCdsEngine}
\label{lst:peconfig_CreditDefaultSwapOption_Black_BlackCdsOptionEngine}
\end{longlisting}

%--------------------------------------------------------
\subsubsection{Product Type: IndexCreditDefaultSwap}
%--------------------------------------------------------

Used by trade type: IndexCreditDefaultSwap

Available Model/Engine pairs:

\begin{itemize}
\item DiscountedCashflows/MidPointIndexCdsEngine
\end{itemize}

Engine description:

DiscountedCashflows/MidPointIndexCdsEngine builds a MidPointIndexCdsEngine. A sample configuration is shown in listing
\ref{lst:peconfig_CreditDefaultSwap_DiscountedCashflows_MidPointIndexCdsEngine}.

The parameters have the following meaning:

\begin{itemize}
\item Curve: Index, Underlying
\item SensitivityDecomposition: Underlying, NotionalWeighted, LossWeighted, DeltaWeighted
\item SensitivityTemplate [optional]: the sensitivity template to use 
\end{itemize}

\begin{longlisting}
\begin{minted}[fontsize=\footnotesize]{xml}
<Product type="IndexCreditDefaultSwap">
    <Model>DiscountedCashflows</Model>
    <ModelParameters/>
    <Engine>MidPointIndexCdsEngine</Engine>
    <EngineParameters>
        <Parameter name="Curve">Index</Parameter>
        <Parameter name="SensitivityDecomposition">DeltaWeighted</Parameter>
        <Parameter name="SensitivityTemplate">IR_Analytical</Parameter>
    </EngineParameters>
</Product>
\end{minted}
\caption{Configuration for Product CreditDefaultSwap, Model DiscountedCashflows, Engine MidPointIndexCdsEngine}
\label{lst:peconfig_CreditDefaultSwap_DiscountedCashflows_MidPointIndexCdsEngine}
\end{longlisting}

%--------------------------------------------------------
\subsubsection{Product Type: IndexCreditDefaultSwapOption}
%--------------------------------------------------------

Used by trade type: IndexCreditDefaultSwapOption

Available Model/Engine pairs:

\begin{itemize}
\item Black/BlackIndexCdsOptionEngine
\item LognormalAdjustedIndexSpread/NumericalIntegrationEngine
\end{itemize}

Engine description:

Black/BlackIndexCdsOptionEngine builds a BlackIndexCdsOptionEngine. A sample configuration is shown in listing
\ref{lst:peconfig_CreditDefaultSwapOption_Black_BlackIndexCdsOptionEngine}.

The parameters have the following meaning:

\begin{itemize}
\item Curve: Index, Underlying
\item FepCurve: Index, Underlying
\item SensitivityDecomposition: Underlying, NotionalWeighted, LossWeighted, DeltaWeighted
\item SensitivityTemplate [optional]: the sensitivity template to use 
\end{itemize}

\begin{longlisting}
\begin{minted}[fontsize=\footnotesize]{xml}
<Product type="IndexCreditDefaultSwapOption">
    <Model>DiscountedCashflows</Model>
    <ModelParameters/>
    <Engine>MidPointIndexCdsEngine</Engine>
    <EngineParameters>
        <Parameter name="Curve">Index</Parameter>
        <Parameter name="FepCurve">Index</Parameter>
        <Parameter name="SensitivityDecomposition">DeltaWeighted</Parameter>
        <Parameter name="SensitivityTemplate">IR_Analytical</Parameter>
    </EngineParameters>
</Product>
\end{minted}
\caption{Configuration for Product CreditDefaultSwap, Model DiscountedCashflows, Engine MidPointIndexCdsEngine}
\label{lst:peconfig_CreditDefaultSwapOption_Black_BlackIndexCdsOptionEngine}
\end{longlisting}

LognormalAdjustedIndexSpread/NumericalIntegrationEngine builds a NumericalIntegrationIndexCdsOptionEngine. A sample
configuration is shown in listing
\ref{lst:peconfig_CreditDefaultSwapOption_LognormalAdjustedIndexSpread_NumericalIntegrationEngine}.

The parameters have the following meaning:

\begin{itemize}
\item Curve: Index, Underlying
\item FepCurve: Index, Underlying
\item SensitivityDecomposition: Underlying, NotionalWeighted, LossWeighted, DeltaWeighted
\item SensitivityTemplate [optional]: the sensitivity template to use 
\end{itemize}

\begin{longlisting}
\begin{minted}[fontsize=\footnotesize]{xml}
<Product type="IndexCreditDefaultSwapOption">
    <Model>DiscountedCashflows</Model>
    <ModelParameters/>
    <Engine>MidPointIndexCdsEngine</Engine>
    <EngineParameters>
        <Parameter name="Curve">Index</Parameter>
        <Parameter name="FepCurve">Index</Parameter>
        <Parameter name="SensitivityDecomposition">DeltaWeighted</Parameter>
        <Parameter name="SensitivityTemplate">IR_Semianalytical</Parameter>
    </EngineParameters>
</Product>
\end{minted}
\caption{Configuration for Product CreditDefaultSwap, Model DiscountedCashflows, Engine MidPointIndexCdsEngine}
\label{lst:peconfig_CreditDefaultSwapOption_LognormalAdjustedIndexSpread_NumericalIntegrationEngine}
\end{longlisting}

%--------------------------------------------------------
\subsubsection{Product Type: CpiCapFloor}
%--------------------------------------------------------

Used by trade type: CapFloor with underlying leg of leg type CPI.

Available Model/Engine pairs: CpiCapModel/CpiCapEngine

Engine description:

CpiCapModel/CpiCapEngine builds a CPIBlackCapFloorEngine or CPIBachelierCapFloorEngine depneding on the volatility type
of the market surface. A sample configuration is shown in listing
\ref{lst:peconfig_CpiCapFloor_CpiCapModel_CpiCapEngine}.

The parameters have the following meaning:

\begin{itemize}
\item useLastFixingDate: if true, use the last known fixing date as the base date of the volatility structure, otherwise
  use observation lag
\item SensitivityTemplate [optional]: the sensitivity template to use 
\end{itemize}

\begin{longlisting}
\begin{minted}[fontsize=\footnotesize]{xml}
<Product type="CpiCapFloor">
    <Model>CpiCapModel</Model>
    <ModelParameters/>
    <Engine>CpiCapEngine</Engine>
    <EngineParameters>
        <Parameter name="useLastFixingDate">true</Parameter>
        <Parameter name="SensitivityTemplate">IR_Analytical</Parameter>
    </EngineParameters>
</Product>
\end{minted}
\caption{Configuration for Product CpiCapFloor, Model CpiCapModel, Engine CpiCapEngine}
\label{lst:peconfig_CpiCapFloor_CpiCapModel_CpiCapEngine}
\end{longlisting}

%--------------------------------------------------------
\subsubsection{Product Type: YYCapFloor}
%--------------------------------------------------------

Used by trade type: CapFloor with underlying leg of leg type YoY.

Available Model/Engine pairs: YYCapModel/YYCapEngine

Engine description:

YYCapModel/YYCapEngine builds a YoYInflationBlackCapFloorEngine, YoYInflationUnitDisplacedBlackCapFloorEngine,
YoYInflationBachelierCapFloorEngine depending on the volatility type of the market surface. A sample configuration is
shown in listing \ref{lst:peconfig_YYCapFloor_YYCapModel_YYCapEngine}.

The parameters have the following meaning:

\begin{itemize}
\item SensitivityTemplate [optional]: the sensitivity template to use 
\end{itemize}

\begin{longlisting}
\begin{minted}[fontsize=\footnotesize]{xml}
<Product type="CpiCapFloor">
    <Model>YYCapModel</Model>
    <ModelParameters/>
    <Engine>YYCapEngine</Engine>
    <EngineParameters>
        <Parameter name="SensitivityTemplate">IR_Analytical</Parameter>
    </EngineParameters>
</Product>
\end{minted}
\caption{Configuration for Product YYCapFloor, Model YYCapModel, Engine YYCapEngine}
\label{lst:peconfig_YYCapFloor_YYCapModel_YYCapEngine}
\end{longlisting}

%--------------------------------------------------------
\subsubsection{Product Type: CappedFlooredYYLeg}
%--------------------------------------------------------

Used by trade type: any trade with a cap / floored YY leg

Available Model/Engine pairs: CapFlooredYYModel/CapFlooredYYCouponPricer

Engine description:

CapFlooredYYModel/CapFlooredYYCouponPricer builds a BlackYoYInflationCouponPricer,
UnitDisplacedBlackYoYInflationCouponPricer, BachelierYoYInflationCouponPricer, depending on the volatility input. A
sample configuration is shown in listing \ref{lst:peconfig_CapFlooredYYLeg_CapFlooredYYModel_CapFlooredYYCouponPricer}

The parameters have the following meaning:

\begin{itemize}
\item SensitivityTemplate [optional]: the sensitivity template to use 
\end{itemize}

\begin{longlisting}
\begin{minted}[fontsize=\footnotesize]{xml}
<Product type="CappedFlooredYYLeg">
    <Model>CapFlooredYYModel</Model>
    <ModelParameters/>
    <Engine>CapFlooredYYCouponPricer</Engine>
    <EngineParameters>
        <Parameter name="SensitivityTemplate">IR_Analytical</Parameter>
    </EngineParameters>
</Product>
\end{minted}
\caption{Configuration for Product CapFlooredYYLeg, Model CapFlooredYYModel, Engine CapFlooredYYCouponPricer}
\label{lst:peconfig_CapFlooredYYLeg_CapFlooredYYModel_CapFlooredYYCouponPricer}
\end{longlisting}

%--------------------------------------------------------
\subsubsection{Product Type: CappedFlooredNonStdYYLeg}
%--------------------------------------------------------

Used by trade type: any trade with a cap / floored YY leg if IrregularYoY is true in YoY leg data

Available Model/Engine pairs: CapFlooredYNonStdYYModel/CapFlooredNonStdYYCouponPricer

Engine description:

CapFlooredYNonStdYYModel/CapFlooredNonStdYYCouponPricer builds a NonStandardBlackYoYInflationCouponPricer,
NonStandardUnitDisplacedBlackYoYInflationCouponPricer, NonStandardBachelierYoYInflationCouponPricer, depending on the
volatility input. A sample configuration is shown in listing
\ref{lst:peconfig_CapFlooredNonStdYYLeg_CapFlooredNonStdYYModel_CapFlooredNonStdYYCouponPricer}

The parameters have the following meaning:

\begin{itemize}
\item SensitivityTemplate [optional]: the sensitivity template to use 
\end{itemize}

\begin{longlisting}
\begin{minted}[fontsize=\footnotesize]{xml}
<Product type="CappedFlooredNonStdYYLeg">
    <Model>CapFlooredNonStdYYModel</Model>
    <ModelParameters/>
    <Engine>CapFlooredNonStdYYCouponPricer</Engine>
    <EngineParameters>
        <Parameter name="SensitivityTemplate">IR_Analytical</Parameter>
    </EngineParameters>
</Product>
\end{minted}
\caption{Configuration for Product CapFlooredNonStdYYLeg, Model CapFlooredNonStdYYModel, Engine
  CapFlooredNonStdYYCouponPricer}
\label{lst:peconfig_CapFlooredNonStdYYLeg_CapFlooredNonStdYYModel_CapFlooredNonStdYYCouponPricer}
\end{longlisting}

%--------------------------------------------------------
\subsubsection{Product Type: CMS}
%--------------------------------------------------------

Used by trade type: any trade referencing a CMS leg, also used by CMS Spread coupon pricers

Available Model/Engine pairs:

\begin{itemize}
  \item LinearTSR/LinearTSRPricer
  \item Hagan/Analytic
  \item Hagan/Numerical
\end{itemize}

Engine description:

LinearTSR/LinearTSRPricer builds a LinearTRSPricer. A sample configuration is shown in listing
\ref{lst:peconfig_CMS_LinearTSR_LinearTSRPricer}.

The parameters have the following meaning:

\begin{itemize}
\item MeanReversion: the mean reversion for the model
\item Policy: RateBound, VegaRatio, PriceThreshold, BsStdDev
\item LowerRateBoundLogNormal, UpperRateBoundLogNormal: rate bounds for ln / sln vol input
\item LowerRateNormal, UpperRateNormal: rate bounds for normal vol input
\item VegaRatio: vega ratio for policy
\item PriceThreshold: price threshold for policy
\item BsStdDev: std devs for BsStdDev
\item SensitivityTemplate [optional]: the sensitivity template to use 
\end{itemize}

\begin{longlisting}
\begin{minted}[fontsize=\footnotesize]{xml}
<Prooduct type="CMS">
    <Model>LinearTSR</Model>
    <ModelParameters/>
    <Engine>LinearTSRPricer</Engine>
    <EngineParameters>
        <Parameter name="MeanReversion">0.0</Parameter>
        <Parameter name="Policy">RateBound</Parameter>
        <Parameter name="LowerRateBoundLogNormal">0.0001</Parameter>
        <Parameter name="UpperRateBoundLogNormal">2.0000</Parameter>
        <Parameter name="LowerRateBoundNormal">-2.0000</Parameter>
        <Parameter name="UpperRateBoundNormal">2.0000</Parameter>
        <Parameter name="VegaRatio">0.01</Parameter>
        <Parameter name="PriceThreshold">0.0000001</Parameter>
        <Parameter name="BsStdDev">3.0</Parameter>
        <Parameter name="SensitivityTemplate">IR_Semianalytical</Parameter>
    </EngineParameters>
</Product>
\end{minted}
\caption{Configuration for Product CMS, Model LinearTSR, Engine LinearTSRPricer}
\label{lst:peconfig_CMS_LinearTSR_LinearTSRPricer}
\end{longlisting}

Hagan/Analytic, Hagan/Numerical build AnalyticHaganPricer, NumericHaganPricer (TODO add parameters and sample config).

%--------------------------------------------------------
\subsubsection{Product Type: SyntheticCDO}
%--------------------------------------------------------

Used by trade type: SyntheticCDO

Available Model/Engine pairs: GaussCopula/Bucketing

Engine description:

GaussCopula/Bucketing builds a IndexCdsTrancheEngine. A sample configuration is shown in listing
\ref{lst:peconfig_SyntheticCDO_GaussCopula_Bucketing}

The parameters have the following meaning:

\begin{itemize}
\item min, max: min max std dev for Gauss copula
\item steps: integration steps
\item useStochasticRecovery: whether to use deterministic (false) or stochastic (true) recovery model
\item recoveryRateGrid: Constant (flat market recovery rate R), Markit2020 (3-pillar recovery grid $[0.1, R, 2R-0.1]$
\item reocveryRateProbabilities: recovery rate probabilities for the recovery rate grid
\item buckets: number of buckets in Hull-White bucketing
\item SensitivityDecomposition: Underlying, NotionalWeighted, LossWeighted, DeltaWeighted
\item useLossDistWhenJustified: whether to use QuantLib::LossDist for determinisitc recovery instead of HulWhiteBucketing
\item homogeneousPoolWhenJustified: whether to use homogeneous pool if possible, applies to QuantLib::LossDist
\item useQuadrature: whether to use quadrature
\item calibrateConstituentCurves: whether to calibrate constituent curves to index level
\item calibrationIndexTerms: terms for constituent curve calibration
\item SensitivityTemplate [optional]: the sensitivity template to use 
\end{itemize}

\begin{longlisting}
\begin{minted}[fontsize=\footnotesize]{xml}
 <Product type="SyntheticCDO">
    <Model>GaussCopula</Model>
    <ModelParameters>
        <Parameter name="min">-5.0</Parameter>
        <Parameter name="max">5.0</Parameter>
        <Parameter name="steps">64</Parameter>
        <Parameter name="useStochasticRecovery">Y</Parameter>
        <Parameter name="recoveryRateGrid">Markit2020</Parameter>
        <Parameter name="recoveryRateProbabilities">0.35,0.3,0.35</Parameter>
    </ModelParameters>
    <Engine>Bucketing</Engine>
    <EngineParameters>
        <Parameter name="buckets">124</Parameter>
        <Parameter name="SensitivityDecomposition">DeltaWeighted</Parameter>
        <Parameter name="useLossDistWhenJustified">N</Parameter>
        <Parameter name="homogeneousPoolWhenJustified">N</Parameter>
        <Parameter name="calibrateConstituentCurves">N</Parameter>
        <Parameter name="calibrationIndexTerms">3Y,5Y</Parameter>
        <Parameter name="SensitivityTemplate">CR_Semianalytical</Parameter>
    </EngineParameters>
</Product>
\end{minted}
\caption{Configuration for Product SyntheticCDO, Model GaussCopula, Engine Bucketing}
\label{lst:peconfig_SyntheticCDO_GaussCopula_Bucketing}
\end{longlisting}

%--------------------------------------------------------
\subsubsection{Product Type: CBO}
%--------------------------------------------------------

Used by trade type: CBO

Available Model/Engine pairs: OneFactorCopula/MonteCarloCBOEngine

Engine description:

OneFactorCopula/MonteCarloCBOEngine builds a MonteCarloCBOEngine. A sample configuration is shown in listing
\ref{lst:peconfig_CBO_OneFactorCopula_MonteCarloCBOEngine}

The parameters have the following meaning:

\begin{itemize}
\item Samples: number of MC samples
\item Bins: number of bins used for discretization
\item Seed: seed for MC simulation
\item LossDistributionPeriods:
\item Correlation: correlation to use
\item SensitivityTemplate [optional]: the sensitivity template to use   
\end{itemize}

\begin{longlisting}
\begin{minted}[fontsize=\footnotesize]{xml}
<Product type="CBO">
    <Model>OneFactorCopula</Model>
    <ModelParameters/>
    <Engine>MonteCarloCBOEngine</Engine>
    <EngineParameters>
        <Parameter name="Samples">1000</Parameter>
        <Parameter name="Bins">20</Parameter>
        <Parameter name="Seed">42</Parameter>
        <Parameter name="LossDistributionPeriods"/>
        <Parameter name="Correlation">0.2</Parameter>
        <Parameter name="SensitivityTemplate">CR_MC</Parameter>
    </EngineParameters>
</Product>
\end{minted}
\caption{Configuration for Product CBO, Model OneFactorCopula, Engine MonteCarloCBOEngine}
\label{lst:peconfig_CBO_OneFactorCopula_MonteCarloCBOEngine}
\end{longlisting}

%--------------------------------------------------------
\subsubsection{Product Type: CMSSpread}
%--------------------------------------------------------

Used by trade type: any trade referencing a CMSSpread leg

Available Model/Engine pairs:

\begin{itemize}
  \item BrigoMercurio/Analytic
\end{itemize}

Engine description:

BrigoMercurio/Analytic builds a LognormalCmsSpreadPricer (following Brigo, Mercurio, Interest Rate Models - Theory and
Practice, section 13.16.2). A sample configuration is shown in listing
\ref{lst:peconfig_CMSSpread_BrigoMercurio_Analytic}.

The parameters have the following meaning:

\begin{itemize}
\item IntegrationPoints: Number of points for Gauss-Hermite numerical integration
\item SensitivityTemplate [optional]: the sensitivity template to use 
\end{itemize}

\begin{longlisting}
\begin{minted}[fontsize=\footnotesize]{xml}
<Product type="CMSSpread">
    <Model>BrigoMercurio</Model>
    <ModelParameters/>
    <Engine>Analytic</Engine>
    <EngineParameters>
        <Parameter name="IntegrationPoints">16</Parameter>
        <Parameter name="SensitivityTemplate">IR_Semianalytical</Parameter>
    </EngineParameters>
</Product>\caption{Configuration for Product CMSSpread, Model BrigoMercurio, Engine Analytic}
\end{minted}
\caption{Configuration for Product CMSSpread, Model BrigoMercurio, Engine Analytic}
\label{lst:peconfig_CMSSpread_BrigoMercurio_Analytic}
\end{longlisting}

%--------------------------------------------------------
\subsubsection{Product Type: DurationAdjustedCMS}
%--------------------------------------------------------

Used by trade type: any trade referencing a DurationAdjustedCMS leg

Available Model/Engine pairs:

\begin{itemize}
  \item LinearTSR/LinearTSRPricer
\end{itemize}

Engine description:

LinearTSR/LinearTSRPricer builds a DurationAdjustedCmsCouponTsrPricer with LinearAnnuityMapping. A sample configuration
is shown in listing \ref{lst:peconfig_DurationAdjustedCMS_LinearTSR_LinearTSRPricer}.

The parameters have the following meaning:

\begin{itemize}
\item MeanReversion: the mean reversion for the model
\item LowerRateBoundLogNormal, UpperRateBoundLogNormal: rate bounds for ln / sln vol input
\item LowerRateNormal, UpperRateNormal: rate bounds for normal vol input
\item SensitivityTemplate [optional]: the sensitivity template to use 
\end{itemize}

\begin{longlisting}
\begin{minted}[fontsize=\footnotesize]{xml}
<Product type="DurationAdjustedCMS">
    <Model>LinearTSR</Model>
    <ModelParameters/>
    <Engine>LinearTSRPricer</Engine>
    <EngineParameters>
        <Parameter name="MeanReversion">0.0</Parameter>
        <Parameter name="LowerRateBoundLogNormal">0.0001</Parameter>
        <Parameter name="UpperRateBoundLogNormal">2.0000</Parameter>
        <Parameter name="LowerRateBoundNormal">-2.0000</Parameter>
        <Parameter name="UpperRateBoundNormal">2.0000</Parameter>
        <Parameter name="SensitivityTemplate">IR_Semianalytical</Parameter>
    </EngineParameters>
</Product>
\end{minted}
\caption{Configuration for Product DurationAdjustedCMS, Model LinearTSR, Engine LinearTSRPricer}
\label{lst:peconfig_DurationAdjustedCMS_LinearTSR_LinearTSRPricer}
\end{longlisting}

%--------------------------------------------------------
\subsubsection{Product Type: CommodityAsianOptionArithmeticPrice}
%--------------------------------------------------------

Used by trade type: CommodityAsianOption if payoffType2 is Arithmetric and payoffType is Asian

Available Model/Engine pairs:

\begin{itemize}
  \item BlackScholesMerton/MCDiscreteArithmeticAPEngine
  \item BlackScholesMerton/TurnbullWakemanAsianEngine
  \item ScriptedTrade/ScriptedTrade
\end{itemize}

Engine description:

BlackScholesMerton/MCDiscreteArithmeticAPEngine builds a MCDiscreteArithmeticAPEngine using Sobol sequences. A sample
configuration is shown in listing
\ref{lst:peconfig_CommodityAsianOptionArithmeticPrice_BlackScholesMerton_MCDiscreteArithmeticAPEngine}.

The parameters have the following meaning:

\begin{itemize}
\item BrownianBridge: whether to use Brownian Bridge for MC simulation
\item AntitheticVariate: whether to use antithetic variates for MC simulation
\item ControlVariate: whether to use control variate for MC simulation
\item RequiredSamples: minimum number of samples for MC simulation, if 0 (not specified) RequiredTolerance must be given
\item RequiredTolernace: max tolerance for MC error, if 0 (not specified) RequiredSamples must be given
\item MaxSamples: max number of samples for MC simulation
\item Seed: seed for random number generator
\item SensitivityTemplate [optional]: the sensitivity template to use 
\end{itemize}

\begin{longlisting}
\begin{minted}[fontsize=\footnotesize]{xml}
<Product type="CommodityAsianOptionArithmeticPrice">
    <Model>BlackScholesMerton</Model>
    <ModelParameters/>
    <Engine>MCDiscreteArithmeticAPEngine</Engine>
    <EngineParameters>
        <Parameter name="BrownianBridge">true</Parameter>    
        <Parameter name="AntitheticVariate">true</Parameter>    
        <Parameter name="ControlVariate">true</Parameter>    
        <Parameter name="RequiredSamples">10000</Parameter>    
        <Parameter name="RequiredTolerance">0</Parameter>    
        <Parameter name="MaxSamples">0</Parameter>    
        <Parameter name="Seed">42</Parameter>    
        <Parameter name="SensitivityTemplate">COMM_MC</Parameter>
    </EngineParameters>
</Product>
\end{minted}
\caption{Configuration for Product CommodityAsianOptionArithmeticPrice, Model BlackScholesMerton, Engine MCDiscreteArithmeticAPEngine}
\label{lst:peconfig_CommodityAsianOptionArithmeticPrice_BlackScholesMerton_MCDiscreteArithmeticAPEngine}
\end{longlisting}

BlackScholesMerton/TurnbullWakemanAsianEngine builds a TurnbullWakemanAsianEngine. A sample configuration is shown in
listing \ref{lst:peconfig_CommodityAsianOptionArithmeticPrice_BlackScholesMerton_TurnbullWakemanAsianEngine}.

The parameters have the following meaning:

\begin{itemize}
\item SensitivityTemplate [optional]: the sensitivity template to use 
\end{itemize}

\begin{longlisting}
\begin{minted}[fontsize=\footnotesize]{xml}
<Product type="CommodityAsianOptionArithmeticPrice">
    <Model>BlackScholesMerton</Model>
    <ModelParameters/>
    <Engine>TurnbullWakemanAsianEngine</Engine>
    <EngineParameters>
        <Parameter name="SensitivityTemplate">COMM_Analytical</Parameter>
    </EngineParameters>
</Product>
\end{minted}
\caption{Configuration for Product CommodityAsianOptionArithmeticPrice, Model BlackScholesMerton, Engine TurnbullWakemanAsianEngine}
\label{lst:peconfig_CommodityAsianOptionArithmeticPrice_BlackScholesMerton_TurnbullWakemanAsianEngine}
\end{longlisting}

ScriptedTrade/ScriptedTrade delegates to the scripted trade engine and the associated pricing engine configuration for
Product Type ScriptedTrade, see there for details. A sample configuration is given in listing
\ref{lst:peconfig_CommodityAsianOptionArithmeticPrice_ScriptedTrade_ScriptedTrade}.

The parameters have the following meaning:

\begin{itemize}
\item SensitivityTemplate [optional]: the sensitivity template to use 
\end{itemize}

\begin{longlisting}
\begin{minted}[fontsize=\footnotesize]{xml}
<Product type="CommodityAsianOptionArithmeticPrice">
    <Model>ScriptedTrade</Model>
    <ModelParameters/>
    <Engine>ScriptedTrade</Engine>
    <EngineParameters>
        <Parameter name="SensitivityTemplate">COMM_MC</Parameter>
    </EngineParameters>
</Product>
\end{minted}
\caption{Configuration for Product CommodityAsianOptionArithmeticPrice, Model ScriptedTrade, Engine ScriptedTrade}
\label{lst:peconfig_CommodityAsianOptionArithmeticPrice_ScriptedTrade_ScriptedTrade}
\end{longlisting}


%--------------------------------------------------------
\subsubsection{Product Type: CommodityAsianOptionArithmeticStrike}
%--------------------------------------------------------

Used by trade type: CommodityAsianOption if payoffType2 is Arithmetric and payoffType is AverageStrike

Available Model/Engine pairs:

\begin{itemize}
\item BlackScholesMerton/MCDiscreteArithmeticASEngine
\item ScriptedTrade/ScriptedTrade
\end{itemize}
  
Engine description:

BlackScholesMerton/MCDiscreteArithmeticASEngine builds a MCDiscreteArithmeticASEngine using Sobol sequences. A sample
configuration is shown in listing
\ref{lst:peconfig_CommodityAsianOptionArithmeticStrike_BlackScholesMerton_MCDiscreteArithmeticASEngine}.

The parameters have the following meaning:

\begin{itemize}
\item BrownianBridge: whether to use Brownian Bridge for MC simulation
\item AntitheticVariate: whether to use antithetic variates for MC simulation
\item ControlVariate: whether to use control variate for MC simulation
\item RequiredSamples: minimum number of samples for MC simulation, if 0 (not specified) RequiredTolerance must be given
\item RequiredTolernace: max tolerance for MC error, if 0 (not specified) RequiredSamples must be given
\item MaxSamples: max number of samples for MC simulation
\item Seed: seed for random number generator
\item SensitivityTemplate [optional]: the sensitivity template to use 
\end{itemize}

\begin{longlisting}
\begin{minted}[fontsize=\footnotesize]{xml}
<Product type="CommodityAsianOptionArithmeticStrike">
    <Model>BlackScholesMerton</Model>
    <ModelParameters/>
    <Engine>MCDiscreteArithmeticASEngine</Engine>
    <EngineParameters>
        <Parameter name="BrownianBridge">true</Parameter>    
        <Parameter name="AntitheticVariate">true</Parameter>    
        <Parameter name="ControlVariate">true</Parameter>    
        <Parameter name="RequiredSamples">10000</Parameter>    
        <Parameter name="RequiredTolerance">0</Parameter>    
        <Parameter name="MaxSamples">0</Parameter>    
        <Parameter name="Seed">42</Parameter>    
        <Parameter name="SensitivityTemplate">COMM_MC</Parameter>
    </EngineParameters>
</Product>
\end{minted}
\caption{Configuration for Product CommodityAsianOptionArithmeticStrike, Model BlackScholesMerton, Engine MCDiscreteArithmeticASEngine}
\label{lst:peconfig_CommodityAsianOptionArithmeticStrike_BlackScholesMerton_MCDiscreteArithmeticASEngine}
\end{longlisting}

ScriptedTrade/ScriptedTrade delegates to the scripted trade engine and the associated pricing engine configuration for
Product Type ScriptedTrade, see there for details. A sample configuration is given in listing
\ref{lst:peconfig_CommodityAsianOptionArithmeticStrike_ScriptedTrade_ScriptedTrade}.

The parameters have the following meaning:

\begin{itemize}
\item SensitivityTemplate [optional]: the sensitivity template to use 
\end{itemize}

\begin{longlisting}
\begin{minted}[fontsize=\footnotesize]{xml}
<Product type="CommodityAsianOptionArithmeticStrike">
    <Model>ScriptedTrade</Model>
    <ModelParameters/>
    <Engine>ScriptedTrade</Engine>
    <EngineParameters>
        <Parameter name="SensitivityTemplate">COMM_MC</Parameter>
    </EngineParameters>
</Product>
\end{minted}
\caption{Configuration for Product CommodityAsianOptionArithmeticStrike, Model ScriptedTrade, Engine ScriptedTrade}
\label{lst:peconfig_CommodityAsianOptionArithmeticStrike_ScriptedTrade_ScriptedTrade}
\end{longlisting}

%--------------------------------------------------------
\subsubsection{Product Type: CommodityAsianOptionGeometricPrice}
%--------------------------------------------------------

Used by trade type: CommodityAsianOption if payoffType2 is Geometric and payoffType is Asian

Available Model/Engine pairs:

\begin{itemize}
  \item BlackScholesMerton/MCDiscreteGeometricAPEngine
  \item BlackScholesMerton/AnalyticDiscreteGeometricAPEngine
  \item BlackScholesMerton/AnalyticContinuousGeometricAPEngine
  \item ScriptedTrade/ScriptedTrade
\end{itemize}

Engine description:

BlackScholesMerton/MCDiscreteGeometricAPEngine builds a MCDiscreteGeometricAPEngine using Sobol sequences. A sample
configuration is shown in listing
\ref{lst:peconfig_CommodityAsianOptionGeometricPrice_BlackScholesMerton_MCDiscreteGeomtetricAPEngine}.

The parameters have the following meaning:

\begin{itemize}
\item BrownianBridge: whether to use Brownian Bridge for MC simulation
\item AntitheticVariate: whether to use antithetic variates for MC simulation
\item ControlVariate: whether to use control variate for MC simulation
\item RequiredSamples: minimum number of samples for MC simulation, if 0 (not specified) RequiredTolerance must be given
\item RequiredTolernace: max tolerance for MC error, if 0 (not specified) RequiredSamples must be given
\item MaxSamples: max number of samples for MC simulation
\item Seed: seed for random number generator
\item SensitivityTemplate [optional]: the sensitivity template to use 
\end{itemize}

\begin{longlisting}
\begin{minted}[fontsize=\footnotesize]{xml}
<Product type="CommodityAsianOptionGeometricPrice">
    <Model>BlackScholesMerton</Model>
    <ModelParameters/>
    <Engine>MCDiscreteGeometricAPEngine</Engine>
    <EngineParameters>
        <Parameter name="BrownianBridge">true</Parameter>    
        <Parameter name="AntitheticVariate">true</Parameter>    
        <Parameter name="ControlVariate">true</Parameter>    
        <Parameter name="RequiredSamples">10000</Parameter>    
        <Parameter name="RequiredTolerance">0</Parameter>    
        <Parameter name="MaxSamples">0</Parameter>    
        <Parameter name="Seed">42</Parameter>    
        <Parameter name="SensitivityTemplate">COMM_MC</Parameter>
    </EngineParameters>
</Product>
\end{minted}
\caption{Configuration for Product CommodityAsianOptionGeometricPrice, Model BlackScholesMerton, Engine MCDiscreteGeometricAPEngine}
\label{lst:peconfig_CommodityAsianOptionGeometricPrice_BlackScholesMerton_MCDiscreteGeomtetricAPEngine}
\end{longlisting}

BlackScholesMerton/AnalyticDiscreteGeometricAPEngine builds a AnalyticDiscreteGeometricAveragePriceAsianEngine. A sample
configuration is shown in listing
\ref{lst:peconfig_CommodityAsianOptionGeometricPrice_BlackScholesMerton_AnalyticDiscreteGeomtetricAPEngine}.

The parameters have the following meaning:

\begin{itemize}
\item SensitivityTemplate [optional]: the sensitivity template to use 
\end{itemize}

\begin{longlisting}
\begin{minted}[fontsize=\footnotesize]{xml}
<Product type="CommodityAsianOptionGeometricPrice">
    <Model>BlackScholesMerton</Model>
    <ModelParameters/>
    <Engine>AnalyticDiscreteGeometricAPEngine</Engine>
    <EngineParameters>
        <Parameter name="SensitivityTemplate">COMM_Analytical</Parameter>
    </EngineParameters>
</Product>
\end{minted}
\caption{Configuration for Product CommodityAsianOptionGeometricPrice, Model BlackScholesMerton, Engine AnalyticDiscreteGeomtetricAPEngine}
\label{lst:peconfig_CommodityAsianOptionGeometricPrice_BlackScholesMerton_AnalyticDiscreteGeomtetricAPEngine}
\end{longlisting}

BlackScholesMerton/AnalyticContinuousGeometricAPEngine builds a AnalyticContinuousGeometricAveragePriceAsianEngine. A sample
configuration is shown in listing
\ref{lst:peconfig_CommodityAsianOptionGeometricPrice_BlackScholesMerton_AnalyticContinuousGeomtetricAPEngine}.

The parameters have the following meaning:

\begin{itemize}
\item SensitivityTemplate [optional]: the sensitivity template to use 
\end{itemize}

\begin{longlisting}
\begin{minted}[fontsize=\footnotesize]{xml}
<Product type="CommodityAsianOptionGeometricPrice">
    <Model>BlackScholesMerton</Model>
    <ModelParameters/>
    <Engine>AnalyticContinuousGeometricAPEngine</Engine>
    <EngineParameters>
        <Parameter name="SensitivityTemplate">COMM_Analytical</Parameter>
    </EngineParameters>
</Product>
\end{minted}
\caption{Configuration for Product CommodityAsianOptionGeometricPrice, Model BlackScholesMerton, Engine AnalyticContinuousGeomtetricAPEngine}
\label{lst:peconfig_CommodityAsianOptionGeometricPrice_BlackScholesMerton_AnalyticContinuousGeomtetricAPEngine}
\end{longlisting}

ScriptedTrade/ScriptedTrade delegates to the scripted trade engine and the associated pricing engine configuration for
Product Type ScriptedTrade, see there for details. A sample configuration is given in listing
\ref{lst:peconfig_CommodityAsianOptionGeometricPrice_ScriptedTrade_ScriptedTrade}.

The parameters have the following meaning:

\begin{itemize}
\item SensitivityTemplate [optional]: the sensitivity template to use 
\end{itemize}

\begin{longlisting}
\begin{minted}[fontsize=\footnotesize]{xml}
<Product type="CommodityAsianOptionGeometricPrice">
    <Model>ScriptedTrade</Model>
    <ModelParameters/>
    <Engine>ScriptedTrade</Engine>
    <EngineParameters>
        <Parameter name="SensitivityTemplate">COMM_MC</Parameter>
    </EngineParameters>
</Product>
\end{minted}
\caption{Configuration for Product CommodityAsianOptionGeometricPrice, Model ScriptedTrade, Engine ScriptedTrade}
\label{lst:peconfig_CommodityAsianOptionGeometricPrice_ScriptedTrade_ScriptedTrade}
\end{longlisting}

%--------------------------------------------------------
\subsubsection{Product Type: CommodityAsianOptionGeometricStrike}
%--------------------------------------------------------

Used by trade type: CommodityAsianOption if payoffType2 is Geometric and payoffType is AverageStrike

Available Model/Engine pairs:

\begin{itemize}
  \item BlackScholesMerton/MCDiscreteGeometricASEngine
  \item BlackScholesMerton/AnalyticDiscreteGeometricASEngine
  \item ScriptedTrade/ScriptedTrade
\end{itemize}

Engine description:

BlackScholesMerton/MCDiscreteGeometricASEngine builds a MCDiscreteGeometricASEngine using Sobol sequences. A sample
configuration is shown in listing
\ref{lst:peconfig_CommodityAsianOptionGeometricStrike_BlackScholesMerton_MCDiscreteGeomtetricASEngine}.

The parameters have the following meaning:

\begin{itemize}
\item BrownianBridge: whether to use Brownian Bridge for MC simulation
\item AntitheticVariate: whether to use antithetic variates for MC simulation
\item ControlVariate: whether to use control variate for MC simulation
\item RequiredSamples: minimum number of samples for MC simulation, if 0 (not specified) RequiredTolerance must be given
\item RequiredTolernace: max tolerance for MC error, if 0 (not specified) RequiredSamples must be given
\item MaxSamples: max number of samples for MC simulation
\item Seed: seed for random number generator
\item SensitivityTemplate [optional]: the sensitivity template to use 
\end{itemize}

\begin{longlisting}
\begin{minted}[fontsize=\footnotesize]{xml}
<Product type="CommodityAsianOptionGeometricStrike">
    <Model>BlackScholesMerton</Model>
    <ModelParameters/>
    <Engine>MCDiscreteGeometricASEngine</Engine>
    <EngineParameters>
        <Parameter name="BrownianBridge">true</Parameter>    
        <Parameter name="AntitheticVariate">true</Parameter>    
        <Parameter name="ControlVariate">true</Parameter>    
        <Parameter name="RequiredSamples">10000</Parameter>    
        <Parameter name="RequiredTolerance">0</Parameter>    
        <Parameter name="MaxSamples">0</Parameter>    
        <Parameter name="Seed">42</Parameter>    
        <Parameter name="SensitivityTemplate">COMM_MC</Parameter>
    </EngineParameters>
</Product>
\end{minted}
\caption{Configuration for Product CommodityAsianOptionGeometricStrike, Model BlackScholesMerton, Engine MCDiscreteGeometricASEngine}
\label{lst:peconfig_CommodityAsianOptionGeometricStrike_BlackScholesMerton_MCDiscreteGeomtetricASEngine}
\end{longlisting}

BlackScholesMerton/AnalyticDiscreteGeometricASEngine builds a AnalyticDiscreteGeometricAverageStrikeAsianEngine. A sample
configuration is shown in listing
\ref{lst:peconfig_CommodityAsianOptionGeometricStrike_BlackScholesMerton_AnalyticDiscreteGeomtetricASEngine}.

The parameters have the following meaning:

\begin{itemize}
\item SensitivityTemplate [optional]: the sensitivity template to use 
\end{itemize}

\begin{longlisting}
\begin{minted}[fontsize=\footnotesize]{xml}
<Product type="CommodityAsianOptionGeometricStrike">
    <Model>BlackScholesMerton</Model>
    <ModelParameters/>
    <Engine>AnalyticDiscreteGeometricASEngine</Engine>
    <EngineParameters>
        <Parameter name="SensitivityTemplate">COMM_Analytical</Parameter>
    </EngineParameters>
</Product>
\end{minted}
\caption{Configuration for Product CommodityAsianOptionGeometricStrike, Model BlackScholesMerton, Engine AnalyticDiscreteGeomtetricASEngine}
\label{lst:peconfig_CommodityAsianOptionGeometricStrike_BlackScholesMerton_AnalyticDiscreteGeomtetricASEngine}
\end{longlisting}

ScriptedTrade/ScriptedTrade delegates to the scripted trade engine and the associated pricing engine configuration for
Product Type ScriptedTrade, see there for details. A sample configuration is given in listing
\ref{lst:peconfig_CommodityAsianOptionGeometricStrike_ScriptedTrade_ScriptedTrade}.

The parameters have the following meaning:

\begin{itemize}
\item SensitivityTemplate [optional]: the sensitivity template to use 
\end{itemize}

\begin{longlisting}
\begin{minted}[fontsize=\footnotesize]{xml}
<Product type="CommodityAsianOptionGeometricStrike">
    <Model>ScriptedTrade</Model>
    <ModelParameters/>
    <Engine>ScriptedTrade</Engine>
    <EngineParameters>
        <Parameter name="SensitivityTemplate">COMM_MC</Parameter>
    </EngineParameters>
</Product>
\end{minted}
\caption{Configuration for Product CommodityAsianOptionGeometricStrike, Model ScriptedTrade, Engine ScriptedTrade}
\label{lst:peconfig_CommodityAsianOptionGeometricStrike_ScriptedTrade_ScriptedTrade}
\end{longlisting}

%--------------------------------------------------------
\subsubsection{Product Type: CommoditySpreadOption}
%--------------------------------------------------------

Used by trade type: CommoditySpreadOption

Available Model/Engine pairs: BlackScholes/CommoditySpreadOptionEngine

Engine description:

BlackScholes/CommoditySpreadOptionEngine builds a CommoditySpreadOptionAnalyticalEngine, which uses the Kirk
approximation described in Iain J. Clark, Commodity Option Pricing, Section 2.9. The correlation between two commodities
resp. the intra-asset correlation between two future contracts is parametrized as

$$\rho(s, t) = e^{-\beta |s-t|}$$

where $s$ and $t$ are times to futures expiry. A sample configuration is shown in listing
\ref{lst:peconfig_CommoditySpreadOption_BlackScholes_CommoditySpreadOptionEngine}.

The parameters have the following meaning:

\begin{itemize}
\item beta: the parameter ``beta'' in the correlation parametrization
\item SensitivityTemplate [optional]: the sensitivity template to use 
\end{itemize}

\begin{longlisting}
\begin{minted}[fontsize=\footnotesize]{xml}
<Product type="CommoditySpreadOption">
    <Model>BlackScholes</Model>
    <ModelParameters/>
    <Engine>CommoditySpreadOptionEngine</Engine>
    <EngineParameters>
        <Parameter name="beta">2.05</Parameter>
        <Parameter name="SensitivityTemplate">COMM_Analytical</Parameter>
    </EngineParameters>
    </Product>
\end{minted}
\caption{Configuration for Product CommoditySpreadOption, Model BlackScholes, Engine CommoditySpreadOptionEngine}
\label{lst:peconfig_CommoditySpreadOption_BlackScholes_CommoditySpreadOptionEngine}
\end{longlisting}

%--------------------------------------------------------
\subsubsection{Product Type: CommodityOption}
%--------------------------------------------------------

Used by trade type: CommodityOption

Available Model/Engine pairs: BlackScholes/AnalyticEuropeanEngine

Engine description:

BlackScholes/AnalyticEuropeanEngine builds a AnalyticEuropeanEngine. A sample configuration is shown in listing
\ref{lst:peconfig_CommodityOption_BlackScholes_AnalyticEuropeanEngine}

The parameters have the following meaning:

\begin{itemize}
\item SensitivityTemplate [optional]: the sensitivity template to use 
\end{itemize}

\begin{longlisting}
\begin{minted}[fontsize=\footnotesize]{xml}
<Product type="CommodityOption">
    <Model>BlackScholes</Model>
    <ModelParameters/>
    <Engine>AnalyticEuropeanEngine</Engine>
    <EngineParameters>
        <Parameter name="SensitivityTemplate">COMM_Analytical</Parameter>
    </EngineParameters>
</Product>
\end{minted}
\caption{Configuration for Product CommodityOption, Model BlackScholes, Engine AnalyticEuropeanEngine}
\label{lst:peconfig_CommodityOption_BlackScholes_AnalyticEuropeanEngine}
\end{longlisting}

%--------------------------------------------------------
\subsubsection{Product Type: CommodityOptionForward}
%--------------------------------------------------------

Used by trade type: CommodityOption if a future is referenced

Available Model/Engine pairs: BlackScholes/AnalyticEuropeanForwardEngine

Engine description:

BlackScholes/AnalyticEuropeanForwardEngine builds a AnalyticEuropeanForwardEngine. A sample configuration is shown in listing
\ref{lst:peconfig_CommodityOptionForward_BlackScholes_AnalyticForwardEuropeanEngine}

The parameters have the following meaning:

\begin{itemize}
\item SensitivityTemplate [optional]: the sensitivity template to use 
\end{itemize}

\begin{longlisting}
\begin{minted}[fontsize=\footnotesize]{xml}
<Product type="CommodityOptionForward">
    <Model>BlackScholes</Model>
    <ModelParameters/>
    <Engine>AnalyticForwardEuropeanEngine</Engine>
    <EngineParameters>
        <Parameter name="SensitivityTemplate">COMM_Analytical</Parameter>
    </EngineParameters>
</Product>
\end{minted}
\caption{Configuration for Product CommodityOptionForward, Model BlackScholes, Engine AnalyticForwardEuropeanEngine}
\label{lst:peconfig_CommodityOptionForward_BlackScholes_AnalyticForwardEuropeanEngine}
\end{longlisting}

%--------------------------------------------------------
\subsubsection{Product Type: CommodityOptionEuropeanCS}
%--------------------------------------------------------

Used by trade type: CommodityOption if cash settled

Available Model/Engine pairs: BlackScholes/AnalyticCashSettledEuropeanEngine

Engine description:

BlackScholes/AnalyticCashSettledEuropeanEngine builds a AnalyticCashSettledEuropeanEngine. A sample configuration is shown in listing
\ref{lst:peconfig_CommodityOptionEuropeanCS_BlackScholes_AnalyticCashSettledEuropeanEngine}

The parameters have the following meaning:

\begin{itemize}
\item SensitivityTemplate [optional]: the sensitivity template to use 
\end{itemize}

\begin{longlisting}
\begin{minted}[fontsize=\footnotesize]{xml}
<Product type="CommodityOptionEuropeanCS">
    <Model>BlackScholes</Model>
    <ModelParameters/>
    <Engine>AnalyticCashSettledEuropeanEngine</Engine>
    <EngineParameters>
        <Parameter name="SensitivityTemplate">COMM_Analytical</Parameter>
    </EngineParameters>
</Product>
\end{minted}
\caption{Configuration for Product CommodityOptionEuropeanCS, Model BlackScholes, Engine AnalyticCashSettledEuropeanEngine}
\label{lst:peconfig_CommodityOptionEuropeanCS_BlackScholes_AnalyticCashSettledEuropeanEngine}
\end{longlisting}

%--------------------------------------------------------
\subsubsection{Product Type: CommodityOptionAmerican}
%--------------------------------------------------------

Used by trade type: CommodityOption if exercise stly is american

Available Model/Engine pairs:

\begin{itemize}
\item BlackScholes/FdBlackScholesVanillaEngine
\item BlackScholes/BaroneAdesiWhaleyApproximationEngine
\end{itemize}

Engine description:

BlackScholes/FdBlackScholesVanillaEngine builds a FdBlackScholesVanillaEngine. A sample configuration is shown in listing
\ref{lst:peconfig_CommodityOptionAmerican_BlackScholes_FdBlackScholesVanillaEngine}

The parameters have the following meaning:

\begin{itemize}
\item Scheme: The finite difference scheme to use
\item TimeStepsPerYear: Time grid specification
\item XGrid: State grid specification
\item DampingSteps: Number of damping steps taken by FD solver
\item EnforceMonotoneVariance [optional]: If true variance is modified to be monotone if needed, defaults to true
\item TimeGridMinimumSize [optional]: Minimum number in resulting time grid, defaults to $1$ if not given
\item SensitivityTemplate [optional]: the sensitivity template to use 
\end{itemize}

\begin{longlisting}
\begin{minted}[fontsize=\footnotesize]{xml}
<Product type="CommodityOptionAmerican">
    <Model>BlackScholes</Model>
    <ModelParameters/>
    <Engine>FdBlackScholesVanillaEngine</Engine>
    <EngineParameters>
        <Parameter name="Scheme">Douglas</Parameter>
        <Parameter name="TimeGridPerYear">100</Parameter>
        <Parameter name="XGrid">100</Parameter>
        <Parameter name="DampingSteps">0</Parameter>
        <Parameter name="EnforceMonotoneVariance">true</Parameter>
        <Parameter name="SensitivityTemplate">COMM_FD</Parameter>
    </EngineParameters>
</Product>
\end{minted}
\caption{Configuration for Product CommodityOptionAmerican, Model BlackScholes, Engine FdBlackScholesVanillaEngine}
\label{lst:peconfig_CommodityOptionAmerican_BlackScholes_FdBlackScholesVanillaEngine}
\end{longlisting}

BlackScholes/BaroneAdesiWhaleyApproximationEngine builds a BaroneAdesiWhaleyApproximationEngine. A sample configuration is shown in listing
\ref{lst:peconfig_CommodityOptionAmerican_BlackScholes_BaroneAdesiWhaleyApproximationEngine}

The parameters have the following meaning:

\begin{itemize}
\item SensitivityTemplate [optional]: the sensitivity template to use 
\end{itemize}

\begin{longlisting}
\begin{minted}[fontsize=\footnotesize]{xml}
<Product type="CommodityOptionAmerican">
    <Model>BlackScholes</Model>
    <ModelParameters/>
    <Engine>BaroneAdesiWhaleyApproximationEngine</Engine>
    <EngineParameters>
        <Parameter name="SensitivityTemplate">COMM_Analytical</Parameter>
    </EngineParameters>
</Product>
\end{minted}
\caption{Configuration for Product CommodityOptionAmerican, Model BlackScholes, Engine BaroneAdesiWhaleyApproximationEngine}
\label{lst:peconfig_CommodityOptionAmerican_BlackScholes_BaroneAdesiWhaleyApproximationEngine}
\end{longlisting}

%--------------------------------------------------------
\subsubsection{Product Type: CommoditySwap}
%--------------------------------------------------------

Used by trade type: CommoditySwap

Available Model/Engine pairs: DiscountedCashflows/CommoditySwapEngine

Engine description:

DiscountedCashflows/CommoditySwapEngine builds a DiscountingSwapEngine. A sample configuration is shown in listing
\ref{lst:peconfig_CommoditySwap_DiscountedCashflows_CommoditySwapEngine}

The parameters have the following meaning:

\begin{itemize}
\item 
\item SensitivityTemplate [optional]: the sensitivity template to use 
\end{itemize}

\begin{longlisting}
\begin{minted}[fontsize=\footnotesize]{xml}
<Product type="CommoditySwap">
    <Model>DiscountedCashflows</Model>
    <ModelParameters/>
    <Engine>CommoditySwapEngine</Engine>
    <EngineParameters>
        <Parameter name="SensitivityTemplate">COMM_Analytical</Parameter>
    </EngineParameters>
</Product>
\end{minted}
\caption{Configuration for Product CommoditySwap, Model DiscountedCashflows, Engine CommoditySwapEngine}
\label{lst:peconfig_CommoditySwap_DiscountedCashflows_CommoditySwapEngine}
\end{longlisting}

%--------------------------------------------------------
\subsubsection{Product Type: CommoditySwaption}
%--------------------------------------------------------

Used by trade type: CommoditySwaption

Available Model/Engine pairs:

\begin{itemize}
\item Black/AnalyticalApproximation
\item Black/MonteCarlo
\end{itemize}

Engine description:

Black/AnalyticalApproximation builds a CommoditySwaptionEngine using moment matching. The correlation between two future
contracts is parametrized as

$$\rho(s, t) = e^{-\beta |s-t|}$$

where $s$ and $t$ are times to futures expiry. A sample configuration is shown in listing
\ref{lst:peconfig_CommoditySwaption_Black_AnalyticalApproximation}

The parameters have the following meaning:

\begin{itemize}
\item beta: the parameter ``beta'' in the correlation parametrization
\item SensitivityTemplate [optional]: the sensitivity template to use 
\end{itemize}

\begin{longlisting}
\begin{minted}[fontsize=\footnotesize]{xml}
<Product type="CommoditySwaption">
    <Model>Black</Model>
    <ModelParameters/>
    <Engine>AnalyticalApproximation</Engine>
    <EngineParameters>
        <Parameter name="beta">2.05</Parameter>
        <Parameter name="SensitivityTemplate">COMM_Analytical</Parameter>
    </EngineParameters>
</Product>
\end{minted}
\caption{Configuration for Product CommoditySwaption, Model Black, Engine AnalyticalApproximation}
\label{lst:peconfig_CommoditySwaption_Black_AnalyticalApproximation}
\end{longlisting}

Black/MonteCarlo builds a CommoditySwaptionMonteCarloEngine using Sobol sequences. The same correlation parametrization
as above is used. A sample configuration is shown in listing
\ref{lst:peconfig_CommoditySwaption_Black_MonteCarlo}

The parameters have the following meaning:

\begin{itemize}
\item beta: the parameter ``beta'' in the correlation parametrization
\item samples: the number of MC samples to use
\item seed: the seed to use
\item SensitivityTemplate [optional]: the sensitivity template to use 
\end{itemize}

\begin{longlisting}
\begin{minted}[fontsize=\footnotesize]{xml}
<Product type="CommoditySwaption">
    <Model>Black</Model>
    <ModelParameters/>
    <Engine>MonteCarlo</Engine>
    <EngineParameters>
        <Parameter name="beta">2.05</Parameter>
        <Parameter name="samples">10000</Parameter>
        <Parameter name="seed">42</Parameter>    
        <Parameter name="SensitivityTemplate">COMM_MC</Parameter>
    </EngineParameters>
</Product>
\end{minted}
\caption{Configuration for Product CommoditySwaption, Model Black, Engine MonteCarlo}
\label{lst:peconfig_CommoditySwaption_Black_MonteCarlo}
\end{longlisting}
    
%--------------------------------------------------------
\subsubsection{Product Type: EquityAsianOptionArithmeticPrice}
%--------------------------------------------------------

Used by trade type: EquityAsianOption if payoffType2 is Arithmetric and payoffType is Asian

Available Model/Engine pairs:

\begin{itemize}
  \item BlackScholesMerton/MCDiscreteArithmeticAPEngine
  \item BlackScholesMerton/TurnbullWakemanAsianEngine
  \item ScriptedTrade/ScriptedTrade
\end{itemize}

Engine description:

BlackScholesMerton/MCDiscreteArithmeticAPEngine builds a MCDiscreteArithmeticAPEngine using Sobol sequences. A sample
configuration is shown in listing
\ref{lst:peconfig_EquityAsianOptionArithmeticPrice_BlackScholesMerton_MCDiscreteArithmeticAPEngine}.

The parameters have the following meaning:

\begin{itemize}
\item BrownianBridge: whether to use Brownian Bridge for MC simulation
\item AntitheticVariate: whether to use antithetic variates for MC simulation
\item ControlVariate: whether to use control variate for MC simulation
\item RequiredSamples: minimum number of samples for MC simulation, if 0 (not specified) RequiredTolerance must be given
\item RequiredTolernace: max tolerance for MC error, if 0 (not specified) RequiredSamples must be given
\item MaxSamples: max number of samples for MC simulation
\item Seed: seed for random number generator
\item SensitivityTemplate [optional]: the sensitivity template to use 
\end{itemize}

\begin{longlisting}
\begin{minted}[fontsize=\footnotesize]{xml}
<Product type="EquityAsianOptionArithmeticPrice">
    <Model>BlackScholesMerton</Model>
    <ModelParameters/>
    <Engine>MCDiscreteArithmeticAPEngine</Engine>
    <EngineParameters>
        <Parameter name="BrownianBridge">true</Parameter>    
        <Parameter name="AntitheticVariate">true</Parameter>    
        <Parameter name="ControlVariate">true</Parameter>    
        <Parameter name="RequiredSamples">10000</Parameter>    
        <Parameter name="RequiredTolerance">0</Parameter>    
        <Parameter name="MaxSamples">0</Parameter>    
        <Parameter name="Seed">42</Parameter>    
        <Parameter name="SensitivityTemplate">EQ_MC</Parameter>
    </EngineParameters>
</Product>
\end{minted}
\caption{Configuration for Product EquityAsianOptionArithmeticPrice, Model BlackScholesMerton, Engine MCDiscreteArithmeticAPEngine}
\label{lst:peconfig_EquityAsianOptionArithmeticPrice_BlackScholesMerton_MCDiscreteArithmeticAPEngine}
\end{longlisting}

BlackScholesMerton/TurnbullWakemanAsianEngine builds a TurnbullWakemanAsianEngine. A sample configuration is shown in
listing \ref{lst:peconfig_EquityAsianOptionArithmeticPrice_BlackScholesMerton_TurnbullWakemanAsianEngine}.

The parameters have the following meaning:

\begin{itemize}
\item SensitivityTemplate [optional]: the sensitivity template to use 
\end{itemize}

\begin{longlisting}
\begin{minted}[fontsize=\footnotesize]{xml}
<Product type="EquityAsianOptionArithmeticPrice">
    <Model>BlackScholesMerton</Model>
    <ModelParameters/>
    <Engine>TurnbullWakemanAsianEngine</Engine>
    <EngineParameters>
        <Parameter name="SensitivityTemplate">EQ_Analytical</Parameter>
    </EngineParameters>
</Product>
\end{minted}
\caption{Configuration for Product EquityAsianOptionArithmeticPrice, Model BlackScholesMerton, Engine TurnbullWakemanAsianEngine}
\label{lst:peconfig_EquityAsianOptionArithmeticPrice_BlackScholesMerton_TurnbullWakemanAsianEngine}
\end{longlisting}

ScriptedTrade/ScriptedTrade delegates to the scripted trade engine and the associated pricing engine configuration for
Product Type ScriptedTrade, see there for details. A sample configuration is given in listing
\ref{lst:peconfig_EquityAsianOptionArithmeticPrice_ScriptedTrade_ScriptedTrade}.

The parameters have the following meaning:

\begin{itemize}
\item SensitivityTemplate [optional]: the sensitivity template to use 
\end{itemize}

\begin{longlisting}
\begin{minted}[fontsize=\footnotesize]{xml}
<Product type="EquityAsianOptionArithmeticPrice">
    <Model>ScriptedTrade</Model>
    <ModelParameters/>
    <Engine>ScriptedTrade</Engine>
    <EngineParameters>
        <Parameter name="SensitivityTemplate">EQ_MC</Parameter>
    </EngineParameters>
</Product>
\end{minted}
\caption{Configuration for Product EquityAsianOptionArithmeticPrice, Model ScriptedTrade, Engine ScriptedTrade}
\label{lst:peconfig_EquityAsianOptionArithmeticPrice_ScriptedTrade_ScriptedTrade}
\end{longlisting}


%--------------------------------------------------------
\subsubsection{Product Type: EquityAsianOptionArithmeticStrike}
%--------------------------------------------------------

Used by trade type: EquityAsianOption if payoffType2 is Arithmetric and payoffType is AverageStrike

Available Model/Engine pairs:

\begin{itemize}
\item BlackScholesMerton/MCDiscreteArithmeticASEngine
\item ScriptedTrade/ScriptedTrade
\end{itemize}
  
Engine description:

BlackScholesMerton/MCDiscreteArithmeticASEngine builds a MCDiscreteArithmeticASEngine using Sobol sequences. A sample
configuration is shown in listing
\ref{lst:peconfig_EquityAsianOptionArithmeticStrike_BlackScholesMerton_MCDiscreteArithmeticASEngine}.

The parameters have the following meaning:

\begin{itemize}
\item BrownianBridge: whether to use Brownian Bridge for MC simulation
\item AntitheticVariate: whether to use antithetic variates for MC simulation
\item ControlVariate: whether to use control variate for MC simulation
\item RequiredSamples: minimum number of samples for MC simulation, if 0 (not specified) RequiredTolerance must be given
\item RequiredTolernace: max tolerance for MC error, if 0 (not specified) RequiredSamples must be given
\item MaxSamples: max number of samples for MC simulation
\item Seed: seed for random number generator
\item SensitivityTemplate [optional]: the sensitivity template to use 
\end{itemize}

\begin{longlisting}
\begin{minted}[fontsize=\footnotesize]{xml}
<Product type="EquityAsianOptionArithmeticStrike">
    <Model>BlackScholesMerton</Model>
    <ModelParameters/>
    <Engine>MCDiscreteArithmeticASEngine</Engine>
    <EngineParameters>
        <Parameter name="BrownianBridge">true</Parameter>    
        <Parameter name="AntitheticVariate">true</Parameter>    
        <Parameter name="ControlVariate">true</Parameter>    
        <Parameter name="RequiredSamples">10000</Parameter>    
        <Parameter name="RequiredTolerance">0</Parameter>    
        <Parameter name="MaxSamples">0</Parameter>    
        <Parameter name="Seed">42</Parameter>    
        <Parameter name="SensitivityTemplate">EQ_MC</Parameter>
    </EngineParameters>
</Product>
\end{minted}
\caption{Configuration for Product EquityAsianOptionArithmeticStrike, Model BlackScholesMerton, Engine MCDiscreteArithmeticASEngine}
\label{lst:peconfig_EquityAsianOptionArithmeticStrike_BlackScholesMerton_MCDiscreteArithmeticASEngine}
\end{longlisting}

ScriptedTrade/ScriptedTrade delegates to the scripted trade engine and the associated pricing engine configuration for
Product Type ScriptedTrade, see there for details. A sample configuration is given in listing
\ref{lst:peconfig_EquityAsianOptionArithmeticStrike_ScriptedTrade_ScriptedTrade}.

The parameters have the following meaning:

\begin{itemize}
\item SensitivityTemplate [optional]: the sensitivity template to use 
\end{itemize}

\begin{longlisting}
\begin{minted}[fontsize=\footnotesize]{xml}
<Product type="EquityAsianOptionArithmeticStrike">
    <Model>ScriptedTrade</Model>
    <ModelParameters/>
    <Engine>ScriptedTrade</Engine>
    <EngineParameters>
        <Parameter name="SensitivityTemplate">EQ_MC</Parameter>
    </EngineParameters>
</Product>
\end{minted}
\caption{Configuration for Product EquityAsianOptionArithmeticStrike, Model ScriptedTrade, Engine ScriptedTrade}
\label{lst:peconfig_EquityAsianOptionArithmeticStrike_ScriptedTrade_ScriptedTrade}
\end{longlisting}

%--------------------------------------------------------
\subsubsection{Product Type: EquityAsianOptionGeometricPrice}
%--------------------------------------------------------

Used by trade type: EquityAsianOption if payoffType2 is Geometric and payoffType is Asian

Available Model/Engine pairs:

\begin{itemize}
  \item BlackScholesMerton/MCDiscreteGeometricAPEngine
  \item BlackScholesMerton/AnalyticDiscreteGeometricAPEngine
  \item BlackScholesMerton/AnalyticContinuousGeometricAPEngine
  \item ScriptedTrade/ScriptedTrade
\end{itemize}

Engine description:

BlackScholesMerton/MCDiscreteGeometricAPEngine builds a MCDiscreteGeometricAPEngine using Sobol sequences. A sample
configuration is shown in listing
\ref{lst:peconfig_EquityAsianOptionGeometricPrice_BlackScholesMerton_MCDiscreteGeomtetricAPEngine}.

The parameters have the following meaning:

\begin{itemize}
\item BrownianBridge: whether to use Brownian Bridge for MC simulation
\item AntitheticVariate: whether to use antithetic variates for MC simulation
\item ControlVariate: whether to use control variate for MC simulation
\item RequiredSamples: minimum number of samples for MC simulation, if 0 (not specified) RequiredTolerance must be given
\item RequiredTolernace: max tolerance for MC error, if 0 (not specified) RequiredSamples must be given
\item MaxSamples: max number of samples for MC simulation
\item Seed: seed for random number generator
\item SensitivityTemplate [optional]: the sensitivity template to use 
\end{itemize}

\begin{longlisting}
\begin{minted}[fontsize=\footnotesize]{xml}
<Product type="EquityAsianOptionGeometricPrice">
    <Model>BlackScholesMerton</Model>
    <ModelParameters/>
    <Engine>MCDiscreteGeometricAPEngine</Engine>
    <EngineParameters>
        <Parameter name="BrownianBridge">true</Parameter>    
        <Parameter name="AntitheticVariate">true</Parameter>    
        <Parameter name="ControlVariate">true</Parameter>    
        <Parameter name="RequiredSamples">10000</Parameter>    
        <Parameter name="RequiredTolerance">0</Parameter>    
        <Parameter name="MaxSamples">0</Parameter>    
        <Parameter name="Seed">42</Parameter>    
        <Parameter name="SensitivityTemplate">EQ_MC</Parameter>
    </EngineParameters>
</Product>
\end{minted}
\caption{Configuration for Product EquityAsianOptionGeometricPrice, Model BlackScholesMerton, Engine MCDiscreteGeometricAPEngine}
\label{lst:peconfig_EquityAsianOptionGeometricPrice_BlackScholesMerton_MCDiscreteGeomtetricAPEngine}
\end{longlisting}

BlackScholesMerton/AnalyticDiscreteGeometricAPEngine builds a AnalyticDiscreteGeometricAveragePriceAsianEngine. A sample
configuration is shown in listing
\ref{lst:peconfig_EquityAsianOptionGeometricPrice_BlackScholesMerton_AnalyticDiscreteGeomtetricAPEngine}.

The parameters have the following meaning:

\begin{itemize}
\item SensitivityTemplate [optional]: the sensitivity template to use 
\end{itemize}

\begin{longlisting}
\begin{minted}[fontsize=\footnotesize]{xml}
<Product type="EquityAsianOptionGeometricPrice">
    <Model>BlackScholesMerton</Model>
    <ModelParameters/>
    <Engine>AnalyticDiscreteGeometricAPEngine</Engine>
    <EngineParameters>
        <Parameter name="SensitivityTemplate">EQ_Analytical</Parameter>
    </EngineParameters>
</Product>
\end{minted}
\caption{Configuration for Product EquityAsianOptionGeometricPrice, Model BlackScholesMerton, Engine AnalyticDiscreteGeomtetricAPEngine}
\label{lst:peconfig_EquityAsianOptionGeometricPrice_BlackScholesMerton_AnalyticDiscreteGeomtetricAPEngine}
\end{longlisting}

BlackScholesMerton/AnalyticContinuousGeometricAPEngine builds a AnalyticContinuousGeometricAveragePriceAsianEngine. A sample
configuration is shown in listing
\ref{lst:peconfig_EquityAsianOptionGeometricPrice_BlackScholesMerton_AnalyticContinuousGeomtetricAPEngine}.

The parameters have the following meaning:

\begin{itemize}
\item SensitivityTemplate [optional]: the sensitivity template to use 
\end{itemize}

\begin{longlisting}
\begin{minted}[fontsize=\footnotesize]{xml}
<Product type="EquityAsianOptionGeometricPrice">
    <Model>BlackScholesMerton</Model>
    <ModelParameters/>
    <Engine>AnalyticContinuousGeometricAPEngine</Engine>
    <EngineParameters>
        <Parameter name="SensitivityTemplate">EQ_Analytical</Parameter>
    </EngineParameters>
</Product>
\end{minted}
\caption{Configuration for Product EquityAsianOptionGeometricPrice, Model BlackScholesMerton, Engine AnalyticContinuousGeomtetricAPEngine}
\label{lst:peconfig_EquityAsianOptionGeometricPrice_BlackScholesMerton_AnalyticContinuousGeomtetricAPEngine}
\end{longlisting}

ScriptedTrade/ScriptedTrade delegates to the scripted trade engine and the associated pricing engine configuration for
Product Type ScriptedTrade, see there for details. A sample configuration is given in listing
\ref{lst:peconfig_EquityAsianOptionGeometricPrice_ScriptedTrade_ScriptedTrade}.

The parameters have the following meaning:

\begin{itemize}
\item SensitivityTemplate [optional]: the sensitivity template to use 
\end{itemize}

\begin{longlisting}
\begin{minted}[fontsize=\footnotesize]{xml}
<Product type="EquityAsianOptionGeometricPrice">
    <Model>ScriptedTrade</Model>
    <ModelParameters/>
    <Engine>ScriptedTrade</Engine>
    <EngineParameters>
        <Parameter name="SensitivityTemplate">EQ_MC</Parameter>
    </EngineParameters>
</Product>
\end{minted}
\caption{Configuration for Product EquityAsianOptionGeometricPrice, Model ScriptedTrade, Engine ScriptedTrade}
\label{lst:peconfig_EquityAsianOptionGeometricPrice_ScriptedTrade_ScriptedTrade}
\end{longlisting}

%--------------------------------------------------------
\subsubsection{Product Type: EquityAsianOptionGeometricStrike}
%--------------------------------------------------------

Used by trade type: EquityAsianOption if payoffType2 is Geometric and payoffType is AverageStrike

Available Model/Engine pairs:

\begin{itemize}
  \item BlackScholesMerton/MCDiscreteGeometricASEngine
  \item BlackScholesMerton/AnalyticDiscreteGeometricASEngine
  \item ScriptedTrade/ScriptedTrade
\end{itemize}

Engine description:

BlackScholesMerton/MCDiscreteGeometricASEngine builds a MCDiscreteGeometricASEngine using Sobol sequences. A sample
configuration is shown in listing
\ref{lst:peconfig_EquityAsianOptionGeometricStrike_BlackScholesMerton_MCDiscreteGeomtetricASEngine}.

The parameters have the following meaning:

\begin{itemize}
\item BrownianBridge: whether to use Brownian Bridge for MC simulation
\item AntitheticVariate: whether to use antithetic variates for MC simulation
\item ControlVariate: whether to use control variate for MC simulation
\item RequiredSamples: minimum number of samples for MC simulation, if 0 (not specified) RequiredTolerance must be given
\item RequiredTolernace: max tolerance for MC error, if 0 (not specified) RequiredSamples must be given
\item MaxSamples: max number of samples for MC simulation
\item Seed: seed for random number generator
\item SensitivityTemplate [optional]: the sensitivity template to use 
\end{itemize}

\begin{longlisting}
\begin{minted}[fontsize=\footnotesize]{xml}
<Product type="EquityAsianOptionGeometricStrike">
    <Model>BlackScholesMerton</Model>
    <ModelParameters/>
    <Engine>MCDiscreteGeometricASEngine</Engine>
    <EngineParameters>
        <Parameter name="BrownianBridge">true</Parameter>    
        <Parameter name="AntitheticVariate">true</Parameter>    
        <Parameter name="ControlVariate">true</Parameter>    
        <Parameter name="RequiredSamples">10000</Parameter>    
        <Parameter name="RequiredTolerance">0</Parameter>    
        <Parameter name="MaxSamples">0</Parameter>    
        <Parameter name="Seed">42</Parameter>    
        <Parameter name="SensitivityTemplate">EQ_MC</Parameter>
    </EngineParameters>
</Product>
\end{minted}
\caption{Configuration for Product EquityAsianOptionGeometricStrike, Model BlackScholesMerton, Engine MCDiscreteGeometricASEngine}
\label{lst:peconfig_EquityAsianOptionGeometricStrike_BlackScholesMerton_MCDiscreteGeomtetricASEngine}
\end{longlisting}

BlackScholesMerton/AnalyticDiscreteGeometricASEngine builds a AnalyticDiscreteGeometricAverageStrikeAsianEngine. A sample
configuration is shown in listing
\ref{lst:peconfig_EquityAsianOptionGeometricStrike_BlackScholesMerton_AnalyticDiscreteGeomtetricASEngine}.

The parameters have the following meaning:

\begin{itemize}
\item SensitivityTemplate [optional]: the sensitivity template to use 
\end{itemize}

\begin{longlisting}
\begin{minted}[fontsize=\footnotesize]{xml}
<Product type="EquityAsianOptionGeometricStrike">
    <Model>BlackScholesMerton</Model>
    <ModelParameters/>
    <Engine>AnalyticDiscreteGeometricASEngine</Engine>
    <EngineParameters>
        <Parameter name="SensitivityTemplate">EQ_Analytical</Parameter>
    </EngineParameters>
</Product>
\end{minted}
\caption{Configuration for Product EquityAsianOptionGeometricStrike, Model BlackScholesMerton, Engine AnalyticDiscreteGeomtetricASEngine}
\label{lst:peconfig_EquityAsianOptionGeometricStrike_BlackScholesMerton_AnalyticDiscreteGeomtetricASEngine}
\end{longlisting}

ScriptedTrade/ScriptedTrade delegates to the scripted trade engine and the associated pricing engine configuration for
Product Type ScriptedTrade, see there for details. A sample configuration is given in listing
\ref{lst:peconfig_EquityAsianOptionGeometricStrike_ScriptedTrade_ScriptedTrade}.

The parameters have the following meaning:

\begin{itemize}
\item SensitivityTemplate [optional]: the sensitivity template to use 
\end{itemize}

\begin{longlisting}
\begin{minted}[fontsize=\footnotesize]{xml}
<Product type="EquityAsianOptionGeometricStrike">
    <Model>ScriptedTrade</Model>
    <ModelParameters/>
    <Engine>ScriptedTrade</Engine>
    <EngineParameters>
        <Parameter name="SensitivityTemplate">EQ_MC</Parameter>
    </EngineParameters>
</Product>
\end{minted}
\caption{Configuration for Product EquityAsianOptionGeometricStrike, Model ScriptedTrade, Engine ScriptedTrade}
\label{lst:peconfig_EquityAsianOptionGeometricStrike_ScriptedTrade_ScriptedTrade}
\end{longlisting}

%--------------------------------------------------------
\subsubsection{Product Type: EquityBarrierOption, FxBarrierOption}
%--------------------------------------------------------

Used by trade type: EquityBarrierOption resp. FxBarrierOption

Available Model/Engine pairs:

\begin{itemize}
\item BlackScholesMerton/AnalyticBarrierEngine
\item BlackScholesMerton/FdBlackScholesBarrierEngine
\end{itemize}

Engine description:

BlackScholes/AnalyticBarrierEngine builds a AnalyticBarrierEngine. A sample configuration is shown in listing
\ref{lst:peconfig_EquityBarrierOption_BlackScholesMerton_AnalyticBarrierEngine} for equity. The configuration for fx is
identical except the Model is set to GarmanKohlhagen.

The parameters have the following meaning:

\begin{itemize}
\item SensitivityTemplate [optional]: the sensitivity template to use 
\end{itemize}

\begin{longlisting}
\begin{minted}[fontsize=\footnotesize]{xml}
<Product type="EquityBarrierOption">
    <Model>BlackScholesMerton</Model>
    <ModelParameters/>
    <Engine>AnalyticBarrierEngine</Engine>
    <EngineParameters>
        <Parameter name="SensitivityTemplate">EQ_Analytical</Parameter>
    </EngineParameters>
</Product>
\end{minted}
\caption{Configuration for Product EquityBarrierOption, Model BlackScholesMertong Engine AnalyticBarrierEngine}
\label{lst:peconfig_EquityBarrierOption_BlackScholesMerton_AnalyticBarrierEngine}
\end{longlisting}

BlackScholes/FdBlackScholesBarrierEngine builds a FdBlackScholesBarrierEngine. A sample configuration is shown in
listing \ref{lst:peconfig_EquityBarrierOption_BlackScholesMerton_FdBlackScholesBarrierEngine} for equity. The
configuration for fx is identical except the Model is set to GarmanKohlhagen.

The parameters have the following meaning:

\begin{itemize}
\item Scheme: The finite difference scheme to use
\item TimeStepsPerYear: Time grid specification
\item XGrid: State grid specification
\item DampingSteps: Number of damping steps taken by FD solver
\item EnforceMonotoneVariance [optional]: If true variance is modified to be monotone if needed, defaults to true
\item SensitivityTemplate [optional]: the sensitivity template to use 
\end{itemize}

\begin{longlisting}
\begin{minted}[fontsize=\footnotesize]{xml}
<Product type="EquityBarrierOption">
    <Model>BlackScholesMerton</Model>
    <ModelParameters/>
    <Engine>FdBlackScholesBarrierEngine</Engine>
    <EngineParameters>
        <Parameter name="Scheme">Douglas</Parameter>
        <Parameter name="TimeGridPerYear">100</Parameter>
        <Parameter name="XGrid">100</Parameter>
        <Parameter name="DampingSteps">0</Parameter>
        <Parameter name="EnforceMonotoneVariance">true</Parameter>
        <Parameter name="SensitivityTemplate">EQ_FD</Parameter>
    </EngineParameters>
</Product>
\end{minted}
\caption{Configuration for Product EquityBarrierOption, Model BlackScholesMerton, Engine FdBlackScholesBarrierEngine}
\label{lst:peconfig_EquityBarrierOption_BlackScholesMerton_FdBlackScholesBarrierEngine}
\end{longlisting}

%--------------------------------------------------------
\subsubsection{Product Type: EquityDoubleBarrierOption, FxDoubleBarrierOption}
%--------------------------------------------------------

Used by trade type: EquityDoubleBarrierOption, FxDoubleBarrierOption

Available Model/Engine pairs: GarmanKohlhagen/AnalyticDoubleBarrierEngine (both fx and equity)

Engine description:

GarmanKohlhagen/AnalyticDoubleBarrierEngine builds a AnalyticDoubleBarrierEngine. A sample configuration is shown in
listing \ref{lst:peconfig_FxDoubleBarrierOption_GarmanKohlhagen_AnalyticDoubleBarrierEngine}. The configuration for equity
is identical.

The parameters have the following meaning:

\begin{itemize}
\item SensitivityTemplate [optional]: the sensitivity template to use 
\end{itemize}

\begin{longlisting}
\begin{minted}[fontsize=\footnotesize]{xml}
<Product type="FxDoubleBarrierOption">
    <Model>GarmanKohlhagen</Model>
    <ModelParameters/>
    <Engine>AnalyticDoubleBarrierEngine</Engine>
    <EngineParameters>
        <Parameter name="SensitivityTemplate">FX_Analytical</Parameter>
    </EngineParameters>
</Product>
\end{minted}
\caption{Configuration for Product FxDoubleBarrierOption, Model GarmanKohlhagen, Engine AnalyticDoubleBarrierBinaryEngine}
\label{lst:peconfig_FxDoubleBarrierOption_GarmanKohlhagen_AnalyticDoubleBarrierEngine}
\end{longlisting}


%--------------------------------------------------------
\subsubsection{Product Type: EquityDigitalOption, FxDigitalOption}
%--------------------------------------------------------

Used by trade type: EquityDigitalOption, FxDigitalOption

Available Model/Engine pairs: BlackScholesMerton/AnalyticEuropeanEngine (equity)
resp. GarmanKohlhagen/AnalyticEuropeanEngine (fx)

Engine description:

BlackScholesMerton/AnalyticEuropeanEngine (equity) resp. GarmanKohlhagen/AnalyticEuropeanEngine (fx) builds a
AnalyticEuropeanEngine. A sample configuration is shown in listing
\ref{lst:peconfig_FxDigitalOption_GarmanKohlhagen_AnalyticEuropeanEngine} for fx. The configuration for equity is
identical except the Model is set to BlackScholesMerton.

The parameters have the following meaning:

\begin{itemize}
\item SensitivityTemplate [optional]: the sensitivity template to use 
\end{itemize}

\begin{longlisting}
\begin{minted}[fontsize=\footnotesize]{xml}
 <Product type="FxDigitalOption">
    <Model>GarmanKohlhagen</Model>
    <ModelParameters/>
    <Engine>AnalyticEuropeanEngine</Engine>
    <EngineParameters>
        <Parameter name="SensitivityTemplate">FX_Analytical</Parameter>
    </EngineParameters>
</Product>
\end{minted}
\caption{Configuration for Product FxDigitalOption, Model GarmanKohlhagen, Engine AnalyticEuropeanEngine}
\label{lst:peconfig_FxDigitalOption_GarmanKohlhagen_AnalyticEuropeanEngine}
\end{longlisting}

%--------------------------------------------------------
\subsubsection{Product Type: EquityEuropeanCompositeOption}
%--------------------------------------------------------

Used by trade type: EquityOption if a composite option is built

Available Model/Engine pairs: BlackScholes/AnalyticEuropeanEngine

Engine description:

BlackScholes/AnalyticEuropeanEngine builds a AnalyticEuropeanEngine (with a vol composed from the relevant equity and fx
vol surfaces). A sample configuration is shown in listing
\ref{lst:peconfig_EquityEuropeanCompositeOption_BlackScholes_AnalyticEuropeanEngine}

The parameters have the following meaning:

\begin{itemize}
\item SensitivityTemplate [optional]: the sensitivity template to use 
\end{itemize}

\begin{longlisting}
\begin{minted}[fontsize=\footnotesize]{xml}
<Product type="EquityEuropeanCompositeOption">
    <Model>BlackScholes</Model>
    <ModelParameters/>
    <Engine>AnalyticEuropeanEngine</Engine>
    <EngineParameters>
        <Parameter name="SensitivityTemplate">EQ_Analytical</Parameter>
    </EngineParameters>
</Product>
\end{minted}
\caption{Configuration for Product EquityEuropeanCompositeOption, Model BlackScholes, Engine AnalyticEuropeanEngine}
\label{lst:peconfig_EquityEuropeanCompositeOption_BlackScholes_AnalyticEuropeanEngine}
\end{longlisting}

%--------------------------------------------------------
\subsubsection{Product Type: EquityForward}
%--------------------------------------------------------

Used by trade type: EquityForward

Available Model/Engine pairs: DiscountedCashflows/DiscountingEquityForwardEngine

Engine description:

DiscountedCashflows/DiscountingEquityForwardEngine builds a DiscountingEquityForwardEngine. A sample configuration is
shown in listing \ref{lst:peconfig_EquityForward_DiscountedCashflows_DiscountingEquityForwardEngine}

The parameters have the following meaning:

\begin{itemize}
\item SensitivityTemplate [optional]: the sensitivity template to use 
\end{itemize}

\begin{longlisting}
\begin{minted}[fontsize=\footnotesize]{xml}
<Product type="EquityForward">
    <Model>DiscountedCashflows</Model>
    <ModelParameters/>
    <Engine>DiscountingEquityForwardEngine</Engine>
    <EngineParameters>
        <Parameter name="SensitivityTemplate">EQ_Analytical</Parameter>
    </EngineParameters>
</Product>
\end{minted}
\caption{Configuration for Product EquiytForward, Model DiscountedCashflows, Engine DiscountingEquityForwardEngine}
\label{lst:peconfig_EquityForward_DiscountedCashflows_DiscountingEquityForwardEngine}
\end{longlisting}

%--------------------------------------------------------
\subsubsection{Product Type: EquityFutureOption}
%--------------------------------------------------------

Used by trade type: EquityFuturesOption

Available Model/Engine pairs: BlackScholes/AnalyticEuroepanForwardEngine

Engine description:

BlackScholes/AnalyticEuroepanForwardEngine builds a AnalyticEuropeanForwardEngine. A sample configuration is shown in
listing \ref{lst:peconfig_EquityFutureOption_BlackScholes_AnalyticEuropeanForwardEngine}

The parameters have the following meaning:

\begin{itemize}
\item SensitivityTemplate [optional]: the sensitivity template to use 
\end{itemize}

\begin{longlisting}
\begin{minted}[fontsize=\footnotesize]{xml}
<Product type="EquityFutureOption">
  <Model>BlackScholes</Model>
  <ModelParameters />
  <Engine>AnalyticEuropeanForwardEngine</Engine>
  <EngineParameters>
      <Parameter name="SensitivityTemplate">COMM_Analytical</Parameter>
  </EngineParameters>
</Product>
\end{minted}
\caption{Configuration for Product EquiytFutureOption, Model BlackScholes, Engine AnalyticEuropeanForwardEngine}
\label{lst:peconfig_EquityFutureOption_BlackScholes_AnalyticEuropeanForwardEngine}
\end{longlisting}

%--------------------------------------------------------
\subsubsection{Product Type: EquityOption}
%--------------------------------------------------------

Used by trade type: EquityOption

Available Model/Engine pairs: BlackScholesMerton/AnalyticEuropeanEngine

Engine description:

BlackScholesMerton/AnalyticEuropeanEngine builds a AnalyticEuropeanEngine. A sample configuration is shown in listing
\ref{lst:peconfig_EquityOption_BlackScholesMerton_AnalyticEuropeanEngine}

The parameters have the following meaning:

\begin{itemize}
\item SensitivityTemplate [optional]: the sensitivity template to use 
\end{itemize}

\begin{longlisting}
\begin{minted}[fontsize=\footnotesize]{xml}
<Product type="EquityOption">
    <Model>BlackScholesMerton</Model>
    <ModelParameters/>
    <Engine>AnalyticEuropeanEngine</Engine>
    <EngineParameters>
        <Parameter name="SensitivityTemplate">COMM_Analytical</Parameter>
    </EngineParameters>
</Product>
\end{minted}
\caption{Configuration for Product EquityOption, Model BlackScholesMerton, Engine AnalyticEuropeanEngine}
\label{lst:peconfig_EquityOption_BlackScholesMerton_AnalyticEuropeanEngine}
\end{longlisting}

%--------------------------------------------------------
\subsubsection{Product Type: EquityCliquetOption}
%--------------------------------------------------------

Used by trade type: EquityCliquetOption

Available Model/Engine pairs: BlackScholes/MCScript

Engine description:

BlackScholes/MCScript builds a CliquetOptionMcScriptEngine which uses the scripted trade framework. A sample
configuration is shown in listing \ref{lst:peconfig_EquityCliquetOption_BlackScholes_MCScript}

The parameters have the following meaning:

\begin{itemize}
\item Samples: number of MC samples to use
\item RegressionOrder: the regression order to use for conditional expectations (not used by this engine)
\item Interactive: whether to enable the interactive debugger
\item ScriptedLibraryOverride: whether to override a script 'EquityCliquetOption' in the script library with a hardcoded script
\item SensitivityTemplate [optional]: the sensitivity template to use 
\end{itemize}

\begin{longlisting}
\begin{minted}[fontsize=\footnotesize]{xml}
<Product type="EquityCliquetOption">
    <Model>BlackScholes</Model>
    <ModelParameters/>
    <Engine>MCScript</Engine>
    <EngineParameters>
        <Parameter name="Samples">10000</Parameter>
        <Parameter name="RegressionOrder">6</Parameter>
        <Parameter name="Interactive">false</Parameter>
        <Parameter name="ScriptedLibraryOverride">false</Parameter>
        <Parameter name="SensitivityTemplate">EQ_MC</Parameter>
    </EngineParameters>
</Product>
\end{minted}
\caption{Configuration for Product EquityCliquetOption, Model BlackScholes, Engine MCScript}
\label{lst:peconfig_EquityCliquetOption_BlackScholes_MCScript}
\end{longlisting}

%--------------------------------------------------------
\subsubsection{Product Type: QuantoEquityOption}
%--------------------------------------------------------

Used by trade type: EquityOption with quanto payoff (pay currency does not match equity currency)

Available Model/Engine pairs: BlackScholes/AnalyticEuropeanEngine

Engine description:

BlackScholes/AnalyticEuropeanEngine builds a AnalyticEuropeanEngine. A sample configuration is shown in listing
\ref{lst:peconfig_QuantoEquityOption_BlackScholes_AnalyticEuropeanEngine}

The parameters have the following meaning:

\begin{itemize}
\item SensitivityTemplate [optional]: the sensitivity template to use 
\end{itemize}

\begin{longlisting}
\begin{minted}[fontsize=\footnotesize]{xml}
<Product type="QuantoEquityOption">
    <Model>BlackScholes</Model>
    <ModelParameters>
        <Parameter name="FXSource">GENERIC</Parameter>
    </ModelParameters>
    <Engine>AnalyticEuropeanEngine</Engine>
    <EngineParameters>
        <Parameter name="SensitivityTemplate">EQ_Analytical</Parameter>
    </EngineParameters>
</Product>
\end{minted}
\caption{Configuration for Product QuantoEquityOption, Model BlackScholes, Engine AnalyticEuropeanEngine}
\label{lst:peconfig_QuantoEquityOption_BlackScholes_AnalyticEuropeanEngine}
\end{longlisting}

%--------------------------------------------------------
\subsubsection{Product Type: EquityOptionEuropeanCS}
%--------------------------------------------------------

Used by trade type: EquityOption if cash settled

Available Model/Engine pairs: BlackScholesMerton/AnalyticCashSettledEuropeanEngine

Engine description:

BlackScholesMerton/AnalyticCashSettledEuropeanEngine builds a AnalyticCashSettledEuropeanEngine. A sample configuration is shown in listing
\ref{lst:peconfig_EquityOptionEuropeanCS_BlackScholesMerton_AnalyticCashSettledEuropeanEngine}

The parameters have the following meaning:

\begin{itemize}
\item SensitivityTemplate [optional]: the sensitivity template to use 
\end{itemize}

\begin{longlisting}
\begin{minted}[fontsize=\footnotesize]{xml}
<Product type="EquityOptionEuropeanCS">
    <Model>BlackScholesMerton</Model>
    <ModelParameters/>
    <Engine>AnalyticCashSettledEuropeanEngine</Engine>
    <EngineParameters>
        <Parameter name="SensitivityTemplate">COMM_Analytical</Parameter>
    </EngineParameters>
</Product>
\end{minted}
\caption{Configuration for Product EquityOptionEuropeanCS, Model BlackScholesMerton, Engine AnalyticCashSettledEuropeanEngine}
\label{lst:peconfig_EquityOptionEuropeanCS_BlackScholesMerton_AnalyticCashSettledEuropeanEngine}
\end{longlisting}

%--------------------------------------------------------
\subsubsection{Product Type: EquityOptionAmerican}
%--------------------------------------------------------

Used by trade type: EquityOption if exercise stly is american

Available Model/Engine pairs:

\begin{itemize}
\item BlackScholesMerton/FdBlackScholesVanillaEngine
\item BlackScholesMerton/BaroneAdesiWhaleyApproximationEngine
\end{itemize}

Engine description:

BlackScholesMerton/FdBlackScholesVanillaEngine builds a FdBlackScholesVanillaEngine. A sample configuration is shown in listing
\ref{lst:peconfig_EquityOptionAmerican_BlackScholesMerton_FdBlackScholesVanillaEngine}

The parameters have the following meaning:

\begin{itemize}
\item Scheme: The finite difference scheme to use
\item TimeStepsPerYear: Time grid specification
\item XGrid: State grid specification
\item DampingSteps: Number of damping steps taken by FD solver
\item EnforceMonotoneVariance [optional]: If true variance is modified to be monotone if needed, defaults to true
\item TimeGridMinimumSize [optional]: Minimum number in resulting time grid, defaults to $1$ if not given
\item SensitivityTemplate [optional]: the sensitivity template to use 
\end{itemize}

\begin{longlisting}
\begin{minted}[fontsize=\footnotesize]{xml}
<Product type="EquityOptionAmerican">
    <Model>BlackScholesMerton</Model>
    <ModelParameters/>
    <Engine>FdBlackScholesVanillaEngine</Engine>
    <EngineParameters>
        <Parameter name="Scheme">Douglas</Parameter>
        <Parameter name="TimeGridPerYear">100</Parameter>
        <Parameter name="XGrid">100</Parameter>
        <Parameter name="DampingSteps">0</Parameter>
        <Parameter name="EnforceMonotoneVariance">true</Parameter>
        <Parameter name="SensitivityTemplate">COMM_FD</Parameter>
    </EngineParameters>
</Product>
\end{minted}
\caption{Configuration for Product EquityOptionAmerican, Model BlackScholesMerton, Engine FdBlackScholesVanillaEngine}
\label{lst:peconfig_EquityOptionAmerican_BlackScholesMerton_FdBlackScholesVanillaEngine}
\end{longlisting}

BlackScholesMerton/BaroneAdesiWhaleyApproximationEngine builds a BaroneAdesiWhaleyApproximationEngine. A sample configuration is shown in listing
\ref{lst:peconfig_EquityOptionAmerican_BlackScholesMerton_BaroneAdesiWhaleyApproximationEngine}

The parameters have the following meaning:

\begin{itemize}
\item SensitivityTemplate [optional]: the sensitivity template to use 
\end{itemize}

\begin{longlisting}
\begin{minted}[fontsize=\footnotesize]{xml}
<Product type="EquityOptionAmerican">
    <Model>BlackScholesMerton</Model>
    <ModelParameters/>
    <Engine>BaroneAdesiWhaleyApproximationEngine</Engine>
    <EngineParameters>
        <Parameter name="SensitivityTemplate">COMM_Analytical</Parameter>
    </EngineParameters>
</Product>
\end{minted}
\caption{Configuration for Product EquityOptionAmerican, Model BlackScholesMerton, Engine BaroneAdesiWhaleyApproximationEngine}
\label{lst:peconfig_EquityOptionAmerican_BlackScholesMerton_BaroneAdesiWhaleyApproximationEngine}
\end{longlisting}

%--------------------------------------------------------
\subsubsection{Product Type: EquityTouchOption, FxTouchOption}
%--------------------------------------------------------

Used by trade type: EquityTouchOption, FxTouchOption

Available Model/Engine pairs: BlackScholesMerton/AnalyticDigitalAmericanEngine for equity
resp. GarmanKohlhagen/AnalyticDigitalAmericanEngine for fx

Engine description:

BlackScholesMerton/AnalyticDigitalAmericanEngine (equity) reps. GarmanKohlhagen/AnalyticDigitalAmericanEngine (fx)
builds a AnalyticDigitalAmericanEngine (one touch) or AnalyticDigitalAmericanKOEngine (no touch). A sample configuration
is shown in listing \ref{lst:peconfig_FxTouchOption_GarmanKohlhagen_AnalyticDigitalAmericanEngine} for fx. The
configuration for equity is the same except the Model ist set to BlackScholesMerton.

The parameters have the following meaning:

\begin{itemize}
\item SensitivityTemplate [optional]: the sensitivity template to use 
\end{itemize}

\begin{longlisting}
\begin{minted}[fontsize=\footnotesize]{xml}
<Product type="FxTouchOption">
    <Model>GarmanKohlhagen</Model>
    <ModelParameters/>
    <Engine>AnalyticDigitalAmericanEngine</Engine>
    <EngineParameters>
        <Parameter name="SensitivityTemplate">FX_Analytical</Parameter>
    </EngineParameters>
</Product>
\end{minted}
\caption{Configuration for Product FxTouchOption, Model GarmanKohlhagen, Engine AnalyticDigitalAmericanEngine}
\label{lst:peconfig_FxTouchOption_GarmanKohlhagen_AnalyticDigitalAmericanEngine}
\end{longlisting}

%--------------------------------------------------------
\subsubsection{Product Type: ForwardBond}
%--------------------------------------------------------

Used by trade type: ForwardBond

Available Model/Engine pairs: DiscountedCashflows/DiscountingForwardBondEngine

Engine description:

DiscountedCashflows/DiscountingForwardBondEngine builds a DiscountingForwardBondEngine. A sample configuration is shown
in listing \ref{lst:peconfig_ForwardBond_DiscountedCashflows_DiscountingForwardBondEngine}.

The parameters have the following meaning:

\begin{itemize}
\item TimestepPeriod: discretization interval for zero bond pricing
\item SensitivityTemplate [optional]: the sensitivity template to use 
\end{itemize}

\begin{longlisting}
\begin{minted}[fontsize=\footnotesize]{xml}
<Product type="ForwardBond">
    <Model>DiscountedCashflows</Model>
    <ModelParameters/>
    <Engine>DiscountingForwardBondEngine</Engine>
    <EngineParameters>
        <Parameter name="TimestepPeriod">3M</Parameter>
        <Parameter name="SensitivityTemplate">IR_Analytical</Parameter>
    </EngineParameters>
</Product>
\end{minted}
\caption{Configuration for Product ForwardBond, Model DiscountedCashflows, Engine DiscountingForwardBondEngine}
\label{lst:peconfig_ForwardBond_DiscountedCashflows_DiscountingForwardBondEngine}
\end{longlisting}

%--------------------------------------------------------
\subsubsection{Product Type: EquityVarianceSwap, CommodityVarianceSwap, FxVarianceSwap}
%--------------------------------------------------------

Used by trade type: EquityVarianceSwap, CommodityVarianceSwap, FxVarianceSwap

Available Model/Engine pairs: BlackScholesMerton/EquityVarianceSwap (resp. Commodity, Fx)

Engine description:

BlackScholesMerton/EquityVarianceSwap (resp. Commodity, Fx) builds a GeneralisedReplicatingVarianceSwapEngine or
VolatilityFromVarianceSwapEngine depending on MomentType Variance, Volatility. A sample configuration is shown in
listing \ref{lst:peconfig_EquityVarianceSwap_BlackScholesMerton_ReplicatingVarianceSwapEngine_1} and
\ref{lst:peconfig_EquityVarianceSwap_BlackScholesMerton_ReplicatingVarianceSwapEngine_2}. The example is for Equity, but
holds equally for product types CommodityVarianceSwap, FxVarianceSwap.

The parameters have the following meaning:

\begin{itemize}
\item Scheme [Optional, default GaussLobatto]: GaussLobatto or Segment, this determines the integration scheme used
\item Bounds [Optional, default PriceThreshold]: ``Fixed'': The integration bounds are found by
  
  $$
  \text{Lower} = F e ^ {m \sigma \sqrt{T}},  \text{Upper} = F e ^ {M \sigma \sqrt{T}}
  $$

  where $m$ is the FixedMinStdDevs and $M$ is the FixedMaxStdDevs parameter listed below and $\sigma$ is the ATMF
  volatility of the underlying at maturity $T$.

  ``PriceThreshold'': The integration bounds are found by looking for the largest (smallest) strikes Lower, Upper such
  that the integrand in the replication formula above is below the PriceThreshold parameter listed below. The search
  starts at the forward level $F$ for Lower, Upper and then Lower, Upper are updated by factors $1-p$ resp. $1+p$ where
  $p$ is the PriceThresholdStep parameter below, until either the price threshold criterium is met or the number of
  updates exceeds the MaxPriceThresholdSteps parameter.

\item Accuracy [Optional, default $10^{-5}$]: Numerical accuracy tolerance for the numerical integration. This only
  applies to the GaussLobatto scheme.
\item MaxIterations [Optional, default $1000$]: The maximum number of iterations performed in the numerical
  integration. This only applies to the GaussLobatto scheme.
\item Steps [Optional, default $100$]: The number of steps in the Segment numerical integration scheme (only applies for
  this scheme).
\item PriceThreshold [Optional, default $10^{-10}$]: Used to determine the integration bounds if Bounds = PriceThredshold, see above.
\item MaxPriceThresholdSteps [Optional, default $100$]: Used to determine the integration bounds if Bounds = PriceThredshold, see above.
\item PriceThresholdStep [Option, default $0.1$]: Used to determine the integration bounds if Bounds = PriceThredshold, see above.
\item FixedMinStdDevs [Optional, default $-5$]: Used to determine the integration bounds if Bounds = Fixed, see above.
\item FixedMaxStdDevs [Optional, default $5$]: Used to determine the integration bounds, if Bounds = Fixed, see above.
\item StaticTodaysSpot [Optional, default false]: If true the contribution to the variance from the last day before the
  valuation date to the valuation date is ignored in scenario / sensitivity calculations. See below for more details.
\item SensitivityTemplate [optional]: the sensitivity template to use 
\end{itemize}

\begin{listing}[ht]
\begin{minted}[fontsize=\footnotesize]{xml}
  <Product type="EquityVarianceSwap">
    <Model>BlackScholesMerton</Model>
    <ModelParameters>
      <Parameter name="StaticTodaysSpot">false</Parameter>
    </ModelParameters>
    <Engine>ReplicatingVarianceSwapEngine</Engine>
    <EngineParameters>
      <Parameter name="Scheme">Segment</Parameter>
      <Parameter name="Bounds">PriceThreshold</Parameter>
      <Parameter name="Steps">1000</Parameter>
      <Parameter name="PriceThreshold">1E-10</Parameter>
      <Parameter name="MaxPriceThresholdSteps">500</Parameter>
      <Parameter name="PriceThresholdStep">0.1</Parameter>
    </EngineParameters>
  </Product>
\end{minted}
\caption{Configuration for Product EquityVarianceSwap, Model BlackScholesMerton, Engine ReplicatingVarianceSwapEngine,
  ``Robust'' configuration using PriceThreshold if market smiles are of lower quality.}
\label{lst:peconfig_EquityVarianceSwap_BlackScholesMerton_ReplicatingVarianceSwapEngine_1}
\end{listing}

\begin{listing}[ht]
\begin{minted}[fontsize=\footnotesize]{xml}
  <Product type="EquityVarianceSwap">
    <Model>BlackScholesMerton</Model>
    <ModelParameters>
      <Parameter name="StaticTodaysSpot">false</Parameter>
    </ModelParameters>
    <Engine>ReplicatingVarianceSwapEngine</Engine>
    <EngineParameters>
      <Parameter name="Scheme">Segment</Parameter>
      <Parameter name="Bounds">Fixed</Parameter>
      <Parameter name="Steps">1000</Parameter>
      <Parameter name="FixedMinStdDevs">-5.0</Parameter>
      <Parameter name="FixedMaxStdDevs">5.0</Parameter>
    </EngineParameters>
  </Product>
\end{minted}
\caption{Configuration for Product EquityVarianceSwap, Model BlackScholesMerton, Engine ReplicatingVarianceSwapEngine,
  Alternative ``Robust'' variance swap pricing engine configuration using fixed integration bounds.}
\label{lst:peconfig_EquityVarianceSwap_BlackScholesMerton_ReplicatingVarianceSwapEngine_2}
\end{listing}

%--------------------------------------------------------
\subsubsection{Product Type: FxAsianOptionArithmeticPrice}
%--------------------------------------------------------

Used by trade type: FxAsianOption if payoffType2 is Arithmetric and payoffType is Asian

Available Model/Engine pairs:

\begin{itemize}
  \item BlackScholesMerton/MCDiscreteArithmeticAPEngine
  \item BlackScholesMerton/TurnbullWakemanAsianEngine
  \item ScriptedTrade/ScriptedTrade
\end{itemize}

Engine description:

BlackScholesMerton/MCDiscreteArithmeticAPEngine builds a MCDiscreteArithmeticAPEngine using Sobol sequences. A sample
configuration is shown in listing
\ref{lst:peconfig_FxAsianOptionArithmeticPrice_BlackScholesMerton_MCDiscreteArithmeticAPEngine}.

The parameters have the following meaning:

\begin{itemize}
\item BrownianBridge: whether to use Brownian Bridge for MC simulation
\item AntitheticVariate: whether to use antithetic variates for MC simulation
\item ControlVariate: whether to use control variate for MC simulation
\item RequiredSamples: minimum number of samples for MC simulation, if 0 (not specified) RequiredTolerance must be given
\item RequiredTolernace: max tolerance for MC error, if 0 (not specified) RequiredSamples must be given
\item MaxSamples: max number of samples for MC simulation
\item Seed: seed for random number generator
\item SensitivityTemplate [optional]: the sensitivity template to use 
\end{itemize}

\begin{longlisting}
\begin{minted}[fontsize=\footnotesize]{xml}
<Product type="FxAsianOptionArithmeticPrice">
    <Model>BlackScholesMerton</Model>
    <ModelParameters/>
    <Engine>MCDiscreteArithmeticAPEngine</Engine>
    <EngineParameters>
        <Parameter name="BrownianBridge">true</Parameter>    
        <Parameter name="AntitheticVariate">true</Parameter>    
        <Parameter name="ControlVariate">true</Parameter>    
        <Parameter name="RequiredSamples">10000</Parameter>    
        <Parameter name="RequiredTolerance">0</Parameter>    
        <Parameter name="MaxSamples">0</Parameter>    
        <Parameter name="Seed">42</Parameter>    
        <Parameter name="SensitivityTemplate">FX_MC</Parameter>
    </EngineParameters>
</Product>
\end{minted}
\caption{Configuration for Product FxAsianOptionArithmeticPrice, Model BlackScholesMerton, Engine MCDiscreteArithmeticAPEngine}
\label{lst:peconfig_FxAsianOptionArithmeticPrice_BlackScholesMerton_MCDiscreteArithmeticAPEngine}
\end{longlisting}

BlackScholesMerton/TurnbullWakemanAsianEngine builds a TurnbullWakemanAsianEngine. A sample configuration is shown in
listing \ref{lst:peconfig_FxAsianOptionArithmeticPrice_BlackScholesMerton_TurnbullWakemanAsianEngine}.

The parameters have the following meaning:

\begin{itemize}
\item SensitivityTemplate [optional]: the sensitivity template to use 
\end{itemize}

\begin{longlisting}
\begin{minted}[fontsize=\footnotesize]{xml}
<Product type="FxAsianOptionArithmeticPrice">
    <Model>BlackScholesMerton</Model>
    <ModelParameters/>
    <Engine>TurnbullWakemanAsianEngine</Engine>
    <EngineParameters>
        <Parameter name="SensitivityTemplate">FX_Analytical</Parameter>
    </EngineParameters>
</Product>
\end{minted}
\caption{Configuration for Product FxAsianOptionArithmeticPrice, Model BlackScholesMerton, Engine TurnbullWakemanAsianEngine}
\label{lst:peconfig_FxAsianOptionArithmeticPrice_BlackScholesMerton_TurnbullWakemanAsianEngine}
\end{longlisting}

ScriptedTrade/ScriptedTrade delegates to the scripted trade engine and the associated pricing engine configuration for
Product Type ScriptedTrade, see there for details. A sample configuration is given in listing
\ref{lst:peconfig_FxAsianOptionArithmeticPrice_ScriptedTrade_ScriptedTrade}.

The parameters have the following meaning:

\begin{itemize}
\item SensitivityTemplate [optional]: the sensitivity template to use 
\end{itemize}

\begin{longlisting}
\begin{minted}[fontsize=\footnotesize]{xml}
<Product type="FxAsianOptionArithmeticPrice">
    <Model>ScriptedTrade</Model>
    <ModelParameters/>
    <Engine>ScriptedTrade</Engine>
    <EngineParameters>
        <Parameter name="SensitivityTemplate">FX_MC</Parameter>
    </EngineParameters>
</Product>
\end{minted}
\caption{Configuration for Product FxAsianOptionArithmeticPrice, Model ScriptedTrade, Engine ScriptedTrade}
\label{lst:peconfig_FxAsianOptionArithmeticPrice_ScriptedTrade_ScriptedTrade}
\end{longlisting}


%--------------------------------------------------------
\subsubsection{Product Type: FxAsianOptionArithmeticStrike}
%--------------------------------------------------------

Used by trade type: FxAsianOption if payoffType2 is Arithmetric and payoffType is AverageStrike

Available Model/Engine pairs:

\begin{itemize}
\item BlackScholesMerton/MCDiscreteArithmeticASEngine
\item ScriptedTrade/ScriptedTrade
\end{itemize}
  
Engine description:

BlackScholesMerton/MCDiscreteArithmeticASEngine builds a MCDiscreteArithmeticASEngine using Sobol sequences. A sample
configuration is shown in listing
\ref{lst:peconfig_FxAsianOptionArithmeticStrike_BlackScholesMerton_MCDiscreteArithmeticASEngine}.

The parameters have the following meaning:

\begin{itemize}
\item BrownianBridge: whether to use Brownian Bridge for MC simulation
\item AntitheticVariate: whether to use antithetic variates for MC simulation
\item ControlVariate: whether to use control variate for MC simulation
\item RequiredSamples: minimum number of samples for MC simulation, if 0 (not specified) RequiredTolerance must be given
\item RequiredTolernace: max tolerance for MC error, if 0 (not specified) RequiredSamples must be given
\item MaxSamples: max number of samples for MC simulation
\item Seed: seed for random number generator
\item SensitivityTemplate [optional]: the sensitivity template to use 
\end{itemize}

\begin{longlisting}
\begin{minted}[fontsize=\footnotesize]{xml}
<Product type="FxAsianOptionArithmeticStrike">
    <Model>BlackScholesMerton</Model>
    <ModelParameters/>
    <Engine>MCDiscreteArithmeticASEngine</Engine>
    <EngineParameters>
        <Parameter name="BrownianBridge">true</Parameter>    
        <Parameter name="AntitheticVariate">true</Parameter>    
        <Parameter name="ControlVariate">true</Parameter>    
        <Parameter name="RequiredSamples">10000</Parameter>    
        <Parameter name="RequiredTolerance">0</Parameter>    
        <Parameter name="MaxSamples">0</Parameter>    
        <Parameter name="Seed">42</Parameter>    
        <Parameter name="SensitivityTemplate">FX_MC</Parameter>
    </EngineParameters>
</Product>
\end{minted}
\caption{Configuration for Product FxAsianOptionArithmeticStrike, Model BlackScholesMerton, Engine MCDiscreteArithmeticASEngine}
\label{lst:peconfig_FxAsianOptionArithmeticStrike_BlackScholesMerton_MCDiscreteArithmeticASEngine}
\end{longlisting}

ScriptedTrade/ScriptedTrade delegates to the scripted trade engine and the associated pricing engine configuration for
Product Type ScriptedTrade, see there for details. A sample configuration is given in listing
\ref{lst:peconfig_FxAsianOptionArithmeticStrike_ScriptedTrade_ScriptedTrade}.

The parameters have the following meaning:

\begin{itemize}
\item SensitivityTemplate [optional]: the sensitivity template to use 
\end{itemize}

\begin{longlisting}
\begin{minted}[fontsize=\footnotesize]{xml}
<Product type="FxAsianOptionArithmeticStrike">
    <Model>ScriptedTrade</Model>
    <ModelParameters/>
    <Engine>ScriptedTrade</Engine>
    <EngineParameters>
        <Parameter name="SensitivityTemplate">FX_MC</Parameter>
    </EngineParameters>
</Product>
\end{minted}
\caption{Configuration for Product FxAsianOptionArithmeticStrike, Model ScriptedTrade, Engine ScriptedTrade}
\label{lst:peconfig_FxAsianOptionArithmeticStrike_ScriptedTrade_ScriptedTrade}
\end{longlisting}

%--------------------------------------------------------
\subsubsection{Product Type: FxAsianOptionGeometricPrice}
%--------------------------------------------------------

Used by trade type: FxAsianOption if payoffType2 is Geometric and payoffType is Asian

Available Model/Engine pairs:

\begin{itemize}
  \item BlackScholesMerton/MCDiscreteGeometricAPEngine
  \item BlackScholesMerton/AnalyticDiscreteGeometricAPEngine
  \item BlackScholesMerton/AnalyticContinuousGeometricAPEngine
  \item ScriptedTrade/ScriptedTrade
\end{itemize}

Engine description:

BlackScholesMerton/MCDiscreteGeometricAPEngine builds a MCDiscreteGeometricAPEngine using Sobol sequences. A sample
configuration is shown in listing
\ref{lst:peconfig_FxAsianOptionGeometricPrice_BlackScholesMerton_MCDiscreteGeomtetricAPEngine}.

The parameters have the following meaning:

\begin{itemize}
\item BrownianBridge: whether to use Brownian Bridge for MC simulation
\item AntitheticVariate: whether to use antithetic variates for MC simulation
\item ControlVariate: whether to use control variate for MC simulation
\item RequiredSamples: minimum number of samples for MC simulation, if 0 (not specified) RequiredTolerance must be given
\item RequiredTolernace: max tolerance for MC error, if 0 (not specified) RequiredSamples must be given
\item MaxSamples: max number of samples for MC simulation
\item Seed: seed for random number generator
\item SensitivityTemplate [optional]: the sensitivity template to use 
\end{itemize}

\begin{longlisting}
\begin{minted}[fontsize=\footnotesize]{xml}
<Product type="FxAsianOptionGeometricPrice">
    <Model>BlackScholesMerton</Model>
    <ModelParameters/>
    <Engine>MCDiscreteGeometricAPEngine</Engine>
    <EngineParameters>
        <Parameter name="BrownianBridge">true</Parameter>    
        <Parameter name="AntitheticVariate">true</Parameter>    
        <Parameter name="ControlVariate">true</Parameter>    
        <Parameter name="RequiredSamples">10000</Parameter>    
        <Parameter name="RequiredTolerance">0</Parameter>    
        <Parameter name="MaxSamples">0</Parameter>    
        <Parameter name="Seed">42</Parameter>    
        <Parameter name="SensitivityTemplate">FX_MC</Parameter>
    </EngineParameters>
</Product>
\end{minted}
\caption{Configuration for Product FxAsianOptionGeometricPrice, Model BlackScholesMerton, Engine MCDiscreteGeometricAPEngine}
\label{lst:peconfig_FxAsianOptionGeometricPrice_BlackScholesMerton_MCDiscreteGeomtetricAPEngine}
\end{longlisting}

BlackScholesMerton/AnalyticDiscreteGeometricAPEngine builds a AnalyticDiscreteGeometricAveragePriceAsianEngine. A sample
configuration is shown in listing
\ref{lst:peconfig_FxAsianOptionGeometricPrice_BlackScholesMerton_AnalyticDiscreteGeomtetricAPEngine}.

The parameters have the following meaning:

\begin{itemize}
\item SensitivityTemplate [optional]: the sensitivity template to use 
\end{itemize}

\begin{longlisting}
\begin{minted}[fontsize=\footnotesize]{xml}
<Product type="FxAsianOptionGeometricPrice">
    <Model>BlackScholesMerton</Model>
    <ModelParameters/>
    <Engine>AnalyticDiscreteGeometricAPEngine</Engine>
    <EngineParameters>
        <Parameter name="SensitivityTemplate">FX_Analytical</Parameter>
    </EngineParameters>
</Product>
\end{minted}
\caption{Configuration for Product FxAsianOptionGeometricPrice, Model BlackScholesMerton, Engine AnalyticDiscreteGeomtetricAPEngine}
\label{lst:peconfig_FxAsianOptionGeometricPrice_BlackScholesMerton_AnalyticDiscreteGeomtetricAPEngine}
\end{longlisting}

BlackScholesMerton/AnalyticContinuousGeometricAPEngine builds a AnalyticContinuousGeometricAveragePriceAsianEngine. A sample
configuration is shown in listing
\ref{lst:peconfig_FxAsianOptionGeometricPrice_BlackScholesMerton_AnalyticContinuousGeomtetricAPEngine}.

The parameters have the following meaning:

\begin{itemize}
\item SensitivityTemplate [optional]: the sensitivity template to use 
\end{itemize}

\begin{longlisting}
\begin{minted}[fontsize=\footnotesize]{xml}
<Product type="FxAsianOptionGeometricPrice">
    <Model>BlackScholesMerton</Model>
    <ModelParameters/>
    <Engine>AnalyticContinuousGeometricAPEngine</Engine>
    <EngineParameters>
        <Parameter name="SensitivityTemplate">FX_Analytical</Parameter>
    </EngineParameters>
</Product>
\end{minted}
\caption{Configuration for Product FxAsianOptionGeometricPrice, Model BlackScholesMerton, Engine AnalyticContinuousGeomtetricAPEngine}
\label{lst:peconfig_FxAsianOptionGeometricPrice_BlackScholesMerton_AnalyticContinuousGeomtetricAPEngine}
\end{longlisting}

ScriptedTrade/ScriptedTrade delegates to the scripted trade engine and the associated pricing engine configuration for
Product Type ScriptedTrade, see there for details. A sample configuration is given in listing
\ref{lst:peconfig_FxAsianOptionGeometricPrice_ScriptedTrade_ScriptedTrade}.

The parameters have the following meaning:

\begin{itemize}
\item SensitivityTemplate [optional]: the sensitivity template to use 
\end{itemize}

\begin{longlisting}
\begin{minted}[fontsize=\footnotesize]{xml}
<Product type="FxAsianOptionGeometricPrice">
    <Model>ScriptedTrade</Model>
    <ModelParameters/>
    <Engine>ScriptedTrade</Engine>
    <EngineParameters>
        <Parameter name="SensitivityTemplate">FX_MC</Parameter>
    </EngineParameters>
</Product>
\end{minted}
\caption{Configuration for Product FxAsianOptionGeometricPrice, Model ScriptedTrade, Engine ScriptedTrade}
\label{lst:peconfig_FxAsianOptionGeometricPrice_ScriptedTrade_ScriptedTrade}
\end{longlisting}

%--------------------------------------------------------
\subsubsection{Product Type: FxAsianOptionGeometricStrike}
%--------------------------------------------------------

Used by trade type: FxAsianOption if payoffType2 is Geometric and payoffType is AverageStrike

Available Model/Engine pairs:

\begin{itemize}
  \item BlackScholesMerton/MCDiscreteGeometricASEngine
  \item BlackScholesMerton/AnalyticDiscreteGeometricASEngine
  \item ScriptedTrade/ScriptedTrade
\end{itemize}

Engine description:

BlackScholesMerton/MCDiscreteGeometricASEngine builds a MCDiscreteGeometricASEngine using Sobol sequences. A sample
configuration is shown in listing
\ref{lst:peconfig_FxAsianOptionGeometricStrike_BlackScholesMerton_MCDiscreteGeomtetricASEngine}.

The parameters have the following meaning:

\begin{itemize}
\item BrownianBridge: whether to use Brownian Bridge for MC simulation
\item AntitheticVariate: whether to use antithetic variates for MC simulation
\item ControlVariate: whether to use control variate for MC simulation
\item RequiredSamples: minimum number of samples for MC simulation, if 0 (not specified) RequiredTolerance must be given
\item RequiredTolernace: max tolerance for MC error, if 0 (not specified) RequiredSamples must be given
\item MaxSamples: max number of samples for MC simulation
\item Seed: seed for random number generator
\item SensitivityTemplate [optional]: the sensitivity template to use 
\end{itemize}

\begin{longlisting}
\begin{minted}[fontsize=\footnotesize]{xml}
<Product type="FxAsianOptionGeometricStrike">
    <Model>BlackScholesMerton</Model>
    <ModelParameters/>
    <Engine>MCDiscreteGeometricASEngine</Engine>
    <EngineParameters>
        <Parameter name="BrownianBridge">true</Parameter>    
        <Parameter name="AntitheticVariate">true</Parameter>    
        <Parameter name="ControlVariate">true</Parameter>    
        <Parameter name="RequiredSamples">10000</Parameter>    
        <Parameter name="RequiredTolerance">0</Parameter>    
        <Parameter name="MaxSamples">0</Parameter>    
        <Parameter name="Seed">42</Parameter>    
        <Parameter name="SensitivityTemplate">FX_MC</Parameter>
    </EngineParameters>
</Product>
\end{minted}
\caption{Configuration for Product FxAsianOptionGeometricStrike, Model BlackScholesMerton, Engine MCDiscreteGeometricASEngine}
\label{lst:peconfig_FxAsianOptionGeometricStrike_BlackScholesMerton_MCDiscreteGeomtetricASEngine}
\end{longlisting}

BlackScholesMerton/AnalyticDiscreteGeometricASEngine builds a AnalyticDiscreteGeometricAverageStrikeAsianEngine. A sample
configuration is shown in listing
\ref{lst:peconfig_FxAsianOptionGeometricStrike_BlackScholesMerton_AnalyticDiscreteGeomtetricASEngine}.

The parameters have the following meaning:

\begin{itemize}
\item SensitivityTemplate [optional]: the sensitivity template to use 
\end{itemize}

\begin{longlisting}
\begin{minted}[fontsize=\footnotesize]{xml}
<Product type="FxAsianOptionGeometricStrike">
    <Model>BlackScholesMerton</Model>
    <ModelParameters/>
    <Engine>AnalyticDiscreteGeometricASEngine</Engine>
    <EngineParameters>
        <Parameter name="SensitivityTemplate">FX_Analytical</Parameter>
    </EngineParameters>
</Product>
\end{minted}
\caption{Configuration for Product FxAsianOptionGeometricStrike, Model BlackScholesMerton, Engine AnalyticDiscreteGeomtetricASEngine}
\label{lst:peconfig_FxAsianOptionGeometricStrike_BlackScholesMerton_AnalyticDiscreteGeomtetricASEngine}
\end{longlisting}

ScriptedTrade/ScriptedTrade delegates to the scripted trade engine and the associated pricing engine configuration for
Product Type ScriptedTrade, see there for details. A sample configuration is given in listing
\ref{lst:peconfig_FxAsianOptionGeometricStrike_ScriptedTrade_ScriptedTrade}.

The parameters have the following meaning:

\begin{itemize}
\item SensitivityTemplate [optional]: the sensitivity template to use 
\end{itemize}

\begin{longlisting}
\begin{minted}[fontsize=\footnotesize]{xml}
<Product type="FxAsianOptionGeometricStrike">
    <Model>ScriptedTrade</Model>
    <ModelParameters/>
    <Engine>ScriptedTrade</Engine>
    <EngineParameters>
        <Parameter name="SensitivityTemplate">FX_MC</Parameter>
    </EngineParameters>
</Product>
\end{minted}
\caption{Configuration for Product FxAsianOptionGeometricStrike, Model ScriptedTrade, Engine ScriptedTrade}
\label{lst:peconfig_FxAsianOptionGeometricStrike_ScriptedTrade_ScriptedTrade}
\end{longlisting}

%--------------------------------------------------------
\subsubsection{Product Type: FxDigitalOptionEuropeanCS}
%--------------------------------------------------------

Used by trade type: FxDigitalOption if cash settled, FxEuropeanBarrierOption

Available Model/Engine pairs: GarmanKohlhagen/AnalyticCashSettledEuropeanEngine

Engine description:

GarmanKohlhagen/AnalyticCashSettledEuropeanEngine builds a AnalyticCashSettledEuropeanEngine. A sample configuration is
shown in listing \ref{lst:peconfig_FxDigitalOptionEuropeanCS_GarmanKohlhagen_AnalyticCashSettledEuropeanEngine}

The parameters have the following meaning:

\begin{itemize}
\item SensitivityTemplate [optional]: the sensitivity template to use 
\end{itemize}

\begin{longlisting}
\begin{minted}[fontsize=\footnotesize]{xml}
<Product type="FxDigitalOptionEuropeanCS">
    <Model>GarmanKohlhagen</Model>
    <ModelParameters/>
    <Engine>AnalyticCashSettledEuropeanEngine</Engine>
    <EngineParameters>
        <Parameter name="SensitivityTemplate">FX_Analytical</Parameter>
    </EngineParameters>
</Product>
\end{minted}
\caption{Configuration for Product FxDigitalOptionEuropeanCS, Model GarmanKohlhagen, Engine AnalyticCashSettledEuropeanEngine}
\label{lst:peconfig_FxDigitalOptionEuropeanCS_GarmanKohlhagen_AnalyticCashSettledEuropeanEngine}
\end{longlisting}

%--------------------------------------------------------
\subsubsection{Product Type: FxDigitalBarrierOption}
%--------------------------------------------------------

Used by trade type: FxDigitalBarrierOption

Available Model/Engine pairs: GarmanKohlhagen/FdBlackScholesBarrierEngine

Engine description:

GarmanKohlhagen/FdBlackScholesBarrierEngine builds a FdBlackScholesBarrierEngine. A sample configuration is shown in
listing \ref{lst:peconfig_FxDigitalBarrierOption_GarmanKohlhagen_FdBlackScholesBarrierEngine}

The parameters have the following meaning:

\begin{itemize}
\item Scheme: The finite difference scheme to use
\item TimeStepsPerYear: Time grid specification
\item XGrid: State grid specification
\item DampingSteps: Number of damping steps taken by FD solver
\item EnforceMonotoneVariance [optional]: If true variance is modified to be monotone if needed, defaults to true
\item SensitivityTemplate [optional]: the sensitivity template to use 
\end{itemize}

\begin{longlisting}
\begin{minted}[fontsize=\footnotesize]{xml}
<Product type="FxDigitalBarrierOption">
    <Model>GarmanKohlhagen</Model>
    <ModelParameters/>
    <Engine>FdBlackScholesBarrierEngine</Engine>
    <EngineParameters>
        <Parameter name="Scheme">Douglas</Parameter>
        <Parameter name="TimeGridPerYear">100</Parameter>
        <Parameter name="XGrid">100</Parameter>
        <Parameter name="DampingSteps">0</Parameter>
        <Parameter name="SensitivityTemplate">FX_FD</Parameter>
    </EngineParameters>
</Product>
\end{minted}
\caption{Configuration for Product FxDigitalBarrierOption, Model GarmanKohlhagen, Engine FdBlackScholesBarrierEngine}
\label{lst:peconfig_FxDigitalBarrierOption_GarmanKohlhagen_FdBlackScholesBarrierEngine}
\end{longlisting}

%--------------------------------------------------------
\subsubsection{Product Type: FormulaBasedCoupon}
%--------------------------------------------------------

Used by trade type: leg type FormulaBased (coupon pricer)

Available Model/Engine pairs:

\begin{itemize}
\item BrigoMercurio/MC
\end{itemize}

Engine description:

BrigoMercurio/MC builds a MCGaussianFormulaBasedCouponPricer. A sample configuration is shown in listing
\ref{lst:peconfig_FormulaBasedCoupon_BrigoMercurio_MC}. We refer to the formula based coupon module documentation for
further details.

The parameters have the following meaning:

\begin{itemize}
\item FXSource: specifies the FX index tag to be used to look up FX-Ibor or FX-CMS correlations
\item Samples: number of MC samples to use
\item Sobol: whether to use Sobol numbers for simulation
\item SalvageCorrelationMatrix: whether to make the input correlation matrix positive definite
\item SensitivityTemplate [optional]: the sensitivity template to use 
\end{itemize}

\begin{longlisting}
\begin{minted}[fontsize=\footnotesize]{xml}
<Product type="FormulaBasedCoupon">
    <Model>BrigoMercurio</Model>
    <ModelParameters>
        <Parameter name="FXSource">GENERIC</Parameter>
    </ModelParameters>
    <Engine>MC</Engine>
    <EngineParameters>
        <Parameter name="Samples">10000</Parameter>
        <Parameter name="Seed">42</Parameter>
        <Parameter name="Sobol">Y</Parameter>
        <Parameter name="SalvageCorrelationMatrix">Y</Parameter>
        <Parameter name="SensitivityTemplate">IR_MC</Parameter>
    </EngineParameters>
</Product>
\end{minted}
\caption{Configuration for Product FormulaBasedCoupon, Model BrigoMercurio, Engine MC}
\label{lst:peconfig_FormulaBasedCoupon_BrigoMercurio_MC}
\end{longlisting}

%--------------------------------------------------------
\subsubsection{Product Type: FxForward}
%--------------------------------------------------------

Used by trade type: FxForward

Available Model/Engine pairs:

\begin{itemize}
\item DiscountedCashflows/DiscountingFxForwardEngine
\item CrossAssetModel/AMC (for use in AMC simulations only)
\end{itemize}

Engine description:

DiscountedCashflows/DiscountingFxForwardEngine builds a DiscountingFxForwardEngine. A sample configuration is shown in
listing \ref{lst:peconfig_FxForward_DiscountedCashflows_DiscountingFxForwardEngine}.

The parameters have the following meaning:

\begin{itemize}
\item SensitivityTemplate [optional]: the sensitivity template to use 
\end{itemize}

\begin{longlisting}
\begin{minted}[fontsize=\footnotesize]{xml}
<Product type="FxForward">
    <Model>DiscountedCashflows</Model>
    <ModelParameters/>
    <Engine>DiscountingFxForwardEngine</Engine>
    <EngineParameters>
        <Parameter name="SensitivityTemplate">FX_Analytical</Parameter>
    </EngineParameters>
</Product>
\end{minted}
\caption{Configuration for Product FxForward, Model DiscountedCashflows, Engine DiscountingFxForwardEngine}
\label{lst:peconfig_FxForward_DiscountedCashflows_DiscountingFxForwardEngine}
\end{longlisting}

CrossAssetModel/AMC builds a McCamFxForwardEngine for use in AMC simulations. We refer to the AMC module documentation
for further details.

%--------------------------------------------------------
\subsubsection{Product Type: FxOption}
%--------------------------------------------------------

Used by trade type: FxOption

Available Model/Engine pairs:

\begin{itemize}
\item GarmanKohlhagen/AnalyticEuropeanEngine
\item CrossAssetModel/AMC (for use in AMC simulations only)
\end{itemize}

Engine description:

GarmanKohlhagen/AnalyticEuropeanEngine builds a AnalyticEuropeanEngine. A sample configuration is shown in listing
\ref{lst:peconfig_FxOption_GarmanKohlhagen_AnalyticEuropeanEngine}

The parameters have the following meaning:

\begin{itemize}
\item SensitivityTemplate [optional]: the sensitivity template to use 
\end{itemize}

\begin{longlisting}
\begin{minted}[fontsize=\footnotesize]{xml}
<Product type="FxOption">
    <Model>GarmanKohlhagen</Model>
    <ModelParameters/>
    <Engine>AnalyticEuropeanEngine</Engine>
    <EngineParameters>
        <Parameter name="SensitivityTemplate">COMM_Analytical</Parameter>
    </EngineParameters>
</Product>
\end{minted}
\caption{Configuration for Product FxOption, Model GarmanKohlhagen, Engine AnalyticEuropeanEngine}
\label{lst:peconfig_FxOption_GarmanKohlhagen_AnalyticEuropeanEngine}
\end{longlisting}

CrossAssetModel/AMC builds a McCamFxOptionEngine for use in AMC simulations. We refer to the AMC module documentation
for further details.

%--------------------------------------------------------
\subsubsection{Product Type: FxOptionEuropeanCS}
%--------------------------------------------------------

Used by trade type: FxOption if cash settled

Available Model/Engine pairs: GarmanKohlhagen/AnalyticCashSettledEuropeanEngine

Engine description:

GarmanKohlhagen/AnalyticCashSettledEuropeanEngine builds a AnalyticCashSettledEuropeanEngine. A sample configuration is shown in listing
\ref{lst:peconfig_FxOptionEuropeanCS_GarmanKohlhagen_AnalyticCashSettledEuropeanEngine}

The parameters have the following meaning:

\begin{itemize}
\item SensitivityTemplate [optional]: the sensitivity template to use 
\end{itemize}

\begin{longlisting}
\begin{minted}[fontsize=\footnotesize]{xml}
<Product type="FxOptionEuropeanCS">
    <Model>GarmanKohlhagen</Model>
    <ModelParameters/>
    <Engine>AnalyticCashSettledEuropeanEngine</Engine>
    <EngineParameters>
        <Parameter name="SensitivityTemplate">COMM_Analytical</Parameter>
    </EngineParameters>
</Product>
\end{minted}
\caption{Configuration for Product FxOptionEuropeanCS, Model GarmanKohlhagen, Engine AnalyticCashSettledEuropeanEngine}
\label{lst:peconfig_FxOptionEuropeanCS_GarmanKohlhagen_AnalyticCashSettledEuropeanEngine}
\end{longlisting}

%--------------------------------------------------------
\subsubsection{Product Type: FxOptionAmerican}
%--------------------------------------------------------

Used by trade type: FxOption if exercise stly is american

Available Model/Engine pairs:

\begin{itemize}
\item GarmanKohlhagen/FdGarmanKohlhagenVanillaEngine
\item GarmanKohlhagen/BaroneAdesiWhaleyApproximationEngine
\end{itemize}

Engine description:

GarmanKohlhagen/FdBlackScholesVanillaEngine builds a FdBlackScholesVanillaEngine. A sample configuration is shown in listing
\ref{lst:peconfig_FxOptionAmerican_GarmanKohlhagen_FdBlackScholesVanillaEngine}

The parameters have the following meaning:

\begin{itemize}
\item Scheme: The finite difference scheme to use
\item TimeStepsPerYear: Time grid specification
\item XGrid: State grid specification
\item DampingSteps: Number of damping steps taken by FD solver
\item EnforceMonotoneVariance [optional]: If true variance is modified to be monotone if needed, defaults to true
\item TimeGridMinimumSize [optional]: Minimum number in resulting time grid, defaults to $1$ if not given
\item SensitivityTemplate [optional]: the sensitivity template to use 
\end{itemize}

\begin{longlisting}
\begin{minted}[fontsize=\footnotesize]{xml}
<Product type="FxOptionAmerican">
    <Model>GarmanKohlhagen</Model>
    <ModelParameters/>
    <Engine>FdBlackScholesVanillaEngine</Engine>
    <EngineParameters>
        <Parameter name="Scheme">Douglas</Parameter>
        <Parameter name="TimeGridPerYear">100</Parameter>
        <Parameter name="XGrid">100</Parameter>
        <Parameter name="DampingSteps">0</Parameter>
        <Parameter name="EnforceMonotoneVariance">true</Parameter>
        <Parameter name="SensitivityTemplate">COMM_FD</Parameter>
    </EngineParameters>
</Product>
\end{minted}
\caption{Configuration for Product FxOptionAmerican, Model GarmanKohlhagen, Engine FdBlackScholesVanillaEngine}
\label{lst:peconfig_FxOptionAmerican_GarmanKohlhagen_FdBlackScholesVanillaEngine}
\end{longlisting}

GarmanKohlhagen/BaroneAdesiWhaleyApproximationEngine builds a BaroneAdesiWhaleyApproximationEngine. A sample configuration is shown in listing
\ref{lst:peconfig_FxOptionAmerican_GarmanKohlhagen_BaroneAdesiWhaleyApproximationEngine}

The parameters have the following meaning:

\begin{itemize}
\item SensitivityTemplate [optional]: the sensitivity template to use 
\end{itemize}

\begin{longlisting}
\begin{minted}[fontsize=\footnotesize]{xml}
<Product type="FxOptionAmerican">
    <Model>GarmanKohlhagen</Model>
    <ModelParameters/>
    <Engine>BaroneAdesiWhaleyApproximationEngine</Engine>
    <EngineParameters>
        <Parameter name="SensitivityTemplate">COMM_Analytical</Parameter>
    </EngineParameters>
</Product>
\end{minted}
\caption{Configuration for Product FxOptionAmerican, Model GarmanKohlhagen, Engine BaroneAdesiWhaleyApproximationEngine}
\label{lst:peconfig_FxOptionAmerican_GarmanKohlhagen_BaroneAdesiWhaleyApproximationEngine}
\end{longlisting}

%--------------------------------------------------------
\subsubsection{Product Type: FxDoubleTouchOption}
%--------------------------------------------------------

Used by trade type: FxDoubleTouchOption

Available Model/Engine pairs: GarmanKohlhagen/AnalyticDoubleBarrierBinaryEngine

Engine description:

GarmanKohlhagen/AnalyticDoubleBarrierBinaryEngine builds a AnalyticDoubleBarrierBinaryEngine. A sample configuration is
shown in listing \ref{lst:peconfig_FxDoubleTouchOption_GarmanKohlhagen_AnalyticDoubleBarrierBinaryEngine}

The parameters have the following meaning:

\begin{itemize}
\item SensitivityTemplate [optional]: the sensitivity template to use 
\end{itemize}

\begin{longlisting}
\begin{minted}[fontsize=\footnotesize]{xml}
<Product type="FxDoubleTouchOption">
    <Model>GarmanKohlhagen</Model>
    <ModelParameters/>
    <Engine>AnalyticDoubleBarrierBinaryEngine</Engine>
    <EngineParameters>
        <Parameter name="SensitivityTemplate">FX_Analytical</Parameter>
    </EngineParameters>
</Product>
\end{minted}
\caption{Configuration for Product FxDoubleTouchOption, Model GarmanKohlhagen, Engine AnalyticDoubleBarrierBinaryEngine}
\label{lst:peconfig_FxDoubleTouchOption_GarmanKohlhagen_AnalyticDoubleBarrierBinaryEngine}
\end{longlisting}

%--------------------------------------------------------
\subsubsection{Product Type: MultiLegOption}
%--------------------------------------------------------

Used by trade type: MultiLegOption

Available Model/Engine pairs:

\begin{itemize}
\item CrossAssetModel/MC
\item CrossAssetModel/AMC
\end{itemize}

Engine description:

CrossAssetModel/MC builds a McMultiLegOptionEngine. A sample configuration is shown in listing
\ref{lst:peconfig_MultiLegOption_CrossAssetModel_MC}

The parameters have the following meaning:

\begin{itemize}
\item IrCalibration: Bootstrap, BestFit, None
\item IrCalibrationStrategy: CoterminalDealStrike, CoterminalATM
\item ReferenceCalibrationGrid: An optional grid, only one calibration instrument per interval is kept
\item IrReversion: The mean reversion (given per ccy)
\item IrReversionType: Hagan, HullWhite (given per ccy)
\item IrVolatility: The volatility (start value for calibration if calibrated, given per ccy)
\item IrVolatilityType: Hagan, HullWhite (given per ccy)
\item FxVolatility: The fx volatility (start value for calibration if calibrated, given per ccy)
\item Corr\_Key1\_key2: The correlations for ir/fx keys (multiple entries possible)
\item ShiftHorizon: Shift horizon for LGM model as fraction of deal maturity
\item Tolerance: Error tolerance for calibration
\item Training.Sequence: The sequence type for the traning phase, can be MersenneTwister+, MersenneTwisterAntithetc+,
  Sobol+, Burley2020Sobol+, SobolBrownianBridge+, Burley2020SobolBrownianBridge+
\item Training.Seed: The seed for the random number generation in the training phase
\item Training.Samples: The number of samples to be used for the training phase
\item Pricing.Sequence: The sequence type for the pricing phase, same values allowed as for training
\item Training.BasisFunction: The type of basis function system to be used for the regression analysis, can be
  Monomial+, Laguerre+, Hermite+, Hyperbolic+, Legendre+, Chbyshev+, Chebyshev2nd+
\item BasisFunctionOrder: The order of the basis function system to be used
\item Pricing.Seed: The seed for the random number generation in the pricing
\item Pricing.Samples: The number of samples to be used for the pricing phase. If this number is zero, no pricing run is
  performed, instead the (T0) NPV is estimated from the training phase (this result is used to fill the T0 slice of the
  NPV cube)
\item BrownianBridgeOrdering: variate ordering for Brownian bridges, can be Steps+, Factors+, Diagonal+
\item SobolDirectionIntegers: direction integers for Sobol generator, can be Unit+, Jaeckel+, SobolLevitan+,
  SobolLevitanLemieux+, JoeKuoD5+, JoeKuoD6+, JoeKuoD7+, Kuo+, Kuo2+, Kuo3+
\item MinObsDate: if true the conditional expectation of each cashflow is taken from the minimum possible observation
  date (i.e. the latest exercise or simulation date before the cashflow's event date); recommended setting is true+
\item RegressorModel: Simple, LaggedFX. If not given, it defaults to Simple. Depending on the choice the regressor is
  built as follows:
  \begin{itemize}
    \item Simple: For an observation date the full model state observed on this date is included in the regressor. No
      past states are included though.
    \item LaggedFX: For an observation date the full model state observed on this date is included in the regressor. In
      addition, past FX states that are relevant for future cashflows are included. For example, for a FX resettable
      cashflow the FX state observed on the FX reset date is included.
  \end{itemize}
\item SensitivityTemplate [optional]: the sensitivity template to use
\end{itemize}

\begin{longlisting}
\begin{minted}[fontsize=\footnotesize]{xml}
<Product type="MultiLegOption">
    <Model>CrossAssetModel</Model>
    <ModelParameters>
        <Parameter name="IrCalibration">Bootstrap</Parameter>
        <Parameter name="IrCalibrationStrategy">CoterminalDealStrike</Parameter>
        <Parameter name="FxCalibration">Bootstrap</Parameter>
        <Parameter name="ReferenceCalibrationGrid">400,3M</Parameter>
        <Parameter name="IrReversion_EUR">0.0</Parameter>
        <Parameter name="IrReversionType_EUR">HullWhite</Parameter>
        <Parameter name="IrVolatility_EUR">0.01</Parameter>
        <Parameter name="IrVolatilityType_EUR">Hagan</Parameter>
        <Parameter name="IrReversion_GBP">0.0</Parameter>
        <Parameter name="IrReversionType_GBP">HullWhite</Parameter>
        <Parameter name="IrVolatility_GBP">0.01</Parameter>
        <Parameter name="IrVolatilityType_GBP">Hagan</Parameter>
        <Parameter name="FxVolatility_GBP">0.08</Parameter>
        <Parameter name="Corr_IR:EUR_FX:GBP">0.23</Parameter>
        <Parameter name="ShiftHorizon">0.5</Parameter>
        <Parameter name="Tolerance">0.20</Parameter>
    </ModelParameters>
    <Engine>MC</Engine>
    <EngineParameters>
        <Parameter name="Training.Sequence">MersenneTwisterAntithetic</Parameter>
        <Parameter name="Training.Seed">42</Parameter>
        <Parameter name="Training.Samples">10000</Parameter>
        <Parameter name="Training.BasisFunction">Monomial</Parameter>
        <Parameter name="Training.BasisFunctionOrder">6</Parameter>
        <Parameter name="Pricing.Sequence">SobolBrownianBridge</Parameter>
        <Parameter name="Pricing.Seed">17</Parameter>
        <Parameter name="Pricing.Samples">0</Parameter>
        <Parameter name="BrownianBridgeOrdering">Steps</Parameter>
        <Parameter name="SobolDirectionIntegers">JoeKuoD7</Parameter>
        <Parameter name="MinObsDate">true</Parameter>
        <Parameter name="RegressorModel">Simple</Parameter>
        <Parameter name="SensitivityTemplate">IR_MC</Parameter>
    </EngineParameters>
</Product>
\end{minted}
\caption{Configuration for Product MultiLegSwaption, Model CrossAssetModel, Engine MC}
\label{lst:peconfig_MultiLegOption_CrossAssetModel_MC}
\end{longlisting}

LGM/AMC builds a McMultiLegOptionEngine for use in AMC simulations. We refer to the AMC module documentation for further
details.

%--------------------------------------------------------
\subsubsection{Product Type: Swap}
%--------------------------------------------------------

Used by trade type: Swap for single-currency swaps

Available Model/Engine pairs:

\begin{itemize}
\item DiscountedCashflows/DiscountingSwapEngine
\item DiscountedCashflows/DiscountingSwapEngineOptimised  
\item CrossAssetModel/AMC
\end{itemize}

Engine description:

DiscountedCashflows/DiscountingSwapEngine builds a DiscountingSwapEngine. A sample configuration is shown in listing
\ref{lst:peconfig_Swap_DiscountedCashflows_DiscountingSwapEngine}

The parameters have the following meaning:

\begin{itemize}
\item SensitivityTemplate [optional]: the sensitivity template to use 
\end{itemize}

\begin{longlisting}
\begin{minted}[fontsize=\footnotesize]{xml}
<Product type="Swap">
    <Model>DiscountedCashflows</Model>
    <ModelParameters/>
    <Engine>DiscountingSwapEngine</Engine>
    <EngineParameters>
        <Parameter name="SensitivityTemplate">IR_Analytical</Parameter>
    </EngineParameters>
</Product>
\end{minted}
\caption{Configuration for Product Swap, Model DiscountedCashflows, Engine DiscountingSwapEngine}
\label{lst:peconfig_Swap_DiscountedCashflows_DiscountingSwapEngine}
\end{longlisting}

DiscountedCashflows/DiscountingSwapEngineOptimised builds a DiscountingSwapEngineOptimised. This is a speed-optimized
version of the DiscountingSwapEngine. In general, results should be validated against DiscountingSwapEngine.A sample
configuration is shown in listing \ref{lst:peconfig_Swap_DiscountedCashflows_DiscountingSwapEngineOptimised}

The parameters have the following meaning:

\begin{itemize}
\item SensitivityTemplate [optional]: the sensitivity template to use 
\end{itemize}

\begin{longlisting}
\begin{minted}[fontsize=\footnotesize]{xml}
<Product type="Swap">
    <Model>DiscountedCashflows</Model>
    <ModelParameters/>
    <Engine>DiscountingSwapEngineOptimised</Engine>
    <EngineParameters>
        <Parameter name="SensitivityTemplate">IR_Analytical</Parameter>
    </EngineParameters>
</Product>
\end{minted}
\caption{Configuration for Product Swap, Model DiscountedCashflows, Engine DiscountingSwapEngineOptimised}
\label{lst:peconfig_Swap_DiscountedCashflows_DiscountingSwapEngineOptimised}
\end{longlisting}

CrossAssetModel/AMC builds a McLgmSwapEngine for use in AMC simulations. We refer to the AMC module documentation for
further details.

%--------------------------------------------------------
\subsubsection{Product Type: CurrencySwap}
%--------------------------------------------------------

Used by trade type: Swap for cross-currency swaps

Available Model/Engine pairs:

\begin{itemize}
\item DiscountedCashflows/DiscountingCrossCurrencySwapEngine
\item CrossAssetModel/AMC
\end{itemize}

Engine description:

DiscountedCashflows/DiscountingCrossCurrencySwapEngine builds a DiscountingCurrencySwapEngine. A sample configuration is
shown in listing \ref{lst:peconfig_Swap_DiscountedCashflows_DiscountingCurrencySwapEngine}

The parameters have the following meaning:

\begin{itemize}
\item SensitivityTemplate [optional]: the sensitivity template to use 
\end{itemize}

\begin{longlisting}
\begin{minted}[fontsize=\footnotesize]{xml}
<Product type="CurrencySwap">
    <Model>DiscountedCashflows</Model>
    <ModelParameters/>
    <Engine>DiscountingCurrencySwapEngine</Engine>
    <EngineParameters>
        <Parameter name="SensitivityTemplate">IR_Analytical</Parameter>
    </EngineParameters>
</Product>
\end{minted}
\caption{Configuration for Product Swap, Model DiscountedCashflows, Engine DiscountingCurrencySwapEngine}
\label{lst:peconfig_Swap_DiscountedCashflows_DiscountingCurrencySwapEngine}
\end{longlisting}

CrossAssetModel/AMC builds a McCamCurrencySwapEngine for use in AMC simulations. We refer to the AMC module
documentation for further details.

%--------------------------------------------------------
\subsubsection{Product Type: RiskParticipationAgreement\_Vanilla}
%--------------------------------------------------------

Used by trade type: RiskParticipationAgreement with

\begin{itemize}
\item exactly one fixed and one floating leg with opposite payer flags and
\item only fixed, ibor, ois (comp, avg) coupons allowed, no cap / floors, no in arrears fixings for ibor
\end{itemize}

Available Model/Engine pairs: Black/Analytic

Engine description:

Black/Analytic builds a AnalyticBlackRiskParticipationAgreementEngine. A sample configuration is shown in listing
\ref{lst:peconfig_RiskParticipationAgreement_Vanilla_Black_Analytic}

The parameters have the following meaning:

\begin{itemize}
\item Reversion: the mean reversion parameter to use for underlying matching in a HW1F model
\item MatchUnderlyingTenor: whether to match the trade underlying tenor in scenario / sensi calculations
\item AlwaysRecomputeOptionRepresentation: whether to recompute underlying representation on each scenario or keep the initial representation across all scenarios
\item MaxGapDays: the maximum gap between option expiry dates used for the underlying representation
\item MaxDiscretizationPoints: the maximum number of discretization points used for underlying representation
\item SensitivityTemplate [optional]: the sensitivity template to use 
\end{itemize}

\begin{longlisting}
\begin{minted}[fontsize=\footnotesize]{xml}
 <Product type="RiskParticipationAgreement_Vanilla">
    <Model>Black</Model>
    <ModelParameters>
        <Parameter name="Reversion">0.0</Parameter>
        <Parameter name="MatchUnderlyingTenor">true</Parameter>
    </ModelParameters>
    <Engine>Analytic</Engine>
    <EngineParameters>
        <Parameter name="AlwaysRecomputeOptionRepresentation">false</Parameter>
        <Parameter name="MaxGapDays">400</Parameter>
        <Parameter name="MaxDiscretisationPoints">20</Parameter>
        <Parameter name="SensitivityTemplate">IR_Semianalytical</Parameter>
    </EngineParameters>
</Product>
\end{minted}
\caption{Configuration for Product RiskParticipationAgreement\_Vanilla, Model Black, Engine Analytic}
\label{lst:peconfig_RiskParticipationAgreement_Vanilla_Black_Analytic}
\end{longlisting}

%--------------------------------------------------------
\subsubsection{Product Type: RiskParticipationAgreement\_Vanilla\_XCcy}
%--------------------------------------------------------

Used by trade type: RiskParticipationAgreement with

\begin{itemize}
\item two legs in different currencies with arbitrary coupons allowed, no
  optionData though (as in structured variant)
\end{itemize}

Available Model/Engine pairs: Black/Analytic

Engine description:

Black/Analytic builds a AnalyticXCcyBlackRiskParticipationAgreementEngine. A sample configuration is shown in listing
\ref{lst:peconfig_RiskParticipationAgreement_Vanilla_XCcy_Black_Analytic}

The parameters have the following meaning:

\begin{itemize}
\item AlwaysRecomputeOptionRepresentation: whether to recompute underlying representation on each scenario or keep the initial representation across all scenarios
\item MaxGapDays: the maximum gap between option expiry dates used for the underlying representation
\item MaxDiscretizationPoints: the maximum number of discretization points used for underlying representation
\item SensitivityTemplate [optional]: the sensitivity template to use 
\end{itemize}

\begin{longlisting}
\begin{minted}[fontsize=\footnotesize]{xml}
<Product type="RiskParticipationAgreement_Vanilla_XCcy">
    <Model>Black</Model>
    <ModelParameters/>
    <Engine>Analytic</Engine>
    <EngineParameters>
        <Parameter name="AlwaysRecomputeOptionRepresentation">false</Parameter>
        <Parameter name="MaxGapDays">400</Parameter>
        <Parameter name="MaxDiscretisationPoints">20</Parameter>
        <Parameter name="SensitivityTemplate">FX_Semianalytical</Parameter>
    </EngineParameters>
</Product>
\end{minted}
\caption{Configuration for Product RiskParticipationAgreement\_Vanilla\_XCcy, Model Black, Engine Analytic}
\label{lst:peconfig_RiskParticipationAgreement_Vanilla_XCcy_Black_Analytic}
\end{longlisting}

%--------------------------------------------------------
\subsubsection{Product Type: RiskParticipationAgreement\_Structured}
%--------------------------------------------------------

Used by trade type: RiskParticipationAgreement with

\begin{itemize}
\item arbitrary number of fixed, floating, cashflow legs
\item only fixed, ibor coupons, ois (comp, avg), simple cashflows allowed, but possibly capped / floored,
  as naked option, with in arrears fixing for ibor
\item with optionData (i.e. callable underlying), as naked option (i.e. swaption)
\end{itemize} 

Available Model/Engine pairs: LGM/Grid

Engine description:

LGM/Grid builds a NumericLgmRiskParticipationAgreementEngine. A sample configuration is shown in listing
\ref{lst:peconfig_RiskParticipationAgreement_Structured_LGM_Grid}

The parameters have the following meaning:

\begin{itemize}
\item Calibration: Bootstrap, BestFit, None
\item CalibrationStrategy: CoterminalDealStrike, CoterminalATM
\item ReferenceCalibrationGrid: An optional grid, only one calibration instrument per interval is kept
\item Reversion: The mean reversion
\item ReversionType: Hagan, HullWhite
\item Volatility: The volatility (start value for calibration if calibrated)
\item VolatilityType: Hagan, HullWhite
\item ShiftHorizon: Shift horizon for LGM model as fraction of deal maturity
\item Tolerance: Error tolerance for calibration
\item sy, sx: Number of covered standard deviations (notation as in Hagan's paper)
\item ny, nx: Number of grid points for numerical integration (notation as in Hagan's paper)
\item SensitivityTemplate [optional]: the sensitivity template to use 
\end{itemize}

\begin{longlisting}
\begin{minted}[fontsize=\footnotesize]{xml}
<Product type="RiskParticipationAgreement_Structured">
    <Model>LGM</Model>
    <ModelParameters>
        <Parameter name="Calibration">Bootstrap</Parameter>
        <Parameter name="CalibrationStrategy">CoterminalDealStrike</Parameter>
        <Parameter name="ReferenceCalibrationGrid">400,3M</Parameter>
        <Parameter name="Reversion">0.0</Parameter>
        <Parameter name="ReversionType">HullWhite</Parameter>
        <Parameter name="Volatility">0.01</Parameter>
        <Parameter name="VolatilityType">Hagan</Parameter>
        <Parameter name="ShiftHorizon">0.5</Parameter>
        <Parameter name="Tolerance">0.02</Parameter>
    </ModelParameters>
    <Engine>Grid</Engine>
    <EngineParameters>
        <Parameter name="sy">5.0</Parameter>
        <Parameter name="ny">30</Parameter>
        <Parameter name="sx">5.0</Parameter>
        <Parameter name="nx">30</Parameter>
        <Parameter name="MaxGapDays">400</Parameter>
        <Parameter name="MaxDiscretisationPoints">20</Parameter>
        <Parameter name="SensitivityTemplate">IR_FD</Parameter>
    </EngineParameters>
</Product>
\end{minted}
\caption{Configuration for Product RiskParticipationAgreement\_Structured, Model LGM, Engine Grid}
\label{lst:peconfig_RiskParticipationAgreement_Structured_LGM_Grid}
\end{longlisting}

%--------------------------------------------------------
\subsubsection{Product Type: RiskParticipationAgreement\_TLock}
%--------------------------------------------------------

Used by trade type: RiskParticipationAgreement with TLock underlying.

Available Model/Engine pairs: LGM/Grid

Engine description:

LGM/Grid builds a NumericLgmRiskParticipationAgreementEngine. A sample configuration is shown in listing
\ref{lst:peconfig_RiskParticipationAgreement_TLock_LGM_Grid}

The parameters have the following meaning:

\begin{itemize}
\item Calibration: Bootstrap, BestFit, None
\item CalibrationStrategy: CoterminalDealStrike, CoterminalATM
\item ReferenceCalibrationGrid: An optional grid, only one calibration instrument per interval is kept
\item Reversion: The mean reversion
\item ReversionType: Hagan, HullWhite
\item Volatility: The volatility (start value for calibration if calibrated)
\item VolatilityType: Hagan, HullWhite
\item ShiftHorizon: Shift horizon for LGM model as fraction of deal maturity
\item Tolerance: Error tolerance for calibration
\item sy, sx: Number of covered standard deviations (notation as in Hagan's paper)
\item ny, nx: Number of grid points for numerical integration (notation as in Hagan's paper)
\item SensitivityTemplate [optional]: the sensitivity template to use 
\end{itemize}

\begin{longlisting}
\begin{minted}[fontsize=\footnotesize]{xml}
<Product type="RiskParticipationAgreement_TLock">
    <Model>LGM</Model>
    <ModelParameters>
        <Parameter name="Calibration">Bootstrap</Parameter>
        <Parameter name="CalibrationStrategy">CoterminalATM</Parameter>
        <Parameter name="ReferenceCalibrationGrid">400,3M</Parameter>
        <Parameter name="Reversion">0.0</Parameter>
        <Parameter name="ReversionType">HullWhite</Parameter>
        <Parameter name="Volatility">0.01</Parameter>
        <Parameter name="VolatilityType">Hagan</Parameter>
        <Parameter name="ShiftHorizon">0.5</Parameter>
        <Parameter name="Tolerance">0.02</Parameter>
        <Parameter name="CalibrationInstrumentSpacing">3M</Parameter>
    </ModelParameters>
    <Engine>Grid</Engine>
    <EngineParameters>
        <Parameter name="sy">5.0</Parameter>
        <Parameter name="ny">30</Parameter>
        <Parameter name="sx">5.0</Parameter>
        <Parameter name="nx">30</Parameter>
        <Parameter name="TimeStepsPerYear">24</Parameter>
        <Parameter name="SensitivityTemplate">IR_FD</Parameter>
    </EngineParameters>
</Product>
\end{minted}
\caption{Configuration for Product RiskParticipationAgreement\_TLock, Model LGM, Engine Grid}
\label{lst:peconfig_RiskParticipationAgreement_TLock_LGM_Grid}
\end{longlisting}


%--------------------------------------------------------
\subsubsection{Product Type: ScriptedTrade}
%--------------------------------------------------------

Used by trade type: ScriptedTrade (and trades wrapped as scripted trade)

Available Model/Engine pairs: ScriptedTrade/ScriptedTrade

Engine description:

ScriptedTrade/ScriptedTrade builds a ScriptedInstrumentPricingEngine (and variants). We refer to the scripted trade
module documentation for a sample configuration and more details.

%--------------------------------------------------------
\subsubsection{Global Parameters}
%--------------------------------------------------------

In addition to product specific settings there is also a block with global parameters with the following meaning:

\begin{itemize}
\item ContinueOnCalibrationError: If set to true an exceedence of a prescribed model calibration tolerance (for e.g. the
  LGM model) will not cause the trade building to fail, instead a warning is logged and the trade is processed
  anyway. Optional, defaults to false.
\item Calibrate: If false, model calibration is disabled. This flag is usually not present in a user configuration, but
  only used internally for certain workflows within ORE which do not require a model calibration. Optional, defaults to
  true.
\item GenerateAdditionalResults: If false, the generation of additional results within pricing engines will be
  suppressed (for those pricing engines which support this). This flag is usually not present in a user configuration,
  but only used internally to improve the performance for processes which only rely on the NPV as a result from pricing
  engines, e.g. when repricing trades under sensitivity or stress scenarios. Option, defaults to false.
\item RunType: Set automatically. One of NPV, SensitivityDelta, SensitivityDeltaGamma, Stress, Exposure, Capital,
  TradeDetails, PortfolioAnalyser, HistoricalPnL, BondSpreadImply, AbsMaturityUpdate depending on the context for which
  a portfolio was built. Might also be left empty. This is used by some pricing engines to adapt to certain run
  types. E.g. a first order sensitivity pnl expansion might be used for a SensitivityDelta run by an engine which is
  able to compute analytical or AAD first order sensitivities.
\end{itemize}


%--------------------------------------------------------
\subsection{Simulation: {\tt simulation.xml}}\label{sec:simulation}
%--------------------------------------------------------

This file determines the behaviour of the risk factor simulation (scenario generation) module.
It is structured in three blocks of data.

\begin{listing}[H]
%\hrule\medskip
\begin{minted}[fontsize=\footnotesize]{xml}
<Simulation>
  <Parameters> ... </Parameters>
  <CrossAssetModel> ... </CrossAssetModel>
  <Market> ... </Market>
</Simulation>
\end{minted}
\caption{Simulation configuration}
\label{lst:simulation_configuration}
\end{listing}

Each of the three blocks is sketched in the following.

\subsubsection{Parameters}\label{sec:sim_params}

Let us discuss this section using the following example

\begin{listing}[H]
%\hrule\medskip
\begin{minted}[fontsize=\footnotesize]{xml}
<Parameters>
  <Grid>80,3M</Grid>
  <Calendar>EUR,USD,GBP,CHF</Calendar>
  <DayCounter>ACT/ACT</DayCounter>
  <Sequence>SobolBrownianBridge</Sequence>
  <Seed>42</Seed>
  <Samples>1000</Samples>
  <Ordering>Steps</Ordering>
  <DirectionIntegers>JoeKuoD7</DirectionIntegers>
  <!-- The following two nodes are optional -->
  <CloseOutLag>2W</CloseOutLag>
  <MporMode>StickyDate</MporMode>
</Parameters>
\end{minted}
\caption{Simulation configuration}
\label{lst:simulation_params_configuration}
\end{listing}

\begin{itemize}
\item {\tt Grid:} Specifies the simulation time grid, here 80 quarterly steps.\footnote{For exposure calculation under DIM, the second parameter has to match the Margin Period of Risk, i.e. if {\tt MarginPeriodOfRisk} is set to for instance {\tt 2W} in a netting set definition in {\tt netting.xml}, then one has to set {\tt Grid} to for instance {\tt 80,2W}.}
\item {\tt Calendar:} Calendar or combination of calendars used to adjust the dates of the grid. Date adjustment is
required because the simulation must step over 'good' dates on which index fixings can be stored.
%\item {\tt Scenario: } Choose between {\em Simple } and {\em Complex } implementations, the latter optimized for
% more efficient memory usage. \todo[inline]{Remove Scenario choice}
\item {\tt DayCounter:} Day count convention used to translate dates to times. Optional, defaults to ActualActual ISDA.
\item {\tt Sequence:} Choose random sequence generator ({\em MersenneTwister, MersenneTwisterAntithetic, Sobol,
  Burley2020Sobol, SobolBrownianBridge, Burley2020SobolBrownianBridge}).
\item {\tt Seed:} Random number generator seed
\item {\tt Samples:} Number of Monte Carlo paths to be produced
%\item {\tt Fixings: } Choose whether fixings should be simulated or not, and if so which fixing simulation method to
use ({\em Backward, Forward, BestOfForwardBackward, InterpolatedForwardBackward}), which number of forward horizon days
to use if one of the {\em Forward } related methods is chosen.
\item {\tt Ordering:} If the sequence type {\em SobolBrownianBridge} or {\em Burley2020SobolBrownianBridge} is used,
  ordering of variates ({\em Factors, Steps, Diagonal})
\item {\tt DirectionIntegers:} If the sequence type {\em SobolBrownianBridge}, {\em Burley2020SobolBrownianBridge}, {\em
  Sobol} or {\em Burley2020Sobol} is used, type of direction integers in Sobol generator ({\em Unit, Jaeckel,
  SobolLevitan, SobolLevitanLemieux, JoeKuoD5, JoeKuoD6, JoeKuoD7, Kuo, Kuo2, Kuo3})
\item {\tt CloseOutLag}: If this tag is present, this specifies the close-out period length (e.g. 2W) used; otherwise no close-out grid is built. The close-out grid is an auxiliary time grid that is offset from the main default date grid by the close-out period, typically set to the applicable margin period of risk. If present, it is used to evolve the portfolio value and determine close-out values associated with the preceding default date valuation.
\item {\tt MporMode}: This tag is expected if the previous one is present, permissible values are then {\tt StickyDate} and {\tt ActualDate}. {\tt StickyDate} means that only market data is evolved from the default date to close-out date for close-out date valuation, the valuation as of date remains unchanged and trades do not ``age'' over the period. As a consequence, exposure evolutions will not show spikes caused by cash flows within the close-out period. {\tt ActualDate} means that trades will also age over the close-out period so that one can experience exposure evolution spikes due to cash flows. 
\end{itemize}

\subsubsection{Model}\label{sec:sim_model}

The {\tt CrossAssetModel} section determines the cross asset model's number of currencies covered, composition, and each
component's calibration. It is currently made of 
\begin{itemize}
\item a sequence of LGM models for each currency (say $n_c$ currencies), 
\item $n_c-1$ FX models for each exchange rate to the base currency, 
\item $n_e$ equity models,
\item $n_i$ inflation models, 
\item $n_{cr}$ credit models, 
\item $n_{com}$ commodity models, 
\item a specification of the correlation structure between all components.
\end{itemize}

\medskip The simulated currencies are specified as follows, with clearly identifying the domestic currency which is also
the target currency for all FX models listed subsequently. If the portfolio requires more currencies to be simulated,
this will lead to an exception at run time, so that it is the user's responsibility to make sure that the list of
currencies here is sufficient. The list can be larger than actually required by the portfolio. This will not lead to any
exceptions, but add to the run time of ORE.

\begin{listing}[H]
%\hrule\medskip
\begin{minted}[fontsize=\footnotesize]{xml}
<CrossAssetModel>
  <DomesticCcy>EUR</DomesticCcy>
  <Currencies>
    <Currency>EUR</Currency>
    <Currency>USD</Currency>
    <Currency>GBP</Currency>
    <Currency>CHF</Currency>
    <Currency>JPY</Currency>
  </Currencies>
  <Equities>
	<!-- ... -->
  </Equities>
  <InflationIndices>
	<!-- ... -->
  </InflationIndices>
  <CreditNames>
	<!-- ... -->
  </CreditNames>
  <Commodities>
	<!-- ... -->
  </Commodities>
  <BootstrapTolerance>0.0001</BootstrapTolerance>
  <Measure>LGM</Measure><!-- Choices: LGM, BA -->
  <Discretization>Exact</Discretization>
  <!-- ... -->
</CrossAssetModel>
\end{minted}
\caption{Simulation model currencies configuration}
\label{lst:simulation_model_currencies_configuration}
\end{listing}
 
Bootstrap tolerance is a global parameter that applies to the calibration of all model components. If the calibration
error of any component exceeds this tolerance, this will trigger an exception at runtime, early in the ORE process.

The Measure tag allows switching between the LGM and the Bank Account (BA) measure for the risk-neutral market simulations using the Cross Asset Model. Note that within LGM one can shift the horizon (see ParameterTransformation below) to effectively switch to a T-Forward measure.

The Discretization tag chooses between time discretization schemes for the risk factor evolution. {\em Exact} means
exploiting the analytical tractability of the model to avoid any time discretization error. {\em Euler} uses a naive
time discretization scheme which has numerical error and requires small time steps for accurate results (useful for
testing purposes or if more sophisticated component models are used.)
 
\medskip

Each interest rate model is specified by a block as follows

\begin{listing}[H]
%\hrule\medskip
\begin{minted}[fontsize=\footnotesize]{xml}
<CrossAssetModel>	
  <!-- ... -->
  <InterestRateModels>
    <LGM ccy="default">
      <CalibrationType>Bootstrap</CalibrationType>
      <Volatility>
        <Calibrate>Y</Calibrate>
        <VolatilityType>Hagan</VolatilityType>
        <ParamType>Piecewise</ParamType>
        <TimeGrid>1.0,2.0,3.0,4.0,5.0,7.0,10.0</TimeGrid>
        <InitialValue>0.01,0.01,0.01,0.01,0.01,0.01,0.01,0.01<InitialValue>
      </Volatility>
      <Reversion>
        <Calibrate>N</Calibrate>
        <ReversionType>HullWhite</ReversionType>
        <ParamType>Constant</ParamType>
        <TimeGrid/>
        <InitialValue>0.03</InitialValue>
      </Reversion>
      <CalibrationSwaptions>
        <Expiries>1Y,2Y,4Y,6Y,8Y,10Y,12Y,14Y,16Y,18Y,19Y</Expiries>
        <Terms>19Y,18Y,16Y,14Y,12Y,10Y,8Y,6Y,4Y,2Y,1Y</Terms>
        <Strikes/>
      </CalibrationSwaptions>
      <ParameterTransformation>
        <ShiftHorizon>0.0</ShiftHorizon>
        <Scaling>1.0</Scaling>
      </ParameterTransformation>
    </LGM>
    <LGM ccy="EUR">
      <!-- ... -->
    </LGM>
    <LGM ccy="USD">
      <!-- ... -->
    </LGM>
  </InterestRateModels>	
  <!-- ... -->		
</CrossAssetModel>
\end{minted}
\caption{Simulation model IR configuration}
\label{lst:simulation_model_ir_configuration}
\end{listing}

We have LGM sections by currency, but starting with a section for currency 'default'. As the name implies, this is used
as default configuration for any currency in the currency list for which we do not provide an explicit
parametrisation. Within each LGM section, the interpretation of elements is as follows:

\begin{itemize}
\item {\tt CalibrationType: } Choose between {\em Bootstrap} and {\em BestFit}, where Bootstrap is chosen when we expect
to be able to achieve a perfect fit (as with calibration of piecewise volatility to a series of co-terminal Swaptions)
\item {\tt Volatility/Calibrate: } Flag to enable/disable calibration of this particular parameter
\item {\tt Volatility/VolatilityType: } Choose volatility parametrisation a la {\em HullWhite} or {\em Hagan}
\item {\tt Volatility/ParamType: } Choose between {\em Constant} and {\em Piecewise}
\item {\tt Volatility/TimeGrid: } Initial time grid for this parameter, can be left empty if ParamType is Constant
\item {\tt Volatility/InitialValue: } Vector of initial values, matching number of entries in time (for CalibrationType {\em BestFit} this should be one more entry than the {\tt Volatility/TimeGrid} entries, for {\em Bootstrap} this is ignored), or single value if the time grid is empty
\item {\tt Reversion/Calibrate: } Flag to enable/disable calibration of this particular parameter
\item {\tt Reversion/VolatilityType: } Choose reversion parametrisation a la {\em HullWhite} or {\em Hagan}
\item {\tt Reversion/ParamType: } Choose between {\em Constant} and {\em Piecewise}
\item {\tt Reversion/TimeGrid: } Initial time grid for this parameter, can be left empty if ParamType is Constant
\item {\tt Reversion/InitialValue: } Vector of initial values, matching number of entries in time, or single value if
the time grid is empty
\item {\tt CalibrationSwaptions: } Choice of calibration instruments by expiry, underlying Swap term and strike. There have to be at least one more calibration options configured than {\tt Volatility/TimeGrid} entries were given.
\item {\tt ParameterTransformation: } LGM model prices are invariant under scaling and shift transformations
\cite{Lichters} with advantages for numerical convergence of results in long term simulations. These transformations can
be chosen here. Default settings are shiftHorizon 0 (time in years) and scaling factor 1.
\end{itemize}

The reason for having to specify one more {\tt Volatility/InitialValue} entries than {\tt Volatility/TimeGrid} entries (and at least one more calibration option than {\tt Volatility/TimeGrid} entries) is the fact that the intervals defined by the {\tt Volatility/TimeGrid} entries are spanning from $[0,t_1],[t_1,t_2]\ldots[t_n,\infty]$, which results in $n+1$ intervals.

\medskip

Each FX model is specified by a block as follows

\begin{listing}[H]
%\hrule\medskip
\begin{minted}[fontsize=\footnotesize]{xml}
<CrossAssetModel>	
  <!-- ... -->
  <ForeignExchangeModels>
    <CrossCcyLGM foreignCcy="default">
      <DomesticCcy>EUR</DomesticCcy>
      <CalibrationType>Bootstrap</CalibrationType>
      <Sigma>
        <Calibrate>Y</Calibrate>
        <ParamType>Piecewise</ParamType>
        <TimeGrid>1.0,2.0,3.0,4.0,5.0,7.0,10.0</TimeGrid>
        <InitialValue>0.1,0.1,0.1,0.1,0.1,0.1,0.1,0.1</InitialValue>
      </Sigma>
      <CalibrationOptions>
        <Expiries>1Y,2Y,3Y,4Y,5Y,10Y</Expiries>
        <Strikes/>
      </CalibrationOptions>
    </CrossCcyLGM>
    <CrossCcyLGM foreignCcy="USD">
      <!-- ... -->
    </CrossCcyLGM>
    <CrossCcyLGM foreignCcy="GBP">
      <!-- ... -->
    </CrossCcyLGM>
    <!-- ... -->
  </ForeignExchangeModels>
  <!-- ... -->
<CrossAssetModel>	
\end{minted}
\caption{Simulation model FX configuration}
\label{lst:simulation_model_fx_configuration}
\end{listing}

CrossCcyLGM sections are defined by foreign currency, but we also support a default configuration as above for the IR
model parametrisations.  Within each CrossCcyLGM section, the interpretation of elements is as follows:

\begin{itemize}
\item {\tt DomesticCcy: } Domestic currency completing the FX pair
\item {\tt CalibrationType: } Choose between {\em Bootstrap} and {\em BestFit} as in the IR section
\item {\tt Sigma/Calibrate: } Flag to enable/disable calibration of this particular parameter
\item {\tt Sigma/ParamType: } Choose between {\em Constant} and {\em Piecewise}
\item {\tt Sigma/TimeGrid: } Initial time grid for this parameter, can be left empty if ParamType is Constant
\item {\tt Sigma/InitialValue: } Vector of initial values, matching number of entries in time (for CalibrationType {\em BestFit} this should be one more entry than the {\tt Sigma/TimeGrid} entries, for {\em Bootstrap} this is ignored), or single value if the time grid is empty
\item {\tt CalibrationOptions: } Choice of calibration instruments by expiry and strike, strikes can be empty (implying
the default, ATMF options), or explicitly specified (in terms of FX rates as absolute strike values, in delta notation
such as $\pm 25D$, $ATMF$ for at the money). There have to be at least one more calibration options configured than {\tt Sigma/TimeGrid} entries were given
\end{itemize}


\medskip

Each equity model is specified by a block as follows

\begin{listing}[H]
%\hrule\medskip
\begin{minted}[fontsize=\footnotesize]{xml}
<CrossAssetModel>	
  <!-- ... -->
  <EquityModels>
    <CrossAssetLGM name="default">
      <Currency>EUR</Currency>
      <CalibrationType>Bootstrap</CalibrationType>
      <Sigma>
        <Calibrate>Y</Calibrate>
        <ParamType>Piecewise</ParamType>
        <TimeGrid>1.0,2.0,3.0,4.0,5.0,7.0,10.0</TimeGrid>
        <InitialValue>0.1,0.1,0.1,0.1,0.1,0.1,0.1,0.1</InitialValue>
      </Sigma>
      <CalibrationOptions>
        <Expiries>1Y,2Y,3Y,4Y,5Y,10Y</Expiries>
        <Strikes/>
      </CalibrationOptions>
    </CrossAssetLGM>
    <CrossAssetLGM name="SP5">
      <!-- ... -->
    </CrossAssetLGM>
    <CrossAssetLGM name="Lufthansa">
      <!-- ... -->
    </CrossAssetLGM>
      <!-- ... -->
  </EquityModels>
  <!-- ... -->
<CrossAssetModel>	
\end{minted}
\caption{Simulation model equity configuration}
\label{lst:simulation_model_eq_configuration}
\end{listing}

CrossAssetLGM sections are defined by equity name, but we also support a default configuration as above for the IR and 
FX model parameterisations.  Within each CrossAssetLGM section, the interpretation of elements is as follows:

\begin{itemize}
	\item {\tt Currency: } Currency of denomination
	\item {\tt CalibrationType: } Choose between {\em Bootstrap} and {\em BestFit} as in the IR section
	\item {\tt Sigma/Calibrate: } Flag to enable/disable calibration of this particular parameter
	\item {\tt Sigma/ParamType: } Choose between {\em Constant} and {\em Piecewise}
	\item {\tt Sigma/TimeGrid: } Initial time grid for this parameter, can be left empty if ParamType is Constant
	\item {\tt Sigma/InitialValue: } Vector of initial values, matching number of entries in time (for CalibrationType {\em BestFit} this should be one more entry than the {\tt Sigma/TimeGrid} entries, for {\em Bootstrap} this is ignored), or single value if the time grid is empty
	\item {\tt CalibrationOptions: } Choice of calibration instruments by expiry and strike, strikes can be empty 
	(implying the default, ATMF options), or explicitly specified (in terms of equity prices as absolute strike values). There have to be at least one more calibration options configured than {\tt Sigma/TimeGrid} entries were given
\end{itemize}

\medskip

For the inflation model component, there is a choice between a Dodgson Kainth model and a Jarrow Yildrim model. The Dodgson Kainth 
model is specified in a \lstinline!LGM! or \lstinline!DodgsonKainth! node as outlined in Listing \ref{lst:simulation_model_dk_inflation_configuration}.
The inflation model parameterisation inherits from the LGM parameterisation for interest rate components, in particular the \lstinline!CalibrationType!, 
\lstinline!Volatility! and \lstinline!Reversion! elements. The \lstinline!CalibrationCapFloors! element specify the model's calibration to a selection of 
either CPI caps or CPI floors with specified strike.

\begin{listing}[H]
%\hrule\medskip
\begin{minted}[fontsize=\footnotesize]{xml}
<CrossAssetModel>	
  ...
  <InflationIndexModels>
    <LGM index="EUHICPXT">
      <Currency>EUR</Currency>
      <!-- As in the LGM parameterisation for any IR components -->
      <CalibrationType> ... </CalibrationType>
      <Volatility> ... </Volatility>
      <Reversion> ... </Reversion> 
      <ParameterTransformation> ... </ParameterTransformation>
      <!-- Inflation model specific -->
      <CalibrationCapFloors>
        <!-- not used yet, as there is only one strategy so far -->
        <CalibrationStrategy> ... </CalibrationStrategy> 
        <CapFloor> Floor </CapFloor> <!-- Cap, Floor -->
        <Expiries> 2Y, 4Y, 6Y, 8Y, 10Y </Expiries>
        <!-- can be empty, this will yield calibration to ATM -->
        <Strikes> 0.03, 0.03, 0.03, 0.03, 0.03 </Strikes> 
      </CalibrationCapFloors>
    </LGM>
    <LGM index="USCPI">
      ...
    </LGM>
    ...
  </InflationIndexModels>
  ...
<CrossAssetModel>	
\end{minted}
\caption{Simulation model DK inflation component configuration}
\label{lst:simulation_model_dk_inflation_configuration}
\end{listing}

The calibration instruments may be specified in an alternative way via a \lstinline!CalibrationBaskets! node. In general, a \lstinline!CalibrationBaskets! node 
can contain multiple \lstinline!CalibrationBasket! nodes each containing a list of calibration instruments of the same type. For Dodgson Kainth, only a single 
calibration basket is allowed and the instruments must be of type \lstinline!CpiCapFloor!. So, for example, the \lstinline!CalibrationCapFloors! node in 
Listing \ref{lst:simulation_model_dk_inflation_configuration} could be replaced with the \lstinline!CalibrationBaskets! node in \ref{lst:dk_inflation_calibration_basket}.

\begin{listing}[H]
\begin{minted}[fontsize=\footnotesize]{xml}
<CalibrationBaskets>
  <CalibrationBasket>
    <CpiCapFloor>
      <Type>Floor</Type>
      <Maturity>2Y</Maturity>
      <Strike>0.03</Strike>
    </CpiCapFloor>
    <CpiCapFloor>
      <Type>Floor</Type>
      <Maturity>4Y</Maturity>
      <Strike>0.03</Strike>
    </CpiCapFloor>
    <CpiCapFloor>
      <Type>Floor</Type>
      <Maturity>6Y</Maturity>
      <Strike>0.03</Strike>
    </CpiCapFloor>
    <CpiCapFloor>
      <Type>Floor</Type>
      <Maturity>8Y</Maturity>
      <Strike>0.03</Strike>
    </CpiCapFloor>
    <CpiCapFloor>
      <Type>Floor</Type>
      <Maturity>10Y</Maturity>
      <Strike>0.03</Strike>
    </CpiCapFloor>
  </CalibrationBasket>
</CalibrationBaskets>
\end{minted}
\caption{Calibration basket for DK inflation model component}
\label{lst:dk_inflation_calibration_basket}
\end{listing}

The Jarrow Yildrim model is specified in a \lstinline!JarrowYildirim! node as outlined in Listing \ref{lst:simulation_model_jy_inflation_configuration}. The \lstinline!RealRate! 
node describes the JY real rate process and has \lstinline!Volatility! and \lstinline!Reversion! nodes that follow those outlined in the interest rate LGM section above. The  
\lstinline!Index! node describes the JY index process and has a \lstinline!Volatility! component that follows the \lstinline!Sigma! component of the FX model above. The 
\lstinline!CalibrationBaskets! node is as outlined above for Dodgson Kainth but up to two baskets may be used and extra inflation instruments are supported in the calibration. More 
information is provided below.

The \lstinline!CalibrationType! determines the calibration approach, if any, that is used to calibrate the various parameters of the model i.e.\ the real rate reversion, the real 
rate volatility and the index volatility. If the \lstinline!CalibrationType! is \lstinline!None!, no calibration is attempted and all parameter values must be explicitly specified.
If the \lstinline!CalibrationType! is \lstinline!BestFit!, the parameters that have \lstinline!Calibrate! set to \lstinline!Y! will be calibrated to the instruments specified in 
the \lstinline!CalibrationBaskets! node. If the \lstinline!CalibrationType! is \lstinline!Bootstrap!, there are a number of options:

\begin{enumerate}
\item
The index volatility parameter may be calibrated, indicated by setting \lstinline!Calibrate! to \lstinline!Y! for that parameter, with both of the real rate parameters not calibrated 
and set explicitly in the \lstinline!RealRate! node. There should be exactly one \lstinline!CalibrationBasket! in the \lstinline!CalibrationBaskets! node and its \lstinline!parameter! 
attribute may be set to \lstinline!Index! or omitted.

\item
One of the real rate parameters may be calibrated, indicated by setting \lstinline!Calibrate! to \lstinline!Y! for that parameter, with the index volatility not calibrated and set 
explicitly in the \lstinline!Volatility! node. There should be exactly one \lstinline!CalibrationBasket! in the \lstinline!CalibrationBaskets! node and its \lstinline!parameter! 
attribute may be set to \lstinline!RealRate! or omitted.

\item
One of the real rate parameters and the index volatility parameter may be calibrated together. There should be exactly two \lstinline!CalibrationBasket! nodes in the \lstinline!CalibrationBaskets! 
node. The \lstinline!parameter! attribute should be set to \lstinline!RealRate! on the \lstinline!CalibrationBasket! node that should be used for the real rate parameter calibration. 
Similarly, the \lstinline!parameter! attribute should be set to \lstinline!Index! on the \lstinline!CalibrationBasket! node that should be used for the index volatility parameter calibration. 
The parameters are calibrated iteratively in turn until the root mean squared error over all calibration instruments in the two baskets is below the tolerance specified by the 
\lstinline!RmseTolerance! in the \lstinline!CalibrationConfiguration! node or until the maximum number of iterations as specified by the \lstinline!MaxIterations! in the 
\lstinline!CalibrationConfiguration! node has been reached. The \lstinline!CalibrationConfiguration! node is optional. If it is omitted, the \lstinline!RmseTolerance! defaults to 0.0001 and the 
\lstinline!MaxIterations! defaults to 50.

\end{enumerate}

Note that it is an error to attempt to calibrate both of the real rate parameters together when \lstinline!CalibrationType! is \lstinline!Bootstrap!. If a parameter is being calibrated 
with \lstinline!CalibrationType! set to \lstinline!Bootstrap!, the \lstinline!ParamType! should be \lstinline!Piecewise!. The \lstinline!TimeGrid! will be overridden for that parameter by 
the relevant calibration instrument times and the parameter's initial values are set to the first element of the \lstinline!InitialValue! list. So, leaving the \lstinline!TimeGrid! node 
empty and giving a single value in the \lstinline!InitialValue! node is the clearest XML setup in this case.

\begin{listing}[H]
\begin{minted}[fontsize=\footnotesize]{xml}
<JarrowYildirim index="EUHICPXT">
  <Currency>EUR</Currency>
  <CalibrationType>Bootstrap</CalibrationType>
  <RealRate>
    <Volatility>
      <Calibrate>Y</Calibrate>
      <VolatilityType>Hagan</VolatilityType>
      <ParamType>Piecewise</ParamType>
      <TimeGrid/>
      <InitialValue>0.0001</InitialValue>
    </Volatility>
    <Reversion>
      <Calibrate>N</Calibrate>
      <ReversionType>HullWhite</ReversionType>
      <ParamType>Constant</ParamType>
      <TimeGrid/>
      <InitialValue>0.5</InitialValue>
    </Reversion>
    <ParameterTransformation>
      <ShiftHorizon>0.0</ShiftHorizon>
      <Scaling>1.0</Scaling>
    </ParameterTransformation>
  </RealRate>
  <Index>
    <Volatility>
      <Calibrate>Y</Calibrate>
      <ParamType>Piecewise</ParamType>
      <TimeGrid/>
      <InitialValue>0.0001</InitialValue>
    </Volatility>
  </Index>
  <CalibrationBaskets>
    <CalibrationBasket parameter="Index">
      <CpiCapFloor>
        <Type>Floor</Type>
        <Maturity>2Y</Maturity>
        <Strike>0.0</Strike>
      </CpiCapFloor>
      ...
    </CalibrationBasket>
    <CalibrationBasket parameter="RealRate">
      <YoYSwap>
        <Tenor>2Y</Tenor>
      </YoYSwap>
      ...
    </CalibrationBasket>
  </CalibrationBaskets>
  <CalibrationConfiguration>
    <RmseTolerance>0.00000001</RmseTolerance>
    <MaxIterations>40</MaxIterations>
  </CalibrationConfiguration>
</JarrowYildirim>
\end{minted}
\caption{Simulation model JY inflation component configuration}
\label{lst:simulation_model_jy_inflation_configuration}
\end{listing}

The \lstinline!CpiCapFloor! and \lstinline!YoYSwap! calibration instruments can be seen in Listing \ref{lst:simulation_model_jy_inflation_configuration}. A \lstinline!YoYCapFloor! is 
also allowed and it has the structure shown in Listing \ref{lst:yoy_cf_calibration_inst}. The \lstinline!Type! may be \lstinline!Cap! or \lstinline!Floor!. The \lstinline!Tenor! should 
be a maturity period e.g.\ \lstinline!5Y!. The \lstinline!Strike! should be an absolute strike level for the year on year cap or floor e.g.\ \lstinline!0.01! for 1\%.

\begin{listing}[H]
\begin{minted}[fontsize=\footnotesize]{xml}
<YoYCapFloor>
  <Type>...</Type>
  <Tenor>...</Tenor>
  <Strike>...</Strike>
</YoYCapFloor>
\end{minted}
\caption{Layout for \lstinline!YoYCapFloor! calibration instrument.}
\label{lst:yoy_cf_calibration_inst}
\end{listing}

% credit models: todo

% commodity models

For commodity simulation we currently provide one model, as described in the methodology appendix. 
Commodity model components are specified by commodity name, by a block as follows

\begin{listing}[H]
\begin{minted}[fontsize=\footnotesize]{xml}
<CrossAssetModel>	
  <!-- ... -->
  <CommodityModels>
    <CommoditySchwartz name="default">
      <Currency>EUR</Currency>
      <CalibrationType>None</CalibrationType>
      <Sigma>
        <Calibrate>Y</Calibrate>
        <InitialValue>0.1</InitialValue>
      </Sigma>
      <Kappa>
        <Calibrate>Y</Calibrate>
        <InitialValue>0.1</InitialValue>
      </Kappa>
      <CalibrationOptions>
           ...
      </CalibrationOptions>
      <DriftFreeState>false</DriftFreeState>
    </CommoditySchwartz>
    <CommoditySchwartz name="WTI">
      <!-- ... -->
    </CommoditySchwartz>
    <CommoditySchwartz name="NG">
      <!-- ... -->
    </CommoditySchwartz>
      <!-- ... -->
  </CommodityModels>
  <!-- ... -->
<CrossAssetModel>	
\end{minted}
\caption{Simulation model commodity configuration}
\label{lst:simulation_model_com_configuration}
\end{listing}

CommoditySchwartz sections are defined by commodity name, but we also support a default configuration as above for the IR and 
FX model parameterisations.  Each component is parameterised in terms of two constant, non time-dependent parameters $\sigma$ and $\kappa$ so far (see appendix).
Within each CommoditySchwartz section, the interpretation of elements is as follows:

\begin{itemize}
\item {\tt Currency: } Currency of denomination
\item {\tt CalibrationType:} Choose between {\em BestFit} and {\em None}.  The choice {\em None} will deactivate calibration as usual. {\em BestFit} will attempt to set the model parameter(s) such that the error in matching calibration instrument prices is minimised.  The option  {\em Bootstrap} is not available here because the model parameters are not time-dependent and the model's degrees of freedom in general do not suffice to perfectly match the calibration instrument prices.
\item {\tt Sigma/Calibrate:} Flag to enable/disable calibration of this particular parameter
\item {\tt Sigma/InitialValue:} Initial value of the constant parameter
\item {\tt Kappa/Calibrate:} Flag to enable/disable calibration of this particular parameter
\item {\tt Kappa/InitialValue:} Initial value of the constant parameter
\item {\tt CalibrationOptions:} Choice of calibration instruments by expiry and strike, strikes can be empty 	(implying the default, ATMF options), or explicitly specified (in terms of commodity prices as absolute strike values). 
\item {\tt DriftFreeState[Optional]:} Boolean to switch between the two implementations of the state variable, see appendix. By default this is set to {\tt false}.        
\end{itemize}

\medskip
Finally, the instantaneous correlation structure is specified as follows.

\begin{listing}[H]
%\hrule\medskip
\begin{minted}[fontsize=\footnotesize]{xml}
<CrossAssetModel>
  <!-- ... -->
  <InstantaneousCorrelations>
    <Correlation factor1="IR:EUR" factor2="IR:USD">0.3</Correlation>
    <Correlation factor1="IR:EUR" factor2="IR:GBP">0.3</Correlation>
    <Correlation factor1="IR:USD" factor2="IR:GBP">0.3</Correlation>
    <Correlation factor1="IR:EUR" factor2="FX:USDEUR">0</Correlation>
    <Correlation factor1="IR:EUR" factor2="FX:GBPEUR">0</Correlation>
    <Correlation factor1="IR:GBP" factor2="FX:USDEUR">0</Correlation>
    <Correlation factor1="IR:GBP" factor2="FX:GBPEUR">0</Correlation>
    <Correlation factor1="IR:USD" factor2="FX:USDEUR">0</Correlation>
    <Correlation factor1="IR:USD" factor2="FX:GBPEUR">0</Correlation>
    <Correlation factor1="FX:USDEUR" factor2="FX:GBPEUR">0</Correlation>
    <!-- ... --> 
  </InstantaneousCorrelations>
</CrossAssetModel>
\end{minted}
\caption{Simulation model correlation configuration}
\label{lst:simulation_model_correlation_configuration}
\end{listing}

Any risk factor pair not specified explicitly here will be assumed to have zero correlation. Note that the commodity components can have non-zero correlations among each other, but correlations to all other CAM components must remain set to zero for the time being.

\subsubsection{Market}\label{sec:sim_market}

The last part of the simulation configuration file covers the specification of the simulated market.  Note that the
simulation model will yield the evolution of risk factors such as short rates which need to be translated into entire
yield curves that can be 'understood' by the instruments which we want to price under scenarios.  

Moreover we need to specify how volatility structures evolve even if we do not explicitly simulate volatility. This 
translation happens based on the information in the {\em simulation market} object, which is configured in the section 
within the enclosing tags {\tt <Market>} and {\tt </Market>}, as shown in the following small example.

It should be noted that equity volatilities are taken to be a curve by default. To simulate an equity volatility surface with smile the xml node {\tt <Surface> } must be supplied.
There are two methods in ORE for equity volatility simulation: 
\begin{itemize}
\item Simulating ATM volatilities only (and shifting other strikes relative to this using the $T_{0}$ smile). In this
  case set {\tt <SimulateATMOnly>} to true and no surface node is given.
\item Simulating the full volatility surface. The node {\tt <SimulateATMOnly>} should be omitted or set to false, and
  explicit moneyness levels for simulation should be provided.
\end{itemize}

Swaption volatilities are taken to be a surface by default. To simulate a swaption volatility cube with smile the xml node {\tt <Cube> } must be supplied.
There are two methods in ORE for swaption volatility cube simulation: 
\begin{itemize}
\item Simulating ATM volatilities only (and shifting other strikes relative to this using the $T_{0}$ smile). In this case set {\tt <SimulateATMOnly>} to true.
\item Simulating the full volatility cube. The node {\tt <SimulateATMOnly>} should be omitted or set to false, and
  explicit strike spreads for simulation should be provided.
\end{itemize}

FX volatilities are taken to be a curve by default. To simulate an FX volatility cube with smile the xml node {\tt <Surface> } must be supplied. The surface node contains the moneyness levels to be simulated.

For Yield Curves, Swaption Volatilities, CapFloor Volatilities, Default Curves, Base Correlations and Inflation Curves, a DayCounter may be specified for each risk factor using the node {\tt <DayCounter name="EXAMPLE\_CURVE">}.  
If no day counter is specified for a given risk factor then the default Actual365 is used. To specify a new default for a risk factor type then use the daycounter node without any attribute,  {\tt <DayCounter>}.

For Yield Curves, there are several choices for the interpolation and extrapolation:
\begin{itemize}
\item Interpolation: This can be LogLinear or LinearZero. If not given, the value defaults to LogLinear.
\item Extrapolation: This can be FlatFwd or FlatZero. If not given, the value defaults to FlatFwd.
\end{itemize}

For Default Curve, there is a similar choice for the extrapolation:
\begin{itemize}
\item Extrapolation: This can be FlatFwd or FlatZero. If not given, the value defaults to FlatFwd.
\end{itemize}

For swap, yield, interest cap-floor, yoy inflation cap-floor, zc inflation cap-floor, cds, fx, equity, commodity
volatilities the smile dynamics can be specified as shown in listing \ref{lst:smile_dynamics_configuration} for swap
vols. The empty key serves as a default configuration for all keys for which no own smile dynamics node is present. The
allowed smile dynamics values are StickyStrike and StickyMoneyness. If not given, the smile dynamics defaults to
StickyStrike.

\begin{listing}
  \begin{minted}[fontsize=\footnotesize]{xml}
    <SmileDynamics key="">StickyStrike</SmileDynamics>
    <SmileDynamics key="EUR-ESTER">StickyMoneyness</SmileDynamics>
  \end{minted}
\caption{Smile Configuration Node}
\label{lst:smile_dynamics_configuration}
\end{listing}

\begin{longlisting}
%\hrule\medskip
\begin{minted}[fontsize=\footnotesize]{xml}
<Market>
  <BaseCurrency>EUR</BaseCurrency>
  <Currencies>
    <Currency>EUR</Currency>
    <Currency>USD</Currency>
  </Currencies>
  <YieldCurves>
    <Configuration>
      <Tenors>3M,6M,1Y,2Y,3Y,4Y,5Y,7Y,10Y,12Y,15Y,20Y</Tenors>
      <Interpolation>LogLinear</Interpolation>
      <Extrapolation>FlatFwd</Extrapolation>
      <DayCounter>ACT/ACT</DayCounter> <!-- Sets a new default for all yieldCurves -->
    </Configuration>
  </YieldCurves>
  <Indices>
    <Index>EUR-EURIBOR-6M</Index>
    <Index>EUR-EURIBOR-3M</Index>
    <Index>EUR-EONIA</Index>
    <Index>USD-LIBOR-3M</Index>
  </Indices>
  <SwapIndices>
    <SwapIndex>
      <Name>EUR-CMS-1Y</Name>
      <ForwardingIndex>EUR-EURIBOR-6M</ForwardingIndex>
      <DiscountingIndex>EUR-EONIA</DiscountingIndex>
    </SwapIndex>
  </SwapIndices>
  <DefaultCurves> 
      <Names> 
        <Name>CPTY1</Name> 
        <Name>CPTY2</Name> 
      </Names> 
      <Tenors>6M,1Y,2Y</Tenors> 
      <SimulateSurvivalProbabilities>true</SimulateSurvivalProbabilities> 
      <DayCounter name="CPTY1">ACT/ACT</DayCounter>
      <Extrapolation>FlatFwd</Extrapolation>
  </DefaultCurves> 
  <SwaptionVolatilities>
    <ReactionToTimeDecay>ForwardVariance</ReactionToTimeDecay>
    <Currencies>
      <Currency>EUR</Currency>
      <Currency>USD</Currency>
    </Currencies>
    <Expiries>6M,1Y,2Y,3Y,5Y,10Y,12Y,15Y,20Y</Expiries>
    <Terms>1Y,2Y,3Y,4Y,5Y,7Y,10Y,15Y,20Y,30Y</Terms>
    <SimulateATMOnly>false</SimulateATMOnly>
    <Cube>
     <StrikeSpreads>-0.02,-0.01,0.0,0.01,0.02</StrikeSpreads>
    </Cube>
    <!-- Sets a new daycounter for just the EUR swaptionVolatility surface -->
    <DayCounter ccy="EUR">ACT/ACT</DayCounter> 
  </SwaptionVolatilities> 
  <CapFloorVolatilities>
    <ReactionToTimeDecay>ConstantVariance</ReactionToTimeDecay>
    <Currencies>
      <Currency>EUR</Currency>
      <Currency>USD</Currency>
    </Currencies>
    <DayCounter ccy="EUR">ACT/ACT</DayCounter>
  </CapFloorVolatilities>
  <FxVolatilities>
    <ReactionToTimeDecay>ForwardVariance</ReactionToTimeDecay>
    <CurrencyPairs>
      <CurrencyPair>EURUSD</CurrencyPair>
    </CurrencyPairs>
    <Expiries>6M,1Y,2Y,3Y,4Y,5Y,7Y,10Y</Expiries>
    <Surface>
     <Moneyness>0.5,0.6,0.7,0.8,0.9</Moneyness>
    </Surface>
  </FxVolatilities>
  <EquityVolatilities>
      <Simulate>true</Simulate>
      <ReactionToTimeDecay>ForwardVariance</ReactionToTimeDecay>
      <!-- Alternative: ConstantVariance -->
      <Names>
        <Name>SP5</Name>
        <Name>Lufthansa</Name>
      </Names>
      <Expiries>6M,1Y,2Y,3Y,4Y,5Y,7Y,10Y</Expiries>
      <SimulateATMOnly>false</SimulateATMOnly>
      <Surface>
        <Moneyness>0.1,0.5,1.0,1.5,2.0,3.0</Moneyness>
      </Surface>
      <TimeExtrapolation>Flat</TimeExtrapolation>
      <StrikeExtrapolation>Flat</StrikeExtrapolation>

  </EquityVolatilities>
  ...
  <BenchmarkCurves>
    <BenchmarkCurve>
      <Currency>EUR</Currency>
      <Name>BENCHMARK_EUR</Name>
  </BenchmarkCurve>
  ...
  </BenchmarkCurves>
  <Securities>
    <Simulate>true</Simulate>
    <Names>
      <Name>SECURITY_1</Name>
      ...
    </Names>
  </Securities>
  <ZeroInflationIndexCurves>
    <Names>
      <Name>EUHICP</Name>
      <Name>UKRPI</Name>
      <Name>USCPI</Name>
      ...
    </Names>
    <Tenors>6M,1Y,2Y,3Y,5Y,7Y,10Y,15Y,20Y</Tenors>
  </ZeroInflationIndexCurves>
  <YYInflationIndexCurves>
    <Names>
      <Name>EUHICPXT</Name>
      ...
    </Names>
    <Tenors>1Y,2Y,3Y,5Y,7Y,10Y,15Y,20Y</Tenors>
  </YYInflationIndexCurves>
  <DefaultCurves>
    <Names>
      <Name>ItraxxEuropeCrossoverS26V1</Name>
      ...
    </Names>
    <Tenors>1Y,2Y,3Y,5Y,10Y</Tenors>
    <SimulateSurvivalProbabilities>true</SimulateSurvivalProbabilities>
  </DefaultCurves>
  <BaseCorrelations/>
  <CDSVolatilities/>
  <Correlations>
    <Simulate>true</Simulate>
    <Pairs>
      <Pair>EUR-CMS-10Y,EUR-CMS-2Y</Pair>
    </Pairs>
    <Expiries>1Y,2Y</Expiries>
  </Correlations>
  <AdditionalScenarioDataCurrencies>
    <Currency>EUR</Currency>
    <Currency>USD</Currency>
  </AdditionalScenarioDataCurrencies>
  <AdditionalScenarioDataIndices>
    <Index>EUR-EURIBOR-3M</Index>
    <Index>EUR-EONIA</Index>
    <Index>USD-LIBOR-3M</Index>
  </AdditionalScenarioDataIndices>
</Market>
\end{minted}
\caption{Simulation market configuration}
\label{lst:simulation_market_configuration}
\end{longlisting}

\todo[inline]{Comment on cap/floor surface structure and reaction to time decay}

%--------------------------------------------------------
\subsection{Sensitivity Analysis: {\tt sensitivity.xml}}\label{sec:sensitivity}
%--------------------------------------------------------

ORE currently supports sensitivity analysis with respect to
\begin{itemize}
\item Discount curves  (in the zero rate domain)
\item Index curves (in the zero rate domain)
\item Yield curves including e.g. equity forecast yield curves (in the zero rate domain)
\item FX Spots
\item FX volatilities
\item Swaption volatilities, ATM matrix or cube 
\item Cap/Floor volatility matrices (in the caplet/floorlet domain)
\item Default probability curves (in the ``zero rate'' domain, expressing survival probabilities $S(t)$ in term of zero rates $z(t)$ via $S(t)=\exp(-z(t)\times t)$ with Actual/365 day counter)
\item Equity spot prices
\item Equity volatilities, ATM or including strike dimension 
\item Zero inflation curves
\item Year-on-Year inflation curves
\item CDS volatilities
\item Bond credit spreads
\item Base correlation curves
\item Correlation termstructures
\end{itemize}

The {\tt sensitivity.xml} file specifies how sensitivities are computed for each market component. 
The general structure is shown in listing \ref{lst:sensitivity_config}, for a more comprehensive case see {\tt Examples/Example\_15}. A subset of the following parameters is used in each market component to specify the sensitivity analysis:

\begin{itemize}
\item {\tt ShiftType:} Both absolute or relative shifts can be used to compute a sensitivity, specified by the key words
  {\tt Absolute} resp. {\tt Relative}.
\item {\tt ShiftSize:} The size of the shift to apply.
\item {\tt ShiftScheme:} The finite difference scheme to use ({\tt Forward}, {\tt Backward}, {\tt Central}), if not given, this parameter defaults to {\tt Forward}
\item {\tt ShiftTenors:} For curves, the tenor buckets to apply shifts to, given as a comma separated list of periods.
\item {\tt ShiftExpiries:} For volatility surfaces, the option expiry buckets to apply shifts to, given as a comma
  separated list of periods.
\item {\tt ShiftStrikes:} For cap/floor, FX option and equity option volatility surfaces, the strikes to apply shifts to, given as a comma separated
  list of absolute strikes
\item {\tt ShiftTerms:} For swaption volatility surfaces, the underlying terms to apply shifts to, given as a comma
  separated list of periods.
\item {\tt Index:} For cap / floor volatility surfaces, the index which together with the currency defines the surface.
  list of absolute strikes
\item {\tt CurveType:} In the context of Yield Curves used to identify an equity ``risk free'' rate forecasting curve; set to {\tt EquityForecast} in this case
\end{itemize}

The ShiftType, ShiftSize, ShiftScheme nodes take an optional attribute key that allows to configure different values for
different sensitivity templates. The sensitivity templates are defined in the pricing engine configuration. This is best
explained by an example: In Example 15 the product type BermudanSwaption has a sensitivity template \verb+IR_FD+
attached, see \ref{lst:sensi_template}. This can be used to specify different shifts for trades that were built against
this engine configuration, see \ref{lst:sensi_config_template}: For Bermudan swaptions a larger shift size of 10bp and a
central difference scheme is used to compute discount curve sensitivities in EUR. Since no separate shift type is
specified, the default shift type {\tt Absolute} is used. Note regarding the reports:

\begin{itemize}
\item the sensi scenario report contains scenario NPVs related to the possibly product specific configured shift sizes
\item the sensi report contains renormalized sensitivities, i.e. sensitivities are always expressed w.r.t. the default shift sizes
\item the sensi config report only contains the default configuration
\end{itemize}

\begin{longlisting}
\begin{minted}[fontsize=\scriptsize]{xml}
  <Product type="BermudanSwaption">
    <Model>LGM</Model>
    <ModelParameters>
      ...
    </ModelParameters>
    <Engine>Grid</Engine>
    <EngineParameters>
      ...
      <Parameter name="SensitivityTemplate">IR_FD</Parameter>
    </EngineParameters>
  </Product>
\end{minted}
\caption{Sensitivity template definition}
\label{lst:sensi_template}
\end{longlisting}

\begin{longlisting}
\begin{minted}[fontsize=\scriptsize]{xml}
    <DiscountCurve ccy="EUR">
      <ShiftType>Absolute</ShiftType>
      <ShiftSize>0.0001</ShiftSize>
      <ShiftScheme>Forward</ShiftScheme>
      <ShiftSize key="IR_FD">0.001</ShiftSize>
      <ShiftScheme key="IR_FD">Central</ShiftScheme>
      <ShiftTenors>6M,1Y,2Y,3Y,5Y,7Y,10Y,15Y,20Y</ShiftTenors>
    </DiscountCurve>
\end{minted}
\caption{Sensitivity template definition}
\label{lst:lst:sensi_config_template}
\end{longlisting}

The cross gamma filter section contains a list of pairs of sensitivity keys. For each possible pair of sensitivity keys
matching the given strings, a cross gamma sensitivity is computed. The given pair of keys can be (and usually are)
shorter than the actual sensitivity keys. In this case only the prefix of the actual key is matched. For example, the
pair {\tt DiscountCurve/EUR,DiscountCurve/EUR} matches all actual sensitivity pairs belonging to a cross sensitivity by
one pillar of the EUR discount curve and another (different) pillar of the same curve. We list the possible keys by
giving an example in each category:

\begin{itemize}
\item {\tt DiscountCurve/EUR/5/7Y}: 7y pillar of discounting curve in EUR, the pillar is at position 5 in the list of
  all pillars (counting starts with zero)
\item {\tt YieldCurve/BENCHMARK\_EUR/0/6M}: 6M pillar of yield curve ``BENCHMARK\_EUR'', the index of the 6M pillar is
  zero (i.e. it is the first pillar)
\item {\tt IndexCurve/EUR-EURIBOR-6M/2/2Y}: 2Y pillar of index forwarding curve for the Ibor index ``EUR-EURIBOR-6M'',
  the pillar index is 2 in this case
\item {\tt OptionletVolatility/EUR/18/5Y/0.04}: EUR caplet volatility surface, at 5Y option expiry and $4\%$ strike, the
  running index for this expiry - strike pair is 18; the index counts the points in the surface in lexical order
  w.r.t. the dimensions option expiry, strike
\item {\tt FXSpot/USDEUR/0/spot}: FX spot USD vs EUR (with EUR as base ccy), the index is always zero for FX spots, the
  pillar is labelled as ``spot'' always
\item {\tt SwaptionVolatility/EUR/11/10Y/10Y/ATM}: EUR Swaption volatility surface at 10Y option expiry and 10Y
  underlying term, ATM level, the running index for this expiry, term, strike triple has running index 11; the index
  counts the points in the surface in lexical order w.r.t. the dimensions option expiry, underlying term and strike
\end{itemize}

Additional flags:

\begin{itemize}
\item ComputeGamma: If set to false, second order sensitivity computation is suppressed
\item UseSpreadedTermStructures: If set to true, spreaded termstructures over t0 will be used for sensitivity
  calculation (where supported), to improve the alignment of the scenario sim market and t0 curves
\end{itemize}

\begin{longlisting}
%\hrule\medskip
  \begin{minted}[fontsize=\scriptsize]{xml}
<SensitivityAnalysis>
  <DiscountCurves>
    <DiscountCurve ccy="EUR">
      <ShiftType>Absolute</ShiftType>
      <ShiftSize>0.0001</ShiftSize>
      <ShiftTenors>6M,1Y,2Y,3Y,5Y,7Y,10Y,15Y,20Y</ShiftTenors>
    </DiscountCurve>
    ...
  </DiscountCurves>
  ...
  <IndexCurves>
    <IndexCurve index="EUR-EURIBOR-6M">
      <ShiftType>Absolute</ShiftType>
      <ShiftSize>0.0001</ShiftSize>
      <ShiftTenors>6M,1Y,2Y,3Y,5Y,7Y,10Y,15Y,20Y</ShiftTenors>
    </IndexCurve>
  </IndexCurves>
  ...
  <YieldCurves>
    <YieldCurve name="BENCHMARK_EUR">
      <ShiftType>Absolute</ShiftType>
      <ShiftSize>0.0001</ShiftSize>
      <ShiftTenors>6M,1Y,2Y,3Y,5Y,7Y,10Y,15Y,20Y</ShiftTenors>
    </YieldCurve>
  </YieldCurves>
  ...
  <FxSpots>
    <FxSpot ccypair="USDEUR">
      <ShiftType>Relative</ShiftType>
      <ShiftSize>0.01</ShiftSize>
    </FxSpot>
  </FxSpots>
  ...
  <FxVolatilities>
    <FxVolatility ccypair="USDEUR">
      <ShiftType>Relative</ShiftType>
      <ShiftSize>0.01</ShiftSize>
      <ShiftExpiries>1Y,2Y,3Y,5Y</ShiftExpiries>
      <ShiftStrikes/>
    </FxVolatility>
  </FxVolatilities>
  ...
  <SwaptionVolatilities>
    <SwaptionVolatility ccy="EUR">
      <ShiftType>Relative</ShiftType>
      <ShiftSize>0.01</ShiftSize>
      <ShiftExpiries>1Y,5Y,7Y,10Y</ShiftExpiries>
      <ShiftStrikes/>
      <ShiftTerms>1Y,5Y,10Y</ShiftTerms>
    </SwaptionVolatility>
  </SwaptionVolatilities>
  ...
  <CapFloorVolatilities>
    <CapFloorVolatility ccy="EUR">
      <ShiftType>Absolute</ShiftType>
      <ShiftSize>0.0001</ShiftSize>
      <ShiftExpiries>1Y,2Y,3Y,5Y,7Y,10Y</ShiftExpiries>
      <ShiftStrikes>0.01,0.02,0.03,0.04,0.05</ShiftStrikes>
      <Index>EUR-EURIBOR-6M</Index>
    </CapFloorVolatility>
  </CapFloorVolatilities>
  ...
  <SecuritySpreads>
    <SecuritySpread security="SECURITY_1">
      <ShiftType>Absolute</ShiftType>
      <ShiftSize>0.0001</ShiftSize>
    </SecuritySpread>
  </SecuritySpreads>
  ...
  <Correlations>
    <Correlation index1="EUR-CMS-10Y" index2="EUR-CMS-2Y">
      <ShiftType>Absolute</ShiftType>
      <ShiftSize>0.01</ShiftSize>
      <ShiftExpiries>1Y,2Y</ShiftExpiries>
      <ShiftStrikes>0</ShiftStrikes>
    </Correlation>
  </Correlations>
  ...
  <CrossGammaFilter>
    <Pair>DiscountCurve/EUR,DiscountCurve/EUR</Pair>
    <Pair>IndexCurve/EUR,IndexCurve/EUR</Pair>
    <Pair>DiscountCurve/EUR,IndexCurve/EUR</Pair>
  </CrossGammaFilter>
  ...
  <ComputeGamma>true</ComputeGamma>
  <UseSpreadedTermStructures>false</UseSpreadedTermStructures>
</SensitivityAnalysis>
\end{minted}
\caption{Sensitivity configuration}
\label{lst:sensitivity_config}
\end{longlisting}

\subsubsection*{Par Sensitivity Analysis}

To perform a par sensitivity analysis, additional sensitivity configuration is required that describes the assumed par instruments and related conventions.
This additional data is required for:
\begin{itemize}
\item DiscountCurves
\item IndexCurves
\item CapFloorVolatilities
\item CreditCurves
\item ZeroInflationIndexCurves
\item YYInflationIndexCurves
\item YYCapFloorVolatilities
\end{itemize}

Using DiscountCurves as an example, the full sensitivity specification including par conversion data is as follows:

\begin{longlisting}
  \begin{minted}[fontsize=\footnotesize]{xml}
    <DiscountCurve ccy="EUR">
      <ShiftType>Absolute</ShiftType>
      <ShiftSize>0.0001</ShiftSize>
      <ShiftTenors>2W,1M,3M,6M,9M,1Y,2Y,3Y,4Y,5Y,7Y,10Y,15Y,20Y,25Y,30Y</ShiftTenors>
      <ParConversion>
        <!--DEP, FRA, IRS, OIS, FXF, XBS -->
	<Instruments>OIS,OIS,OIS,OIS,OIS,OIS,OIS,OIS,OIS,OIS,OIS,OIS,OIS,OIS,OIS,OIS</Instruments>
	<SingleCurve>true</SingleCurve>
	<Conventions>
	  <Convention id="DEP">EUR-EURIBOR-CONVENTIONS</Convention>
	  <Convention id="IRS">EUR-6M-SWAP-CONVENTIONS</Convention>
	  <Convention id="OIS">EUR-OIS-CONVENTIONS</Convention>
	</Conventions>
      </ParConversion>
    </DiscountCurve>   
\end{minted}
\caption{Par sensitivity configuration}
\label{lst:par_sensitivity_config}
\end{longlisting}

Note
\begin{itemize}
\item The list of shift tenors needs to match the list of tenors matches the corresponding grid in the simulation (market) configuration
\item The length of list of (par) instruments needs to match the length of the list of shift tenors
\item Permissible codes for the assumed par instruments:
	\begin{itemize}
	\item DEP, FRA, IRS, OIS, TBS, FXF, XBS in the case of DiscountCurves 
	\item DEP, FRA, IRS, OIS, TBS in the case of IndexCurves
	\item DEP, FRA, IRS, OIS, TBS, XBS in the case of YieldCurves 
	\item ZIS, YYS for YYInflationIndexCurves, interpreted as Year-on-Year Inflation Swaps linked to Zero Inflation resp. YoY Inflation curves
	\item ZIS, YYS for YYCapFloorVolatilities, interpreted as Year-on-Year Inflation Cap Floor linked to Zero Inflation resp. YoY Inflation curves
	\item Any code for CreditCurves, interpreted as CDS
	\item Any code for ZeroInflationIndexCurves, interpreted as CPI Swaps linked to Zero Inflation curves
	\item Any code for CapFloorVolatilities, interpreted as flat Cap/Floor
	\end{itemize}
\item One convention needs to be referenced for each of the instrument codes	
\end{itemize}

%--------------------------------------------------------
\subsection{Stress Scenario Analysis: {\tt stressconfig.xml}}\label{sec:stress}
%--------------------------------------------------------

Stress tests can be applied in ORE to the same market segments and with same granularity as described in the sensitivity section \ref{sec:sensitivity}.

\medskip
This file {\tt stressconfig.xml} specifies how stress tests can be configured. The general structure is shown in listing
\ref{lst:stress_config}.

In this example, two stress scenarios ``parallel\_rates'' and ``twist'' are defined. Each scenario definition contains
the market components to be shifted in this scenario in a similar syntax that is also used for the sensitivity
configuration, see \ref{sec:sensitivity}. Components that should not be shifted, can just be omitted in the definition
of the scenario.

However, instead of specifying one shift size per market component, here a whole vector of shifts can be given, with
different shift sizes applied to each point of the curve (or surface / cube).

In case of the swaption volatility shifts, the single value given as {\tt Shift} (without the attributes {\tt expiry}
and {\tt term}) represents a default value that is used whenever no explicit value is given for a expiry / term pair.

UseSpreadedTermStructures: If set to true, spreaded termstructures over t0 will be used for the scenario calculation, to
improve the alignment of the scenario sim market and t0 curves.


\begin{longlisting}
%\hrule\medskip
  \begin{minted}[fontsize=\scriptsize]{xml}
<StressTesting>
  <UseSpreadedTermStructures>false</UseSpreadedTermStructures>
  <StressTest id="parallel_rates">
    <DiscountCurves>
      <DiscountCurve ccy="EUR">
        <ShiftType>Absolute</ShiftType>
        <ShiftTenors>6M,1Y,2Y,3Y,5Y,7Y,10Y,15Y,20Y</ShiftTenors>
        <Shifts>0.01,0.01,0.01,0.01,0.01,0.01,0.01,0.01,0.01</Shifts>
      </DiscountCurve>
      ...
    </DiscountCurves>
    <IndexCurves>
      ...
    </IndexCurves>
    <YieldCurves>
      ...
    </YieldCurves>
    <FxSpots>
      <FxSpot ccypair="USDEUR">
        <ShiftType>Relative</ShiftType>
        <ShiftSize>0.01</ShiftSize>
      </FxSpot>
    </FxSpots>
    <FxVolatilities>
      ...
    </FxVolatilities>
    <SwaptionVolatilities>
      <SwaptionVolatility ccy="EUR">
        <ShiftType>Absolute</ShiftType>
        <ShiftExpiries>1Y,10Y</ShiftExpiries>
        <ShiftTerms>5Y</ShiftTerms>
        <Shifts>
          <Shift>0.0010</Shift>
          <Shift expiry="1Y" term="5Y">0.0010</Shift>
          <Shift expiry="1Y" term="5Y">0.0010</Shift>
          <Shift expiry="1Y" term="5Y">0.0010</Shift>
          <Shift expiry="10Y" term="5Y">0.0010</Shift>
          <Shift expiry="10Y" term="5Y">0.0010</Shift>
          <Shift expiry="10Y" term="5Y">0.0010</Shift>
        </Shifts>
      </SwaptionVolatility>
    </SwaptionVolatilities>
    <CapFloorVolatilities>
      <CapFloorVolatility ccy="EUR">
        <ShiftType>Absolute</ShiftType>
        <ShiftExpiries>6M,1Y,2Y,3Y,5Y,10Y</ShiftExpiries>
        <Shifts>0.001,0.001,0.001,0.001,0.001,0.001</Shifts>
      </CapFloorVolatility>
    </CapFloorVolatilities>
  </StressTest>
  <StressTest id="twist">
    ...
  </StressTest>
</StressTesting>
  \end{minted}
\caption{Stress configuration}
\label{lst:stress_config}
\end{longlisting}

%--------------------------------------------------------
\subsection{Calendar Adjustment: {\tt calendaradjustment.xml}}\label{sec:calendaradjustment}
%--------------------------------------------------------

\medskip
 This file {\tt calendaradjustment.xml} list out all additional holidays and business days that are added to a specified calendar in ORE.
 These dates would originally be missing from the calendar and has to be added.The general structure is shown in listing \ref{lst:calendar_adjustment}.
In this example, two additional dates had been added to the calendar "Japan", one additional holiday and one additional business day. If the user is not certain
wether the date is already included or not, adding it to the {\tt calendaradjustment.xml} to be safe won't raise any errors.
A sample {\tt calendaradjustment.xml} file can be found in the global example input directory. However, it is only used in Example\_1.

\begin{longlisting}
\begin{minted}[fontsize=\scriptsize]{xml}
<CalendarAdjustments>
  <Calendar name="Japan">
    <AdditionalHolidays>
      <Date>2020-01-01</Date>
    </AdditionalHolidays>
    <AdditionalBusinessDays>
      <Date>2020-01-02</Date>
    </AdditionalBusinessDays>
</CalendarAdjustments>
\end{minted}
\caption{Calendar Adjustment}\label{lst:calendar_adjustment}
\end{longlisting}

If the parameter \lstinline!BaseCalendar! is provided then a new calendar will be created using the specified calendar as a base, and adding any \lstinline!AdditionalHolidays! or \lstinline!AdditionalBusinessDays!. In the example below a new calendar \lstinline!CUSTOM_Japan! is being created, it will include any additional holidays or business days specified in the original \lstinline!Japan! calendar plus one additional date.

If a new calendar is added in this way and the schema is being used to validate XML input, the corresponding calendar name must be prefixed with `CUSTOM\_'.

\begin{longlisting}
\hrule\medskip
\begin{minted}[fontsize=\scriptsize]{xml}
<CalendarAdjustments>
  <Calendar name="CUSTOM_Japan">
    <BaseCalendar>Japan</BaseCalendar>
    <AdditionalHolidays>
      <Date>2020-04-06</Date>
    </AdditionalHolidays>
</CalendarAdjustments>
\end{minted}
\caption{Calendar Adjustment creating a new calendar}
\label{lst:calendar_adjustment_2}
\end{longlisting}

\include{parameterisation/curveconfig}

\include{referencedata}

\include{iborfallbackconfig}

\include{simmcalibration}

%--------------------------------------------------------
%\subsection{Conventions: {\tt conventions.xml}}
%\label{sec:conventions}
%--------------------------------------------------------
%--------------------------------------------------------
\subsection{Conventions: {\tt conventions.xml}}
\label{sec:conventions}
%--------------------------------------------------------

The conventions to associate with a set market quotes in the construction of termstructures are specified in another xml
file which we will refer to as {\tt conventions.xml} in the following though the file name can be chosen by the user.
Each separate set of conventions is stored in an XML node. The type of conventions that a node holds is determined by
the node name. Every node has an \lstinline!Id! node that gives a unique identifier for the convention set. The
following sections describe the type of conventions that can be created and the allowed values.

%- - - - - - - - - - - - - - - - - - - - - - - - - - - - - - - - - - - - - - - -
\subsubsection{Zero Conventions}
%- - - - - - - - - - - - - - - - - - - - - - - - - - - - - - - - - - - - - - - -
A node with name \emph{Zero} is used to store conventions for direct zero rate quotes. Direct zero rate quotes can be
given with an explicit maturity date or with a tenor and a set of conventions from which the maturity date is
deduced. The node for a zero rate quote with an explicit maturity date is shown in Listing
\ref{lst:zero_conventions_date}. The node for a tenor based zero rate is shown in Listing
\ref{lst:zero_conventions_tenor}.

\begin{listing}[H]
%\hrule\medskip
\begin{minted}[fontsize=\footnotesize]{xml}
<Zero>
  <Id> </Id>
  <TenorBased>False</TenorBased>
  <DayCounter> </DayCounter>
  <CompoundingFrequency> </CompoundingFrequency>
  <Compounding> </Compounding>
</Zero>
\end{minted}
\caption{Zero conventions}
\label{lst:zero_conventions_date}
\end{listing}

\begin{listing}[H]
%\hrule\medskip
\begin{minted}[fontsize=\footnotesize]{xml}
<Zero>
  <Id> </Id>
  <TenorBased>True</TenorBased>
  <DayCounter> </DayCounter>
  <CompoundingFrequency> </CompoundingFrequency>
  <Compounding> </Compounding>
  <TenorCalendar> </TenorCalendar>
  <SpotLag> </SpotLag>
  <SpotCalendar> </SpotCalendar>
  <RollConvention> </RollConvention>
  <EOM> </EOM>
</Zero>
\end{minted}
\caption{Zero conventions, tenor based}
\label{lst:zero_conventions_tenor}
\end{listing}

The meanings of the various elements in this node are as follows:
\begin{itemize}
\item TenorBased: True if the conventions are for a tenor based zero quote and False if they are
for a zero quote with an explicit maturity date.
\item DayCounter: The day count basis associated with the zero rate quote (for choices see section
\ref{sec:allowable_values})
\item CompoundingFrequency: The frequency of compounding (Choices are {\em Once, Annual, Semiannual, Quarterly,
Bimonthly, Monthly, Weekly, Daily}).
\item Compounding: The type of compounding for the zero rate (Choices are {\em Simple, Compounded, Continuous,
SimpleThenCompounded}).
\item TenorCalendar: The calendar used to advance from the spot date to the maturity date by the zero rate tenor (for
choices see section \ref{sec:allowable_values}).
\item SpotLag [Optional]: The number of business days to advance from the valuation date before applying the zero rate
tenor. If not provided, this defaults to 0.
\item SpotCalendar [Optional]: The calendar to use for business days when applying the \lstinline!SpotLag!. If not
provided, it defaults to a calendar with no holidays.
\item RollConvention [Optional]: The roll convention to use when applying the zero rate tenor. If not provided, it
defaults to Following (Choices are {\em Backward, Forward, Zero, ThirdWednesday, Twentieth, TwentiethIMM, CDS, ThirdThursday, ThirdFriday, MondayAfterThirdFriday, TuesdayAfterThirdFriday, LastWednesday}).
\item EOM [Optional]: Whether or not to use the end of month convention when applying the zero rate tenor. If not
provided, it defaults to false.
\end{itemize}

%- - - - - - - - - - - - - - - - - - - - - - - - - - - - - - - - - - - - - - - -
\subsubsection{Deposit Conventions}
%- - - - - - - - - - - - - - - - - - - - - - - - - - - - - - - - - - - - - - - -

A node with name \emph{Deposit} is used to store conventions for deposit or index fixing quotes. The conventions can be
index based, in which case all necessary conventions are deduced from a given index family. The structure of the index
based node is shown in Listing \ref{lst:deposit_conventions_index}. Alternatively, all the necessary conventions can be
given explicitly without reference to an index family. The structure of this node is shown in Listing
\ref{lst:deposit_conventions_explicit}.

\begin{listing}[H]
%\hrule\medskip
\begin{minted}[fontsize=\footnotesize]{xml}
<Deposit>
  <Id> </Id>
  <IndexBased>True</IndexBased>
  <Index> </Index>
</Deposit>
\end{minted}
\caption{Deposit conventions}
\label{lst:deposit_conventions_index}
\end{listing}

\begin{listing}[H]
%\hrule\medskip
\begin{minted}[fontsize=\footnotesize]{xml}
<Deposit>
  <Id> </Id>
  <IndexBased>False</IndexBased>
  <Calendar> </Calendar>
  <Convention> </Convention>
  <EOM> </EOM>
  <DayCounter> </DayCounter>
</Deposit>
\end{minted}
\caption{Deposit conventions}
\label{lst:deposit_conventions_explicit}
\end{listing}


The meanings of the various elements in this node are as follows:
\begin{itemize}
\item IndexBased: \emph{True} if the deposit conventions are index based and \emph{False} if the conventions are given
explicitly.
\item Index: The index family from which to imply the conventions for the deposit quote. For example, this could be
EUR-EURIBOR, USD-LIBOR etc.
\item Calendar: The business day calendar for the deposit quote.
\item Convention: The roll convention for the deposit quote.
\item EOM: \emph{True} if the end of month roll convention is to be used for the deposit quote and \emph{False} if not.
\item DayCounter: The day count basis associated with the deposit quote.
\end{itemize}

%- - - - - - - - - - - - - - - - - - - - - - - - - - - - - - - - - - - - - - - -
\subsubsection{Future Conventions}\label{ss:conventions_future}
%- - - - - - - - - - - - - - - - - - - - - - - - - - - - - - - - - - - - - - - -

A node with name \emph{Future} is used to store conventions for money market (MM) or overnight index (OI) interest rate
future quotes, for example futures on Euribor 3M or SOFR 3M underlyings. The structure of this node is shown in Listing
\ref{lst:future_conventions}. The fields have the following meaning:

\begin{itemize}
\item Id: The name of the convention.
\item Index: The underlying index of the futures, this is either a MM (i.e. Ibor) index like e.g. EUR-EURIBOR-3M or an
  overnight index like e.g. USD-SOFR.
\item DateGenerationRule [Optional]: This should be set to 'IMM' when the start and end dates of the future are
  following the IMM date logic or 'FirstDayOfMonth' when the start and end date are the first day of a month. If not
  given this field defaults to 'IMM'.
  \begin{itemize}
  \item For MM futures only 'IMM' is allowed and the expiry date is determined as the next 3rd Wednesday of the expiry
    month of a future.
  \item For an overnight index future 'IMM' means that the end date of the future is set to the 3rd Wednesday of the
    expiry month and the start date is set to the 3rd Wednesday of the expiry month minus the future tenor. The setting
    'IMM' applies to SOFR-3M futures for example. 'FirstDayOfMonth' on the other hand means that the end date of the
    future is set to the first day in the month following the future's expiry month and the start date is set to the
    first day of the month lying $n$ months before the end date's month where $n$ is the number of months of the
    future's underlying tenor. The setting 'FirstDayOfMonth' applies to SOFR-1M futures for example. This tenor is
    derived from the market quote, see \ref{ss:market_data_oi_index_future_prices}.
  \end{itemize}
\item OvernightIndexFutureNettingType [Optional]: Only relevant for OI futures. Can be 'Compounding' (which is also the
  default value if no value is given) or 'Averaging'. For example, SOFR 3M futures are compounding while SOFR 1M futures
  are averaging the daily overnight fixings over the calculation period of the future.
\end{itemize}

Listings \ref{lst:future_conventions_euribor_3m}, \ref{lst:future_conventions_sofr_3m},
\ref{lst:future_conventions_sofr_1m} show examples for Euribor-3M, SOFR-3M and SOFR-1M future conventions.

\begin{listing}[H]
%\hrule\medskip
\begin{minted}[fontsize=\footnotesize]{xml}
<Future>
  <Id> </Id>
  <Index> </Index>
  <DateGenerationRule> </DateGenerationRule>
  <OvernightIndexFutureNettingType> </OvernightIndexFutureNettingType>
</Future>
\end{minted}
\caption{Future conventions}
\label{lst:future_conventions}
\end{listing}

\begin{listing}[H]
%\hrule\medskip
\begin{minted}[fontsize=\footnotesize]{xml}
<Future>
  <Id>EURIBOR-3M-FUTURES</Id>
  <Index>EUR-EURIBOR-3M</Index>
</Future>
\end{minted}
\caption{Euribor 3M MM Future conventions}
\label{lst:future_conventions_euribor_3m}
\end{listing}

\begin{listing}[H]
%\hrule\medskip
\begin{minted}[fontsize=\footnotesize]{xml}
  <Future>
    <Id>USD-SOFR-3M-FUTURES</Id>
    <Index>USD-SOFR</Index>
    <DateGenerationRule>IMM</DateGenerationRule>
    <OvernightIndexFutureNettingType>Compounding</OvernightIndexFutureNettingType>
  </Future>
\end{minted}
\caption{USD SOFR 3M OI Future conventions}
\label{lst:future_conventions_sofr_3m}
\end{listing}

\begin{listing}[H]
%\hrule\medskip
\begin{minted}[fontsize=\footnotesize]{xml}
  <Future>
    <Id>USD-SOFR-1M-FUTURES</Id>
    <Index>USD-SOFR</Index>
    <DateGenerationRule>FirstDayOfMonth</DateGenerationRule>
    <OvernightIndexFutureNettingType>Averaging</OvernightIndexFutureNettingType>
  </Future>
\end{minted}
\caption{USD SOFR 1M OI Future conventions}
\label{lst:future_conventions_sofr_1m}
\end{listing}


%- - - - - - - - - - - - - - - - - - - - - - - - - - - - - - - - - - - - - - - -
\subsubsection{FRA Conventions}
%- - - - - - - - - - - - - - - - - - - - - - - - - - - - - - - - - - - - - - - -
A node with name \emph{FRA} is used to store conventions for FRA quotes. The structure of this node is shown in Listing 
\ref{lst:fra_conventions}. The only piece of information needed is the underlying index name and this is given in the 
\lstinline!Index! node. For example, this could be EUR-EURIBOR-6M, CHF-LIBOR-6M etc.

\begin{listing}[H]
%\hrule\medskip
\begin{minted}[fontsize=\footnotesize]{xml}
<FRA>
  <Id> </Id>
  <Index> </Index>
</FRA>
\end{minted}
\caption{FRA conventions}
\label{lst:fra_conventions}
\end{listing}

%- - - - - - - - - - - - - - - - - - - - - - - - - - - - - - - - - - - - - - - -
\subsubsection{OIS Conventions}
%- - - - - - - - - - - - - - - - - - - - - - - - - - - - - - - - - - - - - - - -

A node with name \emph{OIS} is used to store conventions for Overnight Indexed Swap (OIS) quotes. The structure of this
node is shown in Listing \ref{lst:ois_conventions}.

\begin{listing}[H]
%\hrule\medskip
\begin{minted}[fontsize=\footnotesize]{xml}
<OIS>
  <Id> </Id>
  <SpotLag> </SpotLag>
  <Index> </Index>
  <FixedDayCounter> </FixedDayCounter>
  <FixedCalendar> </FixedCalendar>
  <PaymentLag> </PaymentLag>
  <EOM> </EOM>
  <FixedFrequency> </FixedFrequency>
  <FixedConvention> </FixedConvention>
  <FixedPaymentConvention> </FixedPaymentConvention>
  <Rule> </Rule>
  <PaymentCalendar> </PaymentCalendar>
</OIS>
\end{minted}
\caption{OIS conventions}
\label{lst:ois_conventions}
\end{listing}

The meanings of the various elements in this node are as follows:
\begin{itemize}
\item SpotLag: The number of business days until the start of the OIS.
\item Index: The name of the overnight index. For example, this could be EUR-EONIA, USD-FedFunds etc.
\item FixedDayCounter: The day count basis on the fixed leg of the OIS.
\item FixedCalendar [Optional]: The business day calendar on the fixed leg. Optional to retain backwards compatibility
  with older versions, if not given defaults to index fixing calendar.
\item PaymentLag [Optional]: The payment lag, as a number of business days, on both legs. If not provided, this defaults
to 0.
\item EOM [Optional]: \emph{True} if the end of month roll convention is to be used when generating the OIS schedule and
\emph{False} if not. If not provided, this defaults to \emph{False}.
\item FixedFrequency [Optional]: The frequency of payments on the fixed leg. If not provided, this defaults to
\emph{Annual}.
\item FixedConvention [Optional]: The roll convention for accruals on the fixed leg. If not provided, this defaults to
\emph{Following}.
\item FixedPaymentConvention [Optional]: The roll convention for payments on the fixed leg. If not provided, this
defaults to \emph{Following}.
\item Rule [Optional]: The rule used for generating the OIS dates schedule i.e.\ \emph{Backward} or \emph{Forward}. If
not provided, this defaults to \emph{Backward}.
\item PaymentCalendar [Optional]: The business day calendar used for determining coupon payment dates.
If not specified, this defaults to the fixing calendar defined on the overnight index.
\end{itemize}

%- - - - - - - - - - - - - - - - - - - - - - - - - - - - - - - - - - - - - - - -
\subsubsection{Swap Conventions}
%- - - - - - - - - - - - - - - - - - - - - - - - - - - - - - - - - - - - - - - -
A node with name \emph{Swap} is used to store conventions for vanilla interest rate swap (IRS) quotes. The structure of
this node is shown in Listing \ref{lst:swap_conventions}.

\begin{listing}[H]
%\hrule\medskip
\begin{minted}[fontsize=\footnotesize]{xml}
<Swap>
  <Id> </Id>
  <FixedCalendar> </FixedCalendar>
  <FixedFrequency> </FixedFrequency>
  <FixedConvention> </FixedConvention>
  <FixedDayCounter> </FixedDayCounter>
  <Index> </Index>
  <FloatFrequency> </FloatFrequency>
  <SubPeriodsCouponType> </SubPeriodsCouponType>
</Swap>
\end{minted}
\caption{Swap conventions}
\label{lst:swap_conventions}
\end{listing}

The meanings of the various elements in this node are as follows:
\begin{itemize}
\item FixedCalendar: The business day calendar on the fixed leg.
\item FixedFrequency: The frequency of payments on the fixed leg.
\item FixedConvention: The roll convention on the fixed leg.
\item FixedDayCounter: The day count basis on the fixed leg.
\item Index: The Ibor index on the floating leg.
\item FloatFrequency [Optional]: The frequency of payments on the floating leg, to be used if the frequency is different to the tenor of the index (e.g. CAD swaps for BA-3M have a 6M or 1Y payment frequency with a Compounding coupon)
\item SubPeriodsCouponType [Optional]: Defines how coupon rates should be calculated when the float frequency is different to that of the index. Possible values are "Compounding" and "Averaging".
\end{itemize}

%- - - - - - - - - - - - - - - - - - - - - - - - - - - - - - - - - - - - - - - -
\subsubsection{Average OIS Conventions}
%- - - - - - - - - - - - - - - - - - - - - - - - - - - - - - - - - - - - - - - -
A node with name \emph{AverageOIS} is used to store conventions for average OIS quotes. An average OIS is a swap where a
fixed rate is swapped against a daily averaged overnight index plus a spread. The structure of this node is shown in
Listing \ref{lst:average_ois_conventions}.

\begin{listing}[H]
%\hrule\medskip
\begin{minted}[fontsize=\footnotesize]{xml}
<AverageOIS>
  <Id> </Id>
  <SpotLag> </SpotLag>
  <FixedTenor> </FixedTenor>
  <FixedDayCounter> </FixedDayCounter>
  <FixedCalendar> </FixedCalendar>
  <FixedConvention> </FixedConvention>
  <FixedPaymentConvention> </FixedPaymentConvention>
  <FixedFrequency> </FixedFrequency>
  <Index> </Index>
  <OnTenor> </OnTenor>
  <RateCutoff> </RateCutoff>
</AverageOIS>
\end{minted}
\caption{Average OIS conventions}
\label{lst:average_ois_conventions}
\end{listing}


The meanings of the various elements in this node are as follows:
\begin{itemize}
\item SpotLag: Number of business days until the start of the average OIS.
\item FixedTenor: The frequency of payments on the fixed leg.
\item FixedDayCounter: The day count basis on the fixed leg.
\item FixedCalendar: The business day calendar on the fixed leg.
\item FixedFrequency: The frequency of payments on the fixed leg.
\item FixedConvention: The roll convention for accruals on the fixed leg.
\item FixedPaymentConvention: The roll convention for payments on the fixed leg.
\item FixedFrequency [Optional]: The frequency of payments on the fixed leg. If not provided, this defaults to \emph{Annual}.
\item Index: The name of the overnight index.
\item OnTenor: The frequency of payments on the overnight leg.
\item RateCutoff: The rate cut-off on the overnight leg. Generally, the overnight fixing is only observed up to a
certain number of days before the payment date and the last observed rate is applied for the remaining days in the
period. This rate cut-off gives the number of days e.g.\ 2 for Fed Funds average OIS.
\end{itemize}

%- - - - - - - - - - - - - - - - - - - - - - - - - - - - - - - - - - - - - - - -
\subsubsection{Tenor Basis Swap Conventions}
%- - - - - - - - - - - - - - - - - - - - - - - - - - - - - - - - - - - - - - - -
A node with name \emph{TenorBasisSwap} is used to store conventions for tenor basis swap quotes. The structure of this 
node is shown in Listing \ref{lst:tenor_basis_conventions}.

\begin{listing}[H]
%\hrule\medskip
\begin{minted}[fontsize=\footnotesize]{xml}
<TenorBasisSwap>
  <Id> </Id>
  <LongIndex> </LongIndex>
  <LongPayTenor> </ShortPayTenor>
  <ShortIndex> </ShortIndex>
  <ShortPayTenor> </ShortPayTenor>
  <SpreadOnShort> </SpreadOnShort>
  <IncludeSpread> </IncludeSpread>
  <SubPeriodsCouponType> </SubPeriodsCouponType>
</TenorBasisSwap>
\end{minted}
\caption{Tenor basis swap conventions}
\label{lst:tenor_basis_conventions}
\end{listing}


The meanings of the various elements in this node are as follows:
\begin{itemize}
\item LongIndex: The name of the long tenor Ibor index. In the case of basis swaps with equal tenor indexes (like overnight 
indexed vs overnight indexed basis swaps) it should be interpreted as the index of the received leg.
\item LongPayTenor [Optional]: The frequency of payments on the LongIndex leg. This is usually the same as the LongIndex's 
tenor. However, it can also be longer, e.g. overnight indexed vs overnight indexed basis swaps that may be quarterly on 
both legs. If not provided, this defaults to the LongIndex's tenor.
\item ShortIndex: The name of the short tenor Ibor or overnight index.
\item ShortPayTenor [Optional]: The frequency of payments on the ShortIndex leg. This is usually the same as the
ShortIndex's tenor. However, it can also be longer e.g.\ USD tenor basis swaps where the short tenor Ibor
index is compounded and paid on the same frequency as the long tenor Ibor index, or overnight indexed vs overnight 
indexed basis swaps that may be quarterly on both legs. If not provided, this defaults to the ShortIndex's tenor.
\item SpreadOnShort [Optional]: \emph{True} if the tenor basis swap quote has the spread on the short tenor Ibor index
leg and \emph{False} if not. If not provided, this defaults to \emph{True}.
\item IncludeSpread [Optional]: \emph{True} if the tenor basis swap spread is to be included when compounding is
performed on the short tenor Ibor index leg and \emph{False} if not. If not provided, this defaults to \emph{False}.
\item SubPeriodsCouponType [Optional]: This field can have the value \emph{Compounding} or \emph{Averaging}. It applies 
to Ibor vs OI and Ibor vs Ibor basis swaps when the frequency of payments on the short tenor leg does not equal the 
short tenor index's tenor. If \emph{Compounding} is specified, then the short tenor Ibor index is compounded and paid on
the frequency specified in the \lstinline!ShortPayTenor! node. If \emph{Averaging} is specified, then the short tenor 
Ibor index is averaged and paid on the frequency specified in the \lstinline!ShortPayTenor! node. If not provided, this 
defaults to \emph{Compounding}. In the context of overnight indexed vs overnight indexed basis swaps this value will apply 
to both legs.
\end{itemize}

%- - - - - - - - - - - - - - - - - - - - - - - - - - - - - - - - - - - - - - - -
\subsubsection{Tenor Basis Two Swap Conventions}
%- - - - - - - - - - - - - - - - - - - - - - - - - - - - - - - - - - - - - - - -
A node with name \emph{TenorBasisTwoSwap} is used to store conventions for tenor basis swap quotes where the quote is
the spread between the fair fixed rate on two swaps against Ibor indices of different tenors. We call the swap against
the Ibor index of longer tenor the long swap and the remaining swap the short swap. The structure of the tenor basis two
swap conventions node is shown in Listing \ref{lst:tenor_basis_two_conventions}.

\begin{listing}[H]
%\hrule\medskip
\begin{minted}[fontsize=\footnotesize]{xml}
<TenorBasisTwoSwap>
  <Id> </Id>
  <Calendar> </Calendar>
  <LongFixedFrequency> </LongFixedFrequency>
  <LongFixedConvention> </LongFixedConvention>
  <LongFixedDayCounter> </LongFixedDayCounter>
  <LongIndex> </LongIndex>
  <ShortFixedFrequency> </ShortFixedFrequency>
  <ShortFixedConvention> </ShortFixedConvention>
  <ShortFixedDayCounter> </ShortFixedDayCounter>
  <ShortIndex> </ShortIndex>
  <LongMinusShort> </LongMinusShort>
</TenorBasisTwoSwap>
\end{minted}
\caption{Tenor basis two swap conventions}
\label{lst:tenor_basis_two_conventions}
\end{listing}

The meanings of the various elements in this node are as follows:
\begin{itemize}
\item Calendar: The business day calendar on both swaps.
\item LongFixedFrequency: The frequency of payments on the fixed leg of the long swap.
\item LongFixedConvention: The roll convention on the fixed leg of the long swap.
\item LongFixedDayCounter: The day count basis on the fixed leg of the long swap.
\item LongIndex: The Ibor index on the floating leg of the long swap.
\item ShortFixedFrequency: The frequency of payments on the fixed leg of the short swap.
\item ShortFixedConvention: The roll convention on the fixed leg of the short swap.
\item ShortFixedDayCounter: The day count basis on the fixed leg of the short swap.
\item ShortIndex: The Ibor index on the floating leg of the short swap.
\item LongMinusShort [Optional]: \emph{True} if the basis swap spread is to be interpreted as the fair rate on the long
swap minus the fair rate on the short swap and \emph{False} if the basis swap spread is to be interpreted as the fair
rate on the short swap minus the fair rate on the long swap. If not provided, it defaults to \emph{True}.
\end{itemize}

%- - - - - - - - - - - - - - - - - - - - - - - - - - - - - - - - - - - - - - - -
\subsubsection{FX Conventions}\label{sss:fx_convention}
%- - - - - - - - - - - - - - - - - - - - - - - - - - - - - - - - - - - - - - - -
A node with name \emph{FX} is used to store conventions for FX spot and forward quotes for a given currency pair. The
structure of this node is shown in Listing \ref{lst:fx_conventions}.

\begin{listing}[H]
%\hrule\medskip
\begin{minted}[fontsize=\footnotesize]{xml}
<FX>
  <Id> </Id>
  <SpotDays> </SpotDays>
  <SourceCurrency> </SourceCurrency>
  <TargetCurrency> </TargetCurrency>
  <PointsFactor> </PointsFactor>
  <AdvanceCalendar> </AdvanceCalendar>
  <SpotRelative> </SpotRelative>
</FX>
\end{minted}
\caption{FX conventions}
\label{lst:fx_conventions}
\end{listing}


The meanings of the various elements in this node are as follows:
\begin{itemize}
\item SpotDays: The number of business days to spot for the currency pair.
\item SourceCurrency: The source currency of the currency pair. The FX quote is assumed to give the number of units of
target currency per unit of source currency.
\item TargetCurrency: The target currency of the currency pair.
\item PointsFactor: The number by which a points quote for the currency pair should be divided before adding it to the
spot quote to obtain the forward rate.
\item AdvanceCalendar [Optional]: The business day calendar(s) used for advancing dates for both spot and forwards. If
not provided, it defaults to a calendar with no holidays.
\item SpotRelative [Optional]: \emph{True} if the forward tenor is to be interpreted as being relative to the spot date.
\emph{False} if the forward tenor is to be interpreted as being relative to the valuation date. If not provided, it
defaults to \emph{True}.
\end{itemize}

%- - - - - - - - - - - - - - - - - - - - - - - - - - - - - - - - - - - - - - - -
\subsubsection{Cross Currency Basis Swap Conventions}
%- - - - - - - - - - - - - - - - - - - - - - - - - - - - - - - - - - - - - - - -
A node with name \emph{CrossCurrencyBasis} is used to store conventions for cross currency basis swap quotes. The
structure of this node is shown in Listing \ref{lst:xccy_basis_conventions}.

\begin{listing}[H]
%\hrule\medskip
\begin{minted}[fontsize=\footnotesize]{xml}
<CrossCurrencyBasis>
  <Id> </Id>
  <SettlementDays> </SettlementDays>
  <SettlementCalendar> </SettlementCalendar>
  <RollConvention> </RollConvention>
  <FlatIndex> </FlatIndex>
  <SpreadIndex> </SpreadIndex>
  <EOM> </EOM>
  <IsResettable> </IsResettable>
  <FlatIndexIsResettable> </FlatIndexIsResettable>>
  <PaymentLag> </PaymentLag>
  <FlatPaymentLag> </FlatPaymentLag>
  <!-- for OIS only -->
  <IncludeSpread> </IncludeSpread>
  <Lookback> </Lookback>
  <FixingDays> </FixingDays>
  <RateCutoff> </RateCutoff>
  <IsAveraged> </IsAveraged>
  <FlatIncludeSpread> </FlatIncludeSpread>
  <FlatLookback> </FlatLookback>
  <FlatFixingDays> </FlatFixingDays>
  <FlatRateCutoff> </FlatRateCutoff>
  <FlatIsAveraged> </FlatIsAveraged>
</CrossCurrencyBasis>
\end{minted}
\caption{Cross currency basis swap conventions}
\label{lst:xccy_basis_conventions}
\end{listing}


The meanings of the various elements in this node are as follows:
\begin{itemize}
\item SettlementDays: The number of business days to the start of the cross currency basis swap.
\item SettlementCalendar: The business day calendar(s) for both legs and to arrive at the settlement date using the
SettlementDays above.
\item RollConvention: The roll convention for both legs.
\item FlatIndex: The name of the index on the leg that does not have the cross currency basis spread.
\item SpreadIndex: The name of the index on the leg that has the cross currency basis spread.
\item EOM [Optional]: \emph{True} if the end of month convention is to be used when generating the schedule on both legs, and \emph{False} if not. If not provided, it defaults to \emph{False}.
\item IsResettable [Optional]: \emph{True} if the swap is mark-to-market resetting, and \emph{False} otherwise. If not provided, it defaults to \emph{False}.
\item FlatIndexIsResettable [Optional]: \emph{True} if it is the notional on the leg paying the flat index that resets, and \emph{False} otherwise. If not provided, it defaults to \emph{True}.
\item FlatTenor [Optional]: the flat leg period length (typical value is 3M), defaults to index tenor except for ON indices for which it defaults to 3M
\item SpreadTenor [Optional]: the spread leg period length (typical value is 3M), defaults to index tenor except for ON indices for which it defaults to 3M
\item SpreadPaymentLag [Optional]: the payment lag for the spread leg, allowable values are 0, 1, 2, ..., defaults to 0 if not given
\item FlatPaymentLag [Optional]: the payment lag for the flat leg, allowable values are 0, 1, 2, ..., defaults to 0 if nove given
\item SpreadIncludeSpread [Optional]: Only relevant if spread leg is OIS, allowable values are true, false, defaults to false if not given
\item SpreadLookback [Optional]: Only relevant if spread leg is OIS, allowable values are 0D, 1D, ..., defaults to 0D if not given
\item SpreadFixingDays [Optional]: Only relevant if spread leg is OIS, allowable values are 0, 1, 2, ..., defaults to 0 if not given
\item SpreadRateCutoff [Optional]: Only relevant if spread leg is OIS, allowable values are 0, 1, 2, ..., defaults to 0 if not given
\item SpreadIsAveraged [Optional]: Only relevant if spread leg is OIS, allowable values are true, false, defaults to false if not given
\item FlatIncludeSpread [Optional]: Only relevant if spread leg is OIS, allowable values are true, false, defaults to false if not given
\item FlatLookback [Optional]: Only relevant if spread leg is OIS, allowable values are 0D, 1D, ..., defaults to 0D if not given
\item FlatFixingDays [Optional]: Only relevant if spread leg is OIS, allowable values are 0, 1, 2, ..., defaults to 0 if not given
\item FlatRateCutoff [Optional]: Only relevant if spread leg is OIS, allowable values are 0, 1, 2, ..., defaults to 0 if not given
\item FlatIsAveraged [Optional]: Only relevant if spread leg is OIS, allowable values are true, false, defaults to false if not given
\end{itemize}

\subsubsection{Inflation Swap Conventions}
A node with name \lstinline!InflationSwap! is used to store conventions for zero or year on year inflation swap quotes. The structure of this node is shown in Listing \ref{lst:inflation_conventions}

\begin{listing}[H]
%\hrule\medskip
\begin{minted}[fontsize=\footnotesize]{xml}
<InflationSwap>
  <Id>EUHICPXT_INFLATIONSWAP</Id>
  <FixCalendar>TARGET</FixCalendar>
  <FixConvention>MF</FixConvention>
  <DayCounter>30/360</DayCounter>
  <Index>EUHICPXT</Index>
  <Interpolated>false</Interpolated>
  <ObservationLag>3M</ObservationLag>
  <AdjustInflationObservationDates>false</AdjustInflationObservationDates>
  <InflationCalendar>TARGET</InflationCalendar>
  <InflationConvention>MF</InflationConvention>
</InflationSwap>
\end{minted}
\caption{Inflation swap conventions}
\label{lst:inflation_conventions}
\end{listing}

The meaning of the elements is as follows:

\begin{itemize}
\item \lstinline!FixCalendar!: The calendar for the fixed rate leg of the swap.
\item \lstinline!FixConvention!: The rolling convention for the fixed rate leg of the swap.
\item \lstinline!DayCounter!: The payoff or coupon day counter, applied to both legs.
\item \lstinline!Index!: The underlying inflation index.
\item \lstinline!Interpolated!: Flag indicating interpolation of the index in the swap's payoff calculation.
\item \lstinline!ObservationLag!: The index observation lag to be applied.
\item \lstinline!AdjustInflationObservationDates!: Flag indicating whether index observation dates should be adjusted or not.
\item \lstinline!InflationCalendar!: The calendar for the inflation leg of the swap.
\item \lstinline!InflationConvention!: The rolling convention for the inflation leg of the swap.

\item \lstinline!PublicationRoll!:
This is an optional node taking the values \lstinline!None!, \lstinline!OnPublicationDate! or \lstinline!AfterPublicationDate!. If omitted, the value \lstinline!None! is used. Currently, our only known use case for a value other than \lstinline!None! is for Australian zero coupon inflation indexed swaps (ZCIIS). Here, the index is published quarterly on the last Wednesday of the month following the end of the reference quarter. The start date and maturity date of the market quoted ZCIIS roll to the next quarterly date after the publication date of the index. For example, the AU CPI value for Q3 2020, i.e.\ 1 Jul 2020 to 30 Sep 2020 was released on 28 Oct 2020. On 27 Oct 2020, before the index publication date, the market 5Y ZCIIS would start on 15 Sep 2020 and end on 15 Sep 2025 and reference the Q2 inflation index value. On 29 Oct 2020, after the index publication date, the market 5Y ZCIIS would start on 15 Dec 2020 and end on 15 Dec 2025 and reference the Q3 inflation index value. On the release date, i.e. 28 Oct 2020, the market ZCIIS that is set up is determined by whether the \lstinline!PublicationRoll! value is \lstinline!OnPublicationDate! or \lstinline!AfterPublicationDate!. If it is set to \lstinline!OnPublicationDate!, the swap rolls on this date and hence the market 5Y ZCIIS would start on 15 Dec 2020 and end on 15 Dec 2025 and reference the Q3 inflation index value. If it is set to \lstinline!AfterPublicationDate!, the swap does not roll on the publication date and instead rolls on the next day, and hence the market 5Y ZCIIS would start on 15 Sep 2020 and end on 15 Sep 2025 and reference the Q2 inflation index value. The publication schedule for the index must be provided in the \lstinline!PublicationSchedule! node if \lstinline!PublicationRoll! is not \lstinline!None!. An example of the AU CPI conventions set up is given in Listing \ref{lst:aucpi_inflation_conventions}.

\item \lstinline!PublicationSchedule!:
This is an optional node and is not used if \lstinline!PublicationRoll! is \lstinline!None!. If \lstinline!PublicationRoll! is not \lstinline!None!, it must be provided and gives the publication dates for the inflation index. The node fields are the same fields that are described in the Section \ref{ss:schedule_data}, i.e.\ they are \lstinline!ScheduleData! elements. An example of the AU CPI conventions set up is given in Listing \ref{lst:aucpi_inflation_conventions}. The \lstinline!PublicationSchedule! must cover the dates on which you intend to perform valuations, i.e. the first publication schedule date must be less than the smallest valuation date that you intend to use and the last publication schedule date must be greater than the largest valuation date that you intend to use.

\end{itemize}

\begin{listing}[H]
\begin{minted}[fontsize=\footnotesize]{xml}
<InflationSwap>
  <Id>AUCPI_INFLATIONSWAP</Id>
  <FixCalendar>AUD</FixCalendar>
  <FixConvention>F</FixConvention>
  <DayCounter>30/360</DayCounter>
  <Index>AUCPI</Index>
  <Interpolated>false</Interpolated>
  <ObservationLag>3M</ObservationLag>
  <AdjustInflationObservationDates>false</AdjustInflationObservationDates>
  <InflationCalendar>AUD</InflationCalendar>
  <InflationConvention>F</InflationConvention>
  <PublicationRoll>AfterPublicationDate</PublicationRoll>
  <PublicationSchedule>
    <Rules>
      <StartDate>2001-01-24</StartDate>
      <EndDate>2030-01-30</EndDate>
      <Tenor>3M</Tenor>
      <Calendar>AUD</Calendar>
      <Convention>Preceding</Convention>
      <TermConvention>Unadjusted</TermConvention>
      <Rule>LastWednesday</Rule>
    </Rules>
  </PublicationSchedule>
</InflationSwap>
\end{minted}
\caption{AU CPI inflation swap conventions}
\label{lst:aucpi_inflation_conventions}
\end{listing}

%- - - - - - - - - - - - - - - - - - - - - - - - - - - - - - - - - - - - - - - -
\subsubsection{CMS Spread Option Conventions}
%- - - - - - - - - - - - - - - - - - - - - - - - - - - - - - - - - - - - - - - -

A node with name \emph{CmsSpreadOption} is used to store the conventions.

\begin{listing}[H]
%\hrule\medskip
\begin{minted}[fontsize=\footnotesize]{xml}
  <CmsSpreadOption>
    <Id>EUR-CMS-10Y-2Y-CONVENTION</Id>
    <ForwardStart>0M</ForwardStart>
    <SpotDays>2D</SpotDays>
    <SwapTenor>3M</SwapTenor>
    <FixingDays>2</FixingDays>
    <Calendar>TARGET</Calendar>
    <DayCounter>A360</DayCounter>
    <RollConvention>MF</RollConvention>
  </CmsSpreadOption>
\end{minted}
\caption{Inflation swap conventions}
\label{lst:cms_spread_option_conventions}
\end{listing}

The meaning of the elements is as follows:

\begin{itemize}
\item ForwardStart: The calendar for the fixed rate leg of the swap.
\item SpotDays: The number of business days to spot for the CMS Spread Index.
\item SwapTenor: The frequency of payments on the CMS Spread leg.
\item FixingDays: The number of fixing days.
\item Calendar: The calendar for the CMS Spread leg.
\item DayCounter: The day counter for the CMS Spread leg.
\item RollConvention: The rolling convention for the CMS Spread Leg.
\end{itemize}

%- - - - - - - - - - - - - - - - - - - - - - - - - - - - - - - - - - - - - - - -
\subsubsection{Ibor Index Conventions}
%- - - - - - - - - - - - - - - - - - - - - - - - - - - - - - - - - - - - - - - -

A node with name \emph{IborIndex} is used to store conventions for Ibor indices. This can be used to define new Ibor
indices without the need of adding them to the C++ code, or also to override the conventions of existing Ibor indices.

\begin{listing}[H]
%\hrule\medskip
\begin{minted}[fontsize=\footnotesize]{xml}
  <IborIndex>
    <Id>EUR-EURIBOR_ACT365-3M</Id>
    <FixingCalendar>TARGET</FixingCalendar>
    <DayCounter>A365F</DayCounter>
    <SettlementDays>2</SettlementDays>
    <BusinessDayConvention>MF</BusinessDayConvention>
    <EndOfMonth>true</EndOfMonth>
  </IborIndex>
\end{minted}
\caption{Ibor index convention}
\label{lst:ibor_index_conventions}
\end{listing}

The meaning of the elements is as follows:

\begin{itemize}
\item Id: The index name. This must be of the form ``CCY-NAME-TENOR'' with a currency ``CCY'', an index name ``NAME''
  and a string ``TENOR'' representing a period. The name should not be ``GENERIC'', since this is reserved.
\item FixingCalendar: The fixing calendar of the index.
\item DayCounter: The day count convention used by the index.
\item SettlementDays: The settlement days for the index. This must be a non-negative whole number.
\item BusinessDayConvention: The business day convention used by the index.
\item EndOfMonth: A flag indicating whether the index employs the end of month convention.
\end{itemize}

Notice that if another convention depends on an Ibor index convention (because it contains the Ibor index name defined
in the latter convention), the Ibor index convention must appear before the convention that depends on it in the
convention input file.

Also notice that customised indices can not be used in cap / floor volatility surface configurations.

%- - - - - - - - - - - - - - - - - - - - - - - - - - - - - - - - - - - - - - - -
\subsubsection{Overnight Index Conventions}
%- - - - - - - - - - - - - - - - - - - - - - - - - - - - - - - - - - - - - - - -

A node with name \emph{OvernightIndex} is used to store conventions for Overnight indices. This can be used to define
new Overnight indices without the need of adding them to the C++ code, or also to override the conventions of existing
Overnight indices.

\begin{listing}[H]
%\hrule\medskip
\begin{minted}[fontsize=\footnotesize]{xml}
  <OvernightIndex>
    <Id>EUR-ESTER</Id>
    <FixingCalendar>TARGET</FixingCalendar>
    <DayCounter>A360</DayCounter>
    <SettlementDays>0</SettlementDays>
  </OvernightIndex>
\end{minted}
\caption{Overnight index convention}
\label{lst:overnight_index_conventions}
\end{listing}

The meaning of the elements is as follows:

\begin{itemize}
\item Id: The index name. This must be of the form ``CCY-NAME'' with a currency ``CCY'' and an index name ``NAME''. The
  name should not be ``GENERIC'', since this is reserved.
\item FixingCalendar: The fixing calendar of the index.
\item DayCounter: The day count convention used by the index.
\item SettlementDays: The settlement days for the index. This must be a non-negative whole number.
\end{itemize}

Notice that if another convention depends on an Overnight index convention (because it contains the Overnight index name
defined in the latter convention), the Overnight index convention must appear before the convention that depends on it
in the convention input file.

Also notice that customised indices can not be used in cap / floor volatility surface configurations.

\subsubsection{Inflation Index Conventions}
A node with the name \lstinline!ZeroInflationIndex! is used to store data for the creation of a new inflation index. This avoids having to add the index definition to the C++ code and recompile. Note that the \lstinline!ZeroInflationIndex! node should be placed before its use in any other convention, e.g.\ in an \lstinline!InflationSwap! convention, to avoid an error due to the new index itself not being created. If the \lstinline!Id! node matches an existing inflation index, the newly created index will take precedence and its defintion will be used in the code for the given \lstinline!Id!.

\begin{listing}[H]
\begin{minted}[fontsize=\footnotesize]{xml}
<ZeroInflationIndex>
  <Id>...</Id>
  <RegionName>...</RegionName>
  <RegionCode>...</RegionCode>
  <Revised>...</Revised>
  <Frequency>...</Frequency>
  <AvailabilityLag>...</AvailabilityLag>
  <Currency>...</Currency>
</ZeroInflationIndex>
\end{minted}
\caption{\emph{ZeroInflationIndex} node}
\label{lst:zero_inflation_index_conventions}
\end{listing}

The meaning of each element is as follows:
\begin{itemize}
\item \lstinline!Id!: The new inflation index name.
\item \lstinline!RegionName!: The name of the region with which the inflation index is associated.
\item \lstinline!RegionCode!: A code for the region with which the inflation index is associated.
\item \lstinline!Revised!: A boolean flag indicating whether the index is a revised index or not. This is generally set to \lstinline!false! but is left as an option to align with the C++ \lstinline!InflationIndex! class definition.
\item \lstinline!Frequency!: A valid frequency indicating the publication frequency of the inflation index, generally \lstinline!Monthly!, \lstinline!Quarterly! or \lstinline!Annual!.
\item \lstinline!AvailabilityLag!: A valid period indicating the lag between the inflation index publication for a given period and the period itself. For example, if March's inflation index value is published in April, the \lstinline!AvailabilityLag! would be \lstinline!1M!.
\item \lstinline!Currency!: The ISO currency code of the currency associated with the inflation index, generally the currency of the region.
\end{itemize}

%- - - - - - - - - - - - - - - - - - - - - - - - - - - - - - - - - - - - - - - -
\subsubsection{Swap Index Conventions}
%- - - - - - - - - - - - - - - - - - - - - - - - - - - - - - - - - - - - - - - -

A node with name \emph{SwapIndex} is used to store conventions for Swap indices (also known as ``CMS'' indices).

\begin{listing}[H]
%\hrule\medskip
\begin{minted}[fontsize=\footnotesize]{xml}
  <SwapIndex>
    <Id>EUR-CMS-2Y</Id>
    <Conventions>EUR-EURIBOR-6M-SWAP</Conventions>
    <FixingCalendar>TARGET</FixingCalendar>
  </SwapIndex>
\end{minted}
\caption{Swap index convention}
\label{lst:swap_index_conventions}
\end{listing}

The meaning of the elements is as follows:

\begin{itemize}
\item Id: The index name. This must be of the form ``CCY-CMS-TENOR'' with a currency ``CCY'' and a string ``TENOR''
  representing a period. The index name can contain an optional tag ``CCY-CMS-TAG-TENOR'' which is an arbitrary label
  that allows to define more than one swap index per currency.
\item Conventions: A swap convention defining the index conventions.
\item FixingCalendar [Optional]: The fixing calendar for the swap index fixings publication. If not given, the fixed leg
  calendar from the swap conventions will be used as a fall back.
\end{itemize}

%- - - - - - - - - - - - - - - - - - - - - - - - - - - - - - - - - - - - - - - - 
\subsubsection{FX Option Conventions}\label{sss:fx_option_conv}
%- - - - - - - - - - - - - - - - - - - - - - - - - - - - - - - - - - - - - - - - 
A node with name \emph{FxOption} is used to store conventions for FX option quotes for a given currency pair. The 
structure of this node is shown in Listing \ref{lst:fx_option_conventions}. 
 
\begin{listing}[H] 
%\hrule\medskip 
\begin{minted}[fontsize=\footnotesize]{xml}
<FxOption>
  <Id>EUR-USD-FXOPTION</Id>
  <FXConventionID>EUR-USD-FX</FXConventionID>
  <AtmType>AtmDeltaNeutral</AtmType>
  <DeltaType>Spot</DeltaType>
  <SwitchTenor>2Y</SwitchTenor>
  <LongTermAtmType>AtmDeltaNeutral</LongTermAtmType>
  <LongTermDeltaType>Fwd</LongTermDeltaType>
  <RiskReversalInFavorOf>Call</RiskReversalInFavorOf>
  <ButterflyStyle>Broker</ButterflyStyle>
</FxOption>
\end{minted} 
\caption{FX option conventions} 
\label{lst:fx_option_conventions} 
\end{listing} 
 
 
The meanings of the various elements in this node are as follows: 
\begin{itemize}
\item FXConventionID: The FX convention for the currency pair (see \ref{sss:fx_convention}). Optional, if not given, the
  FX spot days default to $2$ and the advance calendar defaults to source ccy + target ccy default calendars.
\item AtmType: Convention of ATM option quote (Choices are {\em AtmNull, AtmSpot, AtmFwd, 
AtmDeltaNeutral, AtmVegaMax, AtmGammaMax, AtmPutCall50}). 
\item DeltaType: Convention of Delta option quote (Choices are {\em Spot, Fwd, PaSpot, 
    PaFwd}).
\item SwitchTenor [Optional]: If given, different ATM and Delta conventions will be used if the option tenor is greater
  or equal the switch tenor (``long term'' atm and delta type)
\item LongTermAtmType [Mandatory if and only if SwitchTenor is given]: ATM type to use for options with tenor > switch
  point, if SwitchTenor is given
\item LongTermDeltaType [Mandatory if and only if SwitchTenor is given]: Delta type to use for options with tenor >
  switch point, if SwitchTenor is given
\item RiskReversalInFavorOf [Optional]: Call (default), Put. Only relevant for BF, RR market data input.
\item ButterflyStyle [Optional]: Broker (default), Smile. Only relevant for BF, RR market data input.
\end{itemize} 

%- - - - - - - - - - - - - - - - - - - - - - - - - - - - - - - - - - - - - - - -
\subsubsection{Commodity Forward Conventions}
%- - - - - - - - - - - - - - - - - - - - - - - - - - - - - - - - - - - - - - - -
A node with name \lstinline!CommodityForward! is used to store conventions for commodity forward price quotes. The
structure of this node is shown in Listing \ref{lst:commodity_forward_conventions}.

\begin{listing}[H]
\begin{minted}[fontsize=\footnotesize]{xml}
<CommodityForward>
  <Id>...</Id>
  <SpotDays>...</SpotDays>
  <PointsFactor>...</PointsFactor>
  <AdvanceCalendar>...</AdvanceCalendar>
  <SpotRelative>...</SpotRelative>
  <BusinessDayConvention>...</BusinessDayConvention>
  <Outright>...</Outright>
</CommodityForward>
\end{minted}
\caption{Commodity forward conventions}
\label{lst:commodity_forward_conventions}
\end{listing}

The meanings of the various elements in this node are as follows:
\begin{itemize}
\item \lstinline!Id!: The identifier for the commodity forward convention. The identifier here should match the \lstinline!Name! that would be provided for the commodity in the trade XML as described in Table \ref{tab:commodity_data}.
\item \lstinline!SpotDays! [Optional]: The number of business days to spot for the commodity. Any non-negative integer is allowed here. If omitted, this takes a default value of 2.
\item \lstinline!PointsFactor! [Optional]: This is only used if \lstinline!Outright! is \lstinline!false!. Any positive real number is allowed here. When \lstinline!Outright! is \lstinline!false!, the commodity forward quotes are provided as points i.e. a number that should be added to the commodity spot to give the outright commodity forward rate. The \lstinline!PointsFactor! is the number by which the points quote should be divided before adding it to the spot quote to obtain the forward price. If omitted, this takes a default value of 1.
\item \lstinline!AdvanceCalendar! [Optional]: The business day calendar(s) used for advancing dates for both spot and forwards. The allowable values are given in Table \ref{tab:calendar}. If omitted, it defaults to \lstinline!NullCalendar! i.e. a calendar where all days are considered good business days.
\item \lstinline!SpotRelative! [Optional]: The allowable values are \lstinline!true! and \lstinline!false!. If \lstinline!true!, the forward tenor is interpreted as being relative to the spot date. If \lstinline!false!, the forward tenor is interpreted as being relative to the valuation date. If omitted, it defaults to \lstinline!True!.
\item \lstinline!BusinessDayConvention! [Optional]: The business day roll convention used to adjust dates when getting from the valuation date to the spot date and the forward maturity date. The allowable values are given in Table \ref{tab:allow_stand_data}. If omitted, it defaults to \lstinline!Following!.
\item \lstinline!Outright! [Optional]: The allowable values are \lstinline!true! and \lstinline!false!. If \lstinline!true!, the forward quotes are interpreted as outright forward prices. If \lstinline!false!, the forward quotes are interpreted as points i.e. as a number that must be added to the spot price to get the outright forward price. If omitted, it defaults to \lstinline!true!.
\end{itemize}

\subsubsection{Commodity Future Conventions}
\label{sec:commodity_future_conventions}
A node with name \lstinline!CommodityFuture! is used to store conventions for commodity future contracts and options on them. These conventions are used in commodity derivative trades and commodity curve construction to calculate contract expiry dates. The structure of this node is shown in Listing \ref{lst:commodity_future_conventions}.

\begin{listing}[h!]
\begin{minted}[fontsize=\footnotesize,breaklines]{xml}
<CommodityFuture>
  <Id>...</Id>
  <AnchorDay>
    ...
  </AnchorDay>
  <ContractFrequency>...</ContractFrequency>
  <Calendar>...</Calendar>
  <ExpiryCalendar>...</ExpiryCalendar>
  <ExpiryMonthLag>...</ExpiryMonthLag>
  <OneContractMonth>...</OneContractMonth>
  <OffsetDays>...</OffsetDays>
  <BusinessDayConvention>...</BusinessDayConvention>
  <AdjustBeforeOffset>...</AdjustBeforeOffset>
  <IsAveraging>...</IsAveraging>
  <OptionExpiryOffset>...</OptionExpiryOffset>
  <ProhibitedExpiries>
    <Dates>
      <Date forFuture="true" convention="Preceding" forOption="true" optionConvention="Preceding">...</Date>
        ...
    </Dates>
  </ProhibitedExpiries>
  <OptionExpiryMonthLag>...</OptionExpiryMonthLag>
  <OptionExpiryDay>...</OptionExpiryDay>
  <OptionContractFrequency>...</OptionContractFrequency>
  <OptionNthWeekday>
    <Nth>...</Nth>
    <Weekday>...</Weekday>
  </OptionNthWeekday>
  <OptionExpiryLastWeekdayOfMonth>...</OptionExpiryLastWeekdayOfMonth>
  <OptionExpiryWeeklyDayOfTheWeek>...</OptionExpiryWeeklyDayOfTheWeek>
  <OptionBusinessDayConvention>...</OptionBusinessDayConvention>
  <FutureContinuationMappings>
    <ContinuationMapping>
      <From>...</From>
      <To>...</To>
    </ContinuationMapping>
    ...
  </FutureContinuationMappings>
  <OptionContinuationMappings>
    <ContinuationMapping>
      <From>...</From>
      <To>...</To>
    </ContinuationMapping>
    ...
  </OptionContinuationMappings>
  <AveragingData>
    ...
  </AveragingData>
  <HoursPerDay>...</HoursPerDay>
  <SavingsTime>...<SavingsTime>
  <ValidContractMonths>
  	<Month>...</Month>
  </ValidContractMonths>
  <OptionUnderlyingFutureConvention>...</OptionUnderlyingFutureConvention>
</CommodityFuture>
\end{minted}
\caption{Commodity future conventions}
\label{lst:commodity_future_conventions}
\end{listing}

The meanings of the various elements in this node are as follows:
\begin{itemize}
\item \lstinline!Id!: The identifier for the commodity future convention. The identifier here should match the \lstinline!Name! that would be provided for the commodity in the trade XML as described in Table \ref{tab:commodity_data}.
\item \lstinline!AnchorDay! [Optional]: This node is not applicable for daily future contracts and hence is optional. It is necessary for future contracts with a monthly cycle or greater or if the option contracts cycle is monthly or greater.  This node is used to give a date in the future contract month to use as a base date for calculating the expiry date. It can contain a \lstinline!DayOfMonth! node, a \lstinline!CalendarDaysBefore! node or an \lstinline!NthWeekday! node:
    \begin{itemize}
    \item The \lstinline!DayOfMonth! This node can contain any integer in the range $1,\ldots,31$ indicating the day of the month. A value of 31 will guarantee that the last day in the month is used a base date.
    \item The \lstinline!CalendarDaysBefore! This node can contain any non-negative integer. The contract expiry date is this number of calendar days before the first calendar day of the contract month.
    \item The \lstinline!NthWeekday! This node has the elements shown in Listing \ref{lst:nth_weekday_node}. This node is used to indicate a date in a given month in the form of the n-th named weekday of that month e.g. 3rd Wednesday. The allowable values for \lstinline!Nth! are ${1,2,3,4}$. The \lstinline!Weekday! node takes a weekday in the form of the first three characters of the weekday with the first character capitalised.
    \item The \lstinline!LastWeekday! [Optional]: This node is used to indicate a date in a given month in the form of the last named weekday of that month e.g. last Wednesday. The node takes a weekday in the form of the first three characters of the weekday with the first character capitalised.
    \item The \lstinline!BusinessDaysAfter! This node can contain any integer. If the number is positive the contract expiry is the n-th business day of the contract month. If the number is negative the contract expiry date is this number of business days before the first calendar day of the contract month.
    \item The \lstinline!WeeklyDayOfTheWeek! [Optional]: This node is used to indicate a date in a given week in the form of the named weekday, e.g. Wednesday. This node is mandatory for weekly contract frequencies and is not allowed with any other frequency.  The node takes a weekday in the form of the first three characters of the weekday with the first character capitalised.
    \end{itemize}
\item \lstinline!ContractFrequency!: This node indicates the frequency of the commodity future contracts. The value here is usually \lstinline!Monthly! or \lstinline!Quarterly!, but allowed values are \lstinline!Daily!, \lstinline!Weekly!, \lstinline!Monthly!, \lstinline!Quaterly! and \lstinline!Annual!.
\item \lstinline!Calendar!: The business day trading calendar(s) applicable for the commodity future contract.
\item \lstinline!ExpiryCalendar! [Optional]: The business day expiry calendar(s) applicable for the commodity future contract. This calendar is used when deriving expiry dates. If omitted, this defaults to the trading day calendar specified in the \lstinline!Calendar! node.
\item \lstinline!ExpiryMonthLag! [Optional]: The allowable values are any integer. This value indicates the number of months from the month containing the expiry date to the contract month. If 0, the commodity future contract expiry date is in the contract month. If the value of \lstinline!ExpiryMonthLag! is $n > 0$, the commodity future contract expires in the $n$-th month prior to the contract month. If the value of \lstinline!ExpiryMonthLag! is $n < 0$, the commodity future contract expires in the $n$-th month after the contract month. The value of \lstinline!ExpiryMonthLag! is generally 0, 1 or 2. For example, \lstinline!NYMEX:CL! has an \lstinline!ExpiryMonthLag! of 1 and \lstinline!ICE:B! has an \lstinline!ExpiryMonthLag! of 2. If omitted, it defaults to 0.
\item \lstinline!OneContractMonth! [Optional]: This node takes a calendar month in the form of the first three characters of the month with the first character capitalised. The month provided should be an arbitrary valid future contract month. It is used in cases where the \lstinline!ContractFrequency! is not \lstinline!Monthly! in order to determine the valid contract months. If omitted, it defaults to January.
\item \lstinline!OffsetDays! [Optional]: The number of business days that the expiry date is before the base date where the base date is implied by the \lstinline!AnchorDay! node above. Any non-negative integer is allowed here. If omitted, this takes a default value of zero.
\item \lstinline!BusinessDayConvention! [Optional]: The business day roll convention used to adjust the expiry date. The allowable values are given in Table \ref{tab:allow_stand_data}. If omitted, it defaults to \lstinline!Preceding!.
\item \lstinline!AdjustBeforeOffset! [Optional]: The allowable values are \lstinline!true! and \lstinline!false!. If \lstinline!true!, if the base date implied by the \lstinline!AnchorDay! node above is not a good business day according to the calendar provided in the \lstinline!Calendar! node, this date is adjusted before the offset specified in the \lstinline!OffsetDays! is applied. If \lstinline!false!, this adjustment does not happen. If omitted, it defaults to \lstinline!true!. 
\item \lstinline!IsAveraging! [Optional]: The allowable values are \lstinline!true! and \lstinline!false!. This node indicates if the future contract is based on the average commodity price of the contract period. If omitted, it defaults to \lstinline!false!.
\item \lstinline!OptionExpiryOffset! [Optional]: The number of business days that the option expiry date is before the future expiry date. Any non-negative integer is allowed here. If omitted, this takes a default value of zero and the expiry date of an option on the future contract is assumed to equal the expiry date of the future contract.
\item \lstinline!ProhibitedExpiries! [Optional]: This node can be used to specify explicit dates which are not allowed as future contract expiry dates or as option expiry dates. A useful example of this is the ICE Brent contract which has the following constraint on expiry dates: \emph{If the day on which trading is due to cease would be either: (i) the Business Day preceding Christmas Day, or (ii) the Business Day preceding New Year’s Day, then trading shall cease on the next preceding Business Day}. Each \lstinline!Date! node can take optional attributes. The default values of these attributes is shown in Listing \ref{lst:commodity_future_conventions}. The \lstinline!convention! attribute accepts a valid business day convention in the list \lstinline!Preceding!, \lstinline!ModifiedPreceding!, \lstinline!Following! and \lstinline!ModifiedFollowing!. This \lstinline!convention! indicates how the future expiry date should be adjusted if it lands on the prohibited expiry \lstinline!Date!. If omitted, the default is \lstinline!Preceding!. Both \lstinline!Preceding! and \lstinline!ModifiedPreceding! indicate that the next available business day before the date is tested. \lstinline!Following! and \lstinline!ModifiedFollowing! indicate that the next available business day after the date is tested. The \lstinline!optionConvention! attribute allows the same values and behaves in the same way to determine how the option expiry date should be adjusted if it lands on the prohibited expiry \lstinline!Date!. The \lstinline!forFuture! and \lstinline!forOption! boolean attributes enable the prohibited expiry to apply only for the future expiry date or the option expiry date respectively by setting the value to \lstinline!false!.
\item \lstinline!OptionExpiryMonthLag! [Optional]: The allowable values are any integer. This value indicates the number of months from the month containing the option expiry date to the month containing the expiry date. If 0, the commodity future option contract expiry date is anchored in the same month as the commodity future contract expiry date. If the value of \lstinline!OptionExpiryMonthLag! is $n > 0$, the commodity option future contract expires in the $n$-th month prior to the commodity future contract expiry month. If the value of \lstinline!OptionExpiryMonthLag! is $n < 0$, the commodity option future contract expires in the $n$-th month after the commodity future contract expiry month. The value of \lstinline!OptionExpiryMonthLag! should be equal to \lstinline!ExpiryMonthLag! when \lstinline!OptionExpiryOffset! is used. The \lstinline!OptionExpiryMonthLag! is rarely used. An example is the Crude Palm Oil contract \lstinline!XKLS:FCPO! where the future contract expiry is in the delivery month and the option expiry is in the month that is 2 months prior to this. In this case, \lstinline!OptionExpiryMonthLag! is 2. If omitted, \lstinline!OptionExpiryMonthLag! defaults to 0.
\item \lstinline!OptionExpiryDay! [Optional]: This node can contain any integer in the range $1,\ldots,31$ indicating the day of the month on which an option expiry date is anchored. A value of 31 will guarantee that the last day in the month is used a base date. If omitted, this is not used. Setting this field takes precedence over \lstinline!OptionExpiryOffset!.\item \lstinline!OptionBusinessDayConvention! [Optional]: The business day convention used to adjust the option expiry date to a good business day if \lstinline!OptionExpiryDay! is used.
\item \lstinline!OptionContractFrequency! [Optional]: This node indicates the frequency of the commodity future options if it differs from the frequency of the underlying future contract. The value here is usually \lstinline!Monthly!
\item \lstinline!OptionNthWeekday! [Optional]: This node has the elements shown in Listing \ref{lst:nth_weekday_node}. This node is used to indicate a date in a given month in the form of the n-th named weekday of that month e.g. 3rd Wednesday. The allowable values for \lstinline!Nth! are ${1,2,3,4}$. The \lstinline!Weekday! node takes a weekday in the form of the first three characters of the weekday with the first character capitalised.
\item \lstinline!OptionBusinessDayConvention! [Optional]: The business day convention used to adjust the option expiry date to a good business day if \lstinline!OptionExpiryDay! is used.
\item \lstinline!OptionExpiryLastWeekdayOfMonth! [Optional]: This node is used to indicate a date in a given month in the form of the last named weekday of that month e.g. last Wednesday. The node takes a weekday in the form of the first three characters of the weekday with the first character capitalised.
\item \lstinline!OptionExpiryWeeklyDayOfTheWeek! [Optional]: This node is used to indicate a date in a given week in the form of the named weekday, e.g. Wednesday. The node takes a weekday in the form of the first three characters of the weekday with the first character capitalised. This node is mandatory for weekly expiring options. The node is not allowed to use with any other option contract frequency.
\item \lstinline!OptionUnderlyingFutureConvention! [Optional]: Sometimes the next contract expiry, as specified in the convention, is not the correct option underlying. For example the base metals options expiries on the 1st Wednesday of the contract month, and during the first 3 months there are daily future contracts available. The option underlying is not the future contract which matures on the option expiry but the one which matures on the 3rd Wednesday of the month. This field is referencing to an commodity future convention which specifies the correct expiry date for the underlying contract.
\item \lstinline!FutureContinuationMappings! [Optional]: When building future curves, we may use market data that has a continuation expiry, i.e. \lstinline!c1!, \lstinline!c2!, etc. , as opposed to an explicit expiry date or tenor. In some cases, the continuation expiries coming from the market data provider may skip serial months and therefore we use the mapping here to map from the market data provider index to the relevant serial month.
\item \lstinline!OptionContinuationMappings! [Optional]: When building option volatility structures, we may use market data that has a continuation expiry, i.e. \lstinline!c1!, \lstinline!c2!, etc. , as opposed to an explicit expiry date or tenor. In some cases, the continuation expiries coming from the market data provider may skip serial months and therefore we use the mapping here to map from the market data provider index to the relevant serial month. For example, for the Crude Palm Oil contract \lstinline!XKLS:FCPO!, the option expiry months are serial up to the 9th month and then alternate months. So, we would add a mapping from 10 to 11, 11 to 13 and so on so that the correct option expiry is arrived at when reading the market data quotes and constructing the option volatility structure.
\item \lstinline!AveragingData! [Optional]: This node is needed for future contracts that are used in a piecewise commodity curve \lstinline!PriceSegment! and whose underlying is the average of other future prices or spot prices over a given period. An example is the ICE PMI power contract with contract specifications outlined \href{https://www.theice.com/products/6590369/PJM-Western-Hub-Real-Time-Peak-1-MW-Fixed-Price-Future}{here}. It is described in detail below.
\item \lstinline!HoursPerDay! [Optional]: For power derivatives, quantities are sometimes given as a quantity per hour. To deduce the quantity for the day which is multiplied by that day's future price, one needs to know the number of hours in the day associated with the future price. For example ICE PDQ is the daily PJM Western Hub Real Time Peak future contract. The price each day for this contract is the average of the locational marginal prices (LMPs) for all hours ending 08:00 to 23:00 Eastern Pacific Time. In other words, there are 16 hours in the day that feed in to the average yielding this settlement price. For this contract, \lstinline!HoursPerDay! would be \lstinline!16!. This field is only needed if a trade XML references this commodity contract, has \lstinline!CommodityQuantityFrequency! set to \lstinline!PerHour! and has no \lstinline!HoursPerDay! value set directly in the XML.
\item \lstinline!SavingsTime! [Optional]: For some derivatives, quantities are given as quantity per calendar day and hour. The monthly quantity is then scaled by the number of calendar days times hours per day (see above) plus or minus a daylight savings correction. To compute the daylight savings correction a convention is needed that describes the dates on which dates one hour is gained resp. lost. Currently supported conventions are US, Null. Default is US if no convention is given.
\item \lstinline!ValidContractMonths! [Optional]: For some commodities the contract frequency is almost monthly but for some calendar months there are no contracts listed. For example Corn Futures are only listed for the expiry months March, May, July, September and December. For those contracts the \emph{ContractFrequency} need to be set to \emph{Monthly} and the valid months have to be added to this node. This node is ignored for all other frequencies and if its omitted all calendar months are valid.
\end{itemize}

\begin{listing}[h!]
\begin{minted}[fontsize=\footnotesize]{xml}
<NthWeekday>
  <Nth>...</Nth>
  <Weekday>...</Weekday>
</NthWeekday>
\end{minted}
\caption{\textnormal{\lstinline!NthWeekday!} node outline}
\label{lst:nth_weekday_node}
\end{listing}

An example \lstinline!CommodityFuture! node for the NYMEX WTI future contract, specified \href{https://www.cmegroup.com/trading/energy/crude-oil/light-sweet-crude_contract_specifications.html}{here}, is provided in Listing \ref{lst:ex_wti_comm_future_convention}.

\begin{listing}[h!]
\begin{minted}[fontsize=\footnotesize]{xml}
<CommodityFuture>
  <Id>NYMEX:CL</Id>
  <AnchorDay>
    <DayOfMonth>25</DayOfMonth>
  </AnchorDay>
  <ContractFrequency>Monthly</ContractFrequency>
  <Calendar>US-NYSE</Calendar>
  <ExpiryMonthLag>1</ExpiryMonthLag>
  <OffsetDays>3</OffsetDays>
  <BusinessDayConvention>Preceding</BusinessDayConvention>
  <IsAveraging>false</IsAveraging>
</CommodityFuture>
\end{minted}
\caption{NYMEX WTI \textnormal{\lstinline!CommodityFuture!} node}
\label{lst:ex_wti_comm_future_convention}
\end{listing}

The \lstinline!AveragingData! node referenced above has the structure shown in Listing \ref{lst:ave_data_comm_future_convention}. The meaning of each of the fields is as follows:

\begin{itemize}
\item \lstinline!CommodityName!: The name of the commodity being averaged.
\item \lstinline!Conventions!: The identifier for the conventions associated with the commodity being averaged.
\item \lstinline!Period!: This indicates the averaging period relative to the future expiry date. The allowable values are:
    \begin{itemize}
    \item \lstinline!PreviousMonth!: The calendar month prior to the month in which the (top level) future contract's expiry date falls is used as the averaging period.
    \item \lstinline!ExpiryToExpiry!: Given a (top level) future contract's expiry date, the averaging period is from and excluding the previous expiry date to and including the expiry date.
    \end{itemize}
\item \lstinline!PricingCalendar!: The pricing calendar(s) used to determine the pricing dates in the averaging period.
\item \lstinline!UseBusinessDays! [Optional]: A boolean flag that defaults to \lstinline!true! if omitted. When set to \lstinline!true!, the pricing dates in the averaging period are the set of \lstinline!PricingCalendar! good business days. When set to \lstinline!false!, the pricing dates in the averaging period are the complement of the set of \lstinline!PricingCalendar! good business days. This may be useful in certain situations. For example, the contract ICE PW2 with specifications \href{https://www.theice.com/products/71090520/PJM-Western-Hub-Real-Time-Peak-2x16-Fixed-Price-Future}{here} averages the PJM Western Hub locational marginal prices over each day in the averaging period that is a Saturday, Sunday or NERC holiday. So, in this case, \lstinline!UseBusinessDays! would be \lstinline!false! and \lstinline!PricingCalendar! would be \lstinline!US-NERC!.
\item \lstinline!DeliveryRollDays! [Optional]: This node allows any non-negative integer value. When averaging a commodity future contract price over the averaging period, the averaging period may include an underlying future contract expiry date. This node's value indicates when we should begin using the next future contract's price in the averaging. If the value is zero, we should include the future contract prices up to and including the contract expiry. If the value is one, we should include the contract prices up to and including the day that is one business day before the contract expiry and then switch to using the next future contract's price thereafter. Similarly for other non-negative integer values. If this node is omitted, it is set to zero.
\item \lstinline!FutureMonthOffset! [Optional]: This node allows any non-negative integer value. If this node is omitted, it is set to zero. This node indicates which future contract is being referenced on each \textit{Pricing Date} in the averaging period by acting as an offset from the next available expiry date. If \lstinline!FutureMonthOffset! is zero, the settlement price of the next available monthly contract that has not expired with respect to the \textit{Pricing Date} is used as the price on that \textit{Pricing Date}. If \lstinline!FutureMonthOffset! is one, the settlement price of the second available monthly contract that has not expired with respect to the \textit{Pricing Date} is used as the price on that \textit{Pricing Date}. Similarly for other positive values of \lstinline!FutureMonthOffset!.
\item \lstinline!DailyExpiryOffset! [Optional]: This node allows any non-negative integer value. It should only be used where the \lstinline!CommodityName! being averaged has a daily contract frequency. If this node is omitted, it is set to zero. This node indicates which future contract is being referenced on each \textit{Pricing Date} in the averaging period by acting as a business day offset, using the \lstinline!CommodityName!'s expiry calendar, from the \textit{Pricing Date}. It is useful in the base metals market where the future contract being averaged on each \textit{Pricing Date} is the cash contract on that \textit{Pricing Date} i.e.\ the contract with expiry date two business days after the \textit{Pricing Date}.
\end{itemize}

\begin{listing}[h!]
\begin{minted}[fontsize=\footnotesize]{xml}
<AveragingData>
  <CommodityName>...</CommodityName>
  <Conventions>...</Conventions>
  <Period>...</Period>
  <PricingCalendar>...</PricingCalendar>
  <UseBusinessDays>...</UseBusinessDays>
  <DeliveryRollDays>...</DeliveryRollDays>
  <FutureMonthOffset>...</FutureMonthOffset>
  <DailyExpiryOffset>...</DailyExpiryOffset>
</AveragingData>
\end{minted}
\caption{\lstinline!AveragingData! node structure}
\label{lst:ave_data_comm_future_convention}
\end{listing}

\subsubsection{Credit Default Swap Conventions}
\label{sss:cds_conventions}
A node with name \lstinline!CDS! is used to store conventions for credit default swaps. The structure of this node is shown in Listing \ref{lst:cds_conventions}.

\begin{listing}[H]
\begin{minted}[fontsize=\footnotesize]{xml}
<CDS>
  <Id>...</Id>
  <SettlementDays>...</SettlementDays>
  <Calendar>...</Calendar>
  <Frequency>...</Frequency>
  <PaymentConvention>...</PaymentConvention>
  <Rule>...</Rule>
  <DayCounter>...</DayCounter>
  <SettlesAccrual>...</SettlesAccrual>
  <PaysAtDefaultTime>...</PaysAtDefaultTime>
</CDS>
\end{minted}
\caption{CDS conventions}
\label{lst:cds_conventions}
\end{listing}

The meanings of the various elements in this node are as follows:
\begin{itemize}

\item \lstinline!Id!:
The identifier for the CDS convention.

\item \lstinline!SettlementDays!:
The number of days after the CDS trade date when protection starts i.e.\ the \textit{Protection effective date} or \textit{step-in date}. Any non-negative integer is allowed here. For standard CDS after, this is generally set to 1.

\item \lstinline!Calendar!:
The calendar associated with the CDS. For non-JPY currencies, this is generally \lstinline!WeekendsOnly! to agree with the ISDA standard. For JPY CDS, the ISDA standard calendar is \lstinline!TYO! documented at \url{https://www.cdsmodel.com/cdsmodel}. This could be set up as an additional calendar or \lstinline!JPN! could be used as a proxy. Allowable calendar values are given in Table \ref{tab:calendar}.

\item \lstinline!Frequency!:
The frequency of fee leg payments for the CDS. The ISDA standard is \lstinline!Quarterly! but any valid frequency is allowed.

\item \lstinline!PaymentConvention!:
The business day convention for payments on the CDS. The ISDA standard is \lstinline!Following! but any valid business day convention from Table \ref{tab:allow_stand_data} is allowed.

\item \lstinline!Rule!:
The date generation rule for the fee leg on the CDS. The ISDA standard is \lstinline!CDS2015! but any valid date generation rule is allowed.

\item \lstinline!DayCounter!:
The day counter for fee leg payments on the CDS. The ISDA standard is \lstinline!A360! but any valid day counter from Table \ref{tab:daycount} is allowed.

\item \lstinline!SettlesAccrual!:
A boolean value indicating if an accrued fee is due on the occurrence of a credit event. Allowable boolean values are given in the Table \ref{tab:boolean_allowable}. In general, this is set \lstinline!true!.

\item \lstinline!PaysAtDefaultTime!:
A boolean value indicating if the accrued fee, on the occurrence of a credit event, is payable at the credit event date or the end of the fee period. A value of \lstinline!true! indicates that the accrued is payable at the credit event date and a value of \lstinline!false! indicates that it is payable at the end of the fee period. In general, this is set \lstinline!true!.

\end{itemize}



%- - - - - - - - - - - - - - - - - - - - - - - - - - - - - - - - - - - - - - - -
\subsubsection{Bond Yield Conventions}
%- - - - - - - - - - - - - - - - - - - - - - - - - - - - - - - - - - - - - - - -
A node with name \lstinline!BondYield! is used to store conventions for the conversion
of bond prices into bond yields.
The structure of this node is shown in Listing \ref{lst:bondyield_conventions}.

\begin{listing}[H]
\begin{minted}[fontsize=\footnotesize]{xml}
<BondYield>
  <Id>CMB-DE-BUND-10Y</Id>
  <Compounding>Compounded</Compounding>
  <Frequency>Annual</Frequency>
  <PriceType>Clean</PriceType>
  <Accuracy>1.0e-8</Accuracy>
  <MaxEvaluations>100</MaxEvaluations>
  <Guess>0.05</Guess>
</BondYield>
\end{minted}
\caption{Bond yield conventions}
\label{lst:bondyield_conventions}
\end{listing}

The meaning of the elements is as follows:

\begin{itemize}
\item Id: The constant maturity index index name. This must be of the form ``CMB-FAMILY-TENOR'' where FAMILY can consist of any number of tags separated by ``-''
\item Compounding: Compounding of the yield - Simple, Compounded, Continuous, SimpleThenCompounded
\item Frequency: Frequency of the cash flows - Annual, Semiannual, Quarterly, Monthly etc.
\item PriceType: Dirty or Clean
\item Accuracy/MaxEvaluations/Guess: QuantLib parameters that control the convergence of the numerical price to yield conversion. 
\end{itemize}





%========================================================
%\section{Trade Data}
%========================================================
\input{tradedata/intro}
\input{tradedata/envelope}
\input{tradedata/nettingsetdetails}
\input{tradedata/tradespecifics}

\input{tradedata/swap}
\input{tradedata/zerocouponswap}
\input{tradedata/capfloor}
\input{tradedata/forwardrateagreement}
\subsubsection{Swaption}
\label{ss:swaption} 

The \lstinline!SwaptionData!  node is the trade data container for the \emph{Swaption} trade type. The \lstinline!SwaptionData!
node has one and exactly one \lstinline!OptionData! trade component sub-node, and at least one \lstinline!LegData! trade
component sub-node.  These trade components are outlined in section \ref{ss:option_data} and section
\ref{ss:leg_data}.\\
\vspace{5mm}
Supported swaption exercise styles are \emph{European}, \emph{Bermudan}, \emph{American}. Swaptions of all exercise styles can have an arbitrary number of legs, with
each leg represented by a \lstinline!LegData! sub-node.  Cross currency swaptions are not supported for either exercise style, i.e. the Currency element must
have the same value for all \lstinline!LegData! sub-nodes of a swaption. There must be at least one full coupon period after the exercise date for European 
Swaptions, and after the last exercise date for Bermudan and American Swaptions. See Table \ref{tab:swaption_requirements} for further details on requirements for
 swaptions.\\
\vspace{5mm}
The structure of an example \lstinline!SwaptionData!  node of a European swaption is shown in Listing
\ref{lst:swaption_data}.

\begin{listing}[H]
%\hrule\medskip
\begin{minted}[fontsize=\footnotesize]{xml}
<SwaptionData>
    <OptionData>
        <LongShort>Long</LongShort>
        <Style>European</Style>
        <Settlement>Physical</Settlement>
        <ExerciseDates>
          <ExerciseDate>2027-03-02</ExerciseDate>
        </ExerciseDates>
        ...
        <Premiums>
          <Premium>
            <Amount>807000</Amount>
            <Currency>GBP</Currency>
            <PayDate>2021-06-15</PayDate>
          </Premium>
        </Premiums>
    </OptionData>
    <LegData>
        <LegType>Fixed</LegType>
        <Payer>false</Payer>    
        <Currency>GBP</Currency>
        ...
    </LegData>
    <LegData>
        <LegType>Floating</LegType>
        <Payer>true</Payer>     
        <Currency>GBP</Currency>
        ...
    </LegData>
</SwaptionData>
\end{minted}
\caption{Swaption data}
\label{lst:swaption_data}
\end{listing}

\begin{table}[H]
\centering
\begin{tabular} {|l|p{10cm}|}
    \hline
        & \bfseries{A  Swaption requires:} \\  \hline
    \lstinline!OptionData! & One \lstinline!OptionData! sub-node  \\  \hline
    \lstinline!Style! &  \emph{Bermudan} or \emph{European} or \emph{American}\\ \hline
    \lstinline!ExerciseDates! & \emph{European} swaptions can only have one \lstinline!ExerciseDate! child element. \emph{American} swaptions must have two \lstinline!ExerciseDate! child elements. \emph{Bermudan} swaptions must have at least two \lstinline!ExerciseDate! child elements, or a Rules or Dates based exercise schedule. \\ \hline
    \lstinline!LegData! &  At least one \lstinline!LegData! sub-node \\ \hline
    \lstinline!Currency! & The same currency for all \lstinline!LegData! sub-nodes.\\ \hline
    \lstinline!LegType! & Allowed types are \emph{Cashflow}, \emph{Fixed} or \emph{Floating}. Floating coupons can be (capped / floored) Ibor, (capped / floored) compounded or averaged OIS, or BMA/SIFMA. Standalone options (nakedOption = true) are not allowed, neither are local OIS cap/floors. \\ \hline
  \end{tabular}
  \caption{Requirements for Swaptions}
  \label{tab:swaption_requirements}
\end{table}

The \lstinline!OptionData! trade component sub-node is outlined in section \ref{ss:option_data}. 
The relevant fields in the \lstinline!OptionData! node for a Swaption are:

\begin{itemize}
\item \lstinline!LongShort!: The allowable values are \emph{Long} or \emph{Short}. Note that the payer and receiver legs in the underlying swap are always from the perspective of the party that is \emph{Long}. E.g. for a \emph{Short} swaption with a fixed leg where the Payer flag is set to \emph{false}, it means that the counterparty receives the fixed flows.  

\begin{table}[H]
\centering
\begin{tabular} {| c | c | c | c |}    \hline
        \lstinline!LongShort! & \makecell{\lstinline!Payer! for Fixed leg \\ on underlying Swap} & \makecell{\lstinline!Payer! for Floating leg \\ on underlying Swap} & Resulting Set Up and Flows \\  \hline
   \emph{Long} & \emph{true} & \emph{false} & \makecell[l]{The Party to the trade buys an \\ option to enter a swap where the \\ Party pays fixed and receives \\ floating}  \\  \hline
    \emph{Short} & \emph{true} & \emph{false} & \makecell[l]{The Party to the trade sells an \\ option to the Counterparty to enter \\ a swap where the Counterparty \\ pays fixed and receives floating}  \\  \hline
    \emph{Long} & \emph{false} & \emph{true} & \makecell[l]{The Party to the trade buys an \\ option to enter a swap where the \\ Party receives fixed and pays \\ floating}  \\  \hline
    \emph{Short} & \emph{false} & \emph{true} & \makecell[l]{The Party to the trade sells an \\ option to the Counterparty to enter \\ a swap where the Counterparty \\ receives fixed and pays floating}  \\  \hline        
  \end{tabular}
  \caption{Swaption set up and resulting flows}
  \label{tab:swaption_setup}
\end{table}




\item \lstinline!OptionType![Optional]: This flag is optional for swaptions, and even if set, has no impact. Whether a swaption is a payer or receiver swaption is determined by the Payer flags on the legs of the underlying swap.

\item  \lstinline!Style!: The exercise style of the Swaption. The allowable values are \emph{European},  \emph{Bermudan} or  \emph{American}. 

\item \lstinline!NoticePeriod![Optional]: The notice period defining the date (relative to the exercise date) on which the exercise
  decision has to be taken. If not given the notice period defaults to \emph{0D}, i.e. the notice date is identical to the
  exercise date. Allowable values: A number followed by \emph{D, W, M, or Y}

\item \lstinline!NoticeCalendar![Optional]: The calendar used to compute the notice date from the exercise date. If not given
  defaults to the \emph{NullCalendar} (no holidays, weekends are no holidays either). Allowable values: See Table \ref{tab:calendar} \lstinline!Calendar!.

\item \lstinline!NoticeConvention![Optional]: The roll convention used to compute the notice date from the exercise date. Defaults to
  \emph{Unadjusted} if not given. Allowable values: See Table \ref{tab:convention} Roll Convention.

\item  \lstinline!Settlement!: Delivery Type. The allowable values are \emph{Cash} or \emph{Physical}. Note that for TradeType \emph{CallableSwap} only \emph{Physical} is allowed.

\item \lstinline!SettlementMethod![Optional]: Specifies the method to calculate the settlement amount for Swaptions and CallableSwaps. Allowable values: \emph{PhysicalOTC}, \emph{PhysicalCleared}, \emph{CollateralizedCashPrice}, \emph{ParYieldCurve}. Defaults to \emph{ParYieldCurve} if Settlement is \emph{Cash} and defaults to \emph{PhysicalOTC} if Settlement is \emph{Physical}.

\emph{PhysicalOTC} = OTC traded swaptions with physical settlement\\
\emph{PhysicalCleared} = Cleared swaptions with physical settlement\\
\emph{CollateralizedCashPrice} = Cash settled swaptions with settlement price calculation using zero coupon curve discounting \\
\emph{ParYieldCurve}  = Cash settled swaptions with settlement price calculation using par yield discounting \footnote{https://www.isda.org/book/2006-isda-definitions/} \footnote{https://www.isda.org/a/TlAEE/Supplement-No-58-to-ISDA-2006-Definitions.pdf} \\

\item \lstinline!ExerciseFees![Optional]: This node contains child elements of type \lstinline!ExerciseFee!. Similar to a list of notionals
  (see \ref{ss:leg_data}) the fees can be given either

  \begin{itemize}
  \item as a list where each entry corresponds to an exercise date and the last entry is used for all remaining exercise
    dates if there are more exercise dates than exercise fee entries, or
  \item using the \verb+startDate+ attribute to specify a change in a fee from a certain day on (w.r.t. the exercise
    date schedule)
  \end{itemize}

  Fees can either be given as an absolute amount or relative to the current notional of the period immediately following
  the exercise date using the \verb+type+ attribute together with specifiers \verb+Absolute+ resp. \verb+Percentage+. If
  not given, the type defaults to \verb+Absolute+. \verb+Percentage+ fees are expressed in decimal form, e.g. 0.05 is a fee of 5\% of notional.

  If a fee is given as a positive number the option holder has to pay a corresponding amount if they exercise the
  option. If the fee is negative on the other hand, the option holder receives an amount on the option exercise.

  Only supported for Swaptions and Callable Swaps currently.

\item \lstinline!ExerciseFeeSettlementPeriod![Optional]: The settlement lag for exercise fee payments. Defaults to 0D if not
  given. This lag is relative to the exercise date (as opposed to the notice date). Allowable values: A number followed by \emph{D, W, M, or Y}

\item \lstinline!ExerciseFeeSettlementCalendar![Optional]: The calendar used to compute the exercise fee settlement date from the
  exercise date. If not given defaults to the \emph{NullCalendar} (no holidays, weekends are no holidays either). Allowable values: See Table \ref{tab:calendar} Calendar.

\item \lstinline!ExerciseFeeSettlementConvention![Optional]: The roll convention used to compute the exercise fee settlement date from
  the exercise date. Defaults to \emph{Unadjusted} if not given. Allowable values: See Table \ref{tab:convention} Roll Convention.

\item An \lstinline!ExerciseDates! node where for \emph{European} style swaptions exactly one \lstinline!ExerciseDate! date element must be given, and for \emph{American} style swaptions  exactly two \lstinline!ExerciseDate! date element must be given, defining the start and the end of the American exercise period.  \emph{Bermudan} style swaptions can have \lstinline!ExerciseDate! elements given directly (at least two  \lstinline!ExerciseDate! elements must be given), or Rules or Dates based exercise dates. See Listings \ref{lst:bermudan_swaption_exercisedates}, \ref{lst:bermudan_swaption_rules} and \ref{lst:bermudan_swaption_dates}.


\begin{listing}[H]
\begin{minted}[fontsize=\footnotesize]{xml}
<SwaptionData>
    <OptionData>
        <LongShort>Long</LongShort>
        <Style>Bermudan</Style>
        <Settlement>Physical</Settlement>
        <ExerciseDates>
          <ExerciseDate>2027-03-02</ExerciseDate>
          <ExerciseDate>2028-03-02</ExerciseDate>
          <ExerciseDate>2029-03-02</ExerciseDate>
        </ExerciseDates>
        ...
    </OptionData>
   ...
\end{minted}
\caption{Bermudan Swaption ExerciseDate:s}
\label{lst:bermudan_swaption_exercisedates}
\end{listing}

\begin{listing}[H]
\begin{minted}[fontsize=\footnotesize]{xml}
<SwaptionData>
    <OptionData>
        <LongShort>Long</LongShort>
        <Style>Bermudan</Style>
        <Settlement>Physical</Settlement>
        <ExerciseDates>
         <Rules>
          <StartDate>2027-03-02</StartDate>
          <EndDate>2029-03-02</EndDate>
          <Tenor>1Y</Tenor>
          <Calendar>US</Calendar>
          <Convention>MF</Convention>
         <Rules>                    
        </ExerciseDates>
        ...
    </OptionData>
   ...
\end{minted}
\caption{Bermudan Swaption Rules based}
\label{lst:bermudan_swaption_rules}
\end{listing}

\begin{listing}[H]
\begin{minted}[fontsize=\footnotesize]{xml}
<SwaptionData>
    <OptionData>
        <LongShort>Long</LongShort>
        <Style>Bermudan</Style>
        <Settlement>Physical</Settlement>
        <ExerciseDates>
         <Dates>
          <Calendar>NullCalendar</Calendar>
          <Convention>Unadjusted</Convention>
          <Dates>
           <Date>2027-03-02</Date>
           <Date>2028-03-02</Date>
           <Date>2029-03-02</Date>
          <Dates>
         <Dates>                    
        </ExerciseDates>
        ...
    </OptionData>
   ...
\end{minted}
\caption{Bermudan Swaption Dates based}
\label{lst:bermudan_swaption_dates}
\end{listing}


\item \lstinline!Premiums! [Optional]: Option premium node with amounts paid by the option buyer to the option seller.

Allowable values:  See section \ref{ss:premiums}

\item An \lstinline!ExerciseData! [Optional] node where one  \lstinline!Date! element must be given, and one \lstinline!Price! element can optionally also be given.  See Listing \ref{lst:exercise_data}

This node marks the Swaption as exercised. If the \lstinline!ExerciseData! node is omitted it is assumed the Swaption has not been exercised. 

The effective exercise date is the next \lstinline!ExerciseDate! in the \lstinline!ExerciseDates! node
  greater or equal the given \lstinline!Date!  in \lstinline!ExerciseData!. 
  
  For a cash-settled Swaption, the \lstinline!Price! given in \lstinline!ExerciseData!
  represents the cash settlement amount. It is paid according to the \lstinline!PaymentData! node: If an explicit list of payment dates is given, the payment takes place on the next date following the effective exercise date. If the \lstinline!PaymentData! is rules-based, the payment date is derived from the effective exercise date using the given calendar, lag and convention.
  
If a Swaption is cash-settled and has an \lstinline!ExerciseData! node with a \lstinline!Date!  but no \lstinline!Price!, then the Swaption is considered exercised on the given date, but without a settlement amount being paid.   
  
 \begin{listing}[H]
\begin{minted}[fontsize=\footnotesize]{xml}
<ExerciseData>
  <Date>2023-09-03</Date>
  <Price>112000</Price>
</ExerciseData>
\end{minted}
\caption{ExerciseData to mark a Swaption or CallableSwap as exercised}
\label{lst:exercise_data}
\end{listing}
  
\item A \lstinline!PaymentData! [Optional] node can be added which defines dates or rules-based settlement date(s) for cash-settled Swaptions.  Note that if rules-based, only \emph{Exercise} is allowed in the \lstinline!RelativeTo! field for Swaptions. See  \lstinline!PaymentData! in \ref{ss:option_data}


\end{itemize}




\input{tradedata/fxforward}
\input{tradedata/fxswap}
\input{tradedata/fxoption}
\input{tradedata/fx_asianoption}
\input{tradedata/fx_barrieroption}
\input{tradedata/fx_digitalbarrieroption}
\input{tradedata/fx_digitaloption}
\input{tradedata/fx_doublebarrieroption}
\input{tradedata/fx_doubletouchoption}
\input{tradedata/fx_europeanbarrieroption}
\input{tradedata/fx_kikobarrieroption}
\input{tradedata/fx_touchoption}
\input{tradedata/fxvarianceswap}
\input{tradedata/equityoption}
\input{tradedata/equityfuturesoption}
\input{tradedata/equityforward}
\input{tradedata/equityswap}
\input{tradedata/eq_asianoption}
\input{tradedata/eq_barrieroption}
\input{tradedata/eq_digitaloption}
\input{tradedata/eq_doublebarrieroption}
\input{tradedata/eq_doubletouchoption}
\input{tradedata/eq_europeanbarrieroption}
\input{tradedata/eq_touchoption}
\input{tradedata/equityvarianceswap}
\input{tradedata/equitycliquetoption}
\input{tradedata/equityposition}
\input{tradedata/equityoptionposition}

\input{tradedata/cpiswap}
\input{tradedata/yyswap}

\input{tradedata/bond}
\input{tradedata/bondposition}
\input{tradedata/forwardbond}
\input{tradedata/bondForward_refdata}
\input{tradedata/bondrepo}
\input{tradedata/bondoption}
\input{tradedata/bondoption_refdata}
\input{tradedata/bondTotalReturnSwap}
\input{tradedata/convertiblebond}
\input{tradedata/ascot}
\input{tradedata/cbodata.tex}

\input{tradedata/compositetrade}

\input{tradedata/creditdefaultswap}
\input{tradedata/indexcds}
\input{tradedata/indexcdsoption}
\input{tradedata/syntheticcdo}
\input{tradedata/creditlinkedswap}

\input{tradedata/commodityforward}
\input{tradedata/commodityswap}
\input{tradedata/commodityswaption}
\input{tradedata/commodityoption}
\input{tradedata/commodityapo}
\input{tradedata/commodityoptionstrip}
\input{tradedata/commodityvarianceswap}
\input{tradedata/commodityposition}

\subsubsection{Generic Total Return Swap / Contract for Difference (CFD)}
\label{ss:GenericTRS}

A generic total return swap / CFD (Trade type: \emph{TotalReturnSwap} or \emph{ContractForDifference}) is set up using a
TotalReturnSwapData (or ContractForDifferenceData) block as shown in listing \ref{lst:trsdata} and
\ref{lst:trsdata_cfd}. Both trade types behave exactly the same.

Usually CFDs are traded without a funding component and captured with only two dates in the return schedule, namely the
start date on which the initial price is fixed and a fictitious closing date usually set to ``tomorrow'' or another
suitable future date. See listing \ref{lst:trsdata_cfd} for the setup of a CFD on STOXX50E with initial price 3399.20 on
2019-09-28.

The generic total return swap is priced using the {\em accrual method} as opposed to a {\em full discounting method} as
it is used for the {\em equity swap} trade type. The accrual method is common practice when daily unwind rights are
present in the trade terms or when the underlying valuation is too complex to allow for future projection.

The TotalReturnSwapData (ContractForDifferenceData) block is comprised of four sub-blocks, which are

\begin{itemize}
\item {\tt UnderlyingData} containing one or more {\tt Trade} subnodes describing the asset position of the TRS
\item {\tt ReturnData} describing the fixing and payment schedule of the return leg and specifying indices for FX conversion if applicable
\item {\tt FundingData} (optional) containing one or more funding legs of the TRS, whose notionals are based on either
  \begin{itemize}
    \item ``PeriodReset'': the underlying price on the last valuation date before or on the accrual start date of the relevant funding
      coupon, this price is converted to the funding currency using the FX rate on this same valuation date for compo /
      cross currency swaps (see below)
    \item ``DailyReset'': the underlying price on each day of the accrual period, again converted to the funding
      currency using the FX rate of the same date for compo / cross currency swaps. This notional type is only
      supported for fixed rate funding legs.
    \item ``Fixed'': a fixed notional given explicitly in the funding leg
  \end{itemize}

\item {\tt AdditionalCashflowData} (optional) a single leg of type Cashflow containing additional payments
\end{itemize}

The {\tt ReturnData} and {\tt FundingData} schedule periods often match, but this is not a strict requirement: In
general, the funding notional is determined as described above dependent on the notional types ``PeriodReset'',
``DailyReset'', ``Fixed''.

Notice that in every case, the {\tt UnderlyingData} schedule (if applicable to the underlying trade type as e.g. for a
bond) is completely independent from the funding / return schedules: The underlying schedule defines the underlying
flows to compute its NPV, and is not directly related to the return swap itself.

Generic TRS can be used to represent total return swaps on a wide range of underlying assets including e.g. single bonds
or equities, CFDs on an underlying basket of EquityPositions, proprietary indices on equity options and equity or bond
indices.

\begin{itemize}
\item The {\tt UnderlyingData} block specifies on or more underlyings, which can be a trades of one of the following
  types. See the trade type specific sections for details on the setup of these underlyings.
  \begin{itemize}
  \item Bond: See \ref{ss:bond}, the trade data is given in a BondData sub node.
  \item ForwardBond: See \ref{ss:BondForward_refdata}, the trade data is given in a ForwardBondData sub node.
  \item CBO: See \ref{ss:CBOData}, the trade data is given in a CBOData sub node.
  \item CommodityPosition: See \ref{ss:commodity_position}, the trade data is given in a CommodityPositionData sub node.
  \item ConvertibleBond: See \ref{ss:convertible_bond}, the trade data is given in a ConvertibleBondData sub
    node. When using reference data, a TRS on a convertible bond can also be captured as a TRS on a bond, i.e. there is
    no need to distinguish between a TRS on a Bond and a TRS on a convertible Bond in this case, the pricer will figure
    out which underlying to set up based on the type of reference data that is set up for the ISIN referenced in the
    security id field.
  \item EquityPosition: See \ref{ss:equity_position}, the trade data is given in a EquityPositionData sub
    node. Notice that the equities given in the basket must be available as quoted market data.
  \item EquityOptionPosition: See \ref{ss:equity_option_position}, the trade data is given in a EquityOptionPositionData
    sub node.
  \item BondPosition: See \ref{ss:bond_position}, the trade data is given in a BondBasketData sub node.
  \item Derivative: An arbitrary underlying derivative trade (of any type covered by ORE), allowing the set up of a so called Portfolio Swap with multiple underlying derivatives. The derivative subnode has
    exactly two subnodes
    \begin{itemize}
      \item Id: A unqiue identifier for the derivative position. Historical prices must be given under the fixing name
        ``GENERIC-$<Id>$''.
      \item Trade: The root node of a derivative trade.
    \end{itemize}

  \end{itemize}

  Each trade is specified by a \verb+TradeType+ and a trade type dependent data block as listed above. Listing
  \ref{lst:trsdata} shows an example for a convertible bond underlying. Listing \ref{lst:trsdata2} shows an example for
  an equity basket underlying. Listing \ref{lst:trsdata3} shows an example for a bond basket underlying. Listing
  \ref{lst:trsdata4} shows an example for a derivative underlying (a swaption in this case).

\item The {\tt ReturnData} block specifies the details of the return leg.
  \begin{itemize}
  \item Payer: Indicates whether the return leg is paid.
  
    Allowable values: \emph{true, false}
    
  \item Currency: The currency in which the return is expressed. This can be different from the underlying currency
    (``composite'' swap) and also from the funding leg currency (``cross currency'' swap). The ``composite'' and ``cross
    currency'' features can occur alone or in combination.
    
    Allowable values: A valid currency code, see \lstinline!Currency! in Table \ref{tab:allow_stand_data}, provided it is the same as on the funding leg.
    
  \item ScheduleData: The reference schedule for the return leg, where the valuation dates are derived from this schedule
    using the ObservationLag, ObservationConvention and ObservationCalendar fields. The payment dates are derived from
    this schedule using the PaymentLag, PaymentConvention and PaymentCalendar fields. The payment dates can also be
    given as an explicit list in the PaymentDates node.
    
    Allowable values: A \lstinline!ScheduleData! block as defined in section \ref{ss:schedule_data}
    
  \item ObservationLag [Optional]: The lag between the valuation date and the reference schedule period start date.
  
    Allowable values: Any valid period, i.e. a non-negative whole number, followed by \emph{D} (days), \emph{W} (weeks), \emph{M} (months), \emph{Y} (years). Defaults to \emph{0D} if left blank or omitted.
    
  \item ObservationConvention [Optional]: The roll convention to be used when applying the observation lag.
  
    Allowable values: A valid roll convention (\emph{F, MF, P, MP, U, NEAREST}), see Table \ref{tab:convention} Roll Convention. Defaults to \emph{U} if left blank or omitted.
    
  \item ObservationCalendar [Optional]: The calendar to be used when applying the observation lag.
  
      Allowable values: Any valid calendar, see Table \ref{tab:calendar} Calendar. Defaults to the \emph{NullCalendar} (no holidays) if left blank or omitted.
      
  \item PaymentLag [Optional]: The lag between the reference schedule period end date and the payment date.
  
    Allowable values: Any valid period, i.e.\ a non-negative whole number, optionally followed by \emph{D} (days), \emph{W} (weeks), \emph{M} (months),
  \emph{Y} (years). Defaults to \emph{0D} if left blank or omitted. If a whole number is given and no letter, it is assumed that it is a number of  \emph{D} (days).
    
  \item PaymentConvention [Optional]: The business day convention to be used when applying the payment lag.
  
    Allowable values: A valid roll convention (\emph{F, MF, P, MP, U, NEAREST}), see Table \ref{tab:convention} Roll Convention. Defaults to \emph{U} if left blank or omitted.
    
  \item PaymentCalendar [Optional]: The calendar to be used when applying the payment lag.
  
    Allowable values: Any valid calendar, see Table \ref{tab:calendar} Calendar. Defaults to the \emph{NullCalendar} (no holidays) if left blank or omitted.
    
  \item PaymentDates [Optional]: This node allows for the specification of a list of explicit payment dates, using
    \lstinline!PaymentDate! elements. The list must contain exactly $n-1$ dates where $n$ is the number of dates in the
    reference schedule given in the ScheduleData node. See Listing \ref{lst:paymentdatestrs} for an example with an
    assumed ScheduleData with 4 dates.
    
    \begin{listing}[H]
%\hrule\medskip
\begin{minted}[fontsize=\footnotesize]{xml}
                <PaymentDates>
                      <PaymentDate>2020-01-15</PaymentDate>
                      <PaymentDate>2021-01-15</PaymentDate>
                      <PaymentDate>2022-01-17</PaymentDate>
                </PaymentDates>
\end{minted}
\caption{Payment dates}
\label{lst:paymentdatestrs}
\end{listing}
    
  \item InitialPrice [Optional]: The equity (or bond) price of the underlying
    on the valuation date associated with the start date. Commonly contractually given. The price can be given in the
    underlying currency or the return currency as specified by the InitialPriceCurrency field and is given as
    \begin{itemize}
    \item a (dirty) price for Bond, ForwardBond and Convertible Bond underlyings, the format is dependent on the price quotation method of the referenced bond:
      \begin{itemize}
      \item Percentage of Par: the InitialPrice should be given as e.g. $1.02$ for $102\%$ relative dirty price
      \item Currency per Unit: the InitialPrice should be given as e.g. $0.51$ for a dirty amount of $51$ USD per unit
        of the bond worth (say) $50.0$ USD.
      \end{itemize}
      \item the weighted price of one unit of the bond underlying basket, notice that this is always a ``percentage of
        par'' price regardless of the quotation style of the single bonds in the basket
      \item the (weighted) price of (one unit of) the equity underlying (basket)
      \item the (weighted) price of (one unit of) the equity option underlying (basket)
      \item an {\em absolute amount in the initial price ccy (``dollar amount'')} if more than one underlying is
        specified and if a derivative is specified
      \item absolute NPV if underlying is a CBO
    \end{itemize}
    Notice that for an equity basket underlying with several currencies involved, the initial price is assumed to be given in the
    return currency in case no InitialPriceCurrency is given.

    Allowable values: A real number. If omitted or left blank it defaults to the equity (or bond) price of the valuation
    date associated with the start date. When this valuation date is in the future there is no fixed price, 
    and in these cases the InitialPrice defaults to the forward price.
    
  \item InitialPriceCurrency [Optional]: Only relevant if InitialPrice is given. This specifies whether the initial
    price is given in the asset currency, the return currency or the funding currency.
    
    Allowable values: One of the currencies in ReturnData / Currency (return currency), FundingData/ LegData / currency
    (funding currency) or the currency of the underlying asset. Defaults to the return currency if omitted.

  \item FXTerms [Mandatory when underlying asset / return / additional cashflow / funding currencies differ]: If the
    underlying asset currency is different from the return currency, an FXIndex for the conversion underlying / return currency
    must be given. The same holds for the funding and additional cashflow currencies: Whenever one of these currencies
    are different from the underlying currency, an FXIndex for the conversion to the underlying currency must be given. If multiple currencies differ, 
    multiple FXIndex elements must be given.

    \begin{itemize}
    \item FXIndex: The fx index to use for the conversion, this must contain the funding / return / additional cashflow
      currency and the underlying asset currency (in the order defined in table \ref{tab:fxindex_data}, i.e. it does not
      matter which one is the funding / return / additional cashflow currency and which is the underlying currency)
        
        Allowable values: see \ref{tab:fxindex_data}
    \end{itemize}

    Notice that for an underlying of type EquityPosition or EquityOptionPosition additional \verb+FXIndex+
    entries are required if there is more than one equity position in a different currency: Eventually, for each equity currency there must be a \verb+FXIndex+ specifying the
    conversion from the equity currency to the funding currency (or for the return/cashflow vs funding currency conversion). In this case multiple \verb+FXIndex+ entries are used within a single \lstinline!FXTerms! node, see \ref{lst:fxterms}. 
    
    
        \begin{listing}[H]
\begin{minted}[fontsize=\footnotesize]{xml}
                <FXTerms>
                      <FXIndex>FX-TR20H-GBP-SEK</FXIndex>
                      <FXIndex>FX-TR20H-GBP-EUR</FXIndex>
                      <FXIndex>FX-TR20H-GBP-USD</FXIndex>
                </FXTerms>
\end{minted}
\caption{FXTerms with multiple FXIndex}
\label{lst:fxterms}
\end{listing}

  \item PayUnderlyingCashFlowsImmediately [Optional]: If true, underlying cashflows like coupon or amortisation payments
    from bonds or dividend payments from equities, are paid when they occur. If false, these cashflows are paid together
    with the next return payment. If omitted, the default value is false for trade type TotalReturnSwap and true for
    trade type ContractForDifference.

  Allowable values: true (immediate payment of underlying cashflwos) or false (underlying cashflows are paid on the next
                    return payment date)

\end{itemize}

\item The {\tt FundingData} block specifies the details of the funding leg(s). The block is optional and can be omitted
  if no funding legs are present in the swap (e.g. for CFDs). It contains one or more LegData nodes, see
  \ref{ss:leg_data}. Allowed leg types are
  \begin{itemize}
  \item Fixed
  \item Floating
  \item CMS
  \item CMB
  \end{itemize}
  The number of coupons defined by the legs often match the number of periods of the return schedule, but this is not a
  strict requirement. All funding legs must share the same payment currency.

  There are several ways to determine the notional of each funding leg, which is determined by additional, optional
  NotionalType tags. If given, there must be exactly one NotionalType tag for each LegData nodes. The types have the
  following meanings:

  \begin{itemize}
    \item ``PeriodReset'': the notional of a funding period is determined by the underlying price on the last valuation
      date before or on the accrual start date of the relevant funding coupon, this price is converted to the funding
      currency using the FX rate on this same valuation date for compo / cross currency swaps.
    \item ``DailyReset'': the notional of a funding period is determined by the underlying price on each day of the
      accrual period, again converted to the funding currency using the FX rate of the same date for compo / cross
      currency swaps. This notional type is only supported for fixed rate funding legs.
    \item ``Fixed'': The notional is explicitly given in the leg data.
  \end{itemize}

  If the NotionalType tags are not given, they default to ``PeriodReset'' in case no explicit notional is given on the
  leg and ``Fixed'' in case an explicit notional is given on the leg. See listing \ref{lst:trsdata} for and example with
  two funding legs, one with a notional of type DailyReset and one with a notional of type PeriodReset.

  If a FundingResetGracePeriod is given, a lag of the given number of calendar days is applied when determining the
  relevant return valuation date that determines the funding notional. For example if FundingResetGracePeriod is set to
  2, a valuation date that lies at most 2 calendar days after the funding accrual start date will be still considered
  eligible for this period.

\item The {\tt AdditionalCashflowData} block is optional and specifies unpaid amounts to be included in the NPV. The
  type of this leg must be Cashflow. The currency of the leg must be either the asset currency or the funding currency
  or the return currency.
\end{itemize}

\begin{listing}[H]
%\hrule\medskip
\begin{minted}[fontsize=\footnotesize]{xml}
<TotalReturnSwapData>
  <UnderlyingData>
    <Trade>
      <TradeType>Bond</TradeType>
      <BondData>
        <SecurityId>ISIN:XY1000000000</SecurityId>
        <BondNotional>1000000.00</BondNotional>
      </BondData>
    </Trade>
  </UnderlyingData>
  <ReturnData>
    <Payer>false</Payer>
    <Currency>EUR</Currency>
    <ScheduleData>...</ScheduleData>
    <ObservationLag>0D</ObservationLag>
    <ObservationConvention>P</ObservationConvention>
    <ObservationCalendar>USD</ObservationCalendar>
    <PaymentLag>2D</PaymentLag>
    <PaymentConvention>F</PaymentConvention>
    <PaymentCalendar>TARGET</PaymentCalendar>
    <!-- <PaymentDates> -->
    <!--   <PaymentDate> ... </PaymentDate> -->
    <!--   <PaymentDate> ... </PaymentDate> -->
    <!-- </PaymentDates> -->
    <InitialPrice>1.05</InitialPrice>
    <InitialPriceCurrency>EUR</InitialPriceCurrency>
    <FXTerms>
      <FXIndex>FX-ECB-EUR-USD</FXIndex>
      <FXIndex>FX-ECB-GBP-USD</FXIndex>
    </FXTerms>
    <PayUnderlyingCashFlowsImmediately>false</PayUnderlyingCashFlowsImmediately>
  </ReturnData>
  <FundingData>
    <FundingResetGracePeriod>2</FundingResetGracePeriod>
    <NotionalType>DailyReset</NotionalType>
    <LegData>
      <Payer>true</Payer>
      <LegType>Fixed</LegType>
      ...
    </LegData>
    <NotionalType>PeriodReset</NotionalType>
    <LegData>
      <Payer>true</Payer>
      <LegType>Floating</LegType>
      ...
    </LegData>
  </FundingData>
  <AdditionalCashflowData>
    <LegData>
      <Payer>false</Payer>
      <LegType>Cashflow</LegType>
      ...
    </LegData>
  </AdditionalCashflowData>
</TotalReturnSwapData>
\end{minted}
\caption{Generic Total Return Swap with Convertible Bond underlying}
\label{lst:trsdata}
\end{listing}

\begin{listing}[H]
%\hrule\medskip
\begin{minted}[fontsize=\footnotesize]{xml}
    <TotalReturnSwapData>
      <UnderlyingData>
        <Trade>
          <TradeType>EquityPosition</TradeType>
          <EquityPositionData>
            <!-- basket price = quantity x sum_i ( weight_i x equityPrice_i x fx_i ) -->
            <Quantity>1000</Quantity>
            <Underlying>
              <Type>Equity</Type>
              <Name>BE0003565737</Name>
              <Weight>0.5</Weight>
              <IdentifierType>ISIN</IdentifierType>
              <Currency>EUR</Currency>
              <Exchange>XFRA</Exchange>
            </Underlying>
            <Underlying>
              <Type>Equity</Type>
              <Name>GB00BH4HKS39</Name>
              <Weight>0.5</Weight>
              <IdentifierType>ISIN</IdentifierType>
              <Currency>GBP</Currency>
              <Exchange>XLON</Exchange>
            </Underlying>
          </EquityPositionData>
        </Trade>
      </UnderlyingData>
      <ReturnData>
        ...
        <InitialPrice>112.0</InitialPrice>
        <InitialPriceCurrency>USD</InitialPriceCurrency>
        <FXTerms>
          <FXIndex>FX-ECB-EUR-USD</FXIndex>
          <FXIndex>FX-TR20H-GBP-USD</FXIndex>
        </FXTerms>
      </ReturnData>
      <FundingData>
        <LegData>
          <Payer>true</Payer>
          <LegType>Floating</LegType>
          <Currency>USD</Currency>
          ...
        </LegData>
      </FundingData>
      <AdditionalCashflowData>
        <LegData>
          <Payer>false</Payer>
          <LegType>Cashflow</LegType>
          ...
        </LegData>
      </AdditionalCashflowData>
    </TotalReturnSwapData>
  </Trade>
\end{minted}
\caption{Generic Total Return Swap with equity basket underlying}
\label{lst:trsdata2}
\end{listing}

\begin{listing}[H]
%\hrule\medskip
\begin{minted}[fontsize=\footnotesize]{xml}
    <TotalReturnSwapData>
      <UnderlyingData>
        <Trade>
          <TradeType>BondPosition</TradeType>
          <BondBasketData>
            <Quantity>100000000</Quantity>
            <Identifier>ISIN:GB00B4KT9Q30</Identifier>
          </BondBasketData>
        </Trade>
      </UnderlyingData>
      <!-- omitting ReturnData, FundingData, AdditionalCashflowData -->
    </TotalReturnSwapData>
  </Trade>
\end{minted}
\caption{Generic Total Return Swap with bond basket underlying}
\label{lst:trsdata3}
\end{listing}

\begin{listing}[H]
%\hrule\medskip
\begin{minted}[fontsize=\footnotesize]{xml}
    <TotalReturnSwapData>
      <UnderlyingData>
        <Derivative>
          <Id>DERIV:XR3JF32BFD</Id>
          <Trade>
            <TradeType>Swaption</TradeType>
              <SwaptionData> ... </SwaptionData>
          </Trade>
        </Derivative>
      </UnderlyingData>
      <!-- omitting ReturnData, FundingData, AdditionalCashflowData -->
    </TotalReturnSwapData>
  </Trade>
\end{minted}
\caption{Generic Total Return Swap on a derivative underlying}
\label{lst:trsdata4}
\end{listing}

\begin{listing}[H]
%\hrule\medskip
\begin{minted}[fontsize=\footnotesize]{xml}
    <TotalReturnSwapData>
      <UnderlyingData>
        <Trade>
          <TradeType>CommodityPosition</TradeType>
          <CommodityPositionData>
            <!-- basket price = quantity x sum_i ( weight_i x price_i x fx_i ) -->
            <Quantity>1000</Quantity>
            <Underlying>
              <Type>Commodity</Type>
              <Name>RIC:.BCOM</Name>
              <Weight>1.0</Weight>
              <PriceType>Spot</PriceType>
            </Underlying>
          </CommodityPositionData>
        </Trade>
      </UnderlyingData>
      <!-- omitting ReturnData, FundingData, AdditionalCashflowData -->
    </TotalReturnSwapData>
  </Trade>
\end{minted}
\caption{Generic Total Return Swap on a commodity index underlying}
\label{lst:trsdata5}
\end{listing}

\begin{listing}[H]
%\hrule\medskip
\begin{minted}[fontsize=\footnotesize]{xml}
    <ContractForDifferenceData>
      <UnderlyingData>
	<Trade>
	  <TradeType>EquityPosition</TradeType>
	  <EquityPositionData>
	    <Quantity>1000</Quantity>
	    <Underlying>
	      <Type>Equity</Type>
	      <Name>.STOXX50E</Name>
	      <Weight>1.0</Weight>
	      <IdentifierType>RIC</IdentifierType>
	    </Underlying>
	  </EquityPositionData>
	</Trade>
      </UnderlyingData>
      <ReturnData>
	<Payer>false</Payer>
	<Currency>EUR</Currency>
	<ScheduleData>
	  <Dates>
	    <Dates>
	      <!-- the start date of the CFD on which the initial price was set -->
	      <Date>2018-09-28</Date>
	      <!-- fictitious closing date, e.g. set to "tomorrow" -->
	      <Date>2019-01-04</Date>
	    </Dates>
	  </Dates>
	</ScheduleData>
	<InitialPrice>3399.20</InitialPrice>
	<InitialPriceCurrency>EUR</InitialPriceCurrency>
      </ReturnData>
      </ContractForDifferenceData>
\end{minted}
\caption{CFD on STOXX50E with initial price 3399.20 EUR}
\label{lst:trsdata_cfd}
\end{listing}


%\include{tradecomponents}
\input{tradecomponents/tradecomponentsintro}
\input{tradecomponents/optiondata}
\input{tradecomponents/premiums}
\input{tradecomponents/legdatanotionals}
\input{tradecomponents/scheduledata}
\input{tradecomponents/fixedlegdatarates}
\input{tradecomponents/floatinglegdata}
\input{tradecomponents/legdataamortisation}
\input{tradecomponents/indexings}
\input{tradecomponents/cashflowleg}
\input{tradecomponents/cmsleg}
\input{tradecomponents/cmbleg}
\subsubsection{Digital CMS Leg Data}
\label{ss:digitalcmslegdata}

Listing \ref{lst:digitalcmslegdata} shows an example for a leg of type \emph{DigitalCMS}.

\begin{listing}[H]
%\hrule\medskip
\begin{minted}[fontsize=\footnotesize]{xml}
      <LegData>
        <LegType>DigitalCMS</LegType>
        <Payer>false</Payer>
        <Currency>GBP</Currency>
        <Notionals>
          <Notional>10000000</Notional>
        </Notionals>
        <DayCounter>ACT/ACT</DayCounter>
        <PaymentConvention>Following</PaymentConvention>
        <ScheduleData>
          ...
        </ScheduleData>
        <DigitalCMSLegData>
          <CMSLegData>
            <Index>EUR-CMS-10Y</Index>
            <FixingDays>2</FixingDays>
            <Gearings>
              <Gearing>3.0</Gearing>
            </Gearings>
            <Spreads>
              <Spread>0.0010</Spread>
            </Spreads>
            <NakedOption>false</NakedOption>
          </CMSLegData>
          <CallPosition>Long</CallPosition>
          <IsCallATMIncluded>false</IsCallATMIncluded>
          <CallStrikes>
            <Strike>0.003</Strike>
          </CallStrikes>
          <CallPayoffs>
            <Payoff>0.003</Payoff>
          </CallPayoffs>
          <PutPosition>Short</PutPosition>
          <IsPutATMIncluded>false</IsPutATMIncluded>
          <PutStrikes>
            <Strike>0.05</Strike>
          </PutStrikes>
          <PutPayoffs>
            <Payoff>0.05</Payoff>
          </PutPayoffs>
        </DigitalCMSLegData>
      </LegData>
\end{minted}
\caption{Digital CMS leg data}
\label{lst:digitalcmslegdata}
\end{listing}

The \lstinline!DigitalCMSLegData! block contains the following elements:

\begin{itemize}
\item CMSLegData: a \lstinline!CMSLegData! block describing the underlying Digital CMS leg (see \ref{ss:cmslegdata}).
Caps and floors in the underlying CMS leg are not supported for Digital CMS Options. The \lstinline!NakedOption! flag in the
\lstinline!CMSLegData! block is supported and can be used to separate the digital option payoff from the underlying CMS coupon.
\item CallPosition: Specifies whether the call option position is long or short.
\item IsCallATMIncluded: inclusion flag on the call payoff if the call option ends at-the-money
\item CallStrikes: strike rate for the call option
\item CallPayoffs: digital call option payoff rate. If included the option is cash-or-nothing, if excluded the option is asset-or-nothing
\item PutPosition: Specifies whether the put option position is long  or short.
\item IsPutATMIncluded: inclusion flag on the put payoff if the put option ends at-the-money
\item PutStrikes: strike rate for the put option
\item PutPayoffs: digital put option payoff rate. If included the option is cash-or-nothing, if excluded the option is asset-or-nothing
\end{itemize}

\input{tradecomponents/durationadjustedcmsleg}
\input{tradecomponents/cmsspreadleg}
\subsubsection{Digital CMS Spread Leg Data}
\label{ss:digitalcmsspreadlegdata}

Listing \ref{lst:digitalcmsspreadlegdata} shows an example for a leg of type \emph{DigitalCMSSpread}.

\begin{listing}[H]
%\hrule\medskip
\begin{minted}[fontsize=\footnotesize]{xml}
      <LegData>
        <LegType>DigitalCMSSpread</LegType>
        <Payer>false</Payer>
        <Currency>GBP</Currency>
        <Notionals>
          <Notional>10000000</Notional>
        </Notionals>
        <DayCounter>ACT/ACT</DayCounter>
        <PaymentConvention>Following</PaymentConvention>
        <ScheduleData>
          ...
        </ScheduleData>
        <DigitalCMSSpreadLegData>
            <CMSSpreadLegData>
              <Index1>EUR-CMS-10Y</Index1>
              <Index2>EUR-CMS-2Y</Index2>
              <Spreads>
                <Spread>0.0010</Spread>
              </Spreads>
              <Gearings>
                <Gearing>8.0</Gearing>
              </Gearings>
              <NakedOption>false</NakedOption>
            </CMSSpreadLegData>
            <CallPosition>Long</CallPosition>
            <IsCallATMIncluded>false</IsCallATMIncluded>
            <CallStrikes>
                <Strike>0.0001</Strike>
            </CallStrikes>
            <CallPayoffs>
                <Payoff>0.0001</Payoff>
            </CallPayoffs>
            <PutPosition>Long</PutPosition>
            <IsPutATMIncluded>false</IsPutATMIncluded>
            <PutStrikes>
                <Strike>0.001</Strike>
            </PutStrikes>
            <PutPayoffs>
                <Payoff>0.001</Payoff>
            </PutPayoffs>
        </DigitalCMSSpreadLegData>
      </LegData>
\end{minted}
\caption{Digital CMS Spread leg data}
\label{lst:digitalcmsspreadlegdata}
\end{listing}

The \lstinline!DigitalCMSSpreadLegData! block contains the following elements:

\begin{itemize}
\item CMSSpreadLegData: a \lstinline!CMSSpreadLegData! block describing the underlying Digital CMS Spread leg (see \ref{ss:cmsspreadlegdata}). 
Caps and floors in the underlying CMS Spread leg are not supported for Digital CMS Spread Options. The \lstinline!NakedOption! flag in the
\lstinline!CMSSpreadLegData! block is supported and can be used to separate the digital option payoff from the underlying CMS Spread coupon.
\item CallPosition: Specifies whether the call option position is long or short.
\item IsCallATMIncluded: inclusion flag on the call payoff if the call option ends at-the-money
\item CallStrikes: strike rate for the call option
\item CallPayoffs: digital call option payoff rate. If included the option is cash-or-nothing, if excluded the option is asset-or-nothing
\item PutPosition: Specifies whether the put option position is long  or short.
\item IsPutATMIncluded: inclusion flag on the put payoff if the put option ends at-the-money
\item PutStrikes: strike rate for the put option
\item PutPayoffs: digital put option payoff rate. If included the option is cash-or-nothing, if excluded the option is asset-or-nothing
\end{itemize}

\input{tradecomponents/equityleg}
\subsubsection{CPI Leg Data}
\label{ss:cpilegdata}

A CPI leg contains a series of CPI-linked coupon payments $N\,r\,({I(t)}/{I_0})\,\delta$ and, if \lstinline!NotionalFinalExchange! is set to \emph{true}, a final
inflation-linked redemption $(I(t)/I_0)\,N$. Each coupon and the final redemption can be
subtracting the (un-inflated) notional $N$, i.e. $(I(t)/I_0-1)\,N$,
see below.

Note that CPI legs with just a final redemption and no coupons, can be set up with a dates-based Schedule containing just a single date - representing the date of the final redemption flow. In this case \lstinline!NotionalFinalExchange! must be set to \emph{true}, otherwise the whole leg is empty, and the Rate is not used and can be set to any value. 

Listing \ref{lst:cpilegdata} shows an example for a leg of type CPI with annual coupons, and \ref{lst:cpilegdatafinal} shows an example for a leg of type CPI with just the final redemption. 

The  \lstinline!CPILegData! block contains the following elements:

\begin{itemize}
\item Index: The underlying zero inflation index.

Allowable values:  See \lstinline!Inflation CPI Index! in Table \ref{tab:cpiindex_data}.
\item Rates: The contractual fixed real rate(s) of the leg, \emph{r}. As usual, this can be a single value, a vector of values or a dated vector of
  values.
 
Note that a CPI leg coupon payment at time $t$ is:
$$
N\,r\,\frac{I(t)}{I_0}\,\delta
$$
where:
\begin{itemize}
\item $N$: notional
\item $r$: the contractual fixed real rate
\item $I(t)$: the relevant CPI fixing for time $t$
\item $I_0$: the BaseCPI
\item $\delta$: the day count fraction for the accrual period up to 
time $t$
\end{itemize}

 Allowable values: Each rate element can take any  real number. The rate is
  expressed in decimal form, e.g. 0.05 is a rate of 5\%.
  
\item BaseCPI [Optional]: The base CPI value $I_0$ used to determine the lifting factor for the fixed coupons. If omitted it will take the observed CPI fixing on startDate - observationLag.

Allowable values:  Any positive real number.

\item StartDate [Optional]: The start date needs to be provided in case the schedule comprises only a single date. If
  the schedule has at least two dates and a start date is given at the same time, the first schedule date is taken as 
  the start date and the supplied \lstinline!StartDate! is ignored.
  
Allowable values:  See \lstinline!Date! in Table \ref{tab:allow_stand_data}. 

\item ObservationLag [Optional]: The observation lag to be applied. It's the amount of time from the fixing at the start or end of the period, moving backward in time, to the inflation index observation date (the inflation fixing). Fallback to the index observation lag as specified in the inflation swap conventions of the underlying index, if not specified. 

Allowable values: An integer followed by \emph{D}, \emph{W}, \emph{M} or \emph{Y}. Interpolation lags are typically expressed in a positive number of  \emph{M}, months. Note that negative values are allowed, but mean that the inflation is observed forward in time from the period start/end date, which is unusual.  

\item Interpolation [Optional]: The type of interpolation that is applied to inflation fixings. \emph{Linear} interpolation means that the inflation fixing for a given date is interpolated linearly between the surrounding - usually monthly - actual fixings, whereas with  \emph{Flat} interpoltion the inflation fixings are constant for each day at the value of the previous/latest actual fixing (flat forward interpolation).  
Fallback to the Interpolation as specified in the inflation swap conventions of the underlying index, if not specified. 
%\emph{AsIndex} means that the underlying inflation index interpolation is applied. Note that if an inflation index have fixing interpolation switched off,  \emph{AsIndex}  is equivalent to \emph{Flat} interpolation.

Allowable values:  \emph{Linear, Flat} 

\item SubtractInflationNotional [Optional]: A flag indicating whether
  the non-inflation adjusted notional amount should be subtracted from
  the final inflation-adjusted notional exchange at maturity.
  Note that the final coupon payment is not affected by this flag. \\ 
Final notional payment if \emph{true}: $N \,(I(T)/I_0-1)$. \\ 
Final notional payment if  \emph{false}: $N \,I(T)/I_0$ 

Allowable values: Boolean node, allowing \emph{Y, N, 1, 0, true, false} etc. The full set of allowable values is given in Table \ref{tab:boolean_allowable}.
\\Defaults to \emph{false}  if left blank or omitted.

\item SubtractInflationNotionalAllCoupons [Optional]: A flag indicating whether
  the non-inflation adjusted notional amount should be subtracted from
  all coupons.
  Note that the final redemption payment is not affected by this flag. \\ 
Coupon payment if \emph{true}: $N \,(I(T)/I_0-1)$. \\ 
Coupon payment if  \emph{false}: $N \,I(T)/I_0$ 

Allowable values: Boolean node, allowing \emph{Y, N, 1, 0, true, false} etc. The full set of allowable values is given in Table \ref{tab:boolean_allowable}.
\\Defaults to \emph{false}  if left blank or omitted.

\item Caps [Optional]: This node contains child elements of type
  \lstinline!Cap! indicating that the inflation indexed payment is
  capped; the cap is applied to the inflation index and expressed as
  an inflation rate, see CPI Cap/Floor in the Product Description. \\
  If the cap is constant over the life of the 
cpi leg, only one cap value should
be entered. If two or more coupons have different caps, multiple cap values
are required, each represented by a \lstinline!Cap! child element. The first cap value
corresponds to the first coupon, the second cap value corresponds to the
second coupon, etc. If the number of coupons exceeds the number of cap
values, the cap will be kept at at the value of last entered spread for the
remaining coupons. The number of entered cap values cannot exceed the
number of coupons. Notice that the caps defined under this node only apply to the cpi coupons,
but not a final notional flow (if present). A cap for the final notional flow can be defined
under the FinalFlowCap node.

Allowable values: Each child element can take any real number. The cap is
expressed in decimal form, e.g. 0.03 is a cap of 3\%.

\item Floors [Optional]: This node contains child elements of type
  \lstinline!Floor! indicating that the inflation indexed payment is
  floored; the floor is applied to the inflation index and expressed as
  an inflation rate. The mode of specification is analogous to caps, see
  above. Notice that the floors defined under this node only apply to the cpi coupons,
  but not a final notional flow (if present). A floor for the final notional flow can be defined
  under the FinalFlowFloor node.

Allowable values: Each child element can take any real number. The floor is
expressed in decimal form, e.g. 0.01 is a cap of 1\%.

\item FinalFlowCap [Optional]: The cap to be applied to the final notional flow of the cpi leg. If not given, no cap
  is applied.

Note that final and non-final inflation cap/floor strikes are quoted as a number K and converted to a price via:

$
(1+K)^t
$

 where

K = the cap/floor rate

t = time to expiry.

So inflation caps/floors are caps/floors on the inflation rate and not the inflation index ratio. For example, to cap the final flow at the initial notional  it should be K=0, i.e. FinalFlowCap should be 0.  


Allowable values: A real number. The FinalFlowCap is expressed in decimal form, e.g. 0.01 is a cap on the final flow at 1\% of the inflation rate over the life of the trade.


\item FinalFlowFloor [Optional]: The floor to be applied to the final notional flow of the cpi leg. If not given, no floor
  is applied.


Allowable values: A real number. The FinalFlowFloor is expressed in decimal form, e.g. 0.01 is a floor on the final flow at 1\% of the inflation rate over the life of the trade.

\item NakedOption [Optional]: Optional node, if \emph{true} the leg represents only the embedded floor, cap or collar. 
By convention these embedded options are considered long if the leg is a receiver leg, otherwise short. 
 
 Allowable values:  \emph{true}, \emph{false}. Defaults to \emph{false} if left blank or omitted.
 
\end{itemize} 

Whether the leg cotains a final redemption flow at all or not depends on the
 notional exchange setting, see section \ref{ss:leg_data} and listing \ref{lst:notional_exchange}.

\begin{listing}[H]
%\hrule\medskip
\begin{minted}[fontsize=\footnotesize]{xml}
      <LegData>
        <LegType>CPI</LegType>
        <Payer>false</Payer>
        <Currency>GBP</Currency>
        <Notionals>
          <Notional>10000000</Notional>
          <Exchanges>
            <NotionalInitialExchange>false</NotionalInitialExchange>
            <NotionalFinalExchange>true</NotionalFinalExchange>
          </Exchanges>          
        </Notionals>
        <DayCounter>ACT/ACT</DayCounter>
        <PaymentConvention>Following</PaymentConvention>
        <ScheduleData>
          <Rules>
            <StartDate>2016-07-18</StartDate>
            <EndDate>2021-07-18</EndDate>
            <Tenor>1Y</Tenor>
            <Calendar>UK</Calendar>
            <Convention>ModifiedFollowing</Convention>
            <TermConvention>ModifiedFollowing</TermConvention>
            <Rule>Forward</Rule>
            <EndOfMonth/>
            <FirstDate/>
            <LastDate/>
          </Rules>
        </ScheduleData>
        <CPILegData>
          <Index>UKRPI</Index>
          <Rates>
            <Rate>0.02</Rate>
          </Rates>
          <BaseCPI>210</BaseCPI>
          <StartDate>2016-07-18</StartDate>
          <ObservationLag>2M</ObservationLag>
          <Interpolation>Linear</Interpolation>
          <Caps>
             <Cap>0.03</Cap>
          </Caps>
          <Floors>
            <Floor>0.0</Floor>
          <Floors>
          <FinalFlowCap>0.03</FinalFlowCap>
          <FinalFlowFloor>0.0</FinalFlowFloor>
          <NakedOption>false</NakedOption>
          <SubtractInflationNotionalAllCoupons>false</SubtractInflationNotionalAllCoupons>         
        </CPILegData>
      </LegData>
\end{minted}
\caption{CPI leg data with capped annual coupons}
\label{lst:cpilegdata}
\end{listing}

\begin{listing}[H]
%\hrule\medskip
\begin{minted}[fontsize=\footnotesize]{xml}
      <LegData>
        <Payer>false</Payer>
        <LegType>CPI</LegType>
        <Currency>GBP</Currency>
        <PaymentConvention>ModifiedFollowing</PaymentConvention>
        <DayCounter>ActActISDA</DayCounter>
        <Notionals>
          <Notional>25000000.0</Notional>
          <Exchanges>
            <NotionalInitialExchange>false</NotionalInitialExchange>
            <NotionalFinalExchange>true</NotionalFinalExchange>
          </Exchanges>
        </Notionals>
        <ScheduleData>
          <Dates>
            <Calendar>GBP</Calendar>
            <Dates>
              <Date>2020-08-17</Date>
            </Dates>
          </Dates>
        </ScheduleData>
        <CPILegData>
          <Index>UKRPI</Index>
          <Rates>
            <Rate>1.0</Rate>
          </Rates>
          <BaseCPI>280.64</BaseCPI>
          <StartDate>2018-08-19</StartDate>
          <ObservationLag>2M</ObservationLag>
          <Interpolation>Linear</Interpolation>
          <SubtractInflationNotional>true</SubtractInflationNotional>
          <SubtractInflationNotionalAllCoupons>false</SubtractInflationNotionalAllCoupons>
        </CPILegData>
      </LegData>
\end{minted}
\caption{CPI leg data with just the final redemption}
\label{lst:cpilegdatafinal}
\end{listing}


\input{tradecomponents/yyleg}
\input{tradecomponents/zerocouponleg}
\input{tradecomponents/commodityfixedleg}
\subsubsection{Commodity Floating Leg}
\label{ss:commodityfloatingleg}

A commodity floating leg is specified in a \lstinline!LegData! node with \lstinline!LegType! set to \lstinline!CommodityFloating!. It is used to define a sequence of cashflows that are linked to the price of a given commodity. Each cashflow has an associated \textit{Calculation Period}. The price that is being referenced may be a commodity spot price or a commodity future contract settlement price. The cashflow may depend on the price observed on a single \textit{Pricing Date} in the \textit{Calculation Period} or it may depend on the arithmetic average of the prices over some or all of the business days in the \textit{Calculation Period}. 

The outline of a commodity floating leg is given in listing \ref{lst:commodityfloatingleg}. It has the usual \lstinline!LegData! elements described in section \ref{ss:leg_data} and a \lstinline!CommodityFloatingLegData! node that is described in section \ref{ss:commodity_floating_leg_data} below. Before describing the \lstinline!CommodityFloatingLegData! node, we devote section \ref{ss:commodity_schedules} to the \lstinline!ScheduleData! node in the context of commodity derivatives.

\begin{listing}[h!]
\begin{minted}[fontsize=\footnotesize]{xml}
<LegData>
  <LegType>CommodityFloating</LegType>
  <Payer>...</Payer>
  <Currency>...</Currency>
  <PaymentConvention>...</PaymentConvention>
  <PaymentLag>...</PaymentLag>
  <PaymentCalendar>...</PaymentCalendar>
  <ScheduleData>
    ...
  </ScheduleData>
  <PaymentDates>
    <PaymentDate>...</PaymentDate>
  </PaymentDates>
  <CommodityFloatingLegData>
    ...
  </CommodityFloatingLegData>
</LegData>
\end{minted}
\caption{Commodity floating leg outline.}
\label{lst:commodityfloatingleg}
\end{listing}

\subsubsection{Commodity Schedules}
\label{ss:commodity_schedules}
The \textit{Calculation Period} in a commodity derivative contract is in general specified as a period from and including a given \textit{Start Date} to and including a given \textit{End Date}. A commodity trade leg consists of a sequence of these \textit{Calculation Period}s. It is important to set up the \lstinline!ScheduleData! in the trade XML such that these periods are correctly represented in the ORE instrument. The \lstinline!ScheduleData! allows for the creation of a list of dates that define the boundaries of the periods from the trade \textit{Effective Date} to the trade \textit{Termination Date}. When the \lstinline!ScheduleData! is used on a commodity leg in the ORE trade XML, the \lstinline!StartDate! is included in the first period and the \lstinline!EndDate! is included in the final period. Each intervening date generated by the \lstinline!ScheduleData! is understood to be the included end date of a period with the subsequent period beginning on the day after the intervening date. The following two examples illustrate the set up of the \lstinline!ScheduleData!.

A common commodity derivative schedule is one that has monthly periods running from and including the first calendar day in the month to and including the last calendar day in the month. For example, the contract periods may be specified as shown in table \ref{tab:comm_schedule_monthly}. The corresponding \lstinline!ScheduleData! node that should be used to represent this in ORE XML is shown in listing \ref{lst:comm_schedule_monthly}. Note that \lstinline!Convention! and \lstinline!TermConvention! are set to \lstinline!Unadjusted! and \lstinline!EndOfMonth! is set to \lstinline!true! to place all dates at the end of the month when generating the dates \lstinline!Backward! from 30 Apr 2020. In general, these values should be used when generating monthly periods for commodity derivatives.

\begin{table}[h!]
\centering
  \begin{tabular}{|c|c|}
  \hline
  Start Date & End Date \\
  \hline
  2020-01-01 & 2020-01-31 \\
  2020-02-01 & 2020-02-29 \\
  2020-03-01 & 2020-03-31 \\
  2020-04-01 & 2020-04-30 \\
  \hline
  \end{tabular}
\caption{Commodity derivative monthly schedule.}
\label{tab:comm_schedule_monthly}
\end{table}

\begin{listing}[h!]
\begin{minted}[fontsize=\footnotesize]{xml}
<ScheduleData>
  <Rules>
    <StartDate>2020-01-01</StartDate>
    <EndDate>2020-04-30</EndDate>
    <Tenor>1M</Tenor>
    <Calendar>NullCalendar</Calendar>
    <Convention>Unadjusted</Convention>
    <TermConvention>Unadjusted</TermConvention>
    <Rule>Backward</Rule>
    <EndOfMonth>true</EndOfMonth>
    <AdjustEndDateToPreviousMonthEnd>false</AdjustEndDateToPreviousMonthEnd>
  </Rules>
</ScheduleData>
\end{minted}
\caption{\textnormal{\lstinline!ScheduleData!} node for monthly periods.}
\label{lst:comm_schedule_monthly}
\end{listing}

Note that for fixed and floating commodity legs, the AdjustEndDateToPreviousMonthEnd field can be added to automatically adjust the end date to the end of the previous month:

\lstinline!AdjustEndDateToPreviousMonthEnd! [Optional]: Only relevant for commodity legs. Allows for the \lstinline!EndDate! to be on a date other than the end of the month. If set to \emph{true} the given \lstinline!EndDate! is restated to the end date to the end of previous month.

Allowable values: \emph{true} or \emph{false}. Defaults to false if left blank or omitted.

In certain cases, a sequence of periods may be provided which do not fit within the \lstinline!Rules! provided by \lstinline!ScheduleData!. In this case, one may use the \lstinline!Dates! node provided by \lstinline!ScheduleData!. As an example of such a case, consider table \ref{tab:comm_schedule_explicit} which shows the periods for a commodity swap leg on the arithmetic average of the nearby month NYMEX WTI future contract settlement price. In this example, the \textit{Calculation Period} runs from the day after the previous future contract expiry to and including the nearby month's contract expiry. In this case, we need to use explicit dates as shown in listing \ref{lst:comm_schedule_explicit}.

\begin{table}[h!]
\centering
  \begin{tabular}{|c|c|}
  \hline
  Start Date & End Date \\
  \hline
  2019-11-21 & 2019-12-19 \\
  2019-12-20 & 2020-01-21 \\
  2020-01-22 & 2020-02-20 \\
  2020-02-21 & 2020-03-20 \\
  \hline
  \end{tabular}
\caption{Commodity derivative explicit schedule.}
\label{tab:comm_schedule_explicit}
\end{table}

\begin{listing}[h!]
\begin{minted}[fontsize=\footnotesize]{xml}
<ScheduleData>
  <Dates>
    <Calendar>NullCalendar</Calendar>
    <Convention>Unadjusted</Convention>
    <Dates>
      <Date>2019-11-21</Date>
      <Date>2019-12-19</Date>
      <Date>2020-01-21</Date>
      <Date>2020-02-20</Date>
      <Date>2020-03-20</Date>
    </Dates>
  </Dates>
</ScheduleData>
\end{minted}
\caption{\textnormal{\lstinline!ScheduleData!} node for explicit periods.}
\label{lst:comm_schedule_explicit}
\end{listing}

\subsubsection{Commodity Floating Leg Data}
\label{ss:commodity_floating_leg_data}
The \lstinline!CommodityFloatingLegData! node outline is shown in listing \ref{lst:commodity_floating_leg_data}. The meaning and allowable values for each node are as follows:

\begin{itemize}

\item
\lstinline!Name!: An identifier specifying the commodity being referenced in the leg. 
% The following needs to move into client-specific documentation of allowable values: 
%The \lstinline!Name! is of the form \lstinline!Prefix:Identifier!. The \lstinline!Prefix! is either \lstinline!PM! for precious metal or a code representing the exchange on which the commodity is traded. For precious metals, the \lstinline!Identifier! is the precious metal code followed by the precious metal price currency. For future contracts, the \lstinline!Identifier! is the exchange code for the future contract. 
Table \ref{tab:commodity_data} lists the allowable values for \lstinline!Name! and gives a description.

\item
\lstinline!PriceType!:  It is \emph{Spot} if the leg is referencing a commodity spot price. It is \emph{FutureSettlement} if the leg is referencing a commodity future contract settlement price.

Allowable values: \emph{Spot}, \emph{FutureSettlement} 

\item
\lstinline!Quantities!: This node is used to specify a constant quantity or a quantity that varies over the calculation periods. The usage of this node is analogous to the usage of the \lstinline!Notionals! node as outlined in section \ref{ss:leg_data}. 

Each \lstinline!Quantity! is the number of units of the underlying commodity covered by the transaction or calculation period. The unit type is defined in the underlying contract specs for the commodity name in question. For avoidance of doubt, the \lstinline!Quantity! is the number of units of the underlying commodity, not the number of contracts.

\item
\lstinline!CommodityQuantityFrequency! [Optional]: In some cases, the quantity in a commodity derivatives contract is given as a quantity per time period. This quantity is then multiplied by the number of such time periods in each calculation period to give the quantity relevant for that full calculation period. The \lstinline!CommodityQuantityFrequency! can be set to

    \begin{itemize}
    \item \emph{PerCalculationPeriod}: This indicates that quantitie(s) as given are for the full calculation period and that no multiplication or alteration is required. This is the default setting if this node is omitted.
    \item  \emph{PerPricingDay}: This indicates that the quantitie(s) are to be considered per pricing date. In general, this can be seen on averaging contracts where the quantity provided must be multiplied by the number of pricing dates in the averaging period to give the quantity applicable for the full calculation period i.e.\ the quantity to which the average price over the period is applied.
    \item \emph{PerHour}: This indicates that quantitie(s) are to be considered per hour. This is common in the electricity markets. The quantity then must be multiplied by the hours per day to give the quantity for a given pricing date. Also, if the contract is averaging, the resulting daily amount is multiplied by the number of pricing dates in the period to give the quantity for the full calculation period. Note that the hours per day may be specified in the \lstinline!HoursPerDay! node directly. If it is omitted, it is looked up in the conventions associated with the commodity. If it is not found there and \emph{PerHour} is used, an exception is thrown during trade building.
    \item \emph{PerCalendarDay}: This indicates that quantitie(s) are to be considered per calendar day in the period. In other words, the quantity provided is multiplied by the number of calendar days in the period to give the quantity applicable for the full calculation period. 
    \item \emph{PerHourAndCalendarDay}: This indicates that quantitie(s) are to be considered per hour and per calendar day in the period. In other words, the quantity provided is multiplied by the number of calendar days and number of hours per day in the period to give the quantity applicable for the full calculation period. The number of hours per period is corrected by daylight saving hours as specified in the conventions of the commodity.
    \end{itemize}

Allowable values: \emph{PerCalculationPeriod}, \emph{PerPricingDay}, \emph{PerHour}, \emph{PerCalendarDay}, \emph{PerHourAndCalendarDay}. Defaults to \emph{PerCalculationPeriod} if omitted.

\item
\lstinline!CommodityPayRelativeTo! [Optional]: The allowable values for this node are \\
\lstinline!CalculationPeriodStartDate!, \lstinline!CalculationPeriodEndDate!, \lstinline!TerminationDate!, \lstinline!FutureExpiryDate!. They specify whether payment is relative to the calculation period start date, calculation period end date, leg maturity date or the future expiry date (not allowed for averaging legs) respectively. The default is \lstinline!CalculationPeriodEndDate!. The payment date is then further adjusted by the payment conventions outlined in section \ref{ss:leg_data} i.e.\ \lstinline!PaymentConvention! and \lstinline!PaymentLag!. If explicit payment dates are given via the \lstinline!PaymentDates! node described in section \ref{ss:leg_data}, then those explicit payment dates are used instead and adjusted by the \lstinline!PaymentCalendar! and \lstinline!PaymentConvention!.

Allowable values: \emph{CalculationPeriodStartDate}, \emph{CalculationPeriodEndDate}, \emph{TerminationDate}. Defaults to  \emph{CalculationPeriodEndDate} if omitted.

\item
\lstinline!Spreads! [Optional]: This node allows for the addition of an optional spread to the referenced commodity price in each calculation period. The usage of this node is exactly as described in section \ref{ss:floatingleg_data}, except that for a Commodity leg, the Spread is not a percentage but an amount in the currency the commodity is quoted in. 

Allowable values: Each child \lstinline!Spread! element can take any real number. Defaults to zero spread in each calculation period if the \lstinline!Spreads! node is omitted.

\item
\lstinline!Gearings! [Optional]: This node allows for the multiplication of the referenced commodity price in each calculation period by an optional gearing factor. The usage of this node is exactly as described in section \ref{ss:floatingleg_data}. If the \lstinline!Gearings! node is omitted, the gearing is one in each calculation period. Note that any spread is added to the referenced price before the gearing is applied.

\item
\lstinline!PricingDateRule! [Optional]: The allowable values are \emph{FutureExpiryDate} and \emph{None}. This setting is ignored when \lstinline!IsAveraged! is \emph{true}  or when \lstinline!PriceType! is \emph{Spot}. In particular, when there is no averaging and the leg is referencing a commodity future contract price, setting \lstinline!PricingDateRule! to \emph{FutureExpiryDate} ensures that the future contract price is observed on its expiry date i.e. that the \textit{Pricing Date} is the future contract expiry date. The particular future contract being referenced is determined by the \lstinline!IsInArrears! node and the \lstinline!FutureMonthOffset! node. If \lstinline!IsInArrears! is  \emph{true}, a base date is set as the calculation period end date. If \lstinline!IsInArrears! is  \emph{false} a base date is set as the calculation period start date. The base date's month and year is then possibly moved forward by an integral number of months using the \lstinline!FutureMonthOffset! node value. If this node value is zero, the base date's month and year are unchanged. The \textit{Pricing Date} is then the expiry date of the future contract with base date month and base date year. Setting \lstinline!PricingDateRule! to \emph{None} allows the \textit{Pricing Date} to be determined using the \lstinline!PricingCalendar! and \lstinline!PricingLag! below.

Allowable values:  \emph{FutureExpiryDate}, \emph{None}. Defaults to \emph{FutureExpiryDate} if omitted.

\item
\lstinline!PricingCalendar! [Optional]: This is the business day calendar used to determine pricing date(s) and in the application of the \lstinline!PricingLag! if provided. If it is omitted, the calendar that has been set up for the reference commodity future contract or referenced commodity spot price will be used.

\item
\lstinline!PricingLag! [Optional]: Any non-negative integer is allowed here. This node indicates that the \textit{Pricing Date} is this number of business days before a given base date. The base date is the period start date if \lstinline!IsInArrears! is \emph{true} and it is the period end date if \lstinline!IsInArrears! is \emph{false}. This setting is not used when \lstinline!IsAveraged! is \emph{true}.

Allowable values: Any non-negative integer. Defaults to zero if omitted.

\item
\lstinline!PricingDates! [Optional]: This node is not used when \lstinline!IsAveraged! is \emph{true}. When \lstinline!IsAveraged! is \emph{false}, this node allows the \textit{Pricing Date} in each period to be given an explicit value. If this node is included, it must contain the same number of \lstinline!PricingDate! nodes as calculation periods. In general, this node is omitted but is used when the other options do not give the desired \textit{Pricing Date} as specified in the trade's contractual terms.

\item
\lstinline!IsAveraged! [Optional]: This node is set to \emph{true} if the \textit{Floating Price} is the arithmetic average of the commodity reference price over each business day in the calculation period. This node is set to \emph{false} if there is no averaging of the underlying commodity price.  Note that \lstinline!IsAveraged! must be set to \emph{true} if the \lstinline!Name! given references a future contract that is averaging itself. There is more on this below.

Allowable values: \emph{true}, \emph{false}. Defaults to \emph{false} if omitted.

\item
\lstinline!IsInArrears! [Optional]: This node is not used when \lstinline!IsAveraged! is \emph{true}. Although, if the observed underlying is averaging itself, having \lstinline!IsAveraged! set to \emph{true} would be ignored with regards this node. As noted above, this setting determines a base date from which the \textit{Pricing Date} is determined. The base date is the period end date if \lstinline!IsInArrears! is \emph{true} and it is the period start date if \lstinline!IsInArrears! is \lstinline!false!. How the \textit{Pricing Date} is then determined from this base date is determined by the \lstinline!PricingDateRule! node or the \lstinline!PricingCalendar! and \lstinline!PricingLag! nodes. 

Allowable values: \emph{true}, \emph{false}. Defaults to \emph{true} if omitted.

\item
\lstinline!FutureMonthOffset! [Optional]: This node allows any non-negative integer value. If this node is omitted, it is set to zero. The node has a different usage depending on whether \lstinline!IsAveraged! is \emph{true} or \emph{false}:
\begin{itemize}
  \item If \lstinline!IsAveraged! is \emph{true}, this node indicates which future contract is being referenced on each \textit{Pricing Date} in the calculation period by acting as an offset from the next available expiry date. If \lstinline!FutureMonthOffset! is zero, the settlement price of the next available monthly contract that has not expired with respect to the \textit{Pricing Date} is used as the price on that \textit{Pricing Date}. If \lstinline!FutureMonthOffset! is one, the settlement price of the second available monthly contract that has not expired with respect to the \textit{Pricing Date} is used as the price on that \textit{Pricing Date}. Similarly for other positive values of \lstinline!FutureMonthOffset!.
  \item If \lstinline!IsAveraged! is \emph{false}, this node acts as an offset for the contract month and is used in conjunction with the \lstinline!IsInArrears! setting to determine the future contract being referenced. If \lstinline!IsInArrears! is \lstinline!true!, a base date is set as the calculation period end date. If \lstinline!IsInArrears! is \emph{false}, a base date is set as the calculation period start date. If \lstinline!FutureMonthOffset! is zero, the future contract month and year is taken as the base date's month and year. If \lstinline!FutureMonthOffset! is one, the future contract month and year is taken as the month following the base date's month and year and so on for all positive values of \lstinline!FutureMonthOffset!.
\end{itemize}

\item
\lstinline!DeliveryRollDays! [Optional]: This node allows any non-negative integer value and is only applicable when \lstinline!IsAveraged! is \emph{true}. When averaging a commodity future contract price during a calculation period, where the calculation period includes the contract expiry date, this node's value indicates when we should begin using the next future contract prices in the averaging. If the value is zero, we should include the contract prices up to and including the contract expiry. If the value is one, we should include the contract prices up to and including the day that is one business day before the contract expiry and then switch to using the next contract prices thereafter. Similarly for other non-negative integer values. 

Allowable values: Any non-negative integer. Defaults to zero if omitted. 

\item
\lstinline!IncludePeriodEnd! [Optional]: If this node is set to \emph{true}, the period end date is included in the calculation period. If it is set to \emph{false}, the period end date is excluded from the calculation period. There is more about this in the section \ref{ss:commodity_schedules}. If this node is omitted, it is set to \emph{true}. In general, this node should be omitted and allowed to take its default value.

\item
\lstinline!ExcludePeriodStart! [Optional]: If this node is set to \emph{true}, the period start date is excluded from the calculation period. If it is set to \emph{false}, the period start date is included from the calculation period. There is more about this in the section \ref{ss:commodity_schedules}. If this node is omitted, it is set to \emph{true}. In general, this node should be omitted and allowed to take its default value.

\item
\lstinline!HoursPerDay! [Optional]: This node is used if \lstinline!CommodityQuantityFrequency! is set to \emph{PerHour} or \emph{PerHourAndCalendarDay}. It is described above under \lstinline!CommodityQuantityFrequency!. 

Allowable values: A number between 0 and 24. If omitted it defaults to the value of the \lstinline!HoursPerDay! node  in the conventions for the referenced commodity.

\item
\lstinline!UseBusinessDays! [Optional]: A boolean flag that defaults to \emph{true} if omitted. It is not applicable if \lstinline!IsAveraged! is \emph{false}. When set to \emph{true}, the pricing dates in the averaging period are the set of \lstinline!PricingCalendar! good business days. When set to \lstinline!false!, the pricing dates in the averaging period are the complement of the set of \lstinline!PricingCalendar! good business days. This may be useful in certain situations. For example, the contract ICE PW2 with specifications \href{https://www.theice.com/products/71090520/PJM-Western-Hub-Real-Time-Peak-2x16-Fixed-Price-Future}{here} averages the PJM Western Hub locational marginal prices over each day in the averaging period that is a Saturday, Sunday or NERC holiday. So, in this case, \lstinline!UseBusinessDays! would be \emph{false} and \lstinline!PricingCalendar! would be \lstinline!US-NERC! to generate the correct pricing dates in the averaging period.

Allowable values: \emph{true}, \emph{false}. Defaults to \emph{true} if omitted.

\item
\lstinline!UnrealisedQuantity! [Optional]: A boolean flag that defaults to \emph{false} if omitted. This is a rarely used flag. When set to \emph{true}, it allows the user, on a given valuation date, to enter the current period quantity as an amount remaining in the current period after the valuation date i.e. the unrealised portion of the current period's quantity. This unrealised quantity is then scaled up internally to give the quantity over the full period.

Allowable values: \emph{true}, \emph{false}. Defaults to \emph{false} if omitted.

\item
\lstinline!LastNDays! [Optional]: This node allows a positive integer value less than or equal to 31 and is currently only supported when \lstinline!PriceType! is \lstinline!FutureSettlement!. When included, instead of the commodity future price being observed on the single \textit{Pricing Date} in the period, it is observed on the \lstinline!LastNDays! \textit{Pricing Date}s, up to and including the original \textit{Pricing Date}, for which future settlement prices are available.

\item
\lstinline!Tag! [Optional]: This node takes any string and can be used to link the floating leg with a fixed leg that has not explicitly provided its own quantities. This can be useful in situations where the quantities on the floating leg are specified with a \lstinline!CommodityQuantityFrequency! that is not simply \lstinline!PerCalculationPeriod!. The fixed leg does not have the \lstinline!CommodityQuantityFrequency! field. In these cases, the fixed leg can omit its \lstinline!Quantities! node and take the quantities from the floating leg. This \lstinline!Tag! node allows the fixed leg to link to a specific floating leg if there is more than one floating leg on the trade i.e.\ the fixed leg must just have the same \lstinline!Tag!. The link is also used to set the payment dates of the fixed leg if CommodityPayRelativeTo is set to FutureExpiryDate.

\item
\lstinline!DailyExpiryOffset! [Optional]: This node allows any non-negative integer value. It only has effect the
underlying commodity \lstinline!Name! is not being averaged and has a daily contract frequency.

If this node is omitted, it defaults to zero. This node indicates which future contract is being referenced on each
\textit{Pricing Date} by acting as a business day offset, using the commodity \lstinline!Name!'s expiry calendar, from
the \textit{Pricing Date}. It is useful e.g. in the base metals market where a future contract on each \textit{Pricing
  Date} is the cash contract on that \textit{Pricing Date} i.e.\ the contract with expiry date two business days after
the \textit{Pricing Date}. In this case, the \lstinline!DailyExpiryOffset! would be set to \lstinline!2!.

\item \lstinline!FXIndex! [Optional]: If \lstinline!IsAveraged! is \emph{true} this node allows the fx conversion to be applied daily in the computation of averaged cash flows. It cannot be used with the \lstinline!Indexing! node.

Allowable values:  See Table \ref{tab:fxindex_data} for supported fx indices.
\end{itemize}

\begin{listing}[h!]
\begin{minted}[fontsize=\footnotesize]{xml}
<CommodityFloatingLegData>
  <Name>...</Name>
  <PriceType>...</PriceType>
  <Quantities>
    <Quantity>...</Quantity>
  </Quantities>
  <CommodityQuantityFrequency>...</CommodityQuantityFrequency>
  <CommodityPayRelativeTo>...</CommodityPayRelativeTo>
  <Spreads>
    <Spread>...</Spread>
  </Spreads>
  <Gearings>
    <Gearing>...</Gearing>
  </Gearings>
  <PricingDateRule>...</PricingDateRule>
  <PricingCalendar>...</PricingCalendar>
  <PricingLag>...</PricingLag>
  <PricingDates>
    <PricingDate>...</PricingDate>
  </PricingDates>
  <IsAveraged>...</IsAveraged>
  <IsInArrears>...</IsInArrears>
  <FutureMonthOffset>...</FutureMonthOffset>
  <DeliveryRollDays>...</DeliveryRollDays>
  <IncludePeriodEnd>...</IncludePeriodEnd>
  <ExcludePeriodStart>...</ExcludePeriodStart>
  <HoursPerDay>...</HoursPerDay>
  <UseBusinessDays>...</UseBusinessDays>
  <Tag>...</Tag>
  <DailyExpiryOffset>...</DailyExpiryOffset>
</CommodityFloatingLegData>
\end{minted}
\caption{Commodity floating leg data outline.}
\label{lst:commodity_floating_leg_data}
\end{listing}

We note above that \lstinline!IsAveraged! must be set to \emph{true}  if the \lstinline!Name! given references a future contract that is averaging itself. For the avoidance of doubt, this does not lead to the prices of the averaging future contract being averaged in each calculation period. Instead, a check is performed in the code if the contract defined by \lstinline!Name! is averaging, and if the leg itself is averaging we switch to observing the averaging future contract price on the single \textit{Pricing Date} determined by the \lstinline!PricingDateRule! node or the \lstinline!PricingCalendar! and \lstinline!PricingLag! nodes or the \lstinline!PricingDates! node. This is best illustrated using an example. Suppose that we have a commodity swap with the schedule shown in table \ref{tab:comm_ex_swap_schedule}. Suppose that the \textit{Floating Price} for the swap is specified as \textit{For each Calculation Period, the arithmetic average of the Commodity Reference Price, for each Commodity Business Day in the Calculation Period} and that the \textit{Commodity Reference Price} is specified as \textit{OIL-WTI-NYMEX} with \textit{Delivery Date} of \textit{First Nearby Month}. There are two approaches to setting up the XML for this commodity floating leg:
\begin{enumerate}

\item
The first approach is shown in listing \ref{lst:example_ave_floating_leg_1}. Note that the \lstinline!Name! is \lstinline!NYMEX:CL! to indicate the NYMEX WTI future contract, \lstinline!IsAveraged! is \lstinline!true! and \lstinline!FutureMonthOffset! is \lstinline!0! to indicate that we are using the nearby month contract price in the averaging. This approach is clear.

\item
The second approach is to use the \lstinline!CommodityFloatingLegData! shown in listing \ref{lst:example_ave_floating_leg_2}. Note that we have changed the \lstinline!Name! to \lstinline!NYMEX:CSX! to reference the NYMEX WTI Financial Futures contract. This future contract settlement price at expiry is the exact payoff of the swap leg in that it is the arithmetic average of the nearby month NYMEX WTI future contract settlement prices over the calendar month. The contract details are given \href{https://www.cmegroup.com/trading/energy/crude-oil/west-texas-intermediate-wti-crude-oil-calendar-swap-futures_contract_specifications.html}{here}. We keep \lstinline!IsAveraged! set to \lstinline!true!. If we set \lstinline!IsAveraged! to \lstinline!false!, an error will be thrown. When \lstinline!IsAveraged! is set to \lstinline!true! and the \lstinline!Name! references a future contract that is averaging, it is understood that the commodity leg is to use the same averaging as the future contract. In this case, we switch to a non-averaging cashflow in the code and read the averaged price directly off the price curve that we have set up using the averaging future contract prices.

\end{enumerate}

In some cases, we will only have an averaging future contract available as an allowable \lstinline!Name! value. For example, \lstinline!NYMEX:A7Q! is one such instance. The contract details are given \href{https://www.cmegroup.com/trading/energy/petrochemicals/mont-belvieu-natural-gasoline-5-decimal-opis-swap_contract_specifications.html}{here}. This future contract's price at the end of each contract month is the \textit{arithmetic average of the OPIS Mt. Belvieu Natural Gasoline (non-LDH) price for each business day during the contract month}. The corresponding commodity floating leg would be set up with \lstinline!Name! set to \lstinline!NYMEX:A7Q! and \lstinline!IsAveraged! set to \lstinline!true!. Again, for the avoidance of doubt, we are not averaging the averaging future contract price. Instead, we switch to a non-averaging cashflow in the code and read the averaged price directly off the price curve that we have built out of \lstinline!NYMEX:A7Q! future contract prices. We are pricing a leg that has the same payoff as the future contract.

If we have an averaging coupon and the valuation date is during the coupon period, the choice between the first and second approach above will have an effect on the sensitivities that are generated for that one single coupon. It should not affect the NPV of the coupon. The effect becomes more pronounced as the number of days remaining in the coupon period reduce. In the first approach, the coupon is priced by reading the expected future prices on future \textit{Pricing Date}s off the non-averaging future price curve and fetching past fixed settlement prices on past \textit{Pricing Date}s. All of these prices are then averaged. It is clear that as the valuation date approaches the final date in the coupon period, the sensitivity decreases because any bump in the curve used for pricing is only affecting the values on the remaining future \textit{Pricing Date}s. In the second approach, the average price relevant for the full coupon period is read directly off the averaging future price curve. Any bump to the averaging future price curve affects the full coupon regardless of the position of the valuation date in the coupon period. The sensitivity will therefore be larger than using the first approach and the difference will become more noticeable as the valuation date moves towards the end of the coupon period. This subtlety can lead to differences that are larger than expected on basis swaps with averaging coupons and short maturities. If one commodity floating leg references a non-averaging price curve and the other leg references an averaging price curve, the differing effects of the bump outlined above on each leg can lead to a larger than expected net sensitivity.

\begin{table}[h!]
\centering
  \begin{tabular}{|c|c|c|}
  \hline
  Start Date & End Date & Quantity Per Period \\
  \hline
  2019-09-01 & 2019-09-30 & 5,000 \\
  2019-10-01 & 2019-10-31 & 5,000 \\
  \hline
  \end{tabular}
\caption{Example commodity swap schedule.}
\label{tab:comm_ex_swap_schedule}
\end{table}

\begin{listing}[h!]
\begin{minted}[fontsize=\footnotesize]{xml}
<LegData>
  <LegType>CommodityFloating</LegType>
  <Payer>true</Payer>
  <Currency>USD</Currency>
  <PaymentLag>2</PaymentLag>
  <PaymentConvention>Following</PaymentConvention>
  <PaymentCalendar>US-NYSE</PaymentCalendar>
  <CommodityFloatingLegData>
    <Name>NYMEX:CL</Name>
    <PriceType>FutureSettlement</PriceType>
    <Quantities>
      <Quantity>5000</Quantity>
    </Quantities>
    <IsAveraged>true</IsAveraged>
    <FutureMonthOffset>0</FutureMonthOffset>
  </CommodityFloatingLegData>
  <ScheduleData>
    <Rules>
      <StartDate>2019-09-01</StartDate>
      <EndDate>2019-10-31</EndDate>
      <Tenor>1M</Tenor>
      <Calendar>NullCalendar</Calendar>
      <Convention>Unadjusted</Convention>
      <TermConvention>Unadjusted</TermConvention>
      <Rule>Backward</Rule>
      <EndOfMonth>true</EndOfMonth>
    </Rules>
  </ScheduleData>
</LegData>
\end{minted}
\caption{Example WTI averaging floating leg, first approach.}
\label{lst:example_ave_floating_leg_1}
\end{listing}

\begin{listing}[h!]
\begin{minted}[fontsize=\footnotesize]{xml}
<CommodityFloatingLegData>
<Name>NYMEX:CSX</Name>
<PriceType>FutureSettlement</PriceType>
<Quantities>
  <Quantity>5000</Quantity>
</Quantities>
<IsAveraged>true</IsAveraged>
<FutureMonthOffset>0</FutureMonthOffset>
</CommodityFloatingLegData>
\end{minted}
\caption{Example WTI averaging floating leg, second approach.}
\label{lst:example_ave_floating_leg_2}
\end{listing}

\input{tradecomponents/equitymarginleg}
\input{tradecomponents/cdsreferenceinformation}
\input{tradecomponents/basketdata}
\subsubsection{Underlying}
\label{ss:underlying}

This trade component can be used to define the underlying entity for an Equity, Commodity or FX trade, but it can also define an underlying interest rate, inflation index, credit name or an underlying bond. It can be used for a single underlying, or within a basket with associated weight.
For an equity underlying a string representation is used to match \lstinline!Underlying! node to required configuration and reference data. The string representation is of the form {IdentifierType}:{Name}:{Currency}:{Exchange}, with all entries optional except for Name.

\begin{listing}[H]
%\hrule\medskip
\begin{minted}[fontsize=\footnotesize]{xml}
<Underlying>
  <Type>...</Type>
  <Name>...</Name>
  <Weight>...</Weight>
  <Currency>...</Currency>
  <IdentifierType>...</IdentifierType>
  <Exchange>...</Exchange>
  <PriceType>...</PriceType>
  <FutureMonthOffset>...</FutureMonthOffset>
  <DeliveryRollDays>...</DeliveryRollDays>
  <DeliveryRollCalendar>...</DeliveryRollCalendar>
</Underlying>
\end{minted}
\caption{Underlying node}
\label{lst:underlying}
\end{listing}

Example structures of the \lstinline!Underlying! trade component node are shown in Listings \ref{lst:equnderlyingric} and \ref{lst:equnderlyingisin} for
an equity underlying, in Listing \ref{lst:fxunderlying} for an fx underlying, in Listing \ref{lst:communderlying} for
a commodity underlying, in Listing \ref{lst:irunderlying} for an underlying interest rate index, in Listing \ref{lst:infunderlying} for an underlying inflation index, in Listing \ref{lst:crunderlying} for an underlying credit name, in listing \ref{lst:bondunderlying} for an underlying bond.

\begin{listing}[H]
%\hrule\medskip
\begin{minted}[fontsize=\footnotesize]{xml}
        <Underlying>
            <Type>Equity</Type>
            <Name>.SPX</Name>
            <Weight>1.0</Weight>
            <IdentifierType>RIC</IdentifierType>
        </Underlying>
\end{minted}
\caption{Equity Underlying - RIC}
\label{lst:equnderlyingric}
\end{listing}

\begin{listing}[H]
%\hrule\medskip
\begin{minted}[fontsize=\footnotesize]{xml}
        <Underlying>
            <Type>Equity</Type>
            <Name>NL0000852580</Name>
            <Weight>1.0</Weight>
            <IdentifierType>ISIN</IdentifierType>
            <Currency>EUR</Currency>
            <Exchange>XAMS</Exchange>
        </Underlying>
\end{minted}
\caption{Equity Underlying - ISIN}
\label{lst:equnderlyingisin}
\end{listing}

\begin{listing}[H]
\begin{minted}[fontsize=\footnotesize]{xml}
        <Underlying>
            <Type>Equity</Type>
            <Name>BBG000BLNNV0</Name>
            <IdentifierType>FIGI</IdentifierType>
        </Underlying>
\end{minted}
\caption{Equity Underlying - FIGI}
\label{lst:equnderlyingfigi}
\end{listing}

\begin{listing}[H]
\begin{minted}[fontsize=\footnotesize]{xml}
        <Underlying>
            <Type>Equity</Type>
            <Name>BARC LN Equity</Name>
            <IdentifierType>BBG</IdentifierType>
        </Underlying>
\end{minted}
\caption{Equity Underlying - Bloomberg Identifier (Parsekey)}
\label{lst:equnderlyingbbg}
\end{listing}

\begin{listing}[H]
%\hrule\medskip
\begin{minted}[fontsize=\footnotesize]{xml}
        <Underlying>
          <Type>FX</Type>
          <Name>ECB-EUR-USD</Name>
          <Weight>1.0</Weight>
        </Underlying>
\end{minted}
\caption{FX Underlying}
\label{lst:fxunderlying}
\end{listing}

\begin{listing}[H]
%\hrule\medskip
\begin{minted}[fontsize=\footnotesize]{xml}
        <Underlying>
          <Type>Commodity</Type>
          <Name>NYMEX:CL</Name>
          <Weight>1.0</Weight>
          <PriceType>FutureSettlement</PriceType>
          <FutureMonthOffset>0</FutureMonthOffset>
          <DeliveryRollDays>0</DeliveryRollDays>
          <DeliveryRollCalendar>TARGET</DeliveryRollCalendar>
          <FutureContractMonth>Nov2023</FutureContractMonth>
        </Underlying>
\end{minted}
\caption{Commodity Underlying}
\label{lst:communderlying}
\end{listing}

\begin{listing}[H]
%\hrule\medskip
\begin{minted}[fontsize=\footnotesize]{xml}
        <Underlying>
          <Type>InterestRate</Type>
          <Name>USD-CMS-10Y</Name>
          <Weight>1.0</Weight>
        </Underlying>
\end{minted}
\caption{InterestRate Underlying}
\label{lst:irunderlying}
\end{listing}

\begin{listing}[H]
%\hrule\medskip
\begin{minted}[fontsize=\footnotesize]{xml}
        <Underlying>
          <Type>Inflation</Type>
          <Name>USCPI</Name>
          <Weight>1.0</Weight>
          <!-- optional -->
          <Interpolation>Linear</Interpolation>
</Underlying>
\end{minted}
\caption{Inflation Index Underlying}
\label{lst:infunderlying}
\end{listing}

\begin{listing}[H]
%\hrule\medskip
\begin{minted}[fontsize=\footnotesize]{xml}
        <Underlying>
          <Type>Credit</Type>
          <Name>ISSUER_A</Name>
          <Weight>1.0</Weight>
        </Underlying>
\end{minted}
\caption{Credit Underlying}
\label{lst:crunderlying}
\end{listing}

\begin{listing}[H]
%\hrule\medskip
\begin{minted}[fontsize=\footnotesize]{xml}
      <Underlying>
        <Type>Bond</Type>
        <Name>US69007TAB08</Name>
        <IdentifierType>ISIN</IdentifierType>
        <Weight>0.5</Weight>
        <BidAskAdjustment>-0.0025</BidAskAdjustment>
      </Underlying>
\end{minted}
\caption{Bond Underlying}
\label{lst:bondunderlying}
\end{listing}

The meanings and allowable values of the elements in the \lstinline!Underlying! node are as follows:

\begin{itemize}

\item \lstinline!Type!: The type of the Underlying asset.

  Allowable values:  \emph{Equity}, \emph{FX}, \emph{Commodity}, \emph{InterestRate}, \emph{Inflation}, \emph{Credit}, \emph{Bond}

\item \lstinline!Name!:
  The name of the Underlying asset. 
  
  Allowable values:  

  \emph{Equity}: See \lstinline!Name! for equity trades in Table \ref{tab:equity_name}

  \emph{FX}: A string on the form SOURCE-CCY1-CCY2, where SOURCE is the FX fixing source, and the fixing is expressed as amount in CCY2 per one unit of CCY1.  See Table \ref{tab:fxindex_data}, and note that the FX- prefix is not included in \lstinline!Name! as it is already included in \lstinline!Type!.

 \emph{InterestRate}: Any valid interest rate index name, see Table \ref{tab:indices}

 \emph{Inflation}: Any valid zero coupon inflation index (CPI) name, See Table \ref{tab:cpiindex_data}

 \emph{Credit}: Any valid credit name with a configured default curve, see Table \ref{tab:equity_credit_data}

 \emph{Bond}: Any valid bond identifier, the bond must be set up in the reference data.

 \emph{Commodity}: An identifier specifying the commodity being referenced in the leg.
% The following needs to move into client-specific documentation of allowable values:
%The \lstinline!Name! is of the form \lstinline!Prefix:Identifier!. The \lstinline!Prefix! is either \lstinline!PM! for precious metal or a code representing the exchange on which the commodity is traded. For precious metals, the \lstinline!Identifier! is the precious metal code followed by the precious metal price currency. For future contracts, the \lstinline!Identifier! is the exchange code for the future contract.
Table \ref{tab:commodity_data} lists the allowable values for \lstinline!Name! and gives a description. \\

\item \lstinline!Weight! [Optional]:
The relative weight of the underlying if part of a basket. For a single underlying this can be omitted or set to 1. 

Allowable values: A real number. Defaults to 1 if left blank or omitted. \\
Notes on negative weights in the \emph{TotalReturnSwap} trade type: \\
Negative weights for EquityOptionPositions are allowed, but not recommended. A negative weight for an EquityOptionPosition is equivalent to inverting the LongShort flag in the respective OptionData node.   \\
For EquityPositions a negative weight means that flows are in the opposite direction of the Payer flag on the return leg. A use case for negative weights is for a basket of EquityPositions that include both long and short positions.

\item \lstinline!IdentifierType! [Optional]:
Only valid when \lstinline!Type! is  \emph{Equity} or \emph{Bond}. The type of the identifier being used.

Allowable values:  \emph{RIC}, \emph{ISIN}, \emph{FIGI}, \emph{BBG}. Defaults to \emph{RIC}, if left blank or omitted, and \lstinline!Type!: is  \emph{Equity}.

\item \lstinline!Currency! [Mandatory when \lstinline!IdentifierType! is  \emph{ISIN}]: Only valid when \lstinline!Type! is  \emph{Equity}. The currency the underlying equity is quoted in. Used when \lstinline!IdentifierType! is  \emph{ISIN}, to - together with the \lstinline!Exchange!  convert a given ISIN to a RIC code.  

Allowable values: See Table \ref{tab:currency} \lstinline!Currency!. Mandatory when \lstinline!IdentifierType! is  \emph{ISIN}, and should not be used for other  \lstinline!IdentifierType!:s  When \lstinline!Type! is \emph{Equity}, Minor Currencies in Table \ref{tab:currency} are also allowable.

\item \lstinline!Exchange! [Mandatory when \lstinline!IdentifierType! is  \emph{ISIN}]:
Only valid when \lstinline!Type! is  \emph{Equity}. A string code representing the exchange the equity is traded on. Used when \lstinline!IdentifierType! is  \emph{ISIN}, to - together with the \lstinline!Currency!  convert a given ISIN to a RIC code.  

Allowable values:  The MIC code of the exchange, see Table \ref{tab:mic}. Mandatory when \lstinline!IdentifierType! is  \emph{ISIN}, and should not be used for other  \lstinline!IdentifierType!:s.

\item \lstinline!PriceType! [Optional]:
Only valid when  \lstinline!Type! is  \emph{Commodity}.  Whether the Spot or Future price is referenced. 

Allowable values:  \emph{Spot}, \emph{FutureSettlement}. Mandatory when  \lstinline!Type! is  \emph{Commodity} .

\item \lstinline!FutureMonthOffset! [Optional]:
Only valid when  \lstinline!Type! is  \emph{Commodity}. Only relevant for the \emph{FutureSettlement} price type, in which case the $N+1$th future with
  expiry greater than ObservationDate for the given commodity underlying will be referenced.

Allowable values:  An integer. Mandatory for when  \lstinline!Type! is  \emph{Commodity} and \lstinline!PriceType! is \emph{FutureSettlement}.

\item \lstinline!DeliveryRollDays! [Optional]:
Only valid when  \lstinline!Type! is  \emph{Commodity}.  The number of days the observation date is rolled forward before the
  next future expiry is looked up.
  
Allowable values: An integer. Defaults to 0 if left blank or omitted, and \lstinline!Type!: is  \emph{Commodity}.

\item \lstinline!DeliveryRollCalendar! [Optional]:
Only valid when  \lstinline!Type! is  \emph{Commodity}.  The calendar used to roll forward the observation date.

Allowable values: See Table \ref{tab:calendar}. Defaults to the null calendar if left blank or omitted, and \lstinline!Type!: is  \emph{Commodity}.

\item \lstinline!FutureContractMonth! [Optional]:
Only valid when  \lstinline!Type! is  \emph{Commodity}, \lstinline!PriceType! is FutureSettlement and there is no \lstinline!FutureExpiryDate! node. It specifies the underlying future contract month  in the format \emph{MonYYYY}, for example Nov2023. 

\item \lstinline!FutureExpiryDate! [Optional]:
Only valid when  \lstinline!Type! is  \emph{Commodity}, \lstinline!PriceType! is FutureSettlement and there is no \lstinline!FutureContractMonth! node. This gives the expiration date of the underlying commodity future contract. 

If the field \lstinline!FutureExpiryDate! and \lstinline!FutureContractMonth! are omitted, the expiration date of the underlying commodity future contract is set to the prompt future, adjusted for any \lstinline!FutureMonthOffset!.

\item \lstinline!Interpolation! [Optional]:
Only valid when \lstinline!Type! is  \emph{Inflation}. The index observation interpolation between fixings.

Allowable values: Flat, Linear

\item \lstinline!BidAskAdjustment! [Optional]: Only valid when \lstinline!Type! is \emph{Bond}. A correction applied to
  the price found in the market data (usually mid), if the bond basket price is defined on the bid or ask side rather
  than mid.

Allowable values: Any real number.

\end{itemize}

\input{tradecomponents/strikedata}
\input{tradecomponents/barrierdata}
\input{tradecomponents/rangebound}
\input{tradecomponents/bondbasketdata.tex}
\input{tradecomponents/cbotranches.tex}
\include{allowablevalues}

%========================================================
%\section{Netting Set Definitions}\label{sec:nettingsetinput}
%========================================================
\include{nettingdata}

%========================================================
%\section{Market Data}\label{sec:market_data}
%========================================================
\include{marketdata}

%========================================================
%\section{Fixing History}
%========================================================
\include{fixingdata}

%========================================================
%\section{Dividend History}
%========================================================
\include{dividenddata}

\newpage
\begin{appendix}

%========================================================
\section{Methodology Summary}
%========================================================

\subsection{Risk Factor Evolution Model}\label{sec:app_rfe}

ORE applies the cross asset model described in detail in \cite{Lichters} to evolve  the market through time. So far the
evolution model in ORE supports IR and FX risk factors for any number of currencies, Equity and Inflation as well as Credit. Extensions to full simulation of Commodity is planned. \\

The Cross Asset Model is based on the Linear Gauss Markov model (LGM) for interest rates, lognormal FX and equity 
processes, Dodgson-Kainth model for inflation, LGM or Extended Cox-Ingersoll-Ross model (CIR++) for credit, and a single-factor log-normal model for commodity curves.
We identify a single {\em domestic} currency; its LGM process,
which is labelled $z_0$; and a set of $n$ foreign currencies with associated LGM processes that are labelled $z_i$, 
$i=1,\dots,n$. 

We denote the equity spot price processes with state variables $s_j$ and the index of the denominating 
currency for the equity process as $\phi(j)$. The dividend yield corresponding to each equity process $s_j$ is denoted 
by $q_j$.

Following \cite{Lichters}, 13.27 - 13.29 we write the inflation processes 
in the domestic LGM measure with state variables $z_{I,k}$ and $y_{I,k}$ for $k=1,\ldots,K$
and the credit processes in the domestic LGM measure with state variables ${C,k}$ and $y_{C,k}$ for $k=1,\ldots,K$.
If we consider $n$ 
foreign exchange rates for converting foreign currency amounts into the single domestic currency by multiplication, 
$x_i$, $i=1,\dots,n$, then the cross asset model is given by the system of SDEs
\begin{eqnarray*}
dz_0 &=& \alpha_0\,dW_0^z \\
dz_i &=& \gamma_i\,dt + \alpha_i\,dW_i^z,  \qquad i>0 \\
\frac{d x_i}{x_i} &=& \mu_i\, dt + \sigma_i\,dW_i^x, \qquad i > 0 \\
\frac{d s_j}{s_j} &=& \mu_j^S\, dt + \sigma_j^S\,dW_j^S \\
dz_{I,k} &=& \alpha_{I,k}(t)dW_k^I \\
dy_{I,k} &=& \alpha_{I,k}(t)H_{I,k}(t)dW_k^I \\
dz_{C,k} &=& \alpha_{C,k}(t)dW_k^C \\
dy_{C,k} &=& H_{C,k}(t)\alpha_{C,k}(t)dW_k^C \\ \\
\gamma_i &=&
-\alpha_i^2\,H_i -\rho_{ii}^{zx}\,\sigma_i\,\alpha_i + \rho_{i0}^{zz}\,\alpha_i\,\alpha_0\,H_0\\
\mu_i &=& r_0 - r_i + \rho_{0i}^{zx}\,\alpha_0\,H_0\,\sigma_i\\
\mu_j^S &=& (r_{\phi(j)}(t) - q_j(t) + \rho_{0j}^{zs} \alpha_0 H_0 \sigma_j^S - \epsilon_{\phi(j)}
\rho_{j \phi(j)}^{sx}\sigma_j^S \sigma_{\phi(j)}) \\
r_i &=& f_i(0,t) + z_i(t)\,H'_i(t) + \zeta_i(t)\,H_i(t)\,H'_i(t),
\quad \zeta_i(t) = \int_0^t \alpha_i^2(s)\,ds  \\ \\
dW^\alpha_a\,dW^\beta_b &=& \rho^{\alpha\beta}_{ij}\,dt, \qquad \alpha, \beta \in \{z, x, I, C\}, \qquad a, b \text{
                              suitable indices }
%\zeta_i(t) &=& \int_0^t \alpha_i^2(s)\,ds,
%\qquad H_i(t) = \int_0^t e^{-\beta_i(s)} \,ds \\
%\beta_i(t) &=& \int_0^t \lambda_i(s)\,ds,
%\qquad \alpha_i(t) = \sigma_i^{HW}(t)\,e^{\beta(t)} \\
\end{eqnarray*}
where we have dropped time dependencies for readability, $f_i(0,t)$ is the instantaneous forward curve in currency $i$, 
and $\epsilon_i$ is an indicator such that $\epsilon_i = 1 - \delta_{0i}$, where $\delta$ is the Kronecker delta.

\medskip Parameters $H_i(t)$ and $\alpha_i(t)$ (or alternatively $\zeta_i(t)$) are LGM model parameters which determine,
together with the stochastic factor $z_i(t)$, the evolution of numeraire and zero bond prices in the LGM model:
\begin{align}
N(t) &= \frac{1}{P(0,t)}\exp\left\{H_t\, z_t + \frac{1}{2}H^2_t\,\zeta_t \right\}
\label{lgm1f_numeraire} \\
P(t,T,z_t)
&= \frac{P(0,T)}{P(0,t)}\:\exp\left\{ -(H_T-H_t)\,z_t - \frac{1}{2} \left(H^2_T-H^2_t\right)\,\zeta_t\right\}.
\label{lgm1f_zerobond}
\end{align}

Note that the LGM model is closely related to the Hull-White model in T-forward measure \cite{Lichters}.

\medskip The parameters $H_{I,k}(t)$ and $\alpha_{I,k}(t)$ determine together with the factors $z_{I,k}(t), y_{I,k}(t)$
the evolution of the spot Index $I(t)$ and the forward index $\hat{I}(t,T) = P_I(t,T) / P_n(t,T)$ defined as the ratio
of the inflation linked zero bond and the nominal zero bond,

\begin{eqnarray*}
  \hat{I}(t,T) &=& \frac{\hat{I}(0,T)}{\hat{I}(0,t)} e^{(H_{I,k}(T)-H_{I,k}(t))z_{I,k}(t)+\tilde{V}(t,T)} \\
  I(t) &=& I(0) \hat{I}(0,t)e^{H_{I,k}(t)z_{I,k}(t)-y_{I,k}(t)-V(0,t)}
\end{eqnarray*}

with, in case of domestic currency inflation,

\begin{eqnarray*}
  V(t,T) &=& \frac{1}{2} \int_t^T (H_{I,k}(T)-H_{I,k}(s))^2 \alpha_{I,k}^2(s) ds \\
         & & - \rho^{zI}_{0,k} H_0(T) \int_t^T (H_{I,k}(t)-H_{I,k}(s))\alpha_0(s)\alpha_{I,k}(s)ds \\
  \tilde{V}(t,T) &=& V(t,T) - V(0,T) -V(0,t) \\
         &=& -\frac{1}{2}(H_{I,k}^2(T)-H_{I,k}^2(t))\zeta_{I,k}(t,0) \\
         & & +(H_{I,k}(T)-H_{I,k}(t)) \zeta_{I,k}(t,1) \\
         & & +(H_0(T)H_{I,k}(T) - H_0(t)H_{I,k}(t))\zeta_{0I}(t,0) \\
         & & -(H_0(T)-H_0(t))\zeta_{0I}(t,1) \\
  V(0,t) &=& \frac{1}{2}H_{I,k}^2(t)\zeta_{I,k}(t,0)-H_{I,k}(t)\zeta_{I,k}(t,1)+\frac{1}{2}\zeta_{I,k}(t,2) \\
         & & -H_0(t)H_{I,k}(t)\zeta_{0I}(t,0)+H_0(t)\zeta_{0I}(t,1) \\
  \zeta_{I,k}(t,k) &=& \int_0^t H_{I,k}^k(s)\alpha_{I,k}^2(s) ds \\
  \zeta_{0I}(t,k) &=& \rho^{zI}_{0,k}\int_0^t H_{I,k}^k(t) \alpha_0(s) \alpha_{I,k}(s) ds
\end{eqnarray*}

and for foreign currency inflation in currency $i>0$, with

\begin{eqnarray*}
  \tilde{V}(t,T) &=& V(t,T) -V(0,T) + V(0,T)
\end{eqnarray*}

and

\begin{eqnarray*}
  V(t,T) &=& \frac{1}{2}\int_t^T (H_{I,k}(T)-H_{I,k}(s))^2 \alpha_{I,k}(s) ds \\
  & & -\rho^{zI}_{0,k} \int_t^T H_0(s)\alpha_0(s)(H_{I,k}(T)-H_{I,k}(s)\alpha_{I,k}(s)) ds \\
  & & -\rho^{zI}_{i,k} \int_t^T (H_i(T)-H_i(s))\alpha_i(s)(H_{I,k}(T)-H_{I,k}(s))\alpha_{I,k}(s) ds \\
  & & +\rho^{xI}_{i,k} \int_t^T \sigma_i(s)(H_{I,k}(T)-H_{I,k}(s))\alpha_{I,k}(s) ds
\end{eqnarray*}

\subsubsection*{Commodity}

Each commodity component models the commodity price curve as 
\begin{eqnarray}
\frac{dF(t,T)}{F(t,T)} &=& \sigma\,e^{-\kappa\,(T-t)}\, dW(t)  \label{gabillon1f}
\end{eqnarray}
which is a single-factor version of the Gabillon (1991) model that is e.g. described in \cite{Lichters}. It can also be seen as the Schwartz (1997) model formulated in terms of forward curve dynamics. The extension to the full Gabillon model with two factors and time-dependent multiplier
\begin{eqnarray}
\frac{dF(t,T)}{F(t,T)} &=& \alpha(t)\,
\left( \sigma_S \,e^{-\kappa\,(T-t)}\, dW_S(t) + \sigma_L\,\left(1-e^{-\kappa\,(T-t)}\right)\,dW_L(t)\right) \label{gabillon2f}
\end{eqnarray}
for richer dynamics of the curve and accurate calibration to options will follow. 

The commodity components' Wiener processes can be correlated. However, the integration of commodity components into the overall CAM assumes zero correlations between commodities and non-commodity drivers for the time being.

To propagate the one-factor model, we can use an artificial (Ornstein-Uhlenbeck) spot price process
\begin{align*}
dX(t) &= -\kappa\,X(t)\,dt + \sigma(t)\,dW(t), \qquad X(0)=0\\
X(t) &= X(s)\,e^{-\kappa(t-s)}+ \int_s^t \sigma\,e^{-\kappa(t-u)}\, dW(u)
\end{align*}
with 
\begin{align*}
F(t,T) &= F(0,T) \:\exp\left( X(t)\,e^{-\kappa\,(T-t)} - \frac{1}{2}\,(V(0,T)-V(t,T))  \right) \\
V(t,T) &= e^{-2\kappa T}\int_t^T\sigma^2\:e^{2\kappa u}\,du.
\end{align*}
Note that 
$$
\V[\ln F(T,T)] = \V[X(T)] 
$$
is the variance that is used in the pricing of a Futures Option which in turn is used in the calibration of the Schwartz model.

Alternatively, one can use the drift-free state variable $Y(t)=e^{\kappa t} X(t)$ with
\begin{align*}
dY(t) &= \sigma \: e^{\kappa \, t} \, dW(t).
\end{align*}
Both choices of state dynamics are possible in ORE. 

\subsection{Analytical Moments of the Risk Factor Evolution Model}\label{sec:app_analytical_moments}

We follow \cite{Lichters}, chapter 16. The expectation of the interest rate process $z_i$ conditional on $\mathcal{F}_{t_0}$ at $t_0+\Delta t$ is

\begin{eqnarray*}
  \mathbb{E}_{t_0}[z_i(t_0+\Delta t)] &=& z_i(t_0) + \mathbb{E}_{t_0}[\Delta z_i],
  \qquad\mbox{with}\quad \Delta z_i = z_i(t_0+\Delta t) - z_i(t_0) \\
  &=& z_i(t_0) -\int_{t_0}^{t_0+\Delta t} H^z_i\,(\alpha^z_i)^2\,du + \rho^{zz}_{0i} \int_{t_0}^{t_0+\Delta t}
  H^z_0\,\alpha^z_0\,\alpha^z_i\,du \\
  & & - \epsilon_i  \rho^{zx}_{ii}\int_{t_0}^{t_0+\Delta t} \sigma_i^x\,\alpha^z_i\,du
\end{eqnarray*}

where $\epsilon_i$ is zero for $i=0$ (domestic currency) and one otherwise.

\bigskip

The expectation of the FX process $x_i$ conditional on $\mathcal{F}_{t_0}$ at $t_0+\Delta t$ is

\begin{eqnarray*}
  \mathbb{E}_{t_0}[\ln x_i(t_0+\Delta t)] &=& \ln x_i(t_0) +  \mathbb{E}_{t_0}[\Delta \ln x_i],
  \qquad\mbox{with}\quad \Delta \ln x_i = \ln x_i(t_0+\Delta t) - \ln x_i(t_0) \\
  &=& \ln x_i(t_0) + \left(H^z_0(t)-H^z_0(s)\right) z_0(s) -\left(H^z_i(t)-H^z_i(s)\right)z_i(s)\\
  &&+ \ln \left( \frac{P^n_0(0,s)}{P^n_0(0,t)} \frac{P^n_i(0,t)}{P^n_i(0,s)}\right) \\
  && - \frac12 \int_s^t (\sigma^x_i)^2\,du \\
  &&+\frac12 \left((H^z_0(t))^2 \zeta^z_0(t) -  (H^z_0(s))^2 \zeta^z_0(s)- \int_s^t (H^z_0)^2
  (\alpha^z_0)^2\,du\right)\\
  &&-\frac12 \left((H^z_i(t))^2 \zeta^z_i(t) -  (H^z_i(s))^2 \zeta^z_i(s)-\int_s^t (H^z_i)^2 (\alpha^z_i)^2\,du
  \right)\\
  && + \rho^{zx}_{0i} \int_s^t H^z_0\, \alpha^z_0\, \sigma^x_i\,du \\
  &&  - \int_s^t \left(H^z_i(t)-H^z_i\right)\gamma_i \,du, \qquad\mbox{with}\quad s = t_0, \quad t = t_0+\Delta t
\end{eqnarray*}

with

\begin{eqnarray*}
  \gamma_i = -H^z_i\,(\alpha^z_i)^2  + H^z_0\,\alpha^z_0\,\alpha^z_i\,\rho^{zz}_{0i} - \sigma_i^x\,\alpha^z_i\,
  \rho^{zx}_{ii}
\end{eqnarray*}

The expectation of the Inflation processes $z_{I,k}, y_{I,k}$ conditional on $\mathcal{F}_{t_0}$ at any time $t>t_0$ is
equal to $z_{I,k}(t_0)$ resp. $y_{I,k}(t_0)$ since both processes are drift free.

\bigskip

The expectation of the equity processes $s_j$ conditional on $\mathcal{F}_{t_0}$ at $t_0+\Delta t$ is
\begin{eqnarray*}
\mathbb{E}_{t_0}[\ln s_j(t_0+\Delta t)] &=& \ln s_j(t_0) +  \mathbb{E}_{t_0}[\Delta \ln s_j],
\qquad\mbox{with}\quad \Delta \ln s_j = \ln s_j(t_0+\Delta t) - \ln s_j(t_0) \\
&=& \ln s_j(t_0) +  \ln \left[\frac{P_{\phi(j)}(0,s)}{P_{\phi(j)}(0,t)} \right] - \int_s^t 
q_j(u) 
du - \frac{1}{2} \int_s^t \sigma_{j}^{S}(u) \sigma_{j}^{S}(u) du\\
&&
+\rho_{0j}^{zs} \int_s^t \alpha_0(u) H_0(u) \sigma_j^S(u) du
- \epsilon_{\phi(j)} \rho_{j \phi(j)}^{sx} \int_s^t \sigma_j^S (u)\sigma_{\phi(j)}(u) du\\
&&+\frac{1}{2} \left( H_{\phi(j)}^2(t) \zeta_{\phi(j)}(t) - H_{\phi(j)}^2(s) \zeta_{\phi(j)}(s)
- \int_s^t H_{\phi(j)}^2(u) \alpha_{\phi(j)}^2(u) du \right)\\
&&  + (H_{\phi(j)}(t) - H_{\phi(j)}(s)) z_{\phi(j)}(s) 
+\epsilon_{\phi(j)} \int_s^t \gamma_{\phi(j)} (u) (H_{\phi(j)}(t) - H_{\phi(j)}(u)) du\\
\end{eqnarray*}

The IR-IR covariance over the interval $[s,t] := [t_0, t_0+\Delta t]$ (conditional on $\mathcal{F}_{t_0}$) is

\begin{eqnarray*}
      \mathrm{Cov} [\Delta z_a, \Delta \ln x_b] &=& \rho^{zz}_{0a}\int_s^t \left(H^z_0(t)-H^z_0\right)
  \alpha^z_0\,\alpha^z_a\,du \nonumber\\
      &&- \rho^{zz}_{ab}\int_s^t \alpha^z_a \left(H^z_b(t)-H^z_b\right) \alpha^z_b \,du \nonumber\\
      &&+\rho^{zx}_{ab}\int_s^t \alpha^z_a \, \sigma^x_b \,du.
\end{eqnarray*}

The IR-FX covariance over the interval $[s,t] := [t_0, t_0+\Delta t]$ (conditional on $\mathcal{F}_{t_0}$) is

\begin{eqnarray*}
      \mathrm{Cov} [\Delta z_a, \Delta \ln x_b] &=& \rho^{zz}_{0a}\int_s^t \left(H^z_0(t)-H^z_0\right)
  \alpha^z_0\,\alpha^z_a\,du \nonumber\\
      &&- \rho^{zz}_{ab}\int_s^t \alpha^z_a \left(H^z_b(t)-H^z_b\right) \alpha^z_b \,du \nonumber\\
      &&+\rho^{zx}_{ab}\int_s^t \alpha^z_a \, \sigma^x_b \,du.
\end{eqnarray*}

The FX-FX covariance over the interval $[s,t] := [t_0, t_0+\Delta t]$ (conditional on $\mathcal{F}_{t_0}$) is

\begin{eqnarray*}
      \mathrm{Cov}[\Delta \ln x_a, \Delta \ln x_b] &=&
      \int_s^t \left(H^z_0(t)-H^z_0\right)^2 (\alpha_0^z)^2\,du \nonumber\\
      && -\rho^{zz}_{0a} \int_s^t \left(H^z_a(t)-H^z_a\right) \alpha_a^z\left(H^z_0(t)-H^z_0\right) \alpha_0^z\,du
  \nonumber\\
      &&- \rho^{zz}_{0b}\int_s^t \left(H^z_0(t)-H^z_0\right)\alpha_0^z \left(H^z_b(t)-H^z_b\right)\alpha_b^z\,du
  \nonumber\\
      &&+ \rho^{zx}_{0b}\int_s^t \left(H^z_0(t)-H^z_0\right)\alpha_0^z \sigma^x_b\,du \nonumber\\
      &&+ \rho^{zx}_{0a}\int_s^t \left(H^z_0(t)-H^z_0\right)\alpha_0^z\,\sigma^x_a\,du \nonumber\\
      &&- \rho^{zx}_{ab}\int_s^t \left(H^z_a(t)-H^z_a\right)\alpha_a^z \sigma^x_b,du\nonumber\\
      &&- \rho^{zx}_{ba}\int_s^t \left(H^z_b(t)-H^z_b\right)\alpha_b^z\,\sigma^x_a\, du \nonumber\\
      &&+ \rho^{zz}_{ab}\int_s^t \left(H^z_a(t)-H^z_a\right)\alpha_a^z \left(H^z_b(t)-H^z_b\right)\alpha_b^z\,du
  \nonumber\\
      &&+ \rho^{xx}_{ab}\int_s^t\sigma^x_a\,\sigma^x_b \,du
\end{eqnarray*}

The IR-INF covariance over the interval $[s,t] := [t_0, t_0+\Delta t]$ (conditional on $\mathcal{F}_{t_0}$) is

\begin{eqnarray*}
  \mathrm{Cov}[ \Delta z_a, \Delta z_{I,b} ] & = & \rho_{ab}^{zI} \int_s^t \alpha_a(s) \alpha_{I,b}(s) ds \\
  \mathrm{Cov}[ \Delta z_a, \Delta y_{I,b} ] & = & \rho_{ab}^{zI} \int_s^t \alpha_a(s) H_{I,b}(s) \alpha_{I,b}(s) ds
\end{eqnarray*}

The FX-INF covariance over the interval $[s,t] := [t_0, t_0+\Delta t]$ (conditional on $\mathcal{F}_{t_0}$) is

\begin{eqnarray*}
  \mathrm{Cov}[ \Delta x_a, \Delta z_{I,b} ] & = & \rho_{0b}^{zI} \int_s^t \alpha_0(s) (H_0(t)-H_0(s)) \alpha_{I,b}(s) ds \\
                                             & & -\rho_{ab}^{zI} \int_s^t \alpha_a(s)(H_a(t)-H_a(s))\alpha_{I,b}(s) ds \\
                                             & & +\rho_{ab}^{xI}\int_s^t \sigma_a(s) \alpha_{I,b}(s) ds \\
  \mathrm{Cov}[ \Delta x_a, \Delta y_{I,b} ] & = & \rho_{0b}^{zI} \int_s^t \alpha_0(s) (H_0(t)-H_0(s)) H_{I,b}(s)\alpha_{I,b}(s) ds \\
                                             & & -\rho_{ab}^{zI} \int_s^t \alpha_a(s)(H_a(t)-H_a(s))H_{I,b}(s)\alpha_{I,b}(s) ds \\
                                             & & +\rho_{ab}^{xI}\int_s^t \sigma_a(s) H_{I,b}(s)\alpha_{I,b}(s) ds
\end{eqnarray*}

The INF-INF covariance over the interval $[s,t] := [t_0, t_0+\Delta t]$ (conditional on $\mathcal{F}_{t_0}$) is

\begin{eqnarray*}
  \mathrm{Cov}[ \Delta z_{I,a}, \Delta z_{I,b} ] & = & \rho_{ab}^{II} \int_s^t \alpha_{I,a}(s) \alpha_{I,b}(s) ds \\
  \mathrm{Cov}[ \Delta z_{I,a}, \Delta y_{I,b} ] & = & \rho_{ab}^{II} \int_s^t \alpha_{I,a}(s) H_{I,b}(s)
                                                       \alpha_{I,b}(s) ds \\
  \mathrm{Cov}[ \Delta y_{I,a}, \Delta y_{I,b} ] & = & \rho_{ab}^{II} \int_s^t H_{I,a}(s) \alpha_{I,a}(s) H_{I,b}(s) \alpha_{I,b}(s) ds
\end{eqnarray*}

The equity/equity covariance over the interval $[s,t] := [t_0, t_0+\Delta t]$ (conditional on $\mathcal{F}_{t_0}$) is
\begin{eqnarray*}
	Cov \left[\Delta ln[s_i], \Delta ln[s_j] \right] &=&
	\rho_{\phi(i) \phi(j)}^{zz}\int_s^t (H_{\phi(i)} (t) - H_{\phi(i)} (u)) (H_{\phi(j)} (t)\\
	&& - H_{\phi(j)} (u)) \alpha_{\phi(i)}(u) \alpha_{\phi(j)}(u) du\\
	&&+ \rho_{\phi(i) j}^{zs} \int_s^t (H_{\phi(i)} (t) - H_{\phi(i)} (u)) \alpha_{\phi(i)}(u) \sigma_j^S(u) du\\
	&&+ \rho_{\phi(j) i}^{zs} \int_s^t (H_{\phi(j)} (t) - H_{\phi(j)} (u)) \alpha_{\phi(j)}(u) \sigma_i^S(u) du\\
	&&+ \rho_{ij}^{ss} \int_s^t \sigma_i^S(u) \sigma_j^S(u) du\\
\end{eqnarray*}

The equity/FX covariance over the interval $[s,t] := [t_0, t_0+\Delta t]$ (conditional on $\mathcal{F}_{t_0}$) is
\begin{eqnarray*}
	Cov \left[\Delta ln[s_i], \Delta ln[x_j] \right] &=&
	\rho_{\phi(i)0}^{zz} \int_s^t (H_{\phi(i)} (t) - H_{\phi(i)} (u)) (H_0 (t) - H_0 (u)) \alpha_{\phi(i)}(u) 
	\alpha_0(u) 
	du\\
	&& - \rho_{\phi(i)j}^{zz} \int_s^t (H_{\phi(i)} (t) - H_{\phi(i)} (u)) (H_j (t) - H_j (u)) \alpha_{\phi(i)} 
	(u)\alpha_j(u) du\\
	&& + \rho_{\phi(i)j}^{zx} \int_s^t (H_{\phi(i)} (t) - H_{\phi(i)} (u)) \alpha_{\phi(i)} (u) \sigma_j(u) du\\
	&&+ \rho_{i0}^{sz} \int_s^t (H_0 (t) - H_0 (u)) \alpha_0 (u) \sigma_i^S(u) du\\
	&&- \rho_{ij}^{sz} \int_s^t (H_j (t) - H_j (u)) \alpha_j (u) \sigma_i^S(u) du\\
	&&+ \rho_{ij}^{sx} \int_s^t \sigma_i^S(u) \sigma_j(u) du\\
\end{eqnarray*}

The equity/IR covariance over the interval $[s,t] := [t_0, t_0+\Delta t]$ (conditional on $\mathcal{F}_{t_0}$) is
\begin{eqnarray*}
	Cov \left[\Delta ln[s_i], \Delta z_j \right] &=&
	\rho_{\phi(i)j}^{zz} \int_s^t (H_{\phi(i)} (t) - H_{\phi(i)} (u)) \alpha_{\phi(i)} (u) \alpha_j (u) du\\
	&&+ \rho_{ij}^{sz} \int_s^t \sigma_i^S (u) \alpha_j (u) du\\
\end{eqnarray*}

The equity/inflation covariances over the interval $[s,t] := [t_0, t_0+\Delta t]$ (conditional on $\mathcal{F}_{t_0}$) are as follows:
\begin{eqnarray*}
	Cov \left[\Delta ln[s_i], \Delta z_{I,j} \right] &=&
	\rho_{\phi(i)j}^{zI} \int_s^t (H_{\phi(i)} (t) - H_{\phi(i)} (u)) \alpha_{\phi(i)} (u) \alpha_{I,j} (u) du\\
	&&+ \rho_{ij}^{sI} \int_s^t \sigma_i^S (u) \alpha_{I,j} (u) du\\	
	Cov \left[\Delta ln[s_i], \Delta y_{I,j} \right] &=&
	\rho_{\phi(i)j}^{zI} \int_s^t (H_{\phi(i)} (t) - H_{\phi(i)} (u)) \alpha_{\phi(i)} (u) H_{I,j} (u) \alpha_{I,j} (u) du\\
	&&+ \rho_{ij}^{sI} \int_s^t \sigma_i^S (u) H_{I,j} (u) \alpha_{I,j} (u) du\\
\end{eqnarray*}

The expectation of the Credit processes $z_{C,k}, y_{C,k}$ conditional on $\mathcal{F}_{t_0}$ at any time $t>t_0$ is
equal to $z_{C,k}(t_0)$ resp. $y_{C,k}(t_0)$ since both processes are drift free.

The credit/credit covariances over the interval $[s,t] := [t_0, t_0+\Delta t]$ (conditional on $\mathcal{F}_{t_0}$) are as follows:
\begin{eqnarray*}
	Cov \left[\Delta z_{C,a}, \Delta z_{C,b} \right] &=&
	\rho_{ab}^{CC}\int_s^t \alpha_{C, a}(u) \alpha_{C, b}(u) du\\
  Cov \left[\Delta z_{C,a}, \Delta y_{C,b} \right] &=&
	\rho_{ab}^{CC}\int_s^t \alpha_{C, a}(u) H_{C,b}(u) \alpha_{C, b}(u) du\\
  Cov \left[\Delta y_{C,a}, \Delta y_{C,b} \right] &=&
	\rho_{ab}^{CC}\int_s^t \alpha_{C, a}(u) H_{C,a}(u) \alpha_{C, b}(u) H_{C,b}(u) du\\
\end{eqnarray*}

The IR/credit covariances over the interval $[s,t] := [t_0, t_0+\Delta t]$ (conditional on $\mathcal{F}_{t_0}$) are as follows:
\begin{eqnarray*}
	Cov \left[\Delta z_a, \Delta z_{C,b} \right] &=&
	\rho_{ab}^{zC}\int_s^t \alpha_a(u) \alpha_{C, b}(u) du\\
  Cov \left[\Delta z_a, \Delta y_{C,b} \right] &=&
	\rho_{ab}^{zC}\int_s^t \alpha_a(u) H_{C,b}(u) \alpha_{C, b}(u) du\\
\end{eqnarray*}

The FX/credit covariances over the interval $[s,t] := [t_0, t_0+\Delta t]$ (conditional on $\mathcal{F}_{t_0}$) are as follows:
\begin{eqnarray*}
  \mathrm{Cov}[ \Delta x_a, \Delta z_{C,b} ] & = & \rho_{0b}^{zC} \int_s^t \alpha_0(s) (H_0(t)-H_0(s)) \alpha_{C,b}(s) ds \\
                                             & & -\rho_{ab}^{zC} \int_s^t \alpha_a(s)(H_a(t)-H_a(s))\alpha_{C,b}(s) ds \\
                                             & & +\rho_{ab}^{xC}\int_s^t \sigma_a(s) \alpha_{C,b}(s) ds \\
  \mathrm{Cov}[ \Delta x_a, \Delta y_{C,b} ] & = & \rho_{0b}^{zC} \int_s^t \alpha_0(s) (H_0(t)-H_0(s)) H_{C,b}(s)\alpha_{C,b}(s) ds \\
                                             & & -\rho_{ab}^{zC} \int_s^t \alpha_a(s)(H_a(t)-H_a(s))H_{C,b}(s)\alpha_{C,b}(s) ds \\
                                             & & +\rho_{ab}^{xC}\int_s^t \sigma_a(s) H_{C,b}(s)\alpha_{C,b}(s) ds
\end{eqnarray*}

The inflation/credit covariances over the interval $[s,t] := [t_0, t_0+\Delta t]$ (conditional on $\mathcal{F}_{t_0}$) are as follows:
\begin{eqnarray*}
  \mathrm{Cov}[ \Delta z_{I,a}, \Delta z_{C,b} ] &=&
  \rho_{ab}^{IC}\int_s^t \alpha_{I,a} \alpha_{C,b}(u) du\\
  \mathrm{Cov}[ \Delta z_{I,a}, \Delta y_{C,b} ] &=&
  \rho_{ab}^{IC}\int_s^t \alpha_{I,a} H_{C,b}(u) \alpha_{C,b}(u) du\\
  \mathrm{Cov}[ \Delta y_{I,a}, \Delta z_{C,b} ] &=&
  \rho_{ab}^{IC}\int_s^t \alpha_{I,a} H_{I,a}(u) \alpha_{C,b}(u) du\\
  \mathrm{Cov}[ \Delta y_{I,a}, \Delta y_{C,b} ] &=&
  \rho_{ab}^{IC}\int_s^t \alpha_{I,a} H_{I,a}(u) \alpha_{C,b}(u) H_{C,b}(u) du\\
\end{eqnarray*}

The equity/credit covariances over the interval $[s,t] := [t_0, t_0+\Delta t]$ (conditional on $\mathcal{F}_{t_0}$) are as follows:
\begin{eqnarray*}
	Cov \left[\Delta ln[s_i], \Delta z_{C,j} \right] &=&
	\rho_{\phi(i)j}^{zC} \int_s^t (H_{\phi(i)} (t) - H_{\phi(i)} (u)) \alpha_{\phi(i)} (u) \alpha_{C,j} (u) du\\
	&&+ \rho_{ij}^{sC} \int_s^t \sigma_i^S (u) \alpha_{C,j} (u) du\\	
	Cov \left[\Delta ln[s_i], \Delta y_{C,j} \right] &=&
	\rho_{\phi(i)j}^{zC} \int_s^t (H_{\phi(i)} (t) - H_{\phi(i)} (u)) \alpha_{\phi(i)} (u) H_{C,j} (u) \alpha_{C,j} (u) du\\
	&&+ \rho_{ij}^{sC} \int_s^t \sigma_i^S (u) H_{C,j} (u) \alpha_{C,j} (u) du\\
\end{eqnarray*}

\subsection{Change of Measure}

We can change measure from LGM to the T-Forward measure by applying a shift transformation to the $H$ parameter of the domestic LGM process, as explained in \cite{Lichters} and shown in Example 12, section \ref{sec:longterm}. This does not involve amending the system of SDEs above.

\medskip
\noindent
In the following we show how to move from the LGM to the Bank Account measure when we start with the Cross Asset Model in the LGM measure. This description and the implementation in ORE is limited so far to the cross currency case.

First note that the stochastic Bank Account (BA) can be written
\begin{align*}
B(t) &= \frac{1}{P(0,t)}\exp\left(\int_0^t (H_t-H_s)\,\alpha_s\,dW_s^B + \frac{1}{2}\int_0^t (H_t-H_s)^2\,\alpha^2_s\,ds \right)
\end{align*} 
with Wiener processes in the BA measure. We can express this in terms of the domestic LGM's state variable $z(t)$ and an auxiliary random variable $y(t)$
\begin{align*}
B(t) &= \frac{1}{P(0,t)}\exp\left(H(t)\,z(t) - y(t) + \frac{1}{2} \left(H^2(t)\,\zeta_0(t) + \zeta_2(t)\right)\right)
\intertext{with}
dz(t) &= \alpha(t)\,dW^B(t) - H(t)\,\alpha^2(t)\,dt \\
dy(t) &= H(t)\,\alpha(t)\,dW^B(t) \\
\zeta_n(t) &= \int_0^t \alpha^2(s)\,H^n(s) \,ds
\end{align*}
Note the drift of LGM state variable $z(t)$ in the BA measure and the auxiliary state variable $y(t)$ which is driven by the same Wiener process as $z(t)$. The instantaneous correlation of $dz$ and $dy$ is one, but the terminal correlation of $z(t)$ and $y(t)$ is less than one because of their different volatility functions. This is all we need to switch measure to BA in a pure domestic currency case.

To change measure in the cross currency case we need to make changes to the SDE beyond adding an auxiliary state variable $y$ and adding a drift to the domestic LGM state. Let us write down the SDEs in the LGM and BA measure with respective drift terms that ensure martingale properties.

SDE in the LGM measure
\begin{align*}
dz_0 &= \alpha_0\,dW_0^z \\
dz_i &= \left(-\alpha_i^2\,H_i -\rho_{ii}^{zx}\,\sigma_i\,\alpha_i + {\color{red} \rho_{i0}^{zz}\,\alpha_i\,\alpha_0\,H_0}\right)\,dt + \alpha_i\,dW_i^z \\
d\ln x_i &= \left(r_0 - r_i - \frac{1}{2}\sigma^2_i + {\color{red} \rho_{0i}^{zx}\,\alpha_0\,H_0\,\sigma_i} \right)\, dt + \sigma_i\,dW_i^x \\
\intertext{SDE in the BA measure}
{\color{blue}dy_0}  & = {\color{blue}\alpha_0\,H_0\,d\widetilde W_0^z} \\
dz_0 &= {\color{blue}-\alpha_0^2\,H_0\,dt} + \alpha_0\,d\widetilde W_0^z \\
dz_i &= \left(-\alpha_i^2\,H_i-\rho_{ii}^{zx}\,\sigma_i\,\alpha_i\right)\,dt + \alpha_i\,d\widetilde W_i^z \\
d\ln x_i &= \left(r_0 - r_i - \frac{1}{2}\sigma^2_i\right)\, dt + \sigma_i\,d\widetilde W_i^x,\qquad 
r_i = f_i(0,t) + z_i(t)\,H'_i(t) + \zeta_i(t)\,H_i(t)\,H'_i(t)
\end{align*}

Blue terms are {\color{blue}added}, red terms are {\color{red}removed} when moving from LGM to BA.

\medskip\noindent

These drift term changes lead to the following changes in conditional expectations 
\begin{align*}
\E[\Delta y_0] =& 0 \\
\E[\Delta z_0] =& - {\color{blue}\int_s^t H_0\,\alpha_0^2\,du}  \\
\E[\Delta z_i] =& - \int_s^t H_i\,\alpha_i^2\,du 
  - \rho^{zx}_{ii}\int_s^t \sigma_i^x\,\alpha_i\,du
  + {\color{red}\rho^{zz}_{0i} \int_s^t H_0\,\alpha_0\,\alpha_i\,du } \\
\E[\Delta \ln x] 
  =& \left(H_0(t)-H_0(s)\right) z_0(s) -\left(H_i(t)-H_i(s)\right)\,z_i(s)\\
  &+ \ln \left( \frac{P^n_0(0,s)}{P^n_0(0,t)} \frac{P^n_i(0,t)}{P^n_i(0,s)}\right) \\
  & - \frac12 \int_s^t (\sigma^x_i)^2\,du \\
  &+\frac12 \left(H^2_0(t)\, \zeta_0(t) -  H^2_0(s) \,\zeta_0(s) - \int_s^t H_0^2 \alpha_0^2\,du\right)\\
  &-\frac12 \left(H^2_i(t) \,\zeta_i(t) -  H^2_i(s) \,\zeta_i(s) - \int_s^t H_i^2 \alpha_i^2\,du\right)\\
  & + {\color{red} \rho^{zx}_{0i} \int_s^t H_0\, \alpha_0\, \sigma^x_i\,du} \\
  &  - \int_s^t \left(H_i(t)-H_i\right)\gamma_i \,du \qquad\mbox{with}\qquad
  \gamma_i = -\alpha_i^2\,H_i -\rho_{ii}^{zx}\,\sigma_i\,\alpha_i + {\color{red}\rho_{i0}^{zz}\,\alpha_i\,\alpha_0\,H_0}   \\
  & + {\color{blue}\int_s^t \left(H_0(t)-H_0\right)\,\gamma_0 \,du \qquad \mbox{with}\qquad \gamma_0 = - H_0\,\alpha_0^2}
\end{align*}
and the following additional variances and covariances
\begin{align*}
\mathrm{Var}[\Delta y_0] =& \int_s^t \alpha_0^2\,H_0^2\,du \\
\mathrm{Cov}[\Delta y_0, \Delta z_i] =& \rho^{zz}_{0i} \int_s^t \alpha_0\,H_0\,\alpha_i\,du \\
\mathrm{Cov}[\Delta y_0, \Delta \ln x_i] =& \int_s^t \left(H_0(t)-H_0\right) \alpha_0^2\,H_0\,du \\
&  - \rho^{zz}_{0i}\int_s^t \alpha_0\,H_0\left(H_i(t)-H_i\right)\, \alpha_i \,du \\
&  +\rho^{zx}_{0i}\int_s^t \alpha_0 \, H_0\,\sigma^x_i \,du 
%\mathrm{Var}[\Delta z_i] =& \int_s^t \alpha_i^2\,du \\
%\mathrm{Var}[\Delta \ln x_i] =&
%      \int_s^t \left(H_0(t)-H_0\right)^2 \alpha_0^2\,du \nonumber\\
%      & -2\rho^{zz}_{0i} \int_s^t \left(H_i(t)-H_i\right) \alpha_i\left(H_0(t)-H_0\right) \alpha_0\,du
%  \nonumber\\
%      &+ 2\rho^{zx}_{0i}\int_s^t \left(H_0(t)-H_0\right)\alpha_0 \,\sigma^x_i\,du \nonumber\\
%      &- 2\rho^{zx}_{ii}\int_s^t \left(H_i(t)-H_i\right)\alpha_i \,\sigma^x_i\,du\nonumber\\
%      &+ \int_s^t \left(H_i(t)-H_i\right)^2\alpha_i^2 \,du
%  \nonumber\\
%      &+ \int_s^t(\sigma^x_i)^2\,du \\
%\mathrm{Cov} [\Delta z_i, \Delta z_j] =& \rho^{zz}_{ij}\int_s^t \alpha_i\,\alpha_j\,du \\
%\mathrm{Cov} [\Delta z_i, \Delta \ln x_j] =& \rho^{zz}_{0i}\int_s^t \left(H_0(t)-H_0\right)
%  \alpha_0\,\alpha_i\,du \nonumber\\
%      &- \rho^{zz}_{ij}\int_s^t \alpha_i \,\alpha_j \,\left(H_j(t)-H_j\right) \,du \nonumber\\
%      &+\rho^{zx}_{ij}\int_s^t \alpha_i \, \sigma^x_j \,du.
\end{align*}

Example 36 in section \ref{example:36} illustrates the effect of the choice of measure on exposure simulations.

\subsection{Exposures}\label{sec:app_exposure}

In ORE we use the following exposure definitions
\begin{align}
\EE(t) = \EPE(t) &= \E^N\left[ \frac{(NPV(t)-C(t))^+}{N(t)} \right] \label{EE}\\
\ENE(t) &= \E^N\left[ \frac{(-NPV(t)+C(t))^+}{N(t)} \right] \label{ENE}
\end{align}
where $\NPV(t)$ stands for the netting set NPV and $C(t)$ is the collateral balance\footnote{$C(t)>0$ means that we have
  {\em received} collateral from the counterparty} at time $t$. Note that these exposures are expectations of values
discounted with numeraire $N$ (in ORE the Linear Gauss Markov model's numeraire) to today, and expectations are taken in
the measure associated with numeraire $N$. These are the exposures which enter into unilateral CVA and DVA calculation,
respectively, see next section. Note that we sometimes label the expected exposure (\ref{EE}) EPE, not to be confused
with the Basel III Expected Positive Exposure below.

\medskip
Basel III defines a number of exposures each of which is a 'derivative' of Basel's Expected Exposure:
\begin{align}
\intertext{Expected Exposure}
EE_B(t) &= \E[\max(NPV(t) - C(t), 0)] \label{basel_ee}\\
\intertext{Expected Positive Exposure}
EPE_B(T) &= \frac{1}{T} \sum_{t<T} EE_B(t)\cdot \Delta t  \label{basel_epe} \\
\intertext{Effective Expected Exposure, recursively defined as running maximum}
EEE_B(t) &= \max(EEE_B(t-\Delta t), EE_B(t)) \label{basel_eee}\\
\intertext{Effective Expected Positive Exposure}
EEPE_B(T) &= \frac{1}{T} \sum_{t<T} EEE_B(t)\cdot \Delta t \label{basel_eepe}
\end{align}
The last definition, Effective EPE, is used in Basel documents since Basel II for Exposure At Default and capital
calculation. Following \cite{bcbs128,bcbs189} the time averages in the EPE and EEPE calculations are taken over {\em the
  first year} of the exposure evolution (or until maturity if all positions of the netting set mature before one year).

\medskip
To compute $EE_B(t)$ consistently in a risk-neutral setting, we compound (\ref{EE}) with the deterministic discount factor $P(t)$ up to horizon $t$:
$$
EE_B(t) = \frac{1}{P(t)} \:\EE(t)
$$

Finally, we define another common exposure measure, the {\em Potential Future Exposure} (PFE), as a (typically high)
quantile $\alpha$ of the NPV distribution through time, similar to Value at Risk but at the upper end of the NPV
distribution:

\begin{align}
  \PFE_\alpha(t) = \left(\inf\left\{ x | F_t(x) \geq \alpha\right\}\right)^+ \label{PFE}
\end{align}

where $F_t$ is the cumulative NPV distribution function at time $t$. Note that we also take the positive part to ensure
that PFE is a positive measure even if the quantile yields a negative value which is possible in extreme cases.
 
\subsection{Exposures using American Monte Carlo}
\label{sec:app_amc}

The exposure analysis implemented in ORE that is used in the bulk of the examples in this user guide, mostly vanilla portfolios, 
is divided into two independent steps:

\begin{enumerate}
\item in a first step a list of NPVs (or a ``NPV cube'') is computed. The list is indexed by the trade ID, the
  simulation time step and the scenario sample number. Each entry of the cube is computed using the same pricers as for
  the T0 NPV calculation by shifting the evaluation date to the relevant time step of the simulation and updating the
  market term structures to the relevant scenario market data. The market data scenarios are generated using a {\em risk
    factor evolution model} which can be a cross asset model, but also be based on e.g. historical simulation.
\item in a second step the generated NPV cube is passed to a post processor that aggregates the results to XVA figures
  of different kinds.
\end{enumerate}

We label this approach in the following as the {\em classic} exposure analysis.

The AMC module in ORE allows to replace the first step by a different approach which works faster in particular for exotic
deals. The second step remains the same. The risk factor evolution model coincides with the pricing models for the
single trades in this approach and is always a cross asset model operated in a pricing measure.

For AMC the entries of the NPV cube are now viewed as conditional NPVs at the simulation time given the information that
is generated by the cross asset model's driving stochastic process up to the simulation time. The conditional
expectations are then computed using a regression analysis of some type. In our current implementation this is chosen to
be a parametric regression analysis.

The regression models are calibrated per trade during a training phase and later on evaluated in the simulation
phase. The set of paths in the two phases is in general different w.r.t. their number, time step structure, and
generation method (Sobol, Mersenne Twister) and seed. Typically the regressand is the (deflated) dirty {\em path} NPV of
the trade in question, or also its underlying NPV or an option continuation value (to take exercise decisions or
represent the physical underlying for physical exercise rights). The regressor is typically the model state. Certain
exotic features that introduce path-dependency (e.g. a TaRN structure) may require an augmentation of the regressor
though (e.g. by the already accumulated amount in case of the TaRN).

The path NPVs are generated at their {\em natural event dates}, like the fixing date for floating rate coupons or the
payment date for fixed cashflows. This reduces the requirements for the cross asset model to provide closed form
expressions for the numeraire and conditional zero bonds only.

Since the evaluation of the regression functions is computationally cheap the overall timings of the NPV cube generation
are generally smaller compared to the classic approach, in particular for exotic deals like Bermudan Swaptions.

From a methodology point of view an important difference between the classic and the AMC exposure analysis lies in the
model consistency: While the conditional NPVs computed with AMC are by construction consistent with the risk factor
evolution model driving the XVA simulation, the scenario NPVs in the classic approach are in general not consistent in
this sense unless the market scenarios are fully implied by the cross asset model. Here ``fully implied'' means that not
only rate curves, but also market volatility and correlation term structures like FX volatility surfaces, Swaption
volatilities or CMS correlation term structures as well as other parameters used by the single trade pricers have to be
deduced from the cross asset model, e.g. the mean reversion of the Hull White 1F model and a suitable model volatility
feeding into a Bermudan Swaption pricer.

We note that the generation of such implied term structures can be computationally expensive even for simple versions of
a cross asset model like one composed from LGM IR and Black-Scholes FX components etc., and even more so for more exotic
component flavours like Cheyette IR components, Heston FX components etc.

In the current implementation only a subset of all ORE trade types can be simulated using AMC while all other trade types
are still simulated using the classic engine. The separation of the trades and the joining of the resulting classic and AMC
cubes is automatic. The post processing step is run on the joint cube from the classic and AMC simulations as before.

Trade types supported by AMC so far:
\begin{enumerate}
\item Swap
\item CrossCurrencySwap
\item FxOption
\item BermudanSwaption
\item MultiLegOption
\end{enumerate}

\subsubsection{Implementation Details}\label{sec:implementation_details}

\subsubsection*{AMC valuation engine and AMC pricing engines}

The \verb+AMCValuationEngine+ is responsible for generating a NPV cube for a portfolio of AMC enabled trades and
(optionally) to populate a \verb+AggregationScenarioData+ instance with simulation data for post processing, very
similar to the classic \verb+ValuationEngine+ in ORE.

The AMC valuation engine takes a cross asset model defining the risk factor evolution. This is set up identically to the
cross asset model used in the \\ \verb+CrossAssetModelScenarioGenerator+. Similarly the same parameters for the path
generation (given as a \verb+ScenarioGeneratorData+ instance) are used, so that it is guaranteed that both the AMC
engine and the classic engine produce the same paths, hence can be combined to a single cube for post processing. It is
checked, that a non-zero seed for the random number generation is used.

The portfolio that the AMC engine consumes is build against an engine factory set up by a pricing engine configuration
given in the amc analytics type (see \ref{sec:amc_applicationconfig}). This configuration should select special AMC engine
builders which (by a pure naming convention) have the engine type ``AMC''. These engine builders are retrieved from
\verb+getAmcEngineBuilders()+ in \verb+oreappplus.cpp+ and are special in that unlike usual engine builders they take
two parameters

\begin{enumerate}
\item the cross asset model which serves as a risk factor evolution model in the AMC valuation engine
\item the date grid used within the AMC valuation engine
\end{enumerate}

For technical reasons, the configuration also contains configurations for \\ \verb+CapFlooredIborLeg+ and \verb+CMS+
because those are used within the trade builders (more precisely the leg builders called from these) to build the
trade. The configuration can be the same as for T0 pricing for them, it is actually not used by the AMC pricing engines.

The AMC engine builders build a smaller version of the global cross asset model only containing the model components
required to price the specific trade. Note that no deal specific calibration of the model is performed.

The AMC pricing engines perform a T0 pricing and - as a side product - can be used as usual T0 pricing engines if a
corresponding engine builder is supplied, see \ref{sec:amc_sideproducts}.

In addition the AMC pricing engines perform the necessary calculations to yield conditional NPVs on the given global
simulation grid. How these calcuations are performed is completely the responsibility of the pricing engines, altough
some common framework for many trade types is given by a base engine, see \ref{sec:amc_base_engine}. This way the
approximation of conditional NPVs on the simulation grid can be taylored to each product and also each single trade,
with regards to

\begin{enumerate}
\item the number of traning paths and the required date grid for the training (e.g. containing all relevant coupon and
  exercise event dates of a trade)
\item the order and type of regressoin basis functions to be used
\item the choice of the regressor (e.g. a TaRN might require a regressor augmented by the accumulated coupon amount)
\end{enumerate}

The AMC pricing engines then provide an additional result labelled \verb+amcCalculator+ which is a class implementing
the \verb+AmcCalculator+ interface which consists of two methods: The method \verb+simulatePath()+ takes a
\verb+MultiPath+ instance representing one simulated path from the global risk factor evolution model and returns an
array of conditional, deflated NPVs for this path. The method \verb+npvCurrency()+ returns the currency $c$ of the
calculated conditional NPVs. This currency can be different from the base currency $b$ of the global risk factor
evolution model. In this case the conditional NPVs are converted to the global base currency within the AMC valuation
engine by multiplying them with the conversion factor

\begin{equation}\label{currency_conversion_factor}
\frac{N_c(t) X_{c,b}(t)}{N_b(t)}
\end{equation}

where $t$ is the simulation time, $N_c(t)$ is the numeraire in currency $c$, $N_b(t)$ is the numeraire in currency
$b$ and $X_{c,b}(t)$ is the FX rate at time $t$ converting from $c$ to $b$.

The technical criterion for a trade to be processed within the AMC valuation is engine is that a) it can be built
against the AMC engine factory described above and b) it provides an additional result \verb+amcCalculator+. If a trade
does not meet these criteria it is simulated using the classic valuation engine. The logic that does this is located in
the overide of the method \verb+OREAppPlus::generateNPVCube()+.

The AMC valuation engine can also populate an aggregation scenario data instance. This is done only if necessary,
i.e. only if no classic simulation is performed anyway. The numeraire and fx spot values produced by the AMC valuation
engine are identical to the classic engine. Index fixings are close, but not identical, because the AMC engine used the
T0 curves for projection while the classic engine uses scenario simulation market curves, which are not exactly matching
those of the T0 market. In this sense the AMC valuation engine produces more precise values compared to the classic
engine.

\subsubsection*{The multileg option AMC base engine and derived engines}\label{sec:amc_base_engine}

Table \ref{tbl:amcconfig} provides an overview of the implemented AMC engine builders. These builders use the following
QuantExt pricing engines

\begin{enumerate}
\item \verb+McLgmSwapEngine+ for single currency swaps
\item \verb+McCamCurrencySwapEngine+ for cross currency swaps
\item \verb+McCamFxOptionEngine+ for fx options
\item \verb+McLgmSwaptionEngine+ for Bermudan swaptions
\item \verb+McMultiLegOptionEngine+ for Multileg option
\end{enumerate}

All these engine are based on a common \verb+McMultiLegBaseEngine+ which does all the computations. For this each of the
engines sets up the following protected member variables (serving as parameters for the base engine) in their
\verb+calculate()+ method:

\begin{enumerate}
\item \verb+leg_+: a vector of \verb+QuantLib::Leg+
\item \verb+currency_+: a vector of \verb+QuantLib::Currency+ corresponding to the leg vector
\item \verb+payer_+: a vector of $+1.0$ or $-1.0$ double values indicating receiver or payer legs
\item \verb+exercise_+: a \verb+QuantLib::Exercise+ instance describing the exercise dates (may be \verb+nullptr+, if
  the underlying represents the deal already)
\item \verb+optionSettlement_+: a \verb+Settlement::Type+ value indicating whether the option is settled physically or
  in cash
\end{enumerate}

A call to \verb+McMultiLegBaseEngine::calculate()+ will set the result member variables

\begin{enumerate}
\item \verb+resultValue_+: T0 NPV in the base currency of the cross asset model passed to the pricing engine
\item \verb+underlyingValue_+: T0 NPV of the underlying (again in base ccy)
\item *\verb+amcCalculator_+: the AMC calculator engine to be used in the AMC valuation engine
\end{enumerate}

The specific engine implementations should convert the \verb+resultValue_+ to the npv currency of the trade (as defined
by the (ORE) trade builder) so that they can be used as regular pricing engine consistently within ORE. Note that only
the additional \verb+amcCalculator+ result is used by the AMC valuation engine, not any of the T0 NPVs directly.

\subsubsection{Limitations and Open Points}
\label{sec:amc_limitations}

This sections lists known limitations of the AMC simulation engine.

\subsubsection*{Trade Features}

Some trade features are not yet supported by the multileg option engine:

\begin{enumerate}
\item exercise flows (like a notional exchange common to cross currency swaptions) are not supported
\end{enumerate}

\subsubsection*{Flows Generation (for DIM Analysis)}

At the current stage the AMC engine does not generate flows which are required for the DIM analysis in the post
processor.

\subsubsection*{State interpolation for exercise decisions}

During the simulation phase exercise times of a specific trade are not necessarily part of the simulated time
grid. Therefore the model state required to take the exercise decision has in to be interpolated in general on the
simulated path. Currently this is done using a simple linear interpolation while from a pure methodology point of view a
Brownian Bridge would be preferable. In our tests we do not see a big impact of this approximation though.

\subsubsection*{Basis Function Selection}

Currently the basis function system is generated by specifying the type of the functions and the order, see
\ref{sec:amc_pricingengineconfig}. The number of independent variables varies by product type and details. Depending on
the number of independent variables and the order the number of generated basis functions can get quite big which slows
down the computation of regression coefficients. It would be desirable to have the option to filter the full set of
basis functions, e.g. by explicitly enumerating them in the configuration, so that a high order can be chosen even for
products with a relatively large number of independent variables (like e.g. FX Options or Cross Currency Swaps).

\subsubsection{Outlook: Trade Compression}

For vanilla trades where the regression is only required to produce the NPV cube entries (and not to take exercise
decisions etc.) it is not strictly necessary to do the regression analysis on a single trade level\footnote{except
  single trade exposures are explicitly required of course}. Although in the current implementation there is no direct
way to do the regression analysis on whole (sub-)portfolios instead of single trades, one can represent such a
subportfolio as a single technical trade (e.g. as a single swap or multileg option trade) to achieve a similar
result. This might lead to better performance than the usual single trade calculation. However one should also try to
keep the regressions as low-dimensional as possible (for performance and accuracy reasons) and therefore define the
sub-portfolios by e.g. currency, i.e. as big as possible while at the same time keeping the associated model dimension as
small as possible.

\subsection{CVA and DVA}\label{sec:app_cvadva}

Using the expected exposures in \ref{sec:app_exposure} unilateral discretised CVA and DVA are given by \cite{Lichters}
\begin{align}
\CVA &= \sum_{i} \PD(t_{i-1},t_i)\times\LGD\times \EPE(t_i) \label{CVA}\\
\DVA &= \sum_{i} \PD_{Bank}(t_{i-1},t_i)\times\LGD_{Bank}\times \ENE(t_i) \label{DVA}
\end{align}
where
\begin{align*}
\EPE(t) & \mbox{ expected exposure (\ref{EE})}\\
\ENE(t) & \mbox{ expected negative exposure (\ref{ENE})}\\
PD(t_i,t_j) & \mbox{ counterparty probability of default in } [t_i;t_j]\\
PD_{Bank}(t_i,t_j) & \mbox{ our probability of default in } [t_i;t_j]\\
LGD & \mbox{ counterparty loss given default}\\
LGD_{Bank} & \mbox{ our loss given default}\\
\end{align*}

Note that the choice $t_i$ in the arguments of $\EPE(t_i)$ and $\ENE(t_i)$ means we are choosing the {\em advanced}
rather than the {\em postponed} discretization of the CVA/DVA integral \cite{BrigoMercurio}. This choice can be easily
changed in the ORE source code or made configurable. \\

Moreover, formulas (\ref{CVA}, \ref{DVA}) assume independence of credit and other market risk factors, so that $\PD$ and
$\LGD$ factors are outside the expectations. With the extension of ORE to credit asset classes and in particular for
wrong-way-risk analysis, CVA/DVA formulas is generaised and is applicable to calculations with dynamic credit

\begin{align}
\CVA^{dyn} &= \sum_{i} \E^N\left[\frac{\PD^{dyn}(t_{i-1},t_i)\times \PE(t_i)}{N(t)} \right]\times\LGD \label{CVA_dynamic} \\
\DVA^{dyn} &= \sum_{i} \E^N\left[\frac{\PD^{dyn}_{Bank}(t_{i-1},t_i)\times \NE(t_i)}{N(t)} \right]\times\LGD_{Bank} \label{DVA_dynamic}
\end{align}
where
\begin{align*}
\PE(t) & \mbox{ random variables representing positive exposure at } t: (NPV(t)-C(t))^+\\
\NE(t) & \mbox{ random variables representing negative exposure at } t: (-NPV(t)+C(t))^+\\
PD^{dyn}(t_i,t_j) & \mbox{ random variables representing counterparty probability of default in } [t_i;t_j]\\
PD^{dyn}_{Bank}(t_i,t_j) & \mbox{ random variables representing our probability of default in } [t_i;t_j]\\
LGD & \mbox{ counterparty loss given default}\\
LGD_{Bank} & \mbox{ our loss given default}\\
\end{align*}

\subsection{FVA}\label{sec:fva}

%Any exposure (uncollateralised or residual after taking collateral into account) gives rise to funding cost or benefits
%depending on the sign of the residual position. This can be expressed as a Funding Value Adjustment (FVA). A simple
%definition of FVA can be given in a very similar fashion as the sum of unilateral CVA and DVA which we defined by
%(\ref{CVA},\ref{DVA}), namely as an expectation of exposures times funding spreads:
%\begin{align}
%  \FVA &= \underbrace{\sum_{i=1}^n f_b(t_{i-1},t_i)\,\delta_i \, \E^N\left[S_C(t_{i-1})\, S_B(t_{i-1})\, (\NPV(t_i))^+\,
%         D(t_i)\right]}_{\mbox{Funding Benefit Adjustment (FBA)}}\nonumber\\
%       & {} - \underbrace{\sum_{i=1}^n f_l(t_{i-1},t_i)\,\delta_i \, \E^N\left[S_C(t_{i-1})\, S_B(t_{i-1})\, (-\NPV(t_i))^+\, D(t_i)\right]}_{\mbox{Funding Cost Adjustment (FCA)}}\label{eq_simple_fva}
%\end{align}
%where
%\begin{align*}
%D(t_i) & \mbox{ stochastic discount factor, $1/N(t_i)$ in LGM}\\
%\NPV(t_i) & \mbox{ portfolio value after potential collateralization}\\
%S_C(t_j) & \mbox{ survival probability of the counterparty}\\
%S_B(t_j) & \mbox{ survival probability of the bank}\\
%f_b(t_j) & \mbox{ borrowing spread for the bank relative to the collateral compounding rate}\\
%f_l(t_j) & \mbox{ lending spread for the bank relative to the collateral compounding rate}
%\end{align*}
%For details see e.g. Chapter 14 in Gregory \cite{Gregory12} and the discussion in \cite{Lichters}.

Any exposure (uncollateralised or residual after taking collateral into account) gives rise to funding cost or benefits
depending on the sign of the residual position. This can be expressed as a Funding Value Adjustment (FVA). A simple
definition of FVA can be given in a very similar fashion as the sum of unilateral CVA and DVA which we defined by
(\ref{CVA},\ref{DVA}), namely as an expectation of exposures times funding spreads:
\begin{align}
  \FVA &= \underbrace{\sum_{i=1}^n f_l(t_{i-1},t_i)\,\delta_i \, \E^N\left\{S_C(t_{i-1})\, S_B(t_{i-1})\, [-\NPV(t_i)+C(t_i)]^+\,
         D(t_i)\right\}}_{\mbox{Funding Benefit Adjustment (FBA)}}\nonumber\\
       & {} - \underbrace{\sum_{i=1}^n f_b(t_{i-1},t_i)\,\delta_i \, \E^N\left\{S_C(t_{i-1})\, S_B(t_{i-1})\, [\NPV(t_i)-C(t_i)]^+\, D(t_i)\right\}}_{\mbox{Funding Cost Adjustment (FCA)}}\label{eq_simple_fva}
%  \FVA &= - \underbrace{\sum_{i=1}^n f_b(t_{i-1},t_i)\,\delta_i \, \E^N\left[S_C(t_{i-1})\, S_B(t_{i-1})\, (\NPV(t_i))^+\,
 %        D(t_i)\right]}_{\mbox{Funding Cost Adjustment (FCA)}}\nonumber\\
 %      & {} \underbrace{\sum_{i=1}^n f_l(t_{i-1},t_i)\,\delta_i \, \E^N\left[S_C(t_{i-1})\, S_B(t_{i-1})\, (-\NPV(t_i))^+\, D(t_i)\right]}_{\mbox{Funding Benefit Adjustment (FBA)}}\label{eq_simple_fva}
\end{align}
where
\begin{align*}
D(t_i) & \mbox{ stochastic discount factor, $1/N(t_i)$ in LGM}\\
\NPV(t_i) & \mbox{ portfolio value at time } t_i\\
C(t_i) & \mbox{Collateral account balance at time } t_i \\ 
S_C(t_j) & \mbox{ survival probability of the counterparty}\\
S_B(t_j) & \mbox{ survival probability of the bank}\\
f_b(t_j) & \mbox{ borrowing spread for the bank relative to OIS flat}\\
f_l(t_j) & \mbox{ lending spread for the bank relative to OIS flat}
\end{align*}
For details see e.g. Chapter 14 in Gregory \cite{Gregory12} and the discussion in \cite{Lichters}.

\medskip
The reasoning leading to the expression above is as follows. Consider, for example, a single partially collateralised derivative (no collateral at all or CSA with a significant threshold) between us (the Bank) and counterparty 1 (trade 1). 

We assume that we enter into an offsetting trade with (hypothetical) counterparty 2 which is perfectly collateralised (trade 2). We label the NPV of trade 1 and 2 $\NPV_{1,2}$ respectively (from our perspective, excluding CVA). Then $\NPV_2=-\NPV_1$. The respective collateral amounts due to trade 1 and 2 are $C_1$ and $C_2$ from our perspective. Because of the perfect collateralisation of trade 2 we assume $C_2=\NPV_2$. The imperfect collateralisation of trade 1 means $C_1 \ne \NPV_1$. The net collateral balance from our perspective is then $C=C_1+C_2$ which can be written $C=C_1+C_2 = C_1 + \NPV_2 = -\NPV_1 + C_1$.

\begin{itemize}
\item If $C>0$ we receive net collateral and pay the overnight rate on this notional amount. On the other hand we can invest the received collateral and earn our lending rate, so that we have a benefit proportional to the lending spread $f_l$ (lending rate minus overnight rate). It is a benefit assuming $f_l >0$. $C>0$ means $-\NPV_1 + C_1 > 0$ so that we can cover this case with ``lending notional'' $[-\NPV_1 + C_1]^+$.
\item If $C<0$ we post collateral amount $-C$ and receive the overnight rate on this amount. Amount $-C$ needs to be funded in the market, and we pay our borrowing rate on it. This leads to a funding cost proportional to the borrowing spread $f_b$ (borrowing rate minus overnight). $C<0$ means $\NPV_1 - C_1 > 0$, so that we can cover this case with ``borrowing notional'' $[\NPV_1 - C_1]^+$. If the borrowing spread is positive, this term proportional to $f_b \times [\NPV_1 - C_1]^+$ is indeed a cost and therefore needs to be subtracted from the benefit above.
\end{itemize}
   
Formula \eqref{eq_simple_fva} evaluates these funding cost components on the basis of the original trade's or portfolio's $\NPV$. Perfectly collateralised portfolios hence do not contribute to FVA because under the hedging fiction, they are hedged with a perfectly collateralised opposite portfolio, so any collateral payments on portfolio 1 are canceled out by those of the opposite sign on portfolio 2.

\subsection{COLVA}

When the CSA defines a collateral compounding rate that deviates from the overnight rate, this gives rise to another
value adjustment labeled COLVA \cite{Lichters}. In the simplest case the deviation is just given by a constant spread
$\Delta$:
\begin{align}
\COLVA &= \E^N\left[ \sum_i -C(t_i)\cdot \Delta \cdot \delta_i \cdot D(t_{i+1}) \right]
\label{COLVA}
\end{align}
where $C(t)$ is the collateral balance\footnote{see \ref{sec:app_exposure}, $C(t)>0$ means that we have {\em received}
  collateral from the counterparty} at time $t$ and $D(t)$ is the stochastic discount factor $1/N(t)$ in LGM. Both
$C(t)$ and
$N(t)$ are computed in ORE's Monte Carlo framework, and the expectation yields the desired adjustment. \\
 
Replacing the constant spread by a time-dependent deterministic function in ORE is straight forward. 
  
\subsection{Collateral Floor Value}

A less trivial extension of the simple COLVA calculation above, also covered in ORE, is the case where the deviation
between overnight rate and collateral rate is stochastic itself. A popular example is a CSA under which the collateral
rate is the overnight rate {\em floored at zero}. To work out the value of this CSA feature one can take the difference
of discounted margin cash flows with and without the floor feature. It is shown in \cite{Lichters} that the following
formula is a good approximation to the collateral floor value
\begin{align}
\Pi_{Floor} &= \E^N\left[ \sum_i -C(t_i)\cdot (-r(t_i))^+\cdot\delta_i \cdot D(t_{i+1}) \right]
\label{CSA_floor_value_approx}
\end{align}
where $r$ is the stochastic overnight rate and $(-r)^+ = r^+ - r$ is the difference between floored and 'un-floored' compounding rate. \\

Taking both collateral spread and floor into account, the value adjustment is 
\begin{align}
\Pi_{Floor,\Delta} &= \E^N\left[ \sum_i -C(t_i)\cdot ((r(t_i)-\Delta)^+-r(t_i))\cdot\delta_i \cdot D(t_{i+1}) \right] 
\label{CSA_floor_value_approx_2}
\end{align}

\subsection{Dynamic Initial Margin and MVA}\label{sec:app_dim}

The introduction of Initial Margin posting in non-cleared OTC derivatives business reduces residual credit exposures and
the associated value adjustments, {\bf CVA} and {\bf DVA}.

On the other hand, it gives rise to additional funding cost. The value of the latter is referred to as Margin Value Adjustment ({\bf MVA}).\\

To quantify these two effects one needs to model Initial Margin under future market scenarios, i.e. Dynamic Initial Margin ({\bf DIM}). Potential approaches comprise 
\begin{itemize}
\item Monte Carlo VaR embedded into the Monte Carlo simulation
\item Regression-based methods
\item Delta VaR under scenarios
\item ISDA's Standard Initial Margin (SIMM) under scenarios
\end{itemize} 

We skip the first option as too computationally expensive for ORE. In the current ORE release we focus on a relatively
simple regression approach as in \cite{Anfuso2016,LichtersEtAl}. Consider the netting set values $\NPV(t)$ and $\NPV(t+\Delta)$ that
are spaced one margin period of risk $\Delta$ apart. Moreover, let $F(t,t+\Delta)$ denote cumulative netting set cash
flows between time $t$ and $t+\Delta$, converted into the NPV currency. Let $X(t)$ then denote the netting set value
change during the margin period of risk excluding cash flows in that period:
$$
X(t) = \NPV(t+\Delta) + F(t, t+\Delta) - \NPV(t) 
$$  
ignoring discounting/compounding over the margin period of risk. We actually want to determine the distribution of
$X(t)$ conditional on the `state of the world' at time $t$, and pick a high (99\%) quantile to determine the Initial
Margin amount for each time $t$. Instead of working out the distribution, we content ourselves with estimating the
conditional variance $\V(t)$ or standard deviation $S(t)$ of $X(t)$, assuming a normal distribution and scaling $S(t)$
to the desired 99\% quantile by multiplying with the usual factor $\alpha=2.33$ to get an estimate of the Dynamic
Initial Margin $\DIM$:
$$
\V(t) = \E_t[X^2] - \E_t^2[X], \qquad S(t)=\sqrt{\V(t)}, \qquad \DIM(t) = \alpha \,S(t)
$$ 
We further assume that $\E_t[X]$ is small enough to set it to the expected value of $X(t)$ across all Monte Carlo
samples $X$ at time $t$ (rather than estimating a scenario dependent mean). The remaining task is then to estimate the
conditional expectation $\E_t[X^2]$. We do this in the spirit of the Longstaff Schwartz method using regression of
$X^2(t)$ across all Monte Carlo samples at a given time. As a regressor (in the one-dimensional case) we could use
$\NPV(t)$ itself. However, we rather choose to use an adequate market point (interest rate, FX spot rate) as regression
variable $x$, because this is generalised more easily to the multi-dimensional case. As regression basis functions we
use polynomials, i.e. regression functions of the form $c_0 + c_1\,x + c_2\,x^2 + ...+ c_n\,x^n$ where the order $n$ of
the polynomial can be selected by the user. Choosing the lowest order $n=0$, we obtain the simplest possible estimate,
the variance of $X$ across all samples at time $t$, so that we apply a single $\DIM(t)$ irrespective of the 'state of
the world' at time $t$ in that case.  The extension to multi-dimensional regression is also implemented in ORE. The user
can choose several regressors simultaneously (e.g. a EUR rate, a USD rate, USD/EUR spot FX rate, etc.) in order order to
cover complex multi-currency portfolios.

\medskip
Given the DIM estimate along all paths, we can next work out the Margin Value Adjustment \cite{Lichters} in discrete form
%{\color{red}
\begin{align}
\MVA &= \sum_{i=1}^n (f_b - s_I)\, \delta_i\: S_C(t_i)\: S_B(t_i) \times \E^N\left[
\DIM(t_i)\,D(t_i)\right]. \label{MVA} 
\end{align}
%}
with borrowing spread $f_b$ as in the FVA section \ref{sec:fva} and spread $s_I$ received on initial margin, both
spreads relative to the cash collateral rate.

\subsection{KVA (CCR)}\label{sec:app_kva}

The KVA is calculated for the Counterparty Credit Risk Capital charge
(CCR) following the IRB method concisely  described in
\cite{Gregory15}, Appendix 8A.
It is following the Basel rules by computing risk capital as the
product of alpha weighted  exposure at default, worst case probability
of default at 99.9  and a maturity adjustment factor also described in
the Basel annex 4.
The risk capital charges are discounted with a capital discount factor
and summed up to  give the total CCR KVA after being multiplied with
the risk  weight and a capital charge (following the RWA method).

\medskip Basel II internal rating based (IRB) estimate of worst case
probability of  default: large homogeneous pool (LHP) approximation of
Vasicek (1997), KVA regulatory probability of default is the worst
case probability of default floored at 0.03 (the latter is valid for 
corporates and banks, no such floor applies to sovereign counterparties):
$$
\PD_{99.9\%} = \max\left(floor, N \left(\frac{N^{-1}(\PD) + \sqrt{\rho}
  N^{-1}(0.999)}{\sqrt{1 - \rho}}\right) - \PD\right)
$$
$N$ is the cumulative standard normal distribution,

$$
\rho = 0.12 \frac{1 - e^{-50 \PD}}{1 - e^{-50}} + 0.24 \left(1 - \frac{1 -
  e^{-50 \PD}}{1 - e^{-50}}\right)
$$

\medskip Maturity adjustment factor for RWA method capped at 5, floored at 1:
$$
\MA(\PD, M) = \min\left(5, \max\left(1, \frac{1 + (M - 2.5) B(\PD)}{1 - 1.5 B(\PD)}\right)\right)
$$
\medskip where $B(\PD) = (0.11852 - 0.05478 \ln(\PD))^2$ and M is the
effective  maturity of the portfolio (capped at 5):

$$M = \min\left(5, 1 + \frac{\sum\limits_{t_k > 1yr} \EE_B(t_k)\Delta t_k
  B(0,t_k)}{\sum\limits_{t_k \leq 1yr} \EEE_B(t_k)\Delta t_k B(0,t_k)}\right)
$$

\medskip where $B(0,t_k)$ is the risk-free discount factor from the
simulation  date $t_k$ to today, $\Delta t_k$ is the difference
between time points, 
$\EE_B(t_k)$ is the expected (Basel) exposure at time $t_k$ and $\EEE_B(t_k)$ is the
associated effective expected exposure.

\medskip 
Expected risk capital at $t_i$:
$$
\RC(t_i) = EAD(t_i) \times LGD \times \PD_{99.9\%} \times \MA(\PD, M)
$$
where
\begin{itemize}
\item $\EAD(t_i) = \alpha \times \EEPE(t_i)$
\item $\EEPE(t_i)$ is estimated as the time average of the running maximum of $\EPE(t)$ over the time interval $t_i\leq t\leq t_i+1$
\item $\alpha$ is the multiplier resulting from the IRB calculations (Basel II defines a supervisory alpha of 1.4, but gives banks the option to estimate their own $\alpha$,subject to a floor of 1.2).
\item the maturity adjustment MA is derived from the EPE profile for times $t\geq t_i$
\end{itemize}

\medskip 
$\KVA_{CCR}$ is the sum of the expected risk capital amount discounted at {\em capital discount rate} $r_{cd}$ and compounded at rate given by the product of {\em capital hurdle} $h$ and {\em regulatory adjustment} $a$:
$$
\KVA_{CCR} = \sum_i \RC(t_i) \times \frac{1}{ (1 + r_{cd})^{\delta(t_{i-1}, t_i)}} \times \delta(t_{i-1}, t_i) \times h \times a
$$
assuming Actual/Actual day count to compute the year factions $delta$.

In ORE we compute KVA CCR from both perspectives - ``our'' KVA driven by EPE and the counterparty default risk, and similarly ``their'' KVA driven by ENE and our default risk.

\subsection{KVA (BA-CVA)}\label{sec:app_kva_cva}

This section briefly summarizes the calculation of a capital value adjustment associated with the CVA capital charge (in the basic approach, BA-CVA) as introduced in Basel III \cite{bcbs189, d325, d424}. ORE implements the {\em stand-alone} capital charge $\SCVA$ for a netting set and computes a KVA for it\footnote{In the reduced version of BA-CVA, where hedges are not recognized, the total BA-CVA capital charge across all counterparties $c$ is given by
$$
K = \sqrt{\left(\rho \sum_c \SCVA_c\right)^2 +(1-\rho^2)\sum_c \SCVA_c^2}
$$  
with supervisory correlation $\rho=0.5$ to reflect that the credit spread risk factors across counterparties are not perfectly correlated. Each counterparty $\SCVA_c$ is given by a sum over all netting sets with this counterparty.}. In the basic approach, the stand-alone capital charge for a netting set is given by
$$
\SCVA = \RW_c\cdot M\cdot \EEPE \cdot\DF
$$
with 
\begin{itemize}
\item supervisory risk weight $\RW_c$ for the counterparty;
\item effective netting set maturity $M$ as in section \ref{sec:app_kva} (for a bank using IMM to calculate EAD), but without applying a cap of 5;
\item supervisory discount $\DF$ for the netting set which is equal to one for banks using IMM to calculate $\EEPE$ and $\DF=\left(1-\exp\left(-0.05\,M\right)\right)/(0.05\,M)$ for banks not using IMM to calculate $\EEPE$. 
\end{itemize}

The associated capital value adjustment is then computed for each netting set's stand-alone CVA charge as above
$$
\KVA_{BA-\CVA} = \sum_i \SCVA(t_i) \times \frac{1}{ (1 + r_{cd})^{\delta(t_{i-1}, t_i)}} \times \delta(t_{i-1}, t_i) \times h \times a
$$
with 
$$
\SCVA(t_i) = \RW_c \cdot M(t_i)\cdot \EEPE(t_i)\cdot\DF
$$
where we derive both $M$ and EEPE from the EPE profile for times $t\geq t_i$.

In ORE we compute KVA BA-CVA from both perspectives - ``our'' KVA driven by EPE and the counterparty risk weight, and similarly ``their'' KVA driven by ENE and our risk weight. \\

Note: Banks that use the BA-CVA for calculating CVA capital requirements are allowed to cap the maturity adjustment factor $\MA(\PD,M)$ in section \ref{sec:app_kva} at 1 for netting sets that contribute to CVA capital, if using the IRB approach for CCR capital.

\subsection{Collateral Model}\label{sec:app_collateral}

The collateral model implemented in ORE is based on the evolution of collateral account balances along each Monte Carlo
path taking into account thresholds, minimum transfer amounts and independent amounts defined in the CSA, as well as
margin periods of risk.

ORE computes the collateral requirement (aka \emph{Credit Support Amount}) through time along each Monte Carlo path
\begin{align}\label{eq:CSA}
CSA(t_m) &= 
\begin{cases}
\max(0, \NPV(t_m) + \IA - \Th_{rec}),& \NPV(t_m) + \IA \ge 0 \\
\min(0, \NPV(t_m) + \IA + \Th_{pay}),& \NPV(t_m) + \IA < 0
\end{cases}
\end{align}
where
\begin{itemize}
\item $\NPV(t_m)$ is the value of the netting set as of
  time $t_m$ from our persepctive,
  \item $\Th_{rec}$ is the threshold exposure below which we do not 
  require collateral, likewise $\TH_{pay}$ is the threshold that applies to collateral posted to the counterparty,
%\item $MTA$ is the minimum transfer amount for collateral margin
%  flow requests (possibly asymmetric)
\item $\IA$ is the sum of all collateral independent amounts attached to
  the underlying portfolio of trades (positive amounts imply that we
  have received a net inflow of independent amounts from the
  counterparty), assumed here to be cash.
\end{itemize}

As the collateral account already has a value of $C(t_m)$ at time $t_m$, the collateral shortfall is simply the
difference between $C(t_m)$ and $\CSA(t_m)$. However, we also need to account for the possibility that margin calls
issued in the past have not yet been settled (for instance, because of disputes). If $M(t_m)$ denotes the net value of
all outstanding margin calls at $t_m$, and $\Delta(t)$ is the difference 
$$
\Delta(t) = \CSA(t_m) - C(t_m) - M(t_m)
$$
between the {\em Credit Support Amount} and the current and outstanding collateral, then the actual margin
\emph{Delivery Amount} $D(t_m)$ is calculated as follows:
\begin{align}\label{eq:DA}
D(t_m) &= 
\begin{cases}
\Delta(t),& \left| \Delta(t) \right| \ge MTA \\
0,& \left| \Delta(t) \right| < MTA
\end{cases}
\end{align}
where $MTA$ is the minimum transfer amount. 

Consider the upper case of \eqref{eq:CSA}: If the initial value of the netting set is zero ($\NPV(t_0)=0$) and 
if $\Th_{rec}=0$, but the combined $\IA>0$, then the Credit Support Amount equals the Independent Amount, $\CSA(t_0)=\IA$.
If moreover the initial collateral balance is zero (because the Independent Amount has not been received yet),
then $\Delta(t_0)=\CSA(t_0)=\IA$, and the delivery amount $D(t_0)$ also matches the $\IA$ (assuming this exceeds the MTA),
so that the next call leads to the transfer of the Independent Amount to us. For a positive $\Th_{rec}>0$, the transfer to us is reduced accordingly.
In that case we can view the Independent Amount as an offset to the threshold.

Consider the lower case of \eqref{eq:CSA}: If the netting set value is negative from our perspective and in absolute terms larger than the $\IA$, 
then the Credit Support Amount is just the negative difference $\CSA=-|\NPV| + \IA + \Th_{pay}$ so that we need to post collateral, but only the amount 
beyond the combined threshold $\IA + \Th_{pay}$.

\subsubsection{Margin Period of Risk} \label{sec:mpor}
After a counterparty defaults, it takes time to close out the portfolio. During this time period the portfolio value will change upon market conditions, therefore the portfolio's close-out value is subject to market risk, which is referred also as the close-out risk and the corresponding close-out period is called as the {\em Margin Period of Risk} (MPoR).  

Therefore, when a loss on the defaulted counterparty is realised at time $t_d$, the last time the collateral could be received is $t_d-\tau$, where $\tau$ denotes the MPoR. That is, the collateral at time $t_d$ is determined by the collateral value at $t_d-\tau$, namely $CSA(t_d-\tau)$, see equation \ref{eq:CSA}.

In ORE, we have two approaches to incorporate MPoR in the exposure simulations:
\begin{itemize}
 \item {\em Close-out Approach}: Simulating on an auxiliary close-out grid additional to the default time grid.
 \item {\em Lagged Approach}: Simulating only on a default time grid and delaying the margin calls on the grid.
\end{itemize}

\medskip In the {\em Close-out Approach}, we use an auxiliary ``close-out'' grid in addition to the main simulation grid (see section \ref{sec:simulation}). The main simulation grid is used to compute “default values” which feed into the collateral balance $C(t$) filtered by MTA and Threshold etc. The auxiliary “close-out” grid, offset from the main grid by the MPoR, is used to compute the delayed close-out values $V(t)$ associated with default time $t$\footnote{We note that in ORE when the exposure of an uncollateralised netting-set or a single trade without considering the netting-set is calculated, then the default value is calculated at the main simulation grid, not on the close-out grid.}. The difference between $V(t)$ and $C(t)$ causes a residual exposure $[V (t)-C(t)]^+$ even if minimum transfer amounts and thresholds are zero, see for example \cite{Pykhtin2010}. This approach allows a detailed modelling of what happens in the close-out period by calculating the close-out values in different ways. ORE currently supports two options: 
%
\begin{itemize}
\item the close-out value can be computed as of default date, by just evolving the market from default date to close-out date (“sticky date”), or
\item the close-out value can be computed as of close-out date, by evolving both valuation date and market over the close-out period (“actual date”), i.e., the portfolio ages and cash flows might occur in the close-out period causing spikes in the evolution of exposures.
\end{itemize}

The option ``sticky date'' is more aggressive in that it avoids any exposure evolution spikes due to contractual cashflows that occur in the close-out period after default, the only exposure effect is due to market evolution over the period. The ``actual date'' option is more conservative in that it includes the effect of all contractual cash flows in the close-out period, in particular outgoing cashflows at any time in the period which cause an exposure jump upwards. A more detailed framework for collateralised exposure modelling is introduced in the article \cite{Andersen2016}, indicating a potential route for extending ORE.

\medskip On the other hand, in the {\em Lagged Approach} the simulation is conducted only on a default time grid. The collateral values are calculated, by delaying the delivery amounts between default times, specified by the {\em Margin Period of Risk} (MPoR) which leads to residual exposure. 

In table \ref{table:lagged}, we present a toy example to illustrate how the delayed margin calls lead to residual exposures. In this example, we assume that the default time grid is equally-spaced with time steps that match the MPoR (which is 1M). Further, we assume zero threshold and MTA. At the initial time, the delivery amount is $2.00$, which is the difference between the initial value of the portfolio and the default value at 1M. If this amount were settled immediately, then the collateral value would have been $10$ and hence the residual exposure would habe been zero at 1M. The delay of the delivery amount by MPoR implies a collateral value of $8.00$ until 1M and hence a residual exposure of $2$. 
%
\begin{table}[!ht]
    \centering
    \begin{tabular}{|p{1cm}|p{1.4cm}|p{1.5cm}|p{1.5cm}|p{1.6cm}|p{1.cm}|}
    \hline
        Time Grid & Default Value & Delivery Amount & Delivery Amount Delayed  & Collateral Value   & NPV  \\ \hline
         0 & 8.00 & 2.00 &   True &~ &  ~   \\ \hline
        1M & 10.00 & 5.00& True & 8.00&  10.00   \\ \hline
        2M & 15.00 & -3.00 & True & 10.00 & 15.00 \\ \hline
        3M & 12.00 & -3.00 & True & 15.00 & 12.00  \\ \hline
        4M & 9.00 & 5.00 & True & 12.00 & 9.00   \\ \hline
        5M & 14.00 & 6.00 & True & 9.00 & 14.00  \\ \hline
        6M & 20.00 &  ~  & ~ & 14.00 & 20.00  \\ \hline
    \end{tabular}
    \caption{Toy example for delayed margin calls.}\label{table:lagged}
\end{table}
%

Some remarks and observations:
\begin{itemize}
 \item {\em Lagged Approach}  has the disadvantage that we need to use equally-spaced time grids with time steps that match the MPoR. In the above example, let us assume that the MPoR is 2W. Then, delaying the first delivery amount by 2W would still imply a collateral value of $10.00$ at 1M and hence a zero residual exposure.
  \item In {\em Lagged Approach} approach, we support three calculation (settlement) types where the delay of the {\em Delivery Amount } depends on its sign. The above example corresponds to a ``symmetric'' calculation type where both positive and negative delivery amounts are settled with delay, see section \ref{sec:analytics} for other calculation types.
   \item In ORE, the {\em Close-out Approach} is the preferred method -and the {\em Lagged Approach} is the legacy method- to incorporate MPoR in the collateral model. 
\end{itemize}
\subsection{Exposure Allocation}\label{sec:app_allocation}

XVAs and exposures are typically computed at netting set level. For accounting purposes it is typically required to {\em
  allocate} XVAs from netting set to individual trade level such that the allocated XVAs add up to the netting set
XVA. This distribution is not trivial, since due to netting and imperfect correlation single trade (stand-alone) XVAs
hardly ever add up to the netting set XVA: XVA is sub-additive similar to VaR. ORE provides an allocation method
(labeled {\em marginal allocation } in the following) which slightly generalises the one proposed in
\cite{PykhtinRosen}. Allocation is done pathwise which first leads to allocated expected exposures and then to allocated
CVA/DVA by inserting these exposures into equations (\ref{CVA},\ref{DVA}). The allocation algorithm in ORE is as
follows:
\begin{itemize}
\item Consider the netting set's discounted $\NPV$ after taking collateral into account, on a given path at time $t$:
$$
E(t)=D(0,t)\,(\NPV(t)-C(t))
$$ 
\item On each path, compute contributions $A_i$ of the latter to trade $i$ as
$$
A_{i} (t) = \left\{ \begin{array}{ll} 
E(t) \times \NPV_{i}(t) / \NPV(t), & |\NPV(t)| > \epsilon \\
E(t) / n, & |\NPV(t)| \le \epsilon
\end{array}
\right. 
$$
with number of trades $n$ in the netting set and trade $i$'s value $\NPV_i(t)$.
\item The $\EPE$ fraction allocated to trade $i$ at time $t$ by averaging over paths:
$$
\EPE_i(t) = \E\left[ A_i^+(t) \right]
$$
\end{itemize}
By construction, $\sum_i A_i(t) = E(t)$ and hence $\sum_i \EPE_i(t) = \EPE(t)$.\\

We introduced the {\em cutoff } parameter $\epsilon>0$ above in order to handle the case where the netting set value
$\NPV(t)$ (almost) vanishes due to netting, while the netting set 'exposure' $E(t)$ does not. This is possible in a
model with nonzero MTA and MPoR. Since a single scenario with vanishing $\NPV(t)$ suffices to invalidate the expected
exposure at this time $t$, the cutoff is essential. Despite introducing this cutoff, it is obvious that the marginal
allocation method can lead to spikes in the allocated exposures. And generally, the marginal allocation leads to both
positive and negative $\EPE$ allocations.

\medskip As a an example for a simple alternative to the marginal allocation of $\EPE$ we provide allocation based on
today's single-trade CVAs
$$
w_i = \CVA_i / \sum_i \CVA_i.
$$
This yields allocated exposures proportional to the netting set exposure, avoids spikes and negative $\EPE$, but does
not distinguish the 'direction' of each trade's contribution to $\EPE$ and $\CVA$.

\subsection{Sensitivity Analysis}\label{sec:app_sensi}

ORE's sensitivity analysis framework uses ``bump and revalue'' to compute Interest Rate, FX, Inflation, Equity and Credit sensitivities to
\begin{itemize}
\item Discount curves  (in the zero rate domain)
\item Index curves (in the zero rate domain)
\item Yield curves including e.g. equity forecast yield curves (in the zero rate domain)
\item FX Spots
\item FX volatilities
\item Swaption volatilities, ATM matrix or cube 
\item Cap/Floor volatility matrices (in the caplet/floorlet domain)
\item Default probability curves (in the ``zero rate'' domain, expressing survival probabilities $S(t)$ in term of zero rates $z(t)$ via $S(t)=\exp(-z(t)\times t)$ with Actual/365 day counter)
\item Equity spot prices
\item Equity volatilities, ATM or including strike dimension 
\item Zero inflation curves
\item Year-on-Year inflation curves
\item CDS volatilities
\item Base correlation curves
\end{itemize}

Apart from first order sensitivities (deltas), ORE computes second order sensitivities (gammas and cross gammas) as well. Deltas are computed using up-shifts and base values as
$$
\delta = \frac{f(x+\Delta)-f(x)}{\Delta},
$$ 
where the shift $\Delta$ can be absolute or expressed as a relative move $\Delta_r$ from the current level, $\Delta=x\,\Delta_r$. Gammas are computed using up- and down-shifts
$$
\gamma = \frac{f(x+\Delta)+f(x-\Delta) - 2\,f(x)}{\Delta^2},
$$ 
cross gammas using up-shifts and base values as
$$
\gamma_{cross} = \frac{f(x+\Delta_x,y+\Delta_y)-f(x+\Delta_x,y) -f(x,y+\Delta_y) + f(x,y)}{\Delta_x\,\Delta_y}.
$$ 

From the above it is clear that this involves the application of 1-d shifts (e.g. to discount zero curves) and 2-d shifts (e.g. to Swaption volatility matrices). The structure of the shift curves/matrices does not have to match the structure of the underlying data to be shifted, in particular the shift ``curves/matrices'' can be less granular than the market to be shifted. 
Figure \ref{fig_shiftcurve} illustrates for the one-dimensional case how shifts are applied.
\begin{figure}[h]
\begin{center}
\includegraphics[scale=0.6]{shiftcurve.pdf}
\end{center}
\caption{1-d shift curve (bottom) applied to a more granular underlying curve (top). }
\label{fig_shiftcurve}
\end{figure} 

Shifts at the left and right end of the shift curve are extrapolated flat, i.e. applied to all data of the original curve to the left and to the right of the shift curve ends. In between, all shifts are distributed linearly as indicated to the left and right up to the adjacent shift grid points. As a result, a parallel shift of the all points on the shift curve yields a parallel shift of all points on the underlying curve.   \\

The two-dimensional case is covered in an analogous way, applying flat extrapolation at the boundaries and ``pyramidal-shaped'' linear interpolation for the bulk of the points. 

The details of the computation of sensitivities to implied volatilities in strike direction can be summarised as
follows, see also table \ref{sensi_config_overview} for an overview of the admissible configurations and the results
that are obtained using them.

\medskip
For {\em Swaption Volatilities}, the initial market setup can be an ATM surface only or a full cube. The simulation
market can be set up to simulate ATM only or to simulate the full cube, but the latter choice is only possible if a full cube is set
up in the initial market. The sensitivity set up must match the simulation setup with regards to the strikes (i.e. it
is ATM only if and only if the simulation setup is ATM only, or it must contain exactly the same strike spreads relative
to ATM as the simulation setup). Finally, if the initial market setup is a full cube, and the simulation / sensitivity
setup is to simulate ATM only, then sensitivities are computed by shifting the ATM volatility w.r.t. the given shift size and type and
shifting the non-ATM volatilities by the same absolute amount as the ATM volatility.

\medskip
For {\em Cap/Floor Volatilities}, the initial market setup always contains a set of fixed strikes, i.e. there is no
distinction between ATM only and a full surface. The same holds for the simulation market setup. The sensitivity setup
may contain a different strike grid in this case than the simulation market. Sensitivity are computed per expiry and
per strike in every case.

\medskip
For {\em Equity Volatilities}, the initial market setup can be an ATM curve or a full surface. The simulation market can
be set up to simulate ATM only or to simulate the full surface, where a full surface is allowed even if the initial market setup in an
ATM curve only. If we have a full surface in the initial market and simulate the ATM curve only in the simulation market, sensitivities
are computed as in the case of Swaption Volatilities, i.e. the ATM volatility is shifted w.r.t. the specified shift size
and type and the non-ATM volatilities are shifted by the same absolute amount as the ATM volatility. If the simulation
market is set up to simulate the full surface, then all volatilities are shifted individually using the specified shift size and type. In
every case the sensitivities are aggregated on the ATM bucket in the sensitivity report.

\medskip
For {\em FX Volatilities}, the treatment is similar to Equity Volatilities, except for the case of a full surface
definition in the initial market and an ATM only curve in the simulation market. In this case, the pricing in the
simulation market is using the ATM curve only, i.e. the initial market's smile structure is lost.

\medskip
For {\em CDS Volatilities} only an ATM curve can be defined.

\medskip
In all cases the smile dynamics is ``sticky strike'', i.e. the implied vol used for pricing a deal does not change if
the underlying spot price changes.

\begin{table}[hbt]
  \scriptsize
  \begin{center}
    \begin{tabular}{l | l | l | l | l | l}
      \hline
      Type & Init Mkt. Config. & Sim. Mkt Config. & Sensitivity Config. & Pricing & Sensitivities w.r.t. \\
      \hline
      Swaption & ATM & Simulate ATM only & Shift ATM only & ATM Curve & ATM Shifts \\
      Swaption & Cube & Simulate Cube & Shift Smile Strikes & Full Cube & Smile Strike Shifts\footnote{smile
                                                                          strike spreads must match simulation market configuration} \\
      Swaption & Cube & Simulate ATM only & Shift ATM only & Full Cube & ATM Shifts\footnote{smile is shifted in parallel\label{sensismileparallel}} \\
      \hline
      Cap/Floor & Surface & Simulate Surface & Shift Smile Strikes & Full Surface & Smile Strike Shifts \\
      \hline
      Equity & ATM & Simulate ATM only & Shift ATM only & ATM Curve & ATM Shifts \\
      Equity & ATM & Simulate Surface & Shift ATM only & ATM Curve & Smile Strike Shifts\footnote{result sensitivities
                                                                     are aggregated on ATM\label{sensiaggatm}} \\
      Equity & Surface & Simulate ATM only & Shift ATM only & Full Surface & ATM Shifts\textsuperscript{\ref{sensismileparallel}} \\
      Equity & Surface & Simulate Surface & Shift ATM only & Full Surface & Smile Strike Shifts\textsuperscript{\ref{sensiaggatm}} \\
      \hline
      FX & ATM & Simulate ATM only & Shift ATM only & ATM Curve & ATM Shifts \\
      FX & ATM & Simulate Surface & Shift ATM only & ATM Curve & Smile Strike Shifts\textsuperscript{\ref{sensiaggatm}} \\
      FX & Surface & Simulate ATM only & Shift ATM only & ATM Curve & ATM Shifts \\
      FX & Surface & Simulate Surface & Shift ATM only & Full Surface & Smile Strike Shifts\textsuperscript{\ref{sensiaggatm}} \\
      \hline
      CDS & ATM & Simulate ATM only & Shift ATM only & ATM Curve & ATM Shifts \\
    \end{tabular}
    \caption{Admissible configurations for Sensitivity computation in ORE}
    \label{sensi_config_overview}
  \end{center}
  \end{table}

\subsection{Par Sensitivity Analysis}
\label{app:par_sensi}

The ``raw'' sensitivities in ORE are generated in a computationally convenient domain (such as zero rates, caplet/floorlet volatilities, integrated hazard rates, inflation zero rates). These raw sensitivities are typically further processed in risk analytics such as VaR measures. On the other hand, for hedging purposes one is rather interested in sensitivities with respect to fair rates of hedge instruments such as Forward Rate Agreements, Swaps, flat Caps/Floors, CDS, Zero Coupon Inflation Swaps. \\

It is possible to generate par sensitivities from raw sensitivities using the chain rule as follows, and this is the approach taken in ORE. Recall for example the fair swap rate $c$ for some maturity as a function of zero rates $z_i$ in a single curve setting:
$$
c = \frac{1 - e^{-z_n\,t_n}}{\sum_{i=1}^n \delta_i\,e^{-z_i\, t_i}}
$$
More realistically, a given fair swap rate might be a function of the zero rates spanning the discount and index curves in the chosen currency. In a multi currency curve setting, that swap rate might even be a function of the zero rates spanning a foreign (collateral) currency discount curve, foreign and domestic currency index curves. Generally, we can write any fair par rate $c_i$ as function of raw rates $z_j$,
$$
c_i \equiv c_i(z_1, z_2, ..., z_n)
$$
This function may not be available in closed form, but numerically we can evaluate the sensitivity of $c_i$ with respect to changes in all raw rates,
$$
\frac{\partial c_i}{\partial z_j}.
$$
These sensitivities form a {\em Jacobi} matrix of derivatives. Now let $V$ denote some trade's price. Its sensitivity with respect a raw rate change $\partial V/\partial z_k$ can then be expressed in terms of sensitivities w.r.t. par rates using the chain rule
$$
\frac{\partial V}{\partial z_j} = \sum_{i=1}^n \frac{\partial V}{\partial c_i}\,\frac{\partial c_i}{\partial z_j},
$$
or in vector/matrix form
$$
\nabla_z V = C \cdot \nabla_c V, \qquad C_{ji} = \frac{\partial c_i}{\partial z_j}.
$$
Given the raw sensitivity vector $\nabla_z V$, we need to invert the Jacobi matrix $C$ to obtain the par rate sensitivity vector
$$
\nabla_c V = C^{-1} \cdot \nabla_z V.
$$

We then compute the Jacobi matrix $C$ by
\begin{itemize}
\item setting up par instruments with links to all required term structures expressed in terms of raw rates
\item ``bumping'' all relevant raw rates and numerically computing the par instrument's fair rate shift for each bump
\item thus filling the Jacobi matrix with finite difference approximations of the partial derivatives $\partial c_i/\partial z_j$.
\end{itemize}

The par rate conversion supports the following par instruments:
\begin{itemize}
\item Deposits
\item Forward rate Agreements
\item Interest Rate Swaps (fixed vs. ibor)
\item Overnight Index Swaps
\item Tenor Basis Swaps (ibor vs. ibor)
\item Overnight Index Basis Swaps (ibor vs. OIS)
\item FX Forwards
\item Cross Currency Basis Swaps
\item Credit Default Swaps
\item Caps/Floors
\end{itemize}


\subsection{Value at Risk}\label{sec:app_var}

For the computation of the parametric, or variance-covariance VaR, we rely on a second order sensitivity-based P\&L approximation

\begin{eqnarray}\label{taylorPl2}
  \pi_S & = & \sum_{i=1}^n D^i_{T_i}\,V\cdot Y_i 
        + \frac{1}{2} \sum_{i,j=1}^n D^{i,j}_{T_i,T_j}\,V\cdot Y_i\cdot Y_j
\end{eqnarray}

with 
\begin{itemize}
\item portfolio value $V$
\item random variables $Y_i$ representing risk factor returns; these are assumed to be multivariate normally distributed with zero mean
and covariance matrix matrix $C = \{ \rho_{i,k} \sigma_i \sigma_k \}_{i,k}$, where $\sigma_i$ denotes the standard
deviation of $Y_i$; covariance matrix $C$ may be estimated using the Pearson estimator on historical return data
$\{ r_i(j) \}_{i,j}$. Since the raw estimate might not be positive semidefinite, we apply a salvaging algorithm to
ensure this property, which basically replaces negative Eigenvalues by zero and renormalises the resulting matrix, see
\cite{corrSalv};
\item first or second order derivative operators $D$, depending
on the market factor specific shift type $T_i \in \{ A,R,L \}$ (absolute shifts, relative shifts, absolute log-shifts), i.e.
\begin{eqnarray*}\label{derivs}
  D^i_A \,V(x) &=& \frac{\partial V(x)}{\partial x_i} \\
  D^i_R \,V(x) = D^i_L f(x) &=& x_i\frac{\partial V(x)}{\partial x_i}
\end{eqnarray*}
and using the short hand notation
\begin{equation*}
  D^{i,j}_{T_i,T_j} V(x) = D^i_{T_i} D^j_{T_j} V(x)
\end{equation*}
In ORE, these first and second order sensitivities are computed as finite difference
approximations (``bump and revalue'').
\end{itemize}

To approximate the $p$-quantile of $\pi_S$ in \eqref{taylorPl2} ORE offers the techniques outlined below.

\subsubsection*{Delta Gamma Normal Approximation}
 
The distribution of \eqref{taylorPl2} is non-normal due to the second order terms. 
The delta gamma normal approximation in ORE computes mean $m$ and variance $v$ of the portfolio value change $\pi_S$ (discarding moments higher than two) following \cite{alexander} and provides a simple VaR estimate 
$$
VaR = m + N^{-1}(q)\,\sqrt{v}
$$
for the desired quantile $q$ ($N$ is the cumulative standard normal distribution). Omitting the second order terms in \eqref{taylorPl2} yields the delta normal approximation.
 
\subsubsection*{Cornish-Fisher Expansion}

The first four moments of the distribution of $\pi_S$ in \eqref{taylorPl2} can be computed in closed form using the covariance matrix $C$ and the sensitivities of first and second order $D_i$
and $D_{i,k}$, see e.g. \cite{alexander}. Once these moments are known, an approximation to the true quantile of $\pi_S$ can be computed using the Cornish-Fisher expansion, see also [7], which in practice often gives a decent approximation of the true value, but may also show bigger differences in certain configurations.

\subsubsection*{Saddlepoint Approximation}

Another approximation of the true quantile of $\pi_S$ can be computed using the Saddlepoint approximation using results from \cite{Lugannani} and \cite{Daniels}. This method typically produces more accurate results than the Cornish-Fisher method, while still being fast to evaluate.

\subsubsection*{Monte Carlo Simulation}

By simulating a large number of realisations of the return vector $Y=\{ Y_i \}_i$ and computing the corresponding
realisations of $\pi_S$ in \eqref{taylorPl2} we can estimate the desired quantile as the quantile of the empirical
distribution generated by the Monte Carlo samples. Apart from the Monte Carlo Error no approximation is involved in this
method, so that albeit slow it is well suited to produce values against which any other approximate approaches can be tested. Numerically, the simulation is implemented using a Cholesky Decomposition
of the covariance matrix $C$ in conjunction with a pseudo random number generator (Mersenne Twister) and an
implementation of the inverse cumulative normal distribution to transform $U[0,1]$ variates to $N(0,1)$ variates.

\end{appendix}

%========================================================
%\section{References}
%========================================================

\begin{thebibliography}{*}

\bibitem{ORE} \url{http://www.opensourcerisk.org}

\bibitem{QL} \url{http://www.quantlib.org}
 
\bibitem{QRM} \url{http://www.quaternion.com}

\bibitem{acadia} \url{http://www.acadia.inc}

\bibitem{quantlib-install} \url{http://quantlib.org/install/vc10.shtml}

%\bibitem{confluence} https://confluence.atlassian.com/bitbucket/set-up-git-744723531.html

\bibitem{git-download} \url{https://git-scm.com/downloads}

\bibitem{boost-binaries} \url{https://sourceforge.net/projects/boost/files/boost-binaries}

\bibitem{boost} \url{http://www.boost.org}

\bibitem{jupyter} \url{http://jupyter.org}

\bibitem{Anaconda} \url{https://docs.continuum.io/anaconda}

\bibitem{LO} \url{http://www.libreoffice.org}

%\bibitem{xlwings} \url{http://www.xlwings.org}

\bibitem{bcbs128} Basel Committee on Banking Supervision, {\em International Convergence of Capital Measurement and
    Capital Standards, A Revised Framework}, \url{http://www.bis.org/publ/bcbs128.pdf}, June 2006

\bibitem{bcbs189} Basel Committee on Banking Supervision, {\em Basel III: A global regulatory framework for more
    resilient banks and banking systems}, \url{http://www.bis.org/publ/bcbs189.pdf}, June 2011

\bibitem{d325} Basel Committee on Banking Supervision, {\em Review of the Credit Valuation Adjustment Risk Framework}, \url{https://www.bis.org/bcbs/publ/d325.pdf}, 2015

\bibitem{d424} Basel Committee on Banking Supervision, {\em Basel III: Finalising post-crisis reforms}, \url{https://www.bis.org/bcbs/publ/d424.pdf}, 2017

\bibitem{BrigoMercurio} Damiano Brigo and Fabio Mercurio, {\em Interest Rate Models: Theory and Practice, 2nd Edition},
  Springer, 2006.

\bibitem{Pykhtin2010} Michael Pykhtin, {\em Collateralized Credit Exposure}, in Counterparty Credit Risk, (E. Canabarro,
  ed.), Risk Books, 2010

\bibitem{PykhtinRosen} Michael Pykhtin and Dan Rosen, {\em Pricing Counterparty Risk at the Trade Level and CVA
    Allocations}, Finance and Economics Discussion Series, Divisions of Research \& Statistics and Monetary Affairs,
  Federal Reserve Board, Washington, D.C., 2010

\bibitem{Gregory12} Jon Gregory, {\em Counterparty Credit Risk and Credit Value Adjustment, 2nd Ed.}, Wiley Finance,
  2013.

\bibitem{Gregory15} Jon Gregory, {\em The xVA Challenge, 3rd Ed.}, Wiley Finance, 2015.

\bibitem{Lichters} Roland Lichters, Roland Stamm, Donal Gallagher, {\em Modern Derivatives Pricing and Credit Exposure
    Analysis, Theory and Practice of CSA and XVA Pricing, Exposure Simulation and Backtesting}, Palgrave Macmillan,
  2015.

\bibitem{Anfuso2016} Fabrizio Anfuso, Daniel Aziz, Paul Giltinan, Klearchos Loukopoulos, {\em A Sound Modelling and
    Backtesting Framework for Forecasting Initial Margin Requirements},
  \url{http://papers.ssrn.com/sol3/papers.cfm?abstract_id=2716279}, 2016

\bibitem{Andersen2016} Leif B. G. Andersen, Michael Pykhtin, Alexander Sokol, {\em Rethinking Margin Period of Risk},
  http://papers.ssrn.com/sol3/papers.cfm?abstract\_id=2719964, 2016

\bibitem{SIMM2.5A} ISDA SIMM Methodology, version 2.5A, (based on v2.5a) \\
  \url{https://www.isda.org/a/FBLgE/ISDA-SIMM\_v2.5A.pdf}

  % \bibitem{SIMM}{SIMM Methodology\\ \tiny
  %   http://www2.isda.org/attachment/ODM1Mw==/ISDA\%20SIMM\%20Methodology\_7\%20April\%202016\_v3.15\%20(PUBLIC).pdf}

  % \bibitem{SIMM_Data_Standards}{SIMM Risk Data Standards\\ \tiny
  %   https://www2.isda.org/attachment/ODQzMg==/Risk\%20Data\%20Standards\_24\%20May\%202016\_v1.22\%20(PUBLIC).pdf}

  % \bibitem{OO} http://www.openoffice.org

\bibitem{Andersen_Piterbarg_2010} Andersen, L., and Piterbarg, V. (2010): Interest Rate Modeling, Volume I-III
  
\bibitem{LichtersEtAl} Peter Caspers, Paul Giltinan, Paul; Lichters, Roland; Nowaczyk , Nikolai. {\em Forecasting Initial Margin Requirements – A Model Evaluation}, Journal of Risk Management in Financial Institutions, Vol. 10 (2017), No. 4, \url{https://ssrn.com/abstract=2911167}

\bibitem{corrSalv} R. Rebonato and P. Jaeckel, The most general methodology to create a valid correlation matrix for
  risk management and option pricing purposes, The Journal of Risk, 2(2), Winter 1999/2000,
  \url{http://www.quarchome.org/correlationmatrix.pdf}

\bibitem{alexander} Carol Alexander, Market Risk Analysis, Volume IV, Value at Risk Models, Wiley 2009

\bibitem{Lugannani} Lugannani, R.and S.Rice (1980), Saddlepoint Approximations for the Distribution of the Sum of
  Independent Random Variables, Advances in Applied Probability, 12,475-490.

\bibitem{Daniels} Daniels, H. E. (1987), Tail Probability Approximations, International Statistical Review, 55, 37-48.

\end{thebibliography}

\newpage
\addcontentsline{toc}{section}{Todo}
\listoftodos[Todo]
%\todos

\end{document}
